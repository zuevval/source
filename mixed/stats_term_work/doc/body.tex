\documentclass[main.tex]{subfiles}
\begin{document}
\section{Вступление}
В последние годы много внимания уделяется доступности общественных пространств, сервисов и информации для людей с ограниченными возможностями, в частности, для инвалидов по зрению. Чтобы лучше планировать и организовывать лечение, производство специальной продукции, обучение и рабочие места для лиц с нарушениями зрения, нужно оценивать количество людей с нарушениями зрения, прогнозировать в ближайшие годы их число, а также возраст, возможности и потребности. В частности, интересны следующие вопросы:
\begin{enumerate}
	\item Влияет ли экономическое состояние региона на статистику офтальмологических заболеваний и инвалидности по зрению?
	\item Насколько полно официально зарегистрированные случаи заболеваний  описывают реальную картину в странах с низким уровнем жизни?
\end{enumerate}
Эти вопросы изучаются в данной работе методом сопоставления экономических показателей регионов Земли с данными о нарушениях срения.

\newpage
\section{Постановка задачи}
Из различных источников собраны данные по странам и регионам:
\begin{enumerate}
	\item Внутренний валовой продукт (в миллионах долларов США) (далее сокращённо ВВП) \cite{gdp}.
	\item Уровень занятости населения ($\%$) \cite{productivity-loss}
	\item Число людей, которые имеют работу \cite{productivity-loss}
	\item Суммарное число инвалидов по зрению в регионе \cite{who-health-life-estimates}
	\item Доля полностью незрячих по отношению ко всему населению, распределённая по возрастным группам: дети до 15 лет, от 15 до 50 лет и лица старше 50 лет \cite{who-vi-2002}
	\item Число полностью незрячих, также распределённое по возрастным группам \cite{who-vi-2002}.
\end{enumerate}
Все данные были собраны в 2000-2020 годах, поэтому можно считать их актуальными, относящимися к одному и тому же периоду развития  стран мира.\\
Необходимо сопоставить экономические данные с данными о заболеваниях, проверить наличие зависимостей и интерпретировать полученные результаты.

\newpage
\section{Теория}
При оценке зависимостей между данными были применены модели линейной регрессии $y=a+bx$. Формулы для коэффициентов были вычислены двумя способами: более точным методом наименьших квадратов и более устойчивым к выбросам, но грубым методом наименьших модулей \cite{sevastianov}.\\
Коэффициенты линейной регрессии по методу наименьших квадратов:
\begin{gather}
\label{eq:least_sq_b}
b = \frac{\overline{xy} - \overline{x}\cdot\overline{y}}{\overline{x^2}-(\overline{x})^2}\\
\label{eq:least_sq_a}
a = \overline{y}-b\overline{x}
\end{gather}
Оценки коэффициентов по методу наименьших модулей:
\begin{gather}
\label{eq:least_mod_b}
b = r_Q \frac{q_y}{q_x}\\
\label{eq:least_mod_a}
a = med(y)-b med(x)
\end{gather}
где $r_Q=\frac{1}{n} \sum_{i=1}^{n} sign(x_i - med(x))sign(y_i-med(y))$, $n$ -- мощность выборки;
\begin{equation}\label{eq:interquartile}
q_x = \frac{x_{(\lceil\frac{3n}{4}\rceil)} - x_{(\lceil\frac{n}{4}\rceil)}}{k_q(n)}
\end{equation}
$k_q(n)$ -- параметр, зависящий от мощности выборки, значение которого нам неважно, поскольку при подстановке в формулы для коэффициентов линейной регрессии он сокращается.

\newpage
\section{Реализация}
Для обращения с данными были использованы CSV-таблицы и инструментарий Ms Office Excel (группировка данных, расчёт среднего). Так цифры из разных источников с разными категориями данных были обобщены в одной таблице (\ref{link:data}), где строки -- регионы, которые принято выделять в отчётах Всемирной организации здравоохранения (Африка D и E, Америка A, B, D и так далее). Данные о ВВП в открытых источниках были найдены только по странам, поэтому был написан скрипт на языке Python, который по списку стран и их ВВП, а также по списку регионов и входящих в них стран распределил ВВП по регионам. Страны, которые были по-разному обозначены в списках стран и регионов (например, в одном списке Россия -- Russia, в другом - Russian Federation - всего 33), были выделелены этим же скриптом и затем их ВВП прибавлен к ВВП нужных регионов вручную.

Программа для построения регрессионной зависимости по формулам \eqref{eq:least_sq_b}, \eqref{eq:least_sq_a}, \eqref{eq:least_mod_b}, \eqref{eq:least_mod_a} и визуализации данных написана на языке MATLAB. Использованы встроенные в язык функции сортировки массивов \textit{sort}, вычисления медианы \textit{median} и среднего \textit{mean}.  

Ошибка построенной модели, т. е. массив данных $\{\hat y(x_i) - y_i\}$, был всякий раз подвергнут проверке на согласие с моделью нормального распределения с помощью критерия $\chi^2$ встроенной в язык MATLAB функцией \textit{chi2gof} с уровнем значимости $\alpha=0.05$. Как оказалось, всякий раз распределение ошибки таково, что нулевая гипотеза (ошибка распределена по нормальному закону) не может быть отвергнута (следовательно, построенная модель имеет смысл).\\

\newpage
\section{Результаты}
Построены графики зависимости числа незрячих от числа всех людей с нарушениями зрения, зависимости доли инвалидов по зрению среди всего населения от уровня занятости населения (в трёх возрастных категориях), а также графики зависимости числа незрячих от числа работающих людей (также в трёх категориях).\\
Для тех данных, которые были приведены по регионам не в абсолютном, а в процентном отношении, составлена таблица средних значений и дисперсий \ref{table}.  

\subfile{results}


\section{Выводы}
На графике зависимости числа незрячих от числа лиц с нарушениями зрения \ref{img:blind_vi} отчётливо видно, что эти два показателя прямо пропорциональны, поэтому для оценок зависимости от других показателей можно с равным успехом использовать как число полностью незрячих, так и общее число людей с нарушениями зрения; как долю полностью незрячих, так и общую долю людей с заболеваниями зрительного аппарата. В настоящей работе используются данные о нарушениях зрения.\\
Как и ожидалось, число незрячих в возрасте от 15 до 50 лет пропорционально числу занятых в данном регионе (с двумя серьёзными выбросами -- в регионе <<Юго-восточная Азия>>, куда входят Индия, Бангладеш, Северная Корея и другие страны; Западный Тихоокеанский регион, куда входят Китай, Южная Корея и другие страны), и оценка по методу наименьших модулей даёт линию, которая правдоподобно описывает тренд. Это легко объяснить: чем больше в регионе молодого, трудоспособного населения, тем больше занятых и при том тем больше среди молодых незрячих. В двух других возрастных группах (\ref{img:blind_lfpr_0_15}, \ref{img:blind_lfpr_50_inf}) нет такой отчётливой зависимости.\\
Аппроксимация линейной функцией зависимости доли незрячих от уровня занятости населения не может быть названа аккуратной, но можно сказать наверняка, что в регионах с более высоким уроврем занятости доля незрячих ниже. Уровень занятости можно считать показателем экономического благополучия страны, поэтому результаты можно интерпретировать так: в регионах с более развитыми экономиками уделяется больше внимания заботе о здоровье, предупреждению и лечению заболеваний, люди трудятся в более благоприятных условиях и не травмируют зрение. Большой разброс точек в регионах с низким уровнем занятости можно объяснить проблемами сбора информации: в некоторых странах (в частности, в России) сложно и иногда даже невыгодно объявить себя <<легально незрячим>>.\\
Можно заметить, что число и доля незрячих с возрастом увеличиваются, что также согласуется с ожиданиями: основные причины слепоты -- глаукома и катаракта -- проявляются уже у пожилых людей.
\end{document}