\documentclass[main.tex]{subfiles}

\subsection{Melnikova Anna}
\begin{document}
	Пусть у нас несколько типов отказов, но в каждый момент времени происходит только отказ одного типа (они взаимоисключающие).
	Нас интересует совместное распределение времени жизни и отказа.
	
	Бывают следующие типы наблюдений:
	\begin{enumerate}
		\item ...
		\item ...
		\item $ { T > t, b = j } $ -- отказ произошёл позже, и его тип 
		\item Известно только точное время жизни, тип отказа неизвестен
	\end{enumerate}

Обозначим $ S(t) = P(T > t) $ -- функция выживаемости.
Условная вероятность того, что отказ имеет тип $ j $ при условии того, что отказ случился, $ \delta_j(t) = P(b=j | T=t) $

Вариация EM-алгоритма:
пусть $ t_0 = 0 < t_1 < t_2 < \dots < t_k < t_{k+1} = \infty $ -- различные нецензурированные времена отказов, присутствующие в наших данных.

Каждому интервалу $ [t_s; t_{s+1}] $ соответствуют числа $ TODO $, соответствующие числу отказов каждого типа (в цензурированных данных!).

Два шага: E- и M-шаг.

\begin{enumerate}[noitemsep]
	\item Выбрать изначальное значение $ \hat \phi $ (в нашем случае из равномерного распределения)
	\item Для выбранного $ \hat \phi $ оценить $ a_{jk} $
	\item Для оценки $ \hat a_{jk} $ оценить $ \phi $
	\item Если достигнут критерий остановки, остановиться, иначе вернуться на второй шаг.  
\end{enumerate}

\subsection{Knodel Julia}

\textbf{Confidence Bands for a survival curve from censored data}
(доверительные полосы для функций выживаемости по цензурированным данным)

Считаем полосу по оценке Каплана-Мейера.
В случае отсутствия цензурирования полоса сходится к полосе Колмогорова.

Модель:

\begin{itemize}[noitemsep]
	\item $ S^0 = 1 - F^0 $
	\item $ X_1^0, \dots, X_N^0 \sim F^0 $ -- независимые одинаково распределённые истинные времена жизниж $ F $ -- реальная наблюдаемая по данным функция выживаемости. 
	\item $ TODO  $
\end{itemize}

Есть оценка Каплана-Мейера.
В полных данных на этом месте эмпирическая оценка распределения.
Распределение Колмогорова связано о статистикой Колмогорова, а она связана с эмпирической функцией распределения...
если вспомнить некоторые теоретические результаты, которые приводят нас к оценке Каплана-Мейера.

Ширина полосы в зависимоси от глубины цензурирования, $ N=500 $...

q: почему Вейбулла?
- Можно всё, что угодно.
Вейбулла красиво.

\end{document}