\documentclass[main.tex]{subfiles}
\begin{document}

\section{Экспоненциальное распределение вероятности отказа}

8 сентября 2022 г.
	
Экспоненциальное рпспределение:

\[ S(t) = e^{ - \alpha t }, \alpha > 0, t \ge 0 \]

$ \lambda(t) = \alpha; \quad T_{\text{ср}} = \frac{1}{\alpha} $

Выживаемость на интервале $ [\tau; \tau + t] $

\[ S(t | \tau) =  \]

Это свойство экспоненциального распределения, называемое \textit{отсутствием последействия}.
Т. о. экспоненциальное распределение есть идеализированная модель внезапного отказа, при котором устройство не стареет. \\

\textbf{Второе свойство:} экспоненциальное распределение сохраняется при формировании последовательной системы независимых элементов.

Пусть $ T $ -- время жизни всей системы из последовательных независимых элементов, $ T_i $ распределены экспоненциально с параметрами $ \alpha_i > 0, i = 1, \dots, n $.
Тогда

\[ S(t) = P(T > t) = P(T_1 > t, \dots, T_n > t) = \prod_{i=1}^n P(T_i > t) = e^{- \sum \alpha_i t} \]

\textbf{ Третье свойство: }

Рассмотрим $ T_1, \dots, T_n $, распределённых с одним параметром $ \alpha > 0 $, $ T = T_1 + \dots + T_n $.
Тогда

\[ f(t) = \frac{\alpha^\beta t^{\beta - 1}}{\Gamma(\beta)} e^{ -\alpha t }, t > 0 \]
Это гамма-распределение, если же $ \beta = n \in \mathds N $, то \textit{распределение Эрланга}.

Часто работали с \emph{распределением Вейбулла}: $ S(t) = e^{- \alpha t^\beta} $, $\alpha > 0, \quad \beta > 0, \quad t \ge 0$

Распределение Вейбулла есть предельный закон для такой случайной величины: минимум большого числа неотрицательных (нормальных?) случайных величин. \\

Логарифмически нормальное распределение:
\begin{gather*}
\ln T \sim \mathcal N(a, \sigma), \quad a \in \mathds R^1, \sigma > 0 \\
T \sim  f_T (t) = \frac{1}{ \sqrt{ 2 \pi } \sigma t } \cdot e^{ - \frac{(\ln t - a )^2}{2 \sigma^2}}, t > 0
\end{gather*}

\textbf{Распределение с параметром масштаба:}

\begin{gather*}
	S(t, \alpha) = S(\alpha t) \\
	f(t; \alpha) = \alpha f(\alpha t) \\
	\lambda (t; \alpha) = \alpha \lambda (\alpha t)
\end{gather*}

\textbf{ Семейство распределений Лемана } (распределения с пропорциональными рисками): их будем использовать в моделях влияния факторов на выживаемость.

\begin{gather*}
	S(t; \alpha) = [S(t)]^\alpha, \quad \alpha > 0 \\
	f(t; \alpha) = \alpha [S(t)]^{\alpha - 1} f(t) \\
	\lambda (t, \alpha) = \alpha \lambda (t)
\end{gather*}

Это определение семейства, не свойства.
Говорят, что $ S(t) $ -- порождающая функция семейства.

\section{Неполные данные. Цензурирование}

Какие типы неполных данных могут возникать?

\begin{enumerate}[noitemsep]
	\item Пропущенные данные (missing data), прежде всего многомерные.
	Отдельные признаки могут выпадать у какого-то объекта.
	
	По этому поводу есть целые книжки (например, статистический анализ данных с пропусками -- Р.Д. Литтл и Д.Б. Рубин, "Финансы и статистика", 1991; или книга Н.Г. Загорулько).
	Мы пропущенными данными заниматься не будем.
	
	\item Связанные данные, или совпадения (ties, tied data).
	
	Распределение времени жизни непрерывно, поэтому теоретически времена жизни или времена цензурирования не должны совпадать.
	Но в силу округления или другой неявной группировки иногда в данных появляются одинаковые.
	
	Иногда это не страшно (связанные данные не влияют на результат статистического анализа), но иногда нужна модификация процедур.
	
	\item Группировка данных (grouped data)
	
	\item Наблюдения из \emph{ усечённых распределений } (truncated distribution): пусть по тем или иным причинам величина не может наблюдаться в части своих значений.
	
	Например, есть круговая мишень радиуса $ R $ и нас интересует распределение отклонений попавшей пули от центра.
	Пули, попавшие мимо мишени, не регистрируются.
	Это \emph{распределение, усечённое справа}. \\
	
	Распределение, усечённое с двух сторон:
	
	\[ F_X(x | a \le X \le b) : P(X \le x | a < X \le b) = \begin{cases}
		0, \quad x \le a \\
		\frac{F(x) - F(a)}{ F(b) - F(a) }, \quad a < x \le b \\
		1, \quad x > b
	\end{cases} \]
\end{enumerate}

% TODO some (laptop failed)

Цензурирование справа.

Цензурирование слева возникает, когда наблюдение начинается не с самого начала отсчёта времени и некоторые объекты отказали раньше (когда точно -- не знаем).

Цензурирование интервалом: наблюдаем в каком-то промежутке $ [\tau_3; \tau_4] $

В следующий раз рассмотрим общую модель цензурирования справа.
В ней цензурирование не фиксированное, случайное.

\end{document}