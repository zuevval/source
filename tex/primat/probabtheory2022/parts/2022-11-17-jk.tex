\documentclass[main.tex]{subfiles}
\begin{document}
\subsection{Julia Knodel}

\textbf{Доверительная полоса для функции выживаемости}

Колмогоровская полоса: теорема Колмогорова даёт асимптотическую доверительную полосу для \emph{цензурированной} выборки.

Распределение исходное (нецензурированных данных)
\[ X_1^0, \dots, X_N^0 \sim F^0 \]

Доверительная полоса для функции выживаемости ($ \hat S^0_N $ -- оценка Каплана-Мейера, $ S^0 \equiv 1 - F^0 $).

\[ \left\{ \hat S_N^0(t) - \lambda D_n(t) \le S^0(t) \le S^0(t) \le \hat S_N^0(t) + \lambda D_N(t) \forall 0 \le t \le T \right\} \to G_a(\lambda) > G(\lambda) \]

Обозначим для некой функции распределения $ G $
\[ T_G \overset{def}= \inf \{ t \ge 0 | G(t) = 1 \} < \infty \]

\subsection{Anastasia Protsvetkina. Interval censoring}

Пусть наши наблюдения -- фиксированные времена обследования, в которые мы узнаём, наступил отказ или нет.

Наблюдения, в которых мы не наблюдали отказ, не несут статистической информации (кроме последнего).

Время отказа: в непараметрическом подходе такой алгоритм (Turnbull)

$ 0 < \tau_0 < \dots < \tau_m $ -- (sorted) grid including all $ L_i $, $ U_i $. 

\[ \alpha_{ij} \overset{def}= \mathds 1_{(\tau_{j-1}, \tau_j) \subseteq (L_i, U_i] } \]

Полупараметрический подход: можно оценивать интенсивность отказа или функцию выживаемости

Модель ускоренных испытаний, пропорциональных интенсивностей или пропорциональных шансов (первая всё же часто используется при полностью параметрическом подходе).

Полностью параметрический подход:
	
\end{document}