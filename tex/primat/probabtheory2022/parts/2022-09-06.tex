\documentclass[main.tex]{sufbfiles}
\begin{document}
\section{ Вводное занятие, Анализ типа времени жизни }
Sept 06, 2022

Спецификой данных является \textit{неполнота}.

Список литературы есть, НО курс выстраивался в основном не по книжкам, а по результатам анализа периодической литературы.

Книга Кокса и Оукса: полезна для тех, кто активно практикует методы, но курс по ней строить невозможно.

\subsection{Введение в анализ выживаемости}
Survival (data) analysis:

\begin{enumerate}[noitemsep]
	\item Экономика: период безработицы, отказ = человек снова вышел на работу
	\item Медицина: период ремиссии, отказ = возвращение заболевания
	\item Техника: отказ прибора
	\item Техника: длина пробега автомобиля до первой поломки
	\item Техника: пробег ткацкого станка до обрыва нити
\end{enumerate}

Можно обобщить задачу: рассчитывать время не только до первого события, но и до следующих

Мы рассматриваем первую, не обобщённую задачу. \\

Такие данные можно было бы обрабатывать методами классической статистики, НО у нас есть неполные данные: мы можем не для всех точек наблюдать время отказа.

Какие задачи можно поставить?
\begin{enumerate}[noitemsep]
	\item Построение оценок (точечных и интервальных)
	\item Сравнение выживаемости в разных выборках
	\item Сравнение распределения выживаемости с теоретическим распределением
	\item Исследование влияния факторов на выживаемость
\end{enumerate}

Показатели выживаемости -- функции, которые полностью описывают процесс наступления отказа.
Мы рассмотрим 4: функция распределения, функция выживаемости, функция накопленного риска (интегральные показатели); плотность распределения, функция интенсивности отказов (ФИО) (дифференциальные показатели).

\begin{enumerate}
	\item $ S(t) = 1 - F(t) $ -- функция выживаемости (Survivor function), $ t \ge 0 $.
	Это вероятность того, что время жизни $  T \ge t $: $ F(t) = \mathds P (T \le t) $, $ S(t) = \mathds P (T > t) $
	
	Функция выживаемости может быть интерпретирована также как средняя доля не отказавших к моменту $ t $ элементов выборки.
	
	\item Плотность распределения может быть интерпретирована как относительная частота наступления отказов.
	
	\[ f(t) = \frac{dF(t)}{dt} = \frac{dS(t)}{dt} \]
	
	Связь:
	
	\[ T_{\text{ср}} = M[T] = \int_0^t t \cdot f(t) dt \overset{f(t) dt = - dS(t)}= - tS(t)|_0^\infty + \int_0^\infty S(t) dt \]
	
	Этот интеграл должен абсолютно сходиться.
	
\end{enumerate}

Вероятность наступления отказа до момента $ t $ при условии, что отказ произошёл после $ \tau $:

\[ F(t | \tau) = \frac{P(\tau < T \le \tau + t)}{P(T > \tau)} = \frac{F(\tau + t) - F(\tau)}{S(\tau)} = \frac{1 - S(\tau + t) - 1 + S(\tau)}{S(\tau)} = 1 - \frac{S(\tau + t)}{ S(\tau) } \]

\subsubsection{Интенсивность отказов}

\[ \lambda(\tau) \overset{def}= \lim_{t \to 0} \frac{F(t | \tau)}{t} = \left[ \lim_{t \to 0} \left. \frac{S(\tau) - S(\tau + t)}{ t } \right] \right|_{S(\tau)} = \frac{S'(\tau)}{S(\tau)} = - \frac{d}{d\tau} \ln S(\tau) \]

\[ \int_0^t \lambda(\tau) d \tau = - \ln S(t) \]

Интенсивность отказов широко используется в анализе данных типа времени жизни.

Если функция интенсивности в какой-то точке велика, значит, надо быть в этом месте особенно осторожным: вероятен отказ.
Если же функция интенсивности отказа затем стремится к нулю, значит, после успешного прохождения "опасного" участка без отказа возможно "выздоровление".

Функция интенсивности отказов не нормирована, интеграл от неё может и расходиться.

Функция интенсивности отказов может иметь в технике $ U $-образную выпуклую вниз форму.

\subsubsection{ Функция риска }

Hazard rate -- функция риска

\[ H(t) \overset{def}= \int_0^t \lambda(\tau) d \tau \]

т. о. $ H(t) = - \ln S(t) $

\begin{leftbar}
	Hjort (Хьорт) -- норвежский математик, который это исследовал
\end{leftbar}

\end{document}