\documentclass[main.tex]{subfiles}
\begin{document}
\section{Лекция 3}
14 октября 2021 г.

\subsection{Обсуждение работы Даны}

Пароли лучше не вставлять в код, а 

Валидатор: либо использовать joy, либо ajv (последнее -- json-схема).
Есть и другие, но они менее подходят, а эти примерно равны.
Валидатор -- это некая фукнция, которая содержит внутри все правила валидации.

Manager шлёт response: это не очень хорошо.
Логика представления (это $\approx$view) должна быть отделена от контроллера.
Тогда можно будет легко заменить один фреймворк другим.

Ошибка Unhandled Promise Rejection: это тоже не должно быть.
Стиль с \texttt{then} допустим, но лучше всё завернуть в \texttt{asyncAwait}, это в \texttt{handler}, это в обработчик.

Вопрос: где посмотреть стиль с \texttt{asyncAwait}?
Ответ: есть специальная библиотека, которая все middleware оборачивает в middleware catchers, и те либо возвращают что-то, либо посылают ошибку, которая ловится.

Почитайте про H<...>-ссылки.
Это для более удобной навигации по API.

\subsection{Обсуждение работы Василисы}

Миграции содержат up- и down- методы.

\texttt{new Array} -- так не пишут.
Если хотите изучить синтаксис, то курс на Stepik -- самая-самая база, дальше смотрите learn.javascript.

\begin{leftbar}
	зарегистрироваться на сайте Elephant или запускать через Docker;
	
	смотреть TODO в коде
	
	Следующие 6 человек: данные в базе и есть разделение на слои (req и res должны уходить ниже роутера).
	
	TODO ~ в ПН написать Владиславу -- напомнить прислать ссылку про разделение по слоям.
\end{leftbar}

\end{document}
