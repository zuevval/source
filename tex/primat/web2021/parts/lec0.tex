\documentclass[main.tex]{subfiles}
\begin{document}
\setcounter{section}{-1}

\section{Вводная лекция}

Это предмет по выбору, можно выбирать Python либо JS.

По пятницам в 10 утра -- не вариант, максимум, что можем попробовать -- в пятницу вечером.

<<Я не просто так поставил в табличке числа от 1 до 24>>.
<<Я хороший менеджер, руководитель, но плохой преподаватель>>.
Владислав Ломако приехал из Рязани, из университета, команда из которого единственная в России, кроме ИТМО, выигрывала однажды ICPC.
Работает в EPAM.

<<Вы видели мой план; вряд ли туда добавится что-то новое>>.
<<Лекции проходят так: я даю материалы по какому-то разделу (вроде обзорной лекции) -- даю релевантные материалы, которые вы сами изучаете>>.
В лучшем случае <<воркшоп>> -- что-то показываю на экране.

Точек контроля будет не так много.


Курс отсёкся от курса Новикова по интернет-технологиям.

<<Есть умельцы, которые проходят проект и по Python, и по JavaScript>>.

Отличие между двумя курсами -- в лекциях.

<<Дам вам ещё выходные на то, чтобы определиться. Владислав обычно смотрит на то, чтобы пройденные темы вы использовали в проекте (например, лямбда-функции); >>

План: 16 лекций в год (в среднем 2 лекции в неделю).

<<В Python в Web разрабатывают только из-за того, что на нём разрабатывать просто>>.

\begin{leftbar}

    План первого занятия по Python:

    \begin{enumerate}[noitemsep]
        \item TCP/IP + HTTP/HTTPS + API + RESTfull API + cURL + wget + socket  (HW - socket server via bash, GET and POST methods)
        \item Apache/Nginx  + SwaggerUI (HW wrap socket server with swagger + Nginx)
        \item Postman/Chrome devtools + exploit, CORS, OpenID, OAuth 1.0, OAuth2.0 + reverse engineering (HW crack the instagram API, make the feed post automatically \url{https://github.com/facebookarchive/python-instagram})
    \end{enumerate}

\end{leftbar}

\end{document}
