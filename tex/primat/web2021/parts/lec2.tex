\documentclass[main.tex]{subfiles}
\begin{document}

\section{Лекция 2}

8 октября 2021 г.

Что касается эксплойтов: можно прямо в консоли исполнять JS.
Идём на сайт <<цифрового прорыва>>.

Когда мы находимся на каком-то сайте, нам доступно всё, что есть.
Можно начать воровать данные.
Например, если там есть jQuery, можно попросить любую переменную; можно попросить все атрибуты класса.
\begin{verbatim}
	// TODO
\end{verbatim}

\subsubsection{Postman}

Можно профилировать запросы (создавать макросы, скрипты по проверке запроса).

\subsubsection{OpenID}

Это идеология того, как приложения должны взаимодействовать друг с другом с точки зрения авторизации пользователя.
OpenID (точнее, OAuth 2.0) -- своего рода стандарт  (как и \texttt{SAML}).

\subsubsection{Остальные ссылки по теме эксплойтов и картинка}


% TODO some

Токены авторизации OAuth 2.0 -- о них можно поговорить подробнее.

Что такое авторизация?

Сервер (бэкенд) общается с клиентом (фронтендом).
Как защитить данные, которые
Не хотим хранить пароль во фронтенде.
Токен -- некий секретный ключ, связанный с аккаунтом.
Он генерируется при авторизации по паролю и хранится во фронтенде.

Все поля, доступные для редактирования, должны быть экранированы от исполнения кода.
Иначе можно, к примеру, исполнить код, который передаст куда-то данные, хранящиеся в \texttt{local\_storage}
На сайте цифрового прорыва текстовые поля экранированы, но нехорошо

Правильно хранить токены в специальном разделе в cookies, который доступен только при авторизации по OAuth-протоколу.

Есть множество видов токенов, самые распространённые -- JWT (JSON web token).

% TODO a bit

Есть несколько способов авторизации.

JWT -- base-64-закодированная строка.
% TODO a bit

Есть SHA256-токены.

SHA-256: 

Наш ключ может 

Токены обычно живут час или меньше


\subsubsection{Вопрос про oAuth}


\end{document}
