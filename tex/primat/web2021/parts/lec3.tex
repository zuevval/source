\documentclass[main.tex]{subfiles}
\begin{document}

\section{Лекция 9 ноября}

Пример проекта: укорачиватель ссылок

В HTML-тегах есть теги og\_image, og\_title, og\_description

Нам захотелось, чтобы можно было ссылки на Zoom прикреплять

В базе данных есть табличка ключ-значение и метод, который по http возвращает html-страницу (пустую, но с тегами)

Если нет идеи, приходить в Телеграме к Даниилу Александровичу и спрашивать.

Можно делать телеграм-бота.

techempower

FastAPI в 3 раза

"C++ мы отсюда выкидываем, это читерство, никто никогда не будет делать веб-приложение на C++ -- очень долго и очень дорого".
NodeJS

Рекомендую почитать и поинтересоваться про GIL.

WSGI: Flask -- GUNICORN, FastAPI -- uvcorn.
Они, получая запросы от NGINX, начинают распихивать их на разные копии нашего приложения (иными).
HTTP-запросы не выполняются последовательно.
Каждый раз, когда приложение обрабатывает запрос, он стирает контекст (MVC-фреймворк).
Брокером запросов между приложениями выступает база данных.

Загрузка видео -- это не совсем HTTP.
Там сокетный сервер, где одновременно открывается много соединений.
WhatsApp:  написан на RLang.
Позволяет держать на 32-ядерной машине миллион открытых сокетных соединений.
Это приводит к новым ограничениям

Всё, что больше 40 МБ, лучше передавать через сокеты.
В браузере для этого есть FTP-протокол.

\begin{leftbar}
Ссылка 13 в списке должна стать вашей настольной книгой при разработке проекта.
\end{leftbar}

\subsection{Часть 2 лекции. Реляционные базы данных}

Redis: это не реляционная БД, а таблица "ключ-значение", которая существует только в оперативной памяти.
UDB

Мы рассматриваем реляционные БД.

Сейчас всё быстро меняется.
За год всё стало иначе.
FastAPI: за ним, уже очевидно, будущее веб-разработки на Python.

\end{document}
