\documentclass[main.tex]{subfiles}
\begin{document}

\section{Вводная лекция}

Sept 29, 2021

Перепробовали разные форматы.

Сейчас у нас всё онлайн... Будем действовать по схеме прошлого года, которая зарекомендовала себя довольно хорошо - две части, для первого и второго семестра.

Сама задача описана вот здесь (\texttt{SPBSTU\_Task.docx}).
Здесь всё описано, что не описано, можно делать по своему усмотрению.
Не на фреймворках, всё вручную.

Задание очень маленькое и подразумевается, что вы сделаете всё и хорошо.
Обратите внимание: оценивается именно правильность, а не объём работы.
\begin{enumerate}[noitemsep]
    \item Надо пройти курс на Stepik!
    Обязательно пройти тест.
    Для справки можно почитать \href{https://learn.javascript.ru}{learn.javascript.ru}.
    \item Нововведение этого года: надо решить несколько задач в системе \texttt{Autocode}. "Я думаю, что задач накидаю или вы сами предложите, обсудим".
    Наверное, одна задача в этом семестре, одна в следующем.
\end{enumerate}

По четвергам в шесть часов будут пары, полноценная пара.

\href{https://learn.javascript.ru}{learn.javascript.ru} "В лекциях я лучше ничего рассказать не смогу, чем эта книга"

<TODO> Пункт (6) в списке - на английском, но на понятном, простом. Рекомендую посмотреть (длится около часа).

Node JS: <TODO>
Проверка стиля: плагин, см. ссылку.
nodemon: автоматическая перезагрузка NodeJS

На ближайшее время: рекомендую начать со Степика и начать делать задание по ноде.
Каждое занятие мы будем смотреть работы кого-то из вас по очереди (по алфавиту, чтобы не путаться).

"Посещаемость не проверяется, поэтому можно не ходить. Минимум на тройку в первом семестре: stepik, 1 задача на autocode; 'в самом приложении мы тесты не прикручиваем, поэтому autocode полезно' + приложение на Node: простое <TODO>"

"степик проходится за выходные, автокод тоже; за 4-5 дней у вас гарантированная тройка".

Что касается тех, кто будет ходить и что-то делать: к вам требования построже; вам нужно походить

\begin{enumerate}[noitemsep]
    \item Зарегистрироваться на Stepik
    \item Установить Node, БД Express, и чтобы он что-то вам выдавал (желательно - список сотрудников).
\end{enumerate}



"По моим расчётам, чтобы получилось что-то хорошее, нужно уделять курсу 2-3 часа в неделю каждую неделю. Если нет этого времени, стоит задуматься о более простом варианте".

В следующий раз первые 5 человек показывают. Если Вы понимаете, что не сможете рассказать, сообщайте заранее.

\end{document}
