\documentclass[main.tex]{subfiles}
\begin{document}
\section{ Обобщённые линейные модели}
Nov 16, 2021

Одно из возможных обобщений классической модели.

\subsection{Введение в GLM}

Классическая регрессионная модель подразумевает нормальное распределение ошибки измерения наблюдаемой характеристики.
Предполагается также независимость наблюдений.

Но приходится работать с моделями, которые не укладываются в классическую модель.
Рассмотрим модель с предположением о нарушении нормальности.
В реальных наблюдениях могут встречаться категориальные переменные, ординальные...
Если есть бинарная с.в. $ Y \in [0;1] $, классическая модель никак не подходит.

Параметр распределения для бинарной с.в. есть вероятность принятия величиной значения 1.
Пусть есть ковариата $z$.

\[ \mathds E_\theta (Y|z) = p_z = \mathds P(Y=1|z) \in [0;1] \]

Пусть мы захотели использовать линейную регрессию.

% TODO a bit

Можно модернизировать наш подход и вместо линейной регрессии использовать другую функцию, например, логистическую регрессию.
\[ g(p_z) = \beta_1 + \beta_2 z \]
\[ g(u) = logit(u) = \ln \left( \frac{u}{1-u} \right) \]

Можно брать и другие $g$; к примеру, использовать функцию распределения -- например, функцию нормального распределения.
Для модели логистической регрессии

\[ p_z = \frac{1}{1 + e^{- \beta_1 - \beta_2 z}} \]

\subsection{Обобщённая линейная модель. Правая часть}

Регрессионное соотношение

\[ g(\mathds E_\theta (Y|X)) = X^T \beta \]

В классической модели $ g = I $ (тождественное преобразование)

$ g $ -- \emph{функция связи}, строго монотонная функция с областью значений  $ \mathds R $.

Предполагается, как и с обычной линейной моделью, что распределение $ Y|z $ принадлежит какому-то параметрическому семейству.
Обычно используют экспоненциальные семейства.
Их общий вид таков:
\[ f(y, \theta, \phi) = \exp \left( \frac{y \theta - b(\theta)}{ a(\phi) + c(y, \phi)  } \right) \] % TODO a bit

Примеры экспоненциальных семейств:
\begin{enumerate}[noitemsep]
	\item Нормальные семейства $ \mathcal N (a, \sigma^2) $
	\item Гамма-распределения при фиксированном $ p $
	\item Бета-распределения
	\item Биномиальное распределение
	\begin{align*}
		f^*(y, a, b) & = P(Y=y) = \frac{m!}{y!(m-y)!} p^y (1-p)^{m-y} \\
		& = \exp \left( \ln C_m^y + y \ln p + (m-y) \ln (1-p) \right) \\
		& = \exp \left( y \ln \frac{p}{1-p} + \underbrace{m \ln(1-p)}_{b(\theta)} + \ln C_m^y \right)
	\end{align*}
	$ \theta = \ln \frac{p}{1-p} $
	
	\item Распределение Пуассона
\end{enumerate}

% TODO a bit

Матожидание логарифма функции правдоподобия в квадрате.
Продифференцировав дважды, получаем: правая часть -- минус дисперсия, отнесённая к $ a^2(\phi) $.
Т.о. дисперсия  $ Y_i $ выражается
\[ \mathds D (Y_i) = \mathds (Y_i |) \] % TODO a bit

Представим теперь, что есть модель обобщённая: $ g(\mu(X_i)) = g(\mu_i) = \eta_i $

$ X $ --  \emph{матрица плана} (иначе \emph{матрица-регрессор}) 

$ \mu(X) = (\mu_1, ..., \mu_n)^T = \mathds E (Y|X) $

Если $ \eta_i \equiv \phi_i $, функцию $g$ называют \emph{канонической}.

Чтобы найти оценку МП, нужно решить систему относительно логарифма правдоподобия:

\[ \frac{\partial LL(\beta)}{ \partial \beta_i } = 0 \]
Можно расписать цепочку:

% TODO some

В случае нормального распределения мы допускали, что матрица-регрессор имеет неполный ранг.
Здесь этого лучше не допускать (число строк лучше сделать равным числу параметров в распределении).

\subsection{Асимптотические свойства оценок}

Вспомним асимптотическую нормальность.
При выполнении условий регулярности

\[ \sqrt n (\hat \beta - \beta)  \to \mathcal N (0, \bar{\mathds I}^{-1} (\beta)) \]

где $ \bar {\mathds I } (\beta) = \lim_{ n \to \infty } \frac{ \mathds I (\beta)}{n} $ -- усреднённая матрица информации Фишера.

Напомним, матрица информации Фишера может быть вычислена двумя способами.
$ \mathds I(\beta) =  $

% TODO a bit

Информация Фишера для нашего экспоненциального семейства с параметром $ \beta $ может быть представлена в виде линейной комбинации $ \mathds I (\beta) = X^T W X $, $ W = W(\beta, \phi) $ -- диагональная матрица.

Матрица $ W $ нам неизвестна; оцениваем её по данным: диагональные элементы $ \frac{1}{ \mathds D (Y_r) g'(\mu_r)^2 } $

\subsection{Доверительное оценивание}

Итак, основную идеологическую часть мы прошли...
Осталась техническая часть.

Пусть $ \psi = C' \beta $ -- функция параметра,

% TODO a bit

Оценки будут уже не точными, как в случае обычной линейной модели, а асимптотические.

\[  \] % TODO formula 

\begin{leftbar}
	так и надо строить интервал в домашнем задании,  только в ДЗ квантиль распределения Фишера-Снедекора, а здесь хи-квадрат
\end{leftbar}
 
Нужно помнить, что доверительные интервалы должны быть выбраны до того, как мы посмотрели на данные.

\subsection{Проверки гипотез}

Гипотезы ставятся так же, как и в обычной модели, поскольку общий вид правой части такой же.

Критерий типа Вальда:

Критерий отношения правдоподобия:

В классической модели $ G $ и $ Z $ совпадают.
Здесь они могут различаться, но асимптотическое распределение одно и то же.

\subsection{Специальные случаи}

Нормальное распределение: модель регрессии будет канонической.

Гамма-распределение:
\[ L(y,a,\sigma^2) = \frac{ x^{p-1} a^p \exp(-a y) }{ \Gamma p ...} \] % TODO formula

Дискретные модели:

\begin{itemize}
	\item Распределение Бернулли:
	\begin{itemize}[noitemsep]
		\item Функция правдоподобия
		\[ L(y,p) = p^y (1-p)^{1-y} = \exp \left( y \ln \frac{p}{1-p} - \ln (1-p) \right) \]
		\item Моменты: $ \mathds E_p Y = p $; $ \mathds D_p Y = p(1-p) $
		\item Канонический параметр $ \theta = \ln \left( \frac{p}{1-p} \right) $
		\item Каноническая функция -- 
	\end{itemize}
	\item Распределение 
\end{itemize}

\end{document}