\documentclass[main.tex]{subfiles}
\begin{document}
\section{Анализ лонгитюдных данных (в предположении нормальности)}

Бывает достаточно значений фактора в один момент времени
Также они могут меняться в течение всего времени, но это уже более сложный случай.

При выборе моделей есть два пути:

\begin{enumerate}[noitemsep]
	\item моделирование корреляций непосредственно, для чего можно использовать классическую модель с зависимыми остатками (residuals)
	\item модель с псевдофункциями правдоподобия; поскольку мы говорим о нормальном распределении, это для нас пока неактуально.
\end{enumerate}

Мы используем случайный фактор, который зависит от ID и может быть неизменным с течением времени либо меняться.

Если это смешанная модель, то возникают две задачи:

\begin{enumerate}[noitemsep]
	\item моделирование ковариат (неслучайного эффекта)
	\item моделирование зависимости наблюдений, в которую входит случайный эффект и остатки
\end{enumerate}

Модель может быть стратифицированная (если говорим о категориальных сопутствующих факторах, что довольно часто встречается).
Сопутствующий фактор часто удобно свести к ординальному и использовать модель с группировкой.

Случайный эффект -- инициация траекторий в зависимости от индивида.
Использование случайного эффекта уместно, потому что такие результаты (с использованием случайного эффекта) можно использовать для описания всей популяции, а не только данной нам наблюдаемой когорты.
К тому же, использование модели с группировкой позволяет удерживать размер параметра не очень большим.

\subsection{Модели зависимости от времени}

Полиномиальной модели для описания зависимости оказывается достаточно.

\[ \mathds E_\theta Y(t) = \mathds E _\theta (Y|t)=\alpha_0 + \alpha_1 t + \dots + \alpha_{s-1} t^{s-1} \]
\[ Y_{ij} = \sum_{k=0}^{s-1} \alpha_k t_{ij}^{k} + e_{ij} \]

Что касается степени полинома -- его не нужно делать слишком большим; можно посмотреть глазами на траектории, часто его можно оценить визуально.
Если видим где-то перегиб, можно использовать полином третьей степени; больше -- уже экзотика.

Но это если отсутствует сопутствующий признак.
Если сопутствующий признак присутствует, случайный эффект как таковой отсутствует.
Обычно сопутствующий признак вводят как аддитивную компоненту.

В случае одинаковых времён наблюдений $ t_{ij} = t_j $ % TODO a bit
d=s

\subsubsection{Ортогональные многочлены}

Можно использовать

\[ A = \begin{pmatrix}
	g_0 & g_1(t_1) & \dots & \\
\end{pmatrix} \]

Удобство ортогональных многочленов в том, что матрица $ A $, переводящая полиномиальную модель в модель дисперсионного анализа, выбирается так, что 

Индивиду сопоставляется случайная величина (своё значение для каждого индвида),

\end{document}