\documentclass[main.tex]{subfiles}
\begin{document}

% TODO первая часть лекции -- продолжение предыдущей: про BED, PED, VCF, GDS... (нужно ли? Вроде и так многое известно)
\section{Подготовка данных к анализу генетических ассоциаций}
\begin{leftbar}
GWAS с помощью  ML -- определяющая тема для исследований, которые ведутся в Политехе, статистика в меньшей степени.
\end{leftbar}

Как и во всех биостатистических исследованиях, если мощность критерия, выдвинутого для проверки статистической гипотезы, мала, то и шансы найти отклонения ниже.
Проведение исследований с малым объёмом данных зачастую вредит 

Но мощность критерия определяется не только объёмом выборки, но и тем, насколько разделены основная и альтернативная гипотезы.
То есть нужно понять, насколько сильные отклонения от нулевой гипотезы мы будем считать значимыми отклонениями.

Зачастую именно пороговое значение отклонения фиксируется вначале.

\subsection{Контроль качества}

\subsubsection{Фильтрация маркеров}

Рекомендация: $ >98\% $ образцов должно быть генотипировано по каждому снипу (в малых популяциях можно устанавливать планку ниже: если популяция $<50$, уже одно отсутствующее значение при жёстком пороге привело бы к необходимости выбросить SNP).

Эвристическая процедура: провести фильтрацию маркеров по грубому порогу, затем фильтрацию образцов, затем снова фильтрацию маркеров по более жёсткому порогу.

\subsubsection{Значимость }

Специалисты говорят, что серьёзное отклонение от HWE -- скорее всего, результат неверного генотипирования.
О'Брайен не признавал генетические маркеры, в которых отклонение от равновесия Харди выше определённого порога.

В популяции могут быть <<клоны>>: один и тот же образец случайно два раза попал в машину.
Есть надёжные методы выявления таких образцов (они немного разные из-за ошибок, но образец-то один).

\end{document}