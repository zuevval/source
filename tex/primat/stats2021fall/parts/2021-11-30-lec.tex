\documentclass[main.tex]{subfiles}
\begin{document}
\section{ Смешанная модель  }

В смешанной модели присутствует два типа параметров -- случайный и неслучайный.

В классической постановке смешанная модель может быть представлена так: 

 $ \beta $ -- неслучайный эффект, $ b $ -- случайные параметры регрессии.
К неслучайному эффекту добавляются случайные, определяемые регрессорами, плюс случайная ошибка.
Иначе говоря,  $ \beta $ -- параметр, $ b $ -- не параметр ($ \Upsilon $ -- параметр).
\begin{leftbar}
	Одна и та же ковариата может входить как в случайную, так и в неслучайную часть.
\end{leftbar}

Предположения: матожидание вектора случайных эффектов $ b $ равно нулю (иначе его можно перенести в вектор неслучайных эффектов), матрица ковариации $ \Upsilon $ -- какая-то.
Отклонения и случайные эффекты независимы (или, по крайней мере, некоррелированы)

Тогда % TODO a bit

Если зафиксировать значения $ b $...

Условная матрица ковариации при фиксированном $b$ 
\[ Var_\beta(Y|X,Z;b) = \Sigma \]

В классических предположениях $ Y $ имеет нормальное распределение
\[ Y \sim \mathcal N(X^T\beta \Sigma) \] % TODO formula

Для произвольного вида смешанной модели сложно построить точный критерий, но в некоторых ситуациях это удаётся.

Что касается оценивания случайного параметра:
оцениваем его исходя из формулы, полученной нами

\[ \tilde \beta = (X(\Sigma + Z^T\Upsilon Z)^{-1}X^T)^- X(\Sigma+Z^T\Upsilon Z)^{-1}Y \]

Естественно, она является и ММП-оценкой.

Если матрицы ковариаций неизвестны, оценивать нечего, но можно предсказать.
Здесь можно говорить о наилучшем линейном несмещённом предсказании при известном $ \beta $ (BLUP)

... такая статистика тоже может называться BLUP.
Можно было бы, конечно, её назвать \emph{оценённым}  BLUP, но такой термин обычно используется в случае, когда оцениваемые матрицы ковариации считаются неизвестными.

Что касается характеристик BLUP в случае известного вектора $ \beta $,

\begin{gather*}
	Var(\tilde b (\beta)) = \Upsilon Z V^{-1} Z^T \Upsilon \\
	Var(\tilde b) = Cov(\tilde b (\beta), \tilde b) = Cov(b, \tilde b)
\end{gather*}

(эти формулы мы не выводим)

\end{document}