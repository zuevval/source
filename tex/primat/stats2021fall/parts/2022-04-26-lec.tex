\documentclass[main.tex]{subfiles}
\begin{document}
\section{Подготовка данных для GWAS. Продолжение}

29 апреля 2022 года

Ботсвана: законы не так строги по отношению к сбору данных для генотипирования (данные собирались в начале нулевых годов).
Рассказывали о том, что данные были собраны не очень аккуратно.

Когорта на микрочипах.
Данные: работники индустрии эскорта, а также несколько мужчин (по-видимому, случайно выбранных), один из мужчин болен СПИДом.
Когда начали рассматривать данные, оказалось, что несколько маркеров сцеплены с полом (не на Х-хромосоме, а на аутосомной области), и именно эти маркеры, как оказалось, более всего ассоциированы с болезнью СПИД.

Встречаются в геномах довольно странные регионы, которые почему-то плохо генотипируются.
Их далеко не всегда удаётся отфильтровать на предварительных этапах, но желательно отфильтровать.
Как?

\begin{enumerate}[noitemsep]
	\item Call Rate (процент генотипированных индивидов для конкретного маркера): число известных маркеров, приходящихся на одного индвида, должно составлять значительную долю от общего числа рассматриваемых маркеров.
	Требуется обычно достаточно большое число маркеров на индивида и достаточное число индивидов на маркер, чтобы статистические тесты сохраняли высокую мощность.
	
	Вообще говоря, при исследовании микрочипов обычно Call Rate высок, но бывает, что качество ДНК низкое и показатель низкий.
	
	\item Проверяется равновесность контрольной популяции по Харди (нет миграции, отбора).
	В группе заболевших равновесие Харди-Вайнберга может и не выполняться 
	
	О. Брайен: "Сильное отклонение от равновесия Харди-Вайнберга свидетельствует скорее о том, что "
	
	\item Фильтрация образцов с редкой минорной аллелью.
	К примеру, SNP с частотой 1 из 1000 ведёт к увеличению необходимой поправки и уменьшению мощности статистических процедур.
\end{enumerate}

\subsection{Проверка равновесия Харди-Вайнберга}

HWE: популяция бесконечного размера, отсутствует отбор и мутации $ \Rightarrow $ частоты генотипов $ AA $, $ Aa $, $ aa $ находятся в пропорции $ (1-p)^2 $, $ 2p(1-p) $, $ p^2 $, $ p $ -- частота аллеля $ A $.

Статистика критерия $\chi^2$:

\[ X^2 = \frac{n_{AA} - n \hat p^2 }{n \hat p^2} +  \]

Критерий хи-квадрат с одной степенью свободы.
Асимптотический. \\

Можно также рассмотреть в явном виде распределение частот аллелей.

\[ \begin{pmatrix}
	TODO
\end{pmatrix} ~ \text{Multinomial} TODO \]

\subsubsection{Критерий Халдана}

Вариант точного критерия Фишера.

Считаем, что фиксированное общее число аллелей и число аллелей $ A $: $ n_A $, $ n_a $ фиксированы.

\subsection{Другие критерии}

Поскольку у нас мультиномиальное распределение, можно получить также критерий отношения правдоподобия.

Асимптотическое распределение, естественно, -- $ \chi^2_1 $ (то есть с одной степенью свободы).

Есть и другие критерии, но насколько они нужны, вопрос сложный.
Наиболее распространён сейчас всё-таки критерий Фишера.

\subsection{Отклонения от HWE}

Коэффициент неравновесия Харди-Вайнберга:

$ f^* \overset{def}\Leftrightarrow $ TODO

Равновесие Харди-Вайнберга достигается при $ f^* = 0 $.

\begin{leftbar}
	Но даже если равновесие Харди-Вайнберга выполняется, в силу поправки мы должны брать низкое значение P-value неравновесия при фильтрации, когда у нас много SNP.
	
	Выбирают всё же границу достаточно низкую.
	В контрольной группе p-value $ \approx 10^{-6} $, в исследуемой -- ещё ниже.
\end{leftbar}

Коэффициент оталонения от равновесия по сцеплению -- конечно, полезная характеристика, но мы его не знаем точно, а оцениваем, поэтому это случайная величина, его нельзя рассматривать как достоверную характеристику.

Задачи доверительного оценивания тоже требуют поправки.

\subsection{ Неоднородность популяции }

Вопрос причинно-следственной связи: например, мы обнаруживаем ассоциацию.

Фактор неоднородности в популяции может быть наблюдаемым или не наблюдаемым.
Если мы его наблюдаем, можно просто ввести в модель в качестве ковариаты.

Желательно всё же, чтобы в популяции была однородность (поэтому на первом этапе желательно проверить однородность).
Один из классических способов проверки -- \textbf{метод главных компонент}.

PCA предназначен для аппроксимации векторов, натянутых на пространство $ \mathds R^n $.
Минимизируем сумму квадратов отклонений от точек до подпространства $ \mathds R^m $.

Корреляционная матрица $ n \times n $ (она же матрица косинусов углов между образцами): если некоторые её собственные значения нулевые, возможно вложение в пространство более низкой размерности, чем $ n $, без потери информации.

\textbf{Bath Effect}: часто при генотипировании разными методами получается систематическая ошибка.
Можно ввести метод секвенирования в модель в качестве ковариаты, но это ведёт к потере мощности за счёт более универсальной, более общей модели.
Где-то рекомендуется брать это в качестве случайного эффекта.

\begin{leftbar}
	В проекте <<Российские геномы>> просто провели assotiation study и отфильтровали SNP, наиболее ассоциированные с методом.
\end{leftbar}

Другой способ -- метрическое многомерное масштабирование (Metric Multidimensional Scaling).

Результаты часто похожи, но нельзя сказать, что они совпадают.
Понятно, что разные методы должны показывать схожие результаты.

\end{document}