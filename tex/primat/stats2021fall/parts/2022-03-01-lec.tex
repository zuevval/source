\documentclass[main.tex]{subfiles}
\begin{document}
\section{Применение обобщённых линейных моделей к анализу категориальных данных}

Таблица сопряжённости: в случае независимости экспериментов её элементы имеют мультиномиальное распределение (что следует из функции распределения).

В альтернативном подходе можно ввести некую надстройку.
\emph{Пуассоновские процессы} (пуассоновские потоки событий) широко применяются в различных областях статистики, и одно из применений -- анализ таблиц сопряжённости.

Определение пуассоновского процесса:

\begin{enumerate}[noitemsep]
\item Стационарность: вероятность
\item Ординарность (вероятность появления двух и более событий в малом интервале есть б. м. по отношению к вероятности появления одного).
\item  Отсутствие последействия:
\end{enumerate}

В определении не сказано, но известно, что

\begin{enumerate}[noitemsep]
\item число событий, появляющееся в некий интервал времени, имеет распределение Пуассона.
Число событий в единичном интервале имеет распределение Пуассона с параметром $ \lambda = TODO $.

\item TODO

\end{enumerate}


Основное отличие по сравнению с классическим мультиномиальным подходом: распределение выборки пуассоновское.

В мультиномиальном подходе величины зависимые, поскольку сумма $ p_i $ фиксирована.
В пуассоновской модели...

Можно провести связь между пуассоновской и мультиномиальной моделью: условное распределение $ (n_1, \dots, n_d) $ при условии $ \sum_{i=1}^d n_i = n $

Пуассоновская модель: больше предположений.

\subsection{Многомерный эксперимент}

Рассмотрим TODO

\subsection{Модель логистической регрессии}

\[ logit(\mathds E_\theta (Y|z)) = X^T \beta \]

$ z $ -- некий набор признаков.
В частности, можно при рассмотрении таблиц сопряжённости считать один признак зависимым, второй наблюдаемым.
Если один из них бинарный, можно применять модель логистической регрессии.

В случае нескольких уровней фактора можно использовать логистическую регрессию тоже, но более удобно использовать пуассоновскуб модель.
Если признак ординальный, можно попробовать посмотреть: не квадратичная ли зависимость (и так далее)?

% TODO much

Оценки вероятности не должны быть нулевыми.
Если на некотором уровне точек данных нет, выкидываем его (оценивание вероятности недоступно).

Далее -- стандартная схема построения доверительных эллипсоидов.
Доверительные интервалы предполагается строить для некоторой функции параметра $ \psi = C^T \beta $.

Асимптотическая нормальность:
\[ \sqrt n \left( \hat \psi - \psi \right) \Rightarrow \mathcal N (0, \Gamma_\psi)  \]

% TODO some

\begin{leftbar}
На самом деле, есть много вариантов интерпретации многоуровневой схемы, это лишь один из них, придуманный мной (С. В.), если найдёте какую-то серьёзную ошибку, будет большой плюс на экзамене.
\end{leftbar}

Функция правдоподобия: здесь не удастся применить наработки двухуровневой схемы

% TODO much

Ключевое утверждение, которое можно использовать для оценок максимального правдоподобия -- сходимость квадратичных форм.
Необходимым условием является существование предела $ \bar{ \mathds I }(\beta) $.

\end{document}