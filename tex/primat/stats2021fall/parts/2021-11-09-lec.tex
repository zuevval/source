\documentclass[main.tex]{subfiles}
\begin{document}
\section{Дисперсионный анализ}

Ковариаты не являются...
Часто даже неясно, соответствуют разные типы ковариат большему 
Такие типы ковариат будем называть  \emph{факторами}.

\subsection{Однофакторный дисперсионный анализ}

Простая группировка: ковариата выступает в качестве фактора группировки набора наблюдений $ Y $.
Удобно считать, что ковариата принимает значения $ \{ 1, ..., d \} $, но при этом

Модель:
\[ \mathds E_{\theta_i}(Y|z=i) = \eta_i, i = 1, ..., d \]
$ \eta = (\eta_1, ..., \eta_d)^T $ -- среднее по группам, дисперсия $ \mathds D_\theta Y = \sigma^2 $  одинакова во всех группах.

Соответствующая модель линейной регрессии:

 \[ \mathds E_\theta (Y|z) = X^T \eta \]
 \[ X = \begin{pmatrix}
 	content...
 \end{pmatrix}  \]

% TODO a bit

Каждый регрессор содержит одну единицу, остальные нули.

Порядок величин внутри группы и групп неважен.

Можно ввести нотацию с использованием группировки:
$ Y_{ij} $, $i$ -- номер группы, $j$ -- номер наблюдения в группе.

Запись с использованием сравнений: рассматривают общую модель, где вводятся веса.
Сравниваем параметры $ \eta_1, ..., \eta_d $ % TODO a bit

Стандартный выбор весов:
иногда используют равные веса, иногда -- пропорциональные числу элементов в группах.

\subsubsection{Изучение влияния фактора на результат}

Наибольший интерес представляет гипотеза отсутствия влияния фактора на результат.
Она записывается как равенство средних
\[ H_0 : \eta_1 = ... = \eta_d \]

Эквивалентная форма записи:
\[ \alpha_1 = ... = \alpha_d \]
$ \alpha_i $ --  главные эффекты, т. е. отклонения среднего по группам от общего среднего: $ \mu_i = \mu + \alpha_i $

Это же можно записать с помощью любых $ d-1 $ линейно независимых сравнений
\[ H_0 : \psi_1 = \psi_2 = ... \psi_{d-1} = 0, \vec \psi = C^T  \vec \eta \]

Проверка гипотезы:

\[ \mathds F = \frac{\bar{SS}_H}{\bar{SS}_e} = \frac{SS_H /q}{SS_e /(n-r)} \]

\[ SS_H = SS(\hat \eta_H) - SS_e \]

$ SS_H $ -- наименьшая сумма квадратов при нулевой гипотезе $ - SS_e $, $ SS_e $ -- минимальная суммак квадратов в общих предположениях.

Статистика $ \mathds F $ при нулевой гипотезе имеет распределение Фишера-Снедекора $F_{d-1,n-d}$.
При альтернативе -- нецентральное распределение Фишера-Снедекора % TODO a bit

Метод Шеффе позволяет получать совместные доверительные интервалы для любой комбинации параметров.
Напомним, метод Шеффе не очень эффективен для конечного числа функций параметра, но если хотим посчитать совместные доверительные интервалы для \emph{любых} линейных комбинаций параметров, он весьма эффективен.
Метод Шеффе позволяет выявить сравнения, ответственные за отвержения гипотезы в случае её отвержения.

С другой стороны, также можно ставить односторонние гипотезы $ H_0 : \eta_1 < \eta_2 < ... < \eta_d $ следующим образом: % TODO a bit

Мы можем проверять всевозможные гипотезы и выбирать наиболее подходящую.

\subsection{Модель двухфакторного анализа}

У нас есть два фактора $ (z_1, z_2) $, комбинация которых определяет распределение.
Т.о. наблюдение имеет вид $ (Y,z), z = (z_1, z_2) $.

Фактически задача не отличается от однофакторного анализа с одним индексом, который пробегает столько же значений, сколько значений пар двух факторов.

Можно записать с группировкой, но будет ужетри индекса.

\subsubsection{Особенности двухфакторного анализа}

Понятно, что оценка $ \eta_i $ аналогична однофакторному анализу.
Но есть и отличия.

Выделение главных эффектов: $ \eta_{ij} = \mu + \alpha_i^{(1)} + \alpha_j^{(2)} + \alpha_{ij}^{(12)} $, $\alpha_i^{(1)}, \alpha_j^{(2)}$ -- главные эффекты, $ \alpha_{ij}^{(12)} $ -- взаимодействия ( без $ \alpha_{ij}^{(12)}$ не хватает параметров для учёта разброса во всех $ \eta_{ij} $).

В такой модели накладываем ограничения: % TODO a bit

Взаимодействия -- тоже сравнения, и для них можно использовать те же модели, что и для сравнений.
Аддитивная модель: гипотеза отсутствия взаимодействий -- $ H_{(12)} : \alpha_{ij}^{(12)} $

% TODO a bit

Аддитивная модель: значения не будут зависеть от выбора весов.
Согласованное действие: расхождение; пересечение (synergy).

Чем опасно пересечение?
Проводя однофакторный анализ по фактору 1 или 2, можно сделать ошибочный вывод о том, что фактор не влияет на результат.

\subsubsection{Проверка значимости взаимодействий}

% TODO some

Гипотеза о главных эффектах: когда есть пересечение, при каких-то весах гипотеза может давать нулевой результат.

Отсутствие влияния двух факторов на результат равносильно выполнению трёх гипотез $ H_{(12)}, H_{(1)}, H_{(2)} $

\subsection{Многофакторный дисперсионный анализ}

Здесь постараемся в основном объяснять словами.

Пусть имеется $ k $ факторов группировки.
Как и в двухфакторном анализе, можно объединиить все влияния в один фактор.
Но размерность параметра при этом будет экспоненциально зависить от числа исходных факторов и в какой-то момент может оказаться чересчур мало данных для такой модели.

Влияние факторов / их комбинаций:
веса выбираются для каждого из факторов; помимо взаимодействий двух факторов появляются взаимодействия третьего порядка и так далее;  главные эффекты и взаимодействия определяются рекурсивно для каждой комбинации факторов.

Гипотезы обычно выдвигаются блоками.
Начинают с большего числа факторов, т.к., если выполнена гипотеза отсутствия взаимодействия двух факторов, то гипотезу отсутствия взаимодействия двух факторов нельзя оценивать объективно.

Нет смысла, конечно, рассматривать совсем общую модель.
Как интерпретировать взаимодействие пяти факторов?
В таком случае есть несколько подходов; можем проверить 4 фактора, если она слишком сложная -- 3 фактора, 2 фактора... Есть обратный подход -- строим сначала нулевую гипотезу (среднее), затем вводим один фактор, два фактора... до тех пор, пока не пере

Для выбора оптимальной модели часто используют \emph{информационные критерии}.




\end{document}