\documentclass[main.tex]{subfiles}
\begin{document}
\section{Доверительное оценивание}
19 октября 2021

Предположим, что имеется исходный статистический эксперимент; параметризация задаёт совокупность подмножеств, из которых мы будем выбирать доверительные множества.
Для многомерных параметров можно выбирать прямоугольники, можно эллипсоиды.

\textbf{Определение.}
Статистика $ \hat \Theta : \mathfrak X \to \mathcal Y $, удовлетворяющая условию
\[ P_\theta (\theta \in \hat \Theta) \ge 1 - \alpha \quad \forall \theta \in \Theta  \]
называется \emph{доверительной оценкой} параметра $ \theta $.

($ \hat \Theta $ есть случайное множество, функция от наблюдения; его реализация зависит от $ X \in \mathfrak X $, и мы говорим о вероятности того, что теоретическое значение $ \theta $ будет накрыто множеством $ \hat \Theta $).

% TODO a bit

В случае вещественного параметра доверительные множества есть \emph{доверительные интервалы}.

\subsection{ Построение доверительных интервалов}

... что вероятность будет $ > 1 - \alpha $.
Если это выполнено, можем построить прообраз отображения (он должен быть в данном случае интервалом, а в общем случае -- принадлежать нашему классу доверительных множеств).
Тогда, естественно, $ \hat \Theta $ будет доверительным множеством.
Такая функция $ G(X, \theta) $ называется \emph{генератор доверительного интервала}.

Напомним, \emph{лемма Фишера}: если $ \{ X_1, X_2, ..., X_n \} : X_i \sim \mathcal N(a, \sigma^2) $, то выполнен ряд свойств.
В частности, $\frac{ns^2}{\sigma^2} \sim \chi^2_{n-1} $, что позволяет строить доверительные интервалы в условиях известного параметра $ a $; также % TODO a bit

\subsubsection{Оценивание параметров нормального распределения} 

Пусть параметр $ a $ мы хотим оценить; $ \sigma $ считаем мешающим параметром, он нам неизвестен, но пока не нужен.

Вспомним $ IV $ пункт леммы Фишера: $ G(X, a) := \sqrt{n-1} \frac{\bar X - a}{s} $ имеет распределение Стьюдента $ S_{n-1} $.
Распределение Стьюдента симметричное, поэтому строим симметричное доверительное множество.

Нужно решить систему неравенств... % TODO a bit formulas?

Система легко решается.

Можно отметить, что при таком построении этот доверительный интервал наикратчайший.
Это можно объсянить -- эвристически: распределение Стьюдента сконцентрировано вокруг нуля, поэтому несимметричный интервал всегда будет длиннее. \\

Доверительный интервал для $ \sigma $ можно построить по $ III $ пункту леммы Фишера.
Здесь уже $ a $ является мешающим параметром; но мы строим $ G(X, \sigma) = \frac{ns^2}{\sigma^2} $, где 

Этот интервал уже кратчайшим не будет.
Чтобы найти кратчайший, можно попробовать решить задачу оптимизации по параметру $ \lambda $...
Но обычно там не так критично.

\begin{leftbar}
	Построение генератора доверительного множества -- дело тонкое.
	Надо, чтобы в результате решения получилось не что-то, а доверительный интервал.
\end{leftbar}

Можно ещё рассмотреть задачу с двумя выборками, дисперсии которых одинаковы (иначе не получится построить генератор).

Строим доверительный интервал для линейной комбинации двух матожиданий $ \theta = \beta a_1 + \gamma a_2 $ (как правило, нас интересует разность $ a_1 - a_2 $).
По пункту $ I $  леммы Фишера...

% TODO a bit

В качестве генератора берём функцию трёхмерного параметра

\[ TODO a bit \]

Но всё бы было зря, если бы нам не удалось избавиться от $ \sigma^2 $.
Преобразуем и получаем

\[ G(\vec X, \vec Y; \theta) = \frac{ TODO a bit }{  } \]

\begin{leftbar}
	Можно поупражняться: рассмотреть случай, когда дисперсии неодинаковые, но пропорциональны с коэффициентом $ r $.
\end{leftbar}

При двух неизвестных дисперсиях можно построить доверительный интервал для частного двух дисперсий.

\subsubsection{Использование преобразования Смирнова}

Ещё одна, казалось бы, перспективная разработка.

Если случайная величина $ X $ имеет непрерывную функцию распределения $ F $, то случайная величина $ F(X) \sim \mathcal N[0;1] $ 

Проблема должна быть в том, что функция должна быть монотонна по параметру, но часто это выполнено и можно построить доверительный интервал.
К сожалению, не все функции так просто выписываются, поэтому иногда этот подход неприменим. \\

Есть иной, похожий, но более общий подход.
% TODO a bit

Считаем, что $ T $ непрерывно.
Если $ T $ дискретно, этот метод тоже можно воспроизвести, но с некоторыми ограничениями.

\begin{thrm}
	Пусть
	\begin{enumerate}[noitemsep]
		\item  % TODO
	\end{enumerate}
\end{thrm}

В качестве примера можно привести распределение Бернулли.
$ ] X_i \sim Bi(1, \theta) $, $ \theta $ -- вероятность успеха.
Стро

\begin{leftbar}
	Метод сложно применять на практике, поскольку он непрост для понимания, но для общего развития его полезно понимать.
\end{leftbar}

\subsection{Асимптотические доверительные интервалы}

Сложно выбрать универсальный генератор доверительного множества.
Точные доверительные множества сейчас обычно строят только при очень маленьком числе экспериментальных точек.

Широко применяются универсальные \emph{асимптотические доверительные интервалы}.
В асимптотической схеме статистика зависит от размера выборки $ n $; в зависимости от $ n $ строим правило.
Необходимо, чтобы в пределе вероятность того, что интервал накрывает истинное значение, выполнялось с вероятностью $ \ge 1 - \alpha $

Иногда предел не существует, но можно взять нижний предел.

Довольно часто выполнено условие асимптотической нормальности:
\[ \sqrt n ( \hat \theta (\vec X) - \theta ) \Rightarrow \mathcal N (0, \sigma^2(\theta)) \] 
(в слайдах опечатка, правильно $ 0 $, а не $ \alpha $). \\

Мтеод построения асимптотических доверительных интервалов на базе ОМП:

% TODO some

Получаем ещё более общий метод.
% Впрочем, это не работает для нормального распределения (там не работает неравенство Рао-Крамера). % TODO what have been said?

Пример -- для нормального распределения: получаем в итоге
\[ \sqrt{n}\frac{\bar X - a}{s} \Rightarrow \mathcal N(0, 1) \]

(напомним, когда ранее выводили, был множитель $ \sqrt{n-1} $ и не нормальное распределение, а Стьюдента).

Аналогично,
\[ \]

Конечно, выведенные ранее интервалы использовать предпочтительно, зато этот выведен общим методом.

Ещё один пример -- распределение Бернулли. % TODO

Общая формула центральной асимптотической нормальности (ЦПТ):
\[ \sqrt n \frac{\bar X - p}{\sqrt{ p(1-p) }} \Rightarrow \mathcal N (0, 1) \]


\Large{После перерыва:}
\normalsize

Второй способ: не будем мучиться и подставим вместо $ p $ $ \bar X $.

\subsection{Задача доверительного оценивания векторнозначного параметра}

Можно строить для каждого интервала отдельно и, исходя из поправки Бонферрони, вывести совместные интервалы.

Методы множественного оценивания Шеффе: в линейной регрессии.
Об этом будем говорить позже.

Универсальный метод построения асимптотических доверительных множеств похож на таковой для одномерных величин.

Пусть $ \hat \theta $ (вектор!) -- (асимптотически) нормальная оценка $ \theta $
\[ \sqrt n (\hat \theta - \theta) \Rightarrow \mathcal{N}(0, \Sigma(\theta)) \]

% TODO

Если проведём аналог этого оценивания для одномерного нормального распределения, получим просто квадрат, что согласуется с выведенным ранее результатом.

Доверительное множество в данном случае для многомерного вектора -- эллипсоид.

% TODO a bit

Соответствующее нормальное распределение предельное умножается на матрицу $ B $; получаем тоже нормальное.
Таким образом получаем, что любая линейная функция асимптотически нормального параметра будет также асимптотически нормальной.

Интересный факт: если имеется пространство (например, линейных функций параметра на базе исходного вектора) размерности $q$, то всякая оценка линейной функции, полученная по соответствующей формуле, ...

% TODO

Получаем метод Шеффе.
Вероятность того, что каждая $ \psi $
Теорема строится на том, что берутся эллипсоиды, рассматриваются их сечения...
Оказывается, что...

\subsubsection{Метод Бонферрони}

Идея та же самая: вероятность объединения событий не превышает суммы вероятностей.

Тогда можно строить доверительные интервалы из индивидуальных, но уровня доверия $ 1 - \alpha / d $.

Метод Бонферрони эффективнее использовать, когда 

\end{document}