\documentclass[main.tex]{subfiles}
\begin{document}

\section{Лекция 12 апреля}

\subsection{Моделирование случайного эффекта}
% слайд 11

Случайный эффект тоже моделируется похожим образом.
Считаем, что он имеет нормальное распределение.
Насколько это обоснованно?
Сумма независимых нормированных центрированных величин: <<точка притяжения>> -- нормальное распределение.

Иногда вводится \emph{полиномиальный случайный эффект}.

\[ r(b_i, t) = b_{i1} + b_{i2} t + \dots + b_{is} t^{s-1}\]
$ b_i \sim \mathds N(0, \Upsilon) $

Чаще всего это линейная зависимость; тогда уже можем получить какой-то коэффициент наклона.
Среднее (мат. ожидание) считаем нулевым, оно входит в неслучайный эффект.
Чем больше параметров (выше степень полинома), тем больше размерность матрицы ковариаций.
Впрочем, во временнных рядах часто последовательности небольшие (3, 5, иногда 10 наблюдений на индивида).
Если это одни и те же точки во времени, степень полинома не выше $ n-1 $.
Если же времена разные, проверять какие-то гипотезы (или строить доверительные интервалы) с учётом случайного эффекта.

\begin{leftbar}
	Мнение С.В.: нужно использовать максимально свободную форму корреляции.
\end{leftbar}


\subsection{ Простой и линейный эффекты индивида }
% слайд 12

% TODO a bit

Смешанная модель с линеным случайным эффектом: уже два параметра (и интегрировать уже надо по плоскости, по двум переменным); после замены переменной получаем плотность стандартного нормального распределения.

\subsection{ Статистический анализ }
% слайд 13

В реальных приложениях, как правило, удаётся подобрать регулярные семейства, и тогда можно говорить об асимптотической нормальности ММП.

С учётом независимости наблюдений от разных индивидов функция МП распадается на произведение.
Функция правдоподобия для отдельного индивида строится с учётом замечаний, о которых мы говорили ранее.

Параметры модели состоят из двух частей: параметр неслучайного эффекта (наиболее интересный) и параметр случайного эффекта.
Мешающий параметр тоже может присутствовать (например, если мы используем квазипуассоновское распределение), который совершенно неинтересен с точки зрения исследователя, но который позволяет наилучшим образом описать данные.
Единственное, что нужно от случайного параметра -- состоятельность.

Модель не байесовская, поскольку имеются параметры, отличные от случайных.

Поскольку распределение нормальное, плотность $f=$. % TODO formula

\subsection{ Метод максиамального правдоподобия }
% слайд 14

Можем использовать метод Ньютона.
Также иногда используется EM-алгоритм.
В случае бинарной наблюдаемой величины и простого эффекта индивида мы построили функцию правдоподобия в явном виде (лекция 13 предыдущего курса).

Множественное оценивание: метод Бонферрони, метод Шеффе; эллипсоиды - 

Критерии -- критерии типа Вальда на основе квадратичных форм.
Подставляем гипотетические значения параметров.
Или критерий отношения правдоподобия; при этом гипотеза должна подразумевать, что значения параметров лежат в неком линейном подпространстве.
Это стандартные методы для параметра неслучайного эффекта.
Надо помнить, что методы асимптотические и при малом числе наблюдений возникают вопросы.

\subsection{Случайный эффект}
% слайд 15

Аргументы наименьшей дисперсии, которые были в классической модели, не используется.
Используем эмпирическую байесовскую оценку.

%  слайд 16
При изучении случайного эффекта сталкиваемся с обычными сложностями, о которых неоднократно говорили.
Как и в случае классическом, есть % TODO a bit

Даже в самой классической модели возникают некие эффекты, связанные с проверкой значимости.
Асимптотически для неслучайного эффекта всё сохраняется, но что касается случайного эффекта: гипотеза о том, например, что матрица ковариации имеет особый вид, всё должно быть в порядке, но % TODO a bit

\begin{leftbar}
В наших лабораторных работах могут встретиться модели, при оценке случайных эффектов в которых мы тоже можем (в некоторых вариантах) столкнуться с некоторыми проблемами.
\end{leftbar}

% слайд 17

Акаике, байесовский критерий: классические версии этих критериев не подходят для применения к моделям со случайными эффектами (как мы уже говорили ранее).
Ни о какой статистической значимости речи не идёт.

\begin{leftbar}
Вопрос от Насти: модели NLME - non-linear mixed effect models.
Почему они непопулярны?

Ответ:
Линейные модели покрывают достаточно большую область приложений.
Притом для нелинейных моделей меньше доказанных фактов, больше эвристик и, соответственно, меньше доверия результатам.
\end{leftbar}

\section{Анализ данных генетических исследований}

Генетическая составляющая, клиническая составляющая.

Рассматриваем в этом курсе диплоидные организмы.
У человека $ > 500 $ млн SNP (примерно 1 на 6 нуклеотидов), впрочем, это все известные, а в реальных популяциях их меньше.
Информацию о вариантах можно получить в БД NCBI и Ensembl.

Раньше считалось, что тривариантные SNP -- экзотика, но теперь найдено уже довольно много таких.
Полновариантные (все 4 варианта) - большая редкость.

% TG a bit: G Anton, Anna Ig

\emph{Фазированные данные} (гаплотипы) % TODO a bit

Основной и альтернативный (альтернативные) вариант(ы) гена определяются двумя способами:

\begin{enumerate}[noitemsep]
	\item Основной на референсе
	\item Основной -- тот, которого больше в популяции (в полногеномном анализе ассоциаций это наиболее частый метод)
\end{enumerate}

\end{document}