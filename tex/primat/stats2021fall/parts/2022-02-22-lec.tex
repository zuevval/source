\documentclass[main.tex]{subfiles}
\begin{document}
\section{Лекция 2. Анализ категориальных данных. Продолжение}

% 1. combine SDF files from OneDrive (script in qsar: sdf_merger.py)
% 2. upload to Colab, dock // unable to create Vinardo scores, otherwise worked(?)
% 3. download results to OneDrive // done 11:45
% 3a. the same with `test` // done 11:45
% 3b. write to Petr // done 11:49
% 4. google for CatBoost overfitting
% next:
% try what Anna said for her dots
% maybe optimize SDF with MMFF94
% TXT with true atoms numbers: prettify
% CatBoost CV: 
% - generate `CatBoostMinMaxSplit` column in Train, fill it (add cell in current splitter)
% - add docking results OR annas features OR rdkit features: which is better - one of them or none?
% screening in Shrodinger, 

В  таблицах сопряжённости $ 2 \times 2 $ удобно использовать уровни обоих признаков $ \{ 0, 1\} $ вместо $ \{ 1, 2 \} $.

Напомним: нулевая гипотеза -- признаки независимы.
Если нулевая гипотеза не выполняется, желательно описать альтернативную.
Распределение исходных признаков, как правило, не имеет значения, но иногда учитывается и это: либо с помощью совместного распределения (тогда проверяется гипотеза независимости), либо с помощью условных распределениях (тогда гипотеза однородности -- распределения при различных значениях фактора совпадают).

Есть различные меры для проверки гипотезы независимости (однородности).

Для распределения Бернулли мы можем записать классическую предельную теорему.

При разных значениях $ j $

% TODO

Можем говорить о статистике критерия.

Надо помнить, что гипотеза однородности тонкая, и говорить о том, что она верна, не представляется возможным.

% TODO

Если бы мы знали $ p_{ij} $, имели бы центральное $ \chi^2 $-распределение.
Мы знаем лишь, что $ p_{ij} = p_{i+}+p_{+j} $.
% TODO

\begin{leftbar}
	В дальнейшем нам надо будет в работах вычислять число степеней свободы для более сложных таблиц.
\end{leftbar}

% TODO many

\subsection{Трёхмерные таблицы сопряжённости}

Рассмотрим таблицу, где приведена общая статистика ($ 2 \times 2$).
Есть две больницы, и в каждой считается число пациентов с осложнениями после операции или без осложнений.

% TODO many

\subsubsection{Анализ таблиц сопряжённости 2*2*d}

Фактически мы можем рассматривать условные распределения.

Появляется т.н. \emph{гипотеза однородности отношений шансов}.
Она говорит следующее: все шансы одинаковые при различных значениях $z$.

% TODO a bit

Проверка гипотезы однородности $ H_{HA} $: довольно простая статистика, основанная на $ \chi^2 $ -- критерий Бреслоу-Дэй.
% TODO a bit

Ещё одна интересная гипотеза -- гипотеза условной независимости.
Она более сильная, чем гипотеза независимости: помимо равенства параметров распределений предполагается их равенство единице.

Критерий Кочрана-Мантела-Хензела (CMH)

\end{document}