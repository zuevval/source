\documentclass[main.tex]{subfiles}
\begin{document}

\section{Анализ лонгитюдных данных в общих предположениях}
5 апреля 2022 года \\

Понятно, что для нормальных лонгитюдных данных всё хорошо, там применяем смешанную модель, можем строить ковариационную структуру.

Но всё ломается, если нормальность не предполагать.
Например, если появляются бинарные данные.

Используются \emph{обобщённые линейные модели}.
Например, бинарные данные: параметр -- вероятность (матожидание).
В отличие от классической модели, где матожидание выражается как линейная функция параметров, в обобщённой модели подразумевается возможность некого преобразования.
Так удаётся построить не только оценку параметра, но и доверительные интервалы.

Выбор функции связи: любой разумный выбор даёт возможность делать некие статистические выводы.
Чтобы сделать стат. анализ более стандартизованным, обычно выбирают каноническую функцию связи (напомним, в обообщённых моделях для случая канонических параметров проще вычислить информационную матрицу и так далее).

Вопрос выбора между обобщёнными линейными моделями и классическими методами (например, анализ таблиц сопряжённости): при возможности желательно использовать всё же классические методы, потому что обобщённые модели более сложные и, по сути, для них есть только асимптотические методы (на основе ММП).
Что касается лонгитюдных данных: появляется необходимость включить зависимость наблюдаемой характеристики от времени, и с применением классических методов возникают определённые сложности (как включить зависимость наблюдений от одного индивида)?
В классических подходах решение -- смешанные модели: получается моделировать корреляции между остатками.
Но в случае бинарных наблюдений мы уже не можем говорить

Впрочем, мы помним, что есть подход, который предполагает, что распределение всё-таки нормальное, но группируются так, что получается бинарное распределение.
Но здесь не очень понятно, насколько хорошо это работает, поэтому возникает необходимость разработать специальные методы для лонгитюдных данных; это смешанные модели и семипараметрический подход (GEE).

Обобщённая смешанная модель:

\[ g(\mathds{E}_\theta(Y_i | z_i, t_i)) = f(\beta; t, z, n) + r(b_i; t)  \]

Функция $ g $ выбирается обычно так, чтобы множество значений было множеством вещественных чисел, чтобы не было ограничений на параметры.

Что касается корреляций между наблюдениями, этот инструмент становится сложно применимым.

Обычно предполагается, что наблюдения от одного индивида зависимы, но условно независимы при фиксированном индивиде.

% неплохое упражнение - проверить мощности процедур.

Второй подход -- \emph{семипараметрический} (Generalized Estimating Equations, GEE).

Фактически используется метод М-оценивания (вместо функции правдоподобия).
\begin{leftbar}
	ММП-оценка -- частный случай М-оценки.
\end{leftbar}

Семипараметрический подход:

В общем случае не конкретизируется, как именно строится регрессор, и строим модель покомпонентно (как в классическом варианте обобщённых линейных моделей).

Для каждого наблюдения маожидания

\[ g(\mathds E_\theta(Y_ij)) = X(z_{ij}, t_{ij})^T \beta \]

где  $ z_{ij} $ -- значение ковариаты $z$ в момент $ t_{ij}, i = 1, \dots, k; j = 1, \dots, n_{ij} $.

Чаще всего используют экспоненциальные семейства

\[ f(y; \theta; \phi) = \exp \left( \frac{y \theta - b(\theta)}{a(\phi)} + c(y; \phi) \right) \]

Система уравнений % TODO formulas

Решается, как правило, численными методами.

Если наблюдения зависимые, моделируем покомпонентно.

TODO everything starting ~40 min

Состоятельность оценок не меняется, а эффективность может быть повышена.
Предполагается ввести т.н. "рабочую" матрицу ковариации $ R = diag(R_i) $.

Фиктивная функция правдоподобия не является как таковой функцией правдоподобия для какого-то распределения.
При определённых условиях регулярности получаем асимптотическую нормальность.
Как и в случае с независимыми компонентами, матрица представляется как произведение усреднённой матрицы информации на $ \dots $

% TODO

Квазиинформационный критерий, как утверждается, асимптотически эквивалентен критерию Акаике.

\begin{leftbar}
Всё, что связано с асимптотическими методами, будет здесь работать.
\end{leftbar}

Но такой метод почему-то нечасто используется.

\subsection{Использование обобщённых смешанных моделей}

В случае лонгитюдных данных появляется схема независимых блоков.
Случайный эффект можно вводить по-разному; важно, чтобы была сходимость.

Чтобы в том или ином смысле была сходимость, то есть состоятельность оценок (и второе свойство -- асимптотическая нормальность, которая без состоятельности быть не может), ... (TODO).

\[ g(\mathds E_\theta (Y_i,b_i)) = X_i^T \beta + Z_i^T b_i \]

Значения $ b_i $ -- не параметры модели, они имеют распределение, параметры которого являются параметрами модели.

% TODO some

Можно использовать и модель дисперсионного анализа.
Часто используют так называемые ортогональные многочлены (по аналогии с моделями для нормальных наблюдениях).
Если есть сопутствующий признак, можно ввести аддитивное влияние этого признака.



\end{document}