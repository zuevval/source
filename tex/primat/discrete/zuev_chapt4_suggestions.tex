%XeLaTeX+makeIndex+BibTeX OR LuaLaTeX+...
\documentclass[a4paper,12pt]{article} %14pt - extarticle
\usepackage[utf8]{inputenc} %русский язык, не менять
\usepackage[T2A, T1]{fontenc} %русский язык, не менять
\usepackage[english, russian]{babel} %русский язык, не менять
\usepackage{fontspec} %различные шрифты
\setmainfont{Times New Roman}
\defaultfontfeatures{Ligatures={TeX},Renderer=Basic}
\usepackage{hyperref} %гиперссылки
\hypersetup{pdfstartview=FitH,  linkcolor=blue, urlcolor=blue, colorlinks=true} %гиперссылки
\usepackage{subfiles}%включение тех-текста
\usepackage{graphicx} %изображения
\usepackage{float}%картинки где угодно
\usepackage{textcomp}
\usepackage{amssymb}

\usepackage{listings} %code formatting
\lstset{language=pascal,
	%keywordstyle=\color{blue},
	%commentstyle=\color{},
	%stringstyle=\color{red},
	tabsize=1,
	breaklines=true,
	columns=fullflexible,
	%numbers=left,
	escapechar=@,
	morekeywords={return}}


\begin{document}
\author{Валерий Зуев} %\\ %\href{mailto:valera.zuev.zva@gmail.com}{valera.zuev.zva@gmail.com}}
\title{Предлагаемые уточнения к главе 4}
\maketitle
\begin{enumerate}
	\item{4.1.6 (620/1496)} \textit{Квантор $\exists!$ - <<синтаксический сахар>>}: если продолжить эту мысль, можно заметить, что в алгебре высказываний, ввведённой в 4.1.4 (2/4) как $<{T,F}; \vee, \&, \neg>$ тоже что-то можно назвать синтаксическим сахаром. Например, $a\vee b \Leftrightarrow \neg(\neg a \& \neg b), T \Leftrightarrow \neg F$
	\item{4.2.5 (620/1496)} Строка 1. Предположительно не $\mathcal{T}\rightarrow M$, а  $\mathcal{T}\rightarrow\mathcal{M}$, в соответствии с определением интерпретации формальной теории, данным в 4.2.3 (618/1496)
	\item {4.3.3 (5/6) (632/1496)} Вся нижняя половина страницы - условное ветвление внутри 
	\begin{lstlisting}
		if var(b) & b @$\in$@ A then
	\end{lstlisting}
	никогда не будет достигнута, поскольку в коде в верхней половине слайда все возможные сценарии ветвления в строках
	\begin{lstlisting}
	if var(a) & a @$\in$@ B then
	...
	end if
	\end{lstlisting}
	заканчиваются возвратом чего-либо. Предлагаю модифицировать весь блок:
	\begin{lstlisting}
	if var(a) @$\vee$@ var(b) then
		for <@$\alpha, \beta, \vartheta$@ > in {<a, b, B>,<b, a, A>} do
			if not var(@$\alpha$@) then next for <@$\alpha, \beta, \vartheta$@ >  end if
			if @$\alpha \in \vartheta$@ then return 0 end if
			if @$S[\alpha]=\varnothing$@ then @$S[\alpha]=\vartheta$@ return 1 end if
			return @$S[\alpha]=\vartheta$@
		end for
	end if
	\end{lstlisting}
\end{enumerate}
\end{document}