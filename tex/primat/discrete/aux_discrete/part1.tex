\documentclass[zuev-report.tex]{subfiles}
\begin{document}
\section{Постановка задачи}
Требуется реализовать программы, вычисляющие значения распространённых в комбинаторике функций:\\
\begin{enumerate}
	\item Число перестановок $P(n)=n!$
	\item Число размещений с повторениями $U(m,n)=m^n$
	\item Число размещений без повторений $A(m,n)=\frac{m!}{(m−n)!}$
	\item Число сочетаний $C(m,n)=\frac{m!}{n!(m−n)!}$
	\item Число сочетаний с повторениями $V(m,n)=C(m+n−1,n)=\frac{(m+n−1)!}{n!(m−1)!}$
	\item Числа Фибоначчи 
	$$Fib(n)=Fib(n-1)+Fib(n-2)$$
	$$Fib(2)=Fib(1)=1$$
\end{enumerate}
Программа должна давать правильный ответ для любых входных данных из области определения, если исходные данные и результат по модулю не превосходят $2^{8}-1$ (для большинства компиляторов языка C++ - максимальное целое число без знака). В случае, когда результат по модулю превышает максимальное целое число, следует дать ответ UINT\_MAX (индикатор переполнения, который заведомо не является правильным ответом для каких-либо входных данных).
\newpage
\section{Описание алгоритмов}
\def \txtIn {\textbf{Исходные данные алгоритма:}\\}
\def \txtOut {\textbf{Выходные данные алгоритма:}\\}
\def \txtTxt {\textbf{Текст программы:}}
\subsection{Число перестановок}
\txtIn
$n$ - беззнаковое целое от 0 до UINT\_MAX-1\\
\txtOut
$P(n)=n!$ - беззнаковое целое или UINT\_MAX в случае переполнения\\
\txtTxt
\label{permutations}
\lstinputlisting{code/permutations.c}
\subsection{Число размещений с повторениями}
\txtIn
$n$ - беззнаковое целое от 0 до UINT\_MAX-1\\
$m$ - беззнаковое целое от 0 до UINT\_MAX-1\\
\txtOut
$U(m,n)=m^n$ - беззнаковое целое или UINT\_MAX в случае переполнения\\
\txtTxt
\label{perm_with_repet}
\lstinputlisting{code/umn.c}
\subsection{Число размещений без повторений}
\txtIn
$n$ - беззнаковое целое от 0 до UINT\_MAX-1\\
$m$ - беззнаковое целое от 0 до UINT\_MAX-1\\
\txtOut
$A(m,n)=\frac{m!}{(m−n)!}$ - беззнаковое целое или UINT\_MAX в случае переполнения\\
\txtTxt
\label{perm_no_repet}
\lstinputlisting{code/amn.c}
\subsection{Число сочетаний}
\txtIn
$n$ - беззнаковое целое от 0 до UINT\_MAX-1\\
$m$ - беззнаковое целое от 0 до UINT\_MAX-1\\
\txtOut
$C(m,n)=\frac{(m)!}{n!(m−n)!}$ - беззнаковое целое или UINT\_MAX в случае переполнения\\
\txtTxt
\label{binomial}
\lstinputlisting{code/cmn.c}
\subsection{Число сочетаний с повторениями}
\txtIn
$n$ - беззнаковое целое от 0 до UINT\_MAX-1\\
$m$ - беззнаковое целое от 0 до UINT\_MAX-1\\
\txtOut
$V(m,n)=C(m+n−1,n)=\frac{(m+n−1)!}{n!(m−1)!}$ - беззнаковое целое или UINT\_MAX в случае переполнения\\
\txtTxt
\label{vmn}
\lstinputlisting{code/vmn.c}
\subsection{Числа Фибоначчи}
\txtIn
$n$ - беззнаковое целое от 1 до \\ %TODO: upper bound
\txtOut
$Fib(n)=Fib(n-1)+Fib(n-2)$ - беззнаковое целое или UINT\_MAX в случае переполнения\\
\txtTxt
\label{fib}
\lstinputlisting{code/fib.c}
\section{Описание тестирования}
Для тестирования использовалась библиотека Google Test Framework. Также отсутствие утечек памяти проверено утилитой Valgrind memcheck tool. 
Работа каждой функции проверена на трёх группах входных данных:
\begin{enumerate}
	\item{} Базовые входные данные: несколько значений или пар значений в пределах 1-10
	\item{} Докритические значения: данные, при которых ответ приближается к верхней границе беззнакового целого числа, но не превышает её.
	\item{} Значения выше критических: проверяется правильная обработка данных, результат которых не вмещается в беззнаковое целое число.
\end{enumerate}
\end{document}
