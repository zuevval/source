\documentclass[zuev-report.tex]{subfiles}
\begin{document}
\subsection{Код тестирующей программы}
\lstinputlisting{code/tests.cpp}
\newpage
\section{Результаты тестирования}

Все тесты пройдены, в каждом случае программа выдала корректный ответ. Утечек памяти не обнаружено. Ниже  приведён вывод тестирующей программы.

\begin{verbatim}

==9274== Memcheck, a memory error detector
==9274== Copyright (C) 2002-2017, and GNU GPL'd, by Julian Seward et al.
==9274== Using Valgrind-3.13.0 and LibVEX; rerun with -h for copyright info
==9274== Command: /home/zuevval/source/tex/primat/discrete/aux_discrete
/code/cmake-build-debug/code
==9274== 
[==========] Running 18 tests from 6 test cases.
[----------] Global test environment set-up.
[----------] 3 tests from cmn
[ RUN      ] cmn.basic
[       OK ] cmn.basic (27 ms)
[ RUN      ] cmn.underCrit
[       OK ] cmn.underCrit (5 ms)
[ RUN      ] cmn.overCrit
[       OK ] cmn.overCrit (7 ms)
[----------] 3 tests from cmn (59 ms total)

[----------] 3 tests from amn
[ RUN      ] amn.basic
[       OK ] amn.basic (5 ms)
[ RUN      ] amn.underCrit
[       OK ] amn.underCrit (5 ms)
[ RUN      ] amn.overCrit
[       OK ] amn.overCrit (4 ms)
[----------] 3 tests from amn (16 ms total)

[----------] 3 tests from pn
[ RUN      ] pn.basic
[       OK ] pn.basic (5 ms)
[ RUN      ] pn.underCrit
[       OK ] pn.underCrit (3 ms)
[ RUN      ] pn.overCrit
[       OK ] pn.overCrit (3 ms)
[----------] 3 tests from pn (11 ms total)

[----------] 3 tests from umn
[ RUN      ] umn.basic
[       OK ] umn.basic (5 ms)
[ RUN      ] umn.underCrit
[       OK ] umn.underCrit (5 ms)
[ RUN      ] umn.overCrit
[       OK ] umn.overCrit (5 ms)
[----------] 3 tests from umn (16 ms total)

[----------] 3 tests from vmn
[ RUN      ] vmn.basic
[       OK ] vmn.basic (7 ms)
[ RUN      ] vmn.underCrit
[       OK ] vmn.underCrit (4 ms)
[ RUN      ] vmn.overCrit
[       OK ] vmn.overCrit (6 ms)
[----------] 3 tests from vmn (18 ms total)

[----------] 3 tests from fib
[ RUN      ] fib.basic
[       OK ] fib.basic (5 ms)
[ RUN      ] fib.underCrit
[       OK ] fib.underCrit (3 ms)
[ RUN      ] fib.overCrit
[       OK ] fib.overCrit (4 ms)
[----------] 3 tests from fib (14 ms total)

[----------] Global test environment tear-down
[==========] 18 tests from 6 test cases ran. (223 ms total)
[  PASSED  ] 18 tests.
==9274== 
==9274== HEAP SUMMARY:
==9274==     in use at exit: 0 bytes in 0 blocks
==9274==   total heap usage: 359 allocs, 359 frees, 136,746 bytes allocated
==9274== 
==9274== All heap blocks were freed -- no leaks are possible
==9274== 
==9274== ERROR SUMMARY: 0 errors from 0 contexts (suppressed: 0 from 0)
\end{verbatim}
\section{Выводы}
В работе показана возможность построения алгоритмов, вычисляющих комбинаторные числа (число перестановок (\ref{permutations}), число размещений с повторениями (\ref{perm_with_repet}) и без (\ref{perm_no_repet}), число сочетаний с повторениями (\ref{binomial}) и без (\ref{vmn}), числа Фибоначчи (\ref{fib})). Сконструированные алгоритмы принимают на вход беззнаковые целые числа и способны проводить вычисления в пределах, определяемых величиной ответа: ответ по значению не превосходит максимально возможного беззнакового целого. Если ответ не вмещается в отведённое для беззнакового целого число бит, алгоритмы успешно отслеживают это и выдают зарезервированное значение - максимальное беззнаковое целое число, заведомо не являющееся комбинаторным числом из перечисленного списка.\\ Приведённые алгоритмы не основаны на рекурсии, но оказывается, что накладываемые ограничения на величину ответа приводят к тому, что в процессе вычисления требуется немного итераций (обычно менее ста), и решения с применением рекурсии также могут успешно справляться с задачей.
\end{document}
