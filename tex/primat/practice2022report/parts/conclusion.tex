Программный пакет для моделирования структурными уравнениями \texttt{Semopy} предоставляет пользователям широкий арсенал средств для построения, оценки, визуализации и инспекции линейных моделей с латентными переменными и причинно-следственными связями.
Однако ещё существует пространство для усовершенствования этого пакета, внедрения новых классов моделей и прочих функциональных элементов, исправления недочётов.

В настоящей работе изучены несколько вопросов, возникших в ходе эксплуатации пакета: недетерминированность результатов, возможность и корректность вычисления статистик для класса моделей со случайным эффектом \texttt{ModelEffects}, вычисление широко употребимой статистики $ R^2 $.
Удалось написать код, решающий поставленные задачи; но, однако, вычисление статистик для \texttt{ModelEffects} ещё следует подкорректировать.

Во время выполнения работы автор отчёта стал соавтором в публикации \cite{lanczos}.
