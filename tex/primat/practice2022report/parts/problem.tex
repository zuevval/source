\subsection{Модели данных}

Моделирование структурными уравнениями, помимо связей между наблюдаемыми переменными, позволяет определять \emph{латентные} (не наблюдаемые, но влияющие на них).

Основные виды моделей, реализованные в пакете \texttt{Semopy} -- \texttt{Model} \eqref{eq:model}  (классический SEM, где предполагается, что распределение всех переменных -- и нормальных в том числе -- центральное нормальное), \texttt{ModelMeans} \eqref{eq:modelMeans} (SEM с нецентральным распределением), \texttt{ModelEffects} \eqref{eq:modelEffects} (SEM со случайными эффектами -- для зависимых данных). \\

Система уравнений, определяющая взаимосвязи между переменными в классе \texttt{Model}

\begin{equation}\label{eq:model}
	\begin{cases}
		\begin{pmatrix}
			\eta \\ x
		\end{pmatrix} \equiv \omega = B \omega + \epsilon, \epsilon \sim \mathcal{N}(0, \Psi) \\
		\begin{pmatrix}
			y \\ x^{(1)}
		\end{pmatrix} \equiv z = \Lambda \omega + \delta, \delta \sim \mathcal{N}(0, \Theta)
	\end{cases}
\end{equation}

где $ \eta $ -- вектор введённых латентных переменных, $ x $ -- наблюдаемые входные переменные и $ y $ -- наблюдаемые выходные; $ x^{(1)} $ -- \emph{эндогенные} переменные (те, которые зависят от каких-либо других переменных).
$ B, \Lambda, \Psi, \Theta $ -- матрицы, параметризованные согласно модели (для каждой пары связанных переменных вводится коэффициент; в матрице $ \Lambda $ для каждой латентной переменной один из элементов фиксируется и приравнивается к единице; остальные элементы нулевые). \\

\texttt{ModelMeans} является обобщением \eqref{eq:model}, позволяющим моделировать нецентрированные переменные (в том числе латентные).

\begin{equation}\label{eq:modelMeans}
	\begin{cases}
		\begin{pmatrix}
			\eta \\ x^{(1)}
		\end{pmatrix} \equiv \omega = \Gamma_1 x^{(2)} + B \omega + \epsilon, \epsilon \sim \mathcal{N}(0, \Psi) \\
		\begin{pmatrix}
			y \\ x^{(1)}
		\end{pmatrix} \equiv z = \Gamma_2 x^{(2)} + \Lambda \omega + \delta, \delta \sim \mathcal{N}(0, \Theta)
	\end{cases}
\end{equation}

Здесь все обозначения сохранены, $ \Gamma_1 $ и $ \Gamma_2 $ -- тоже параметризованные матрицы. \\

Рассмотренные модели предполагают, что наблюдения представлены в виде выборки. \texttt{ModelEffects} предполагает, что между наблюдениями есть зависимость, которая выражается матрицей ковариации $ K_{ n \times n } $, где $ n $ -- число точек данных.
Такая модель записывается уже в терминах матриц переменных, а не векторов:

\begin{equation}\label{eq:modelEffects}
	\begin{cases}
		\begin{pmatrix}
			H \\ X^{(1)}
		\end{pmatrix} \equiv W = \Gamma_1 X^{(2)} + BW + E, \quad E \sim \mathcal{MN}(0, \Psi, I_n) \\
		\begin{pmatrix}
			Y \\ X^{(1)}
		\end{pmatrix} \equiv Z = \Gamma_2 X^{(2)} + \Lambda W + \Delta + U, \quad \Delta \sim \mathcal{MN}(0, \Theta, I_n), U \sim \mathcal{MN}(0, D, K)
		
	\end{cases}
\end{equation}

Эти модели были реализованы в пакете \texttt{Semopy}, то есть возможно их построение, нахождение параметров методами оптимизации с различными функциями потерь, а также вычисление статистик ($ \chi^2 $, информационные критерии и так далее).

\subsection{Поставленные задачи}

В рамках практической работы поставлены следующие задачи:

\begin{enumerate}
	\item Выяснить и устранить причины обнаруженной недетерминированности в процессе оптимизации (от запуска к запуску программа выдаёт различные результаты)
	\item Проверить корректность вычисления статистик, оптимизированных для модели \texttt{Model}, в случае использования других моделей
	\item Добавить к статистикам новую -- $ R^2 $ (коэффициент детерминированности)
\end{enumerate}
