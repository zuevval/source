Одна из основных задач, которую решает статистика -- нахождение в данных закономерностей, которые позволяют по набору измерений оценить степень зависимости одной или нескольких наблюдаемых переменных от некоторых ковариат.
Как правило, для этого используются регрессионные модели, дисперсионный анализ, а также различные методы кластеризации и понижения размерностей. \\

Часто исследователь обладает некоторыми априорными знаниями о взаимосвязи переменных (или хочет проверить гипотезы об их взаимосвязи).
Например, имея наборы откликов $ \{ y_1, y_2, y_3 \} $ и ковариат  $ \{ x_1, x_2, x_3 \} $, он может предположить, что $ y_1 $ линейно зависит от $ x_1  $ и $ x_2 $, $ y_2 $ от $ x_2 $ и $ x_3 $, а также $ y_3 $ и $ y_2 $ линейно зависят от некой переменной $ \eta $, которую мы не наблюдаем, но которая является линейной комбинацией $ x_1, x_2 $ и $ x_3 $.

В таких случаях -- когда предполагается сложная структура взаимосвязей между переменными -- можно применять метод \emph{моделирования структурными уравнениями} (Structural Equation Modelling, SEM) \cite{hoyle2021sem}, который был разработан в середине XX века и изначально предназначался для моделирования поведения испытуемых в социологических опросах.
В дальнейшем SEM был применён в других областях, в том числе в биоинформатике.
Так, в Санкт-Петербургском Политехническом университете был разработан программный пакет Semopy \cite{semopy, semopy2}, в котором реализованы различные вариации структурных моделей, подходящие в том числе для исследования связи генотипа организма с его фенотипом. \\

Последние полтора года пакет Semopy всячески модернизируется и дополняется новыми возможностями.
В данной работе описаны изменения, внесённые автором в последние месяцы.
В это же время по итогам промежуточной работы написана статья \cite{lanczos}.
