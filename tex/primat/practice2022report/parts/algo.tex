
\subsection{ Источники недетерминированности }

Язык Python не имеет стандарта, соответственно, выполнение программы может различаться от окружения к окружению и от интерпретатора к интерпретатору.
Однако в подавляющем большинстве случаев используется кросс-платформенная реализация \texttt{CPython} \cite{martelli2017}.
На ней были проведены вычислительные эксперименты и для неё предложены исправления в коде.

Разработчики интерпретатора, как правило, прикладывают усилия к тому, чтобы 
Основные потенциальные источники недетерминированности в программе:

\begin{enumerate}
	\item Псевдослучайные состояния -- например, значения переменных при сэмплировании точек из какого-либо распределения
	\item Случайные состояния (основанные на внешнем источнике случайности) -- не используются в Semopy
	\item Состояния, не являющиеся намеренно случайными или псевдослучайными, но не предопределённые (например, порядок завершения потоков). \label{item:nondeterminism}
\end{enumerate}

В ходе последовательного анализа было установлено, что причина недетерминированности заключается в  \ref{item:nondeterminism}: в некоторых местах программы переменные записывались в словари (dict) и затем последовательно считывались в процессе итерирования по словарю.
Подразумевалось, что порядок записи соответствует порядку чтения, но в \texttt{CPython} это не гарантируется, поэтому результат выполнения получался разный в разных запусках (разница в найденных коэффициентах составляла порядка $ 1 \cdot 10^{-3} $ от запуска к запуску).

Решение проблемы заключается в замене используемых типов \texttt{dict} на \texttt{list}.

\subsection{ Проверка корректности вычисления статистик }

В ходе работы были выявлены ошибки в построении теста  $ \chi^2 $ для класса \texttt{ModelEffects}.
Реализация теста предполагает сравнение построенной модели с базовой, в которой все коэффициенты корреляции равны нулю.
Для классов \texttt{Model} и \texttt{ModelEffects} базовую модель можно построить без использования данных; но в \texttt{ModelEffects} даже в базовой модели предполагается зависимость данных, и поэтому информация о них должна быть сообщена фунции, вычисляющей $ \chi^2 $.

Код, использующий  матрицу корреляции между группами в вычислении $ \chi^2 $ ранее не был дописан, что и исправлено в данной работе. \\

Используемый в программе способ вычисления статистики \texttt{DoF} (degrees of freedom, число степеней свободы модели) также не подходит для класса \texttt{Model\-Effects}.
Реализованный в настоящее время способ вычисления степеней свободы через число параметров и число наблюдаемых переменных

\[ DoF = \frac{p \cdot (p + 1)}{2} - N_{params} \]

где $ p $ -- число наблюдаемых переменных и $ N_{params} $ -- число параметров, подходит в случае, когда данные представлены в виде выборки и вектор имеет многомерное нормальное распределение \eqref{eq:model} \eqref{eq:modelMeans}, но не в случае матричного нормального распределения \eqref{eq:modelEffects}.
Это ещё предстоит исправить в дальнейшем.

\subsection{Коэффициент детерминированности}

Коэффициент детерминированности для переменной $ u $ (доля объяснённой дисперсии)

\begin{equation}\label{eq:rsquare}
	R^2(u) = 1 - \frac{\sum _i (u_i - \bar u)^2}{ \sum_i (u_i - f_i)^2 }
\end{equation}

где сумма берётся по всем индексам $ i $ набора данных, $ f_i $ -- предсказание $ u_i $ согласно модели, $ u_i $ - значения из набора данных, $ \bar u $ -- среднее значение.

Таким образом, $ R^2 $ считается для эндогенных переменных

Если рассчитать стандартизованные оценки параметров и при оценивании использовать метод максимального правдоподобия, то \eqref{eq:rsquare}  вычисляется через стардартизованную оценённую дисперсию переменной (которая, как правило, входит в число параметров):

\begin{equation}\label{eq:r2Simplified}
	R^2(u) = 1 -  \widehat{var(u)}
\end{equation}

Такой подход реализован в программном пакете \texttt{lavaan} \cite{lavaan2012}. 
Он же был применён и в \texttt{Semopy}. \\

Если в модели присутствует группировка, коэффициент детерминированности можно считать как отдельно для каждой группы, так и по всему набору данных.
