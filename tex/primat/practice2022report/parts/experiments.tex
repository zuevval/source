
Для проверки корректности работы программы использован классический набор данных <<Политическая демократия>> \cite{bollen1979}, который часто используется для демонстрации работы методов структурного моделирования.

\begin{figure}[H]
	\centering \includegraphics[width=\myPictWidth]{poldemo.pdf}
	\caption{ Модель взаимосвязей между факторами, определяющими или измеряющими развитость институтов политической демократии }
	\label{img:poldemo}
\end{figure}

Это данные из исследования, посвящённого связи показателей экономического роста с показателями демократичности общественных институтов.
Переменные $ x_1, x_2, x_3 $ обозначают валовой национальный продукт, энергопотребление и долю населения, занятую в индустрии (в 1960 году); $ y_1 $ - $ y_4 $ -- различные метрики демократичности власти и прессы в 1960 году, $ y_4 $ - $ y_7 $ -- в 1965 году.

В модели вводятся латентные переменные и связи, которые определяют представление исследователей о влиянии одних наблюдаемых переменных на другие.
С помощью SEM и статистических тестов в оригинальной работе \cite{bollen1979} было установлено, что модель не противоречит данным.
