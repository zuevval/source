По итогам проведения вычислительных экспериментов для каждой из поставленных задач были получены удовлетворительные результаты, которые приводятся и обсуждаются ниже.

\begin{enumerate}
	\item Детерминированность вычислений после исправления способа хранения переменных проведена для ряда моделей, в том числе линейной регрессии, множественной линейной регрессии и политической демократии.
	В модели политической демократии расхождение в оценке параметров от запуска к запуску в новой версии программы не превышает $ 10^{-16} $ по абсолютному значению.
	
	\item Введение возможности вычислять статистики для моделей класса \texttt{Model\-Effects} позволило получить значения для ряда наборов данных.
	В частности, был проверен сгенерированный набор данных -- линейная регрессия с группировкой по фактору, имеющему 4 уровня.
	Тем не менее, значения числа степеней свободы моделей остаются неправдоподобно низкими, что необходимо исправить в дальнейшем.
	
	\item По формуле \eqref{eq:r2Simplified} для коэффициента недетерминированности $ R^2 $ были произведены расчёты для модели политической демократии (рис. \ref{img:poldemo}).
	В результате найденные значения расходятся с соответствующими значениями, вычисленными пакетом \texttt{lavaan}, на абсолютную величину, не превышающую $ 0.02 $ (см. таблицу \ref{table:r2}), что можно считать хорошим показателем.
\end{enumerate}
