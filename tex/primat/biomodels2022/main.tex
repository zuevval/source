% !TeX spellcheck = en_US
% !TeX program = xelatex

\documentclass[a4paper,12pt]{article}
\usepackage[utf8]{inputenc} % russian, do not change
\usepackage[T2A, T1]{fontenc} % russian, do not change
\usepackage[english, russian]{babel} % russian, do not change

% fonts
\usepackage{fontspec} % different fonts
\setmainfont{Times New Roman}
\usepackage{setspace,amsmath}
\usepackage{amssymb} %common math symbols
\usepackage{dsfont}

\usepackage{enumitem}

\begin{document}

\section{4.1 (Дима Мальцов)}

Модель не предполагает динамики изменения (скорости всех реакций не зависят от времени)

Модель регуляции генов представляется в виде графа, который позволяет учитывать неравновесные состояния.
Микросостояния

Если есть ТД-равновесие, то неважно, каким путём мы пришли в какое-то микросостояние.
Если ТД-равновесия нет...

Микроскопический, макроскопический подход.
Микро: вероятности переходов.
Система всё время находится в каком-то микросостоянии.
Макроскопический подход: вершины -- химические вещества, рёбра -- реакции.
Для равновесия в системе нужна как минимум [обратимость всех реакций] -- может, необязательно? обязательно лишь сильная связность графа?

Микро: изменение вероятности состояния

\[ \frac{d}{dt} u(t) = \mathcal L (G) \cdot u(t) \]

Макро: изменение концентрации

\[ \frac{d}{dt} x(t) = \mathcal L (G) \cdot x(t) \]

$ \mathcal L(G) $ -- лапласиан графа.

В стационарном состоянии все рёбра реверсивны и скорости в прямую и обратную сторону одинаковы.

Если равновесия нет, но % TODO a bit
Ищут остовные подграфы.

В неравновесном случае можно ра

\section{4.3 (Dana)}

У прокариот неравновесная регуляция проходит хорошо, у эукариот -- нет

Множество моделей.
Метод: как

Жизнь в принципе не связана с равновесием.
Равновесная клетка -- это мёртвая клетка.
50\% от всего потраченного клеткой АТФ -- внутренние процессы: перестройка цитоскелета и так далее.
Поэтому процесс транскрипции не самый энергозатратный, и не факт, что равновесные модели разумны (энергетическая выгода не главная при выборе модели, есть другие выгоды, на которые стоит обратить внимание).

\subsection{Основы неравновесных процессов регулирования}

Примеры: миозин шагает по актиновой нити; также миозиновый цикл

Модели системы из трёх компонент:

Fully Reversible: число завершение цикла 1-2-3 колеблется около нуля, может быть

Fully Irreversible (1->2->3->1): только положительное число завершений цикла.

Признаки неравновесной регуляции в экспериментах были найдены.
Казалось бы, ТФ блуждают в клетке и случайно присоединяются или отсоединяются, поэтому за "нулевую" гипотезу можно принять равновесный процесс.

Хи-квадрат-образное распределение -- признак неравновесности

Отсоединение -- proofreading step.

Специфичные сайты (с высокой аффинностью) и неспецифичные сайты (с низкой аффинностью).
Горб: есть некая оптимальность, некий максимум.

Фенотипы: статические, динамические регуляторные и так далее.

Если мы отказываемся от предположения равновесности (которое технически удобно из-за малой размерности параметра), параметров становится слишком много и становится сложно полностью определить модель.
Предлагается решать задачу оптимизации, задавшись величинами, которые для клетки важны.
Среди небольшого ансамбля моделей уже будем искать.
По всем фенотипам utility function -- мера оптимальности

\section{5.1 [Anna K] Interlocking feedback loops govern the dynamic behavior of the floral transition in Arabidopsis}

Apr 26, 2022

Вегетативное состояние.
Продуктивное состояние.

Ищем гены, которые отвечают за цветение Arabidopsis.
Arabidopsis -- модельный организм.

Сначала идёт вегетативный рост.
Когда приходит весна, день увеличивается и ночь укорачивается; начинается переход от вегетативного роста в продуктивное состояние.
Первый шаг при этом -- экспрессия FT.

FT-протеин: мобильный сигнал.
Он образует комплекс с FD.
Этот комплекс, когда образуется, вызывает цветение.

Три этапа: листок, розетка, цветок.

Две архитектуры цветения:
\begin{enumerate}[noitemsep]
	\item Determined (один цветок, больше не цветём)
	\item Indetermined (несколько цветков, и после перехода в "режим цветения" рост продолжается).
	В этом случае нужно, чтобы клетки дифференцировались не на всей меристеме,
	
	TFL-1 репрессирует AP1 в центре меристемы, а по бокам клетки дифференцированные.
\end{enumerate}

Горох в процессе эволюции пережил многократную дупликацию генома, поэтому у него много копий FT (у Arabidopsis -- одна).

Биологи выделили гены, которые влияют на цветение.
Построили математическую модель, но оказалось, что она плохо описывает реальные данные.

Авторы статьи рассмотрели две возможные альтернативные модели:
\begin{enumerate}[noitemsep]
	\item Есть дополнительный фактор -- SOC-1
	\item Есть дополнительное влияние -- LFY активирует TFL-1
\end{enumerate}

Уравнения: функция Хилла

Единица времени: как оказывается, наиболее выгодная -- число цветочков.

\section{5.2 [Настя]}

Исследователи постарались построить простую модель цветения, обладающую двумя свойствами:

\begin{enumerate}[noitemsep]
	\item Цветение должно начинаться не просто когда солнце немного посвветило, а когда началось длительное потепление
	\item Цветение должно закончиться (необратимый процесс)
\end{enumerate}

Функция Coherent Feedforward Loop -- фильтрация шумов: сигнал доходит только когда он долговременный, а не малая флуктуация.

Эффект флуктуации

\section{5.3 [Серёжа] Детерминированные и стохастические модели для Arabidopsis}

Однолетнее, семейство горчичных.

Apetelia1 (AP1), Leafy (LFY), Suppressor of Overexpression of Constants 1 (SOC1), Agamous-Like 24 (AGL24), Flowering Locus T (FT), FD.

Авторы статьи хотели упростить модель, предложенную в предыдущей статье, путём добавления стохастичности.

Временная задержка

Два подхода к добавлению стохастических эффектов в модель:

\begin{enumerate}[noitemsep]
	\item Уравнение формулируется в терминях вероятностей
	\item ...
\end{enumerate}

\end{document}