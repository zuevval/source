% !TeX spellcheck = en_US
% !TeX program = pdflatex

\documentclass{beamer}
\usepackage[utf8]{inputenc} % russian, do not change
\usepackage[T2A, T1]{fontenc} % russian, do not change
\usepackage[english, russian]{babel} % russian, do not change

% fonts
%\usepackage{fontspec} % different fonts
%\setmainfont{Times New Roman}
\usepackage{setspace,amsmath}
\usepackage{amssymb} %common math symbols
\usepackage{dsfont}

% utilities
\usepackage{systeme} % systems of equations
\usepackage{mathtools} % xRightarrow, xrightleftharpoons, etc
\usepackage{array} % utils for tables
\usepackage{makecell} % multirow for tables
\usepackage{subfiles}
\usepackage{hyperref}
\hypersetup{pdfstartview=FitH,  linkcolor=blue, urlcolor=blue, colorlinks=true}
\usepackage{framed} % advanced frames, boxes
\usepackage{graphicx}
\usepackage{caption}
\usepackage{subcaption} % captions for subfigures
\usepackage{color}
\usepackage{chngcntr} % change counters
%\counterwithout*{section}{chapter} % continue sections enumeration with chngcntr


% styling
\usetheme{default}
\usepackage{float} % force pictures position
\floatstyle{plaintop} % force caption on top
\usepackage{enumitem} % itemize and enumerate with [noitemsep]
\setlength{\parindent}{0pt} % no indents!

% misc
\graphicspath{{./img/}}
\newcommand{\myPictWidth}{.95\textwidth}
\newcommand{\phm}{\phantom{-}}

\title{Кинетическая коррекция: Новый механизм уменьшения числа ошибок в биосинтетических процессах, требующих высокой специфичности}
\subtitle{(Kinetic Proofreading <...> [Hopfield, 1974])}
\author{Валерий Зуев}
\begin{document}
\begin{frame}[plain]
    \maketitle
\end{frame}
\begin{frame}{Интуитивные соображения}

\end{frame}
\begin{frame}{Big Picture}
Специфичность при чтении ДНК (и других реакциях биосинтеза) может быть увеличена по сравнению со специфичностью, обусловленной разницей средних кинетических барьеров, с помощью процесса \emph{кинетической коррекции}.

В статье описан простой кинетический путь, приводящий к такой коррекции, когда реакция сильно, но неспецифично катализируется, например, гидролизом фосфата.

Известные реакции, кажущиеся излишними, оказываются важными для осуществления коррекции.

\end{frame}

\begin{frame}{Малое число ошибок при синтезе}

Доля ошибок в простой модели синтеза:
\[ p_{err}^{(1)} = \exp \left( - \frac{\Delta G_{CD}}{RT} \right) \]

Модель с коррекцией (точность коррекции та же, что и специфичность при синтезе):

\[ p_{err}^{(2)} = \left(p_{err}^{(1)}\right)^2 \]


\end{frame}

\begin{frame}{Модель кинетической коррекции}
Классическая схема реакции: кинетика Михаэлиса

\[ C + c \overunderset{k'_C}{k_C}{\rightleftharpoons} Cc \overset{W}\to \text{correct product} \quad K_C = \frac{k'_C}{k_C} \]

\end{frame}

\end{document}
