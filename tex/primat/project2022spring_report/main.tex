% !TeX spellcheck = en_US
% !TeX program = xelatex

\documentclass[a4paper,12pt]{article}
\usepackage[utf8]{inputenc} % russian, do not change
\usepackage[T2A, T1]{fontenc} % russian, do not change
\usepackage[english, russian]{babel} % russian, do not change

\usepackage{setspace} % `setspace` moved from `styling`: https://tex.stackexchange.com/a/577582

% fonts
\usepackage{fontspec} % different fonts
\setmainfont{Times New Roman}
\usepackage{amsmath}
\usepackage{amssymb} %common math symbols
\usepackage{dsfont}

% utilities
\usepackage{systeme} % systems of equations
\usepackage{mathtools} % xRightarrow, xrightleftharpoons, etc
\usepackage{array} % utils for tables
\usepackage{makecell} % multirow for tables
\usepackage{subfiles}
\usepackage{hyperref}
\hypersetup{pdfstartview=FitH,  linkcolor=blue, urlcolor=blue, colorlinks=true}
\usepackage{framed} % advanced frames, boxes
\usepackage{graphicx}
\usepackage{caption}
\usepackage{subcaption} % captions for subfigures
\usepackage{color}
\usepackage{chngcntr} % change counters
%\counterwithout*{section}{chapter} % continue sections enumeration with chngcntr
% \usepackage{theorem}
\usepackage{amsthm} % theorems with proofs
\newtheorem{thrm}{Теорема}

\usepackage[backend=biber, bibstyle=gost-numeric, sorting=none, block=space]{biblatex} % this does not always work: in TeXStudio you have to choose "default bibliography tool = Biber" manually
\addbibresource{bibliography.bib}
\DeclareFieldFormat[inproceedings]{pagetotal}{\mkpagetotal[bookpagination]{#1}}
\renewbibmacro*{chapter+pages}{%
    \printfield{note}%
    \setunit{\bibpagespunct -}%
    \printfield{pagetotal}%
    \setunit{\bibpagespunct}%
    \printfield{pages}%
    \newunit}


% styling
\usepackage{float} % force pictures position
\floatstyle{plaintop} % force caption on top
\usepackage{enumitem} % itemize and enumerate with [noitemsep]
\setlength{\parindent}{0pt} % no indents!

% misc
\graphicspath{{./img/}}
\newcommand{\myPictWidth}{.95\textwidth}
\newcommand{\phm}{\phantom{-}}
\newcommand{\defeq}{\overset{def}=}

\begin{document}

% STU 2022 SPRING PROJECT REPORT

\subfile{titlepage}
\tableofcontents
\newpage

\section{ Введение }
Моделирование структурными уравнениями (Structural Equation Modelling, SEM \cite{hoyle2021sem}) -- комплекс методик, позволяющих описать рассматриваемую систему набором линейных уравнений, оперируя понятиями \emph{скрытых переменных} (latent variables), иначе называемых \emph{факторами}, и \emph{наблюдаемых переменных} (observed variables).
Если исследователь располагает набором данных о своей системе (например, измерением генотипов и фенотипов множества растений), он может построить структурную модель; для этого нужно:
\begin{enumerate}
    \item Возможно, ввести какие-либо переменные, значения которых измерить не удаётся, но которые, по мнению исследователя, могут быть интерпретированы в терминах предметной области (например, размер растения как обобщение длины, ширины, веса)
    \item Указать, какие величины, по мнению исследователя, связаны отношением линейной регрессии
\end{enumerate}

Процедура SEM заключается в нахождении коэффициентов регрессии и соответствующих P-значений, а также дисперсий зависимых переменных (\emph{зависимой} в SEM называют переменную, которая зависит от хотя бы одной другой переменной).
Дисперсии независимых переменных считаются фиксированными и принимаются равными выборочным дисперсиям.

Существуют вариации SEM, которые отличаются, помимо прочего, используемыми при оценке параметров функциями цели.
В 2019-2021 годах на кафедре Прикладной математики Политехнического университета был разработан программный пакет Semopy \cite{semopy, semopy2}, реализующий различные версии SEM; он позволяет оптимизировать функцию максимального правдоподобия Уишарта (Maximum Likelihood, ML), для данных с большим количеством пропущенных значений -- функцию максимального правдоподобия с полной информацией (Full Information Maximum Likelihood, FIML); и другие.

В процессе оптимизации требуется многократно вычислять значение функции цели и её градиента.
Значительные вычислительные ресурсы требуются для расчёта выражений вида $ tr(f(A)), A \in \mathds{R}^{n \times n}, n \in \mathds N $, входящие в целевую функцию, в результате чего процедура в отдельных случаях занимает длительное время даже на современных вычислительных устройствах.
Применение приближённых алгоритмов для расчёта следа от матричной функции может дать выигрыш во времени.

\newpage
\section{ Постановка задачи }
\cite{golub2010matMomentsQuadr}
\cite{golub2013matcomput}
\cite{ubaru2017fast}

\newpage
\section{ Описание данных }
% Использованы следующие данные, полученные из открытых источников:

\begin{enumerate}
    \item Транскриптом льна (\textit{Linum usitatissimum}) сортов Regina и Nike при условиях, описанных в разделе \ref{section:problem} -- \href{https://www.ncbi.nlm.nih.gov/sra/?term=PRJNA504749}{NCBI PRJNA504749}
    \item Геном льна \textit{Linum usitatissimum} -- \href{https://www.ncbi.nlm.nih.gov/datasets/genome/GCA_030674075.2/}{GenBank GCA\_030674075.2}
    \item Геном льна \textit{Linum trigynum} и аннотации в формате GTF -- \href{https://www.ncbi.nlm.nih.gov/datasets/genome/GCA_964030455.1/}{GenBank GCA\_964030455.1}
\end{enumerate}
\section{ Проведённые эксперименты }
\label{section:experiments}

Для проверки корректности работы программы использован классический набор данных <<Политическая демократия>> \cite{bollen1979}, который часто используется для демонстрации работы методов структурного моделирования.

\begin{figure}[H]
	\centering \includegraphics[width=\myPictWidth]{poldemo.pdf}
	\caption{ Модель взаимосвязей между факторами, определяющими или измеряющими развитость институтов политической демократии }
	\label{img:poldemo}
\end{figure}

Это данные из исследования, посвящённого связи показателей экономического роста с показателями демократичности общественных институтов.
Переменные $ x_1, x_2, x_3 $ обозначают валовой национальный продукт, энергопотребление и долю населения, занятую в индустрии (в 1960 году); $ y_1 $ - $ y_4 $ -- различные метрики демократичности власти и прессы в 1960 году, $ y_4 $ - $ y_7 $ -- в 1965 году.

В модели вводятся латентные переменные и связи, которые определяют представление исследователей о влиянии одних наблюдаемых переменных на другие.
С помощью SEM и статистических тестов в оригинальной работе \cite{bollen1979} было установлено, что модель не противоречит данным.

\section{ Результаты }
%\documentclass[main.tex]{subfiles}
\begin{document}

\subsection{Эмпирическая функция распределения}
Результаты представлены на графиках.
\begin{figure}[H]
	\centering \includegraphics[width=\myPictWidth]{norm_cdf.jpg}
	\caption{Нормальное распределение. Эмпирическая функция распределения}
	\label{img:norm_cdf}
\end{figure}
\begin{figure}[H]
	\centering \includegraphics[width=\myPictWidth]{cauchy_cdf.jpg}
	\caption{Распределение Коши. Эмпирическая функция распределения}
	\label{img:cauchy_cdf}
\end{figure}
\begin{figure}[H]
	\centering \includegraphics[width=\myPictWidth]{laplace_cdf.jpg}
	\caption{Распределение Лапласа. Эмпирическая функция распределения}
	\label{img:laplace_cdf}
\end{figure}
\begin{figure}[H]
	\centering \includegraphics[width=\myPictWidth]{uniform_cdf.jpg}
	\caption{Равномерное распределение. Эмпирическая функция распределения}
	\label{img:uniform_cdf}
\end{figure}
\begin{figure}[H]
	%\centering \includegraphics[width=\myPictWidth]{poisson_cdf.jpg}
	\caption{Распределение Пуассона. Эмпирическая функция распределения}
	\label{img:poisson_cdf}
\end{figure}

\subsection{Ядерные оценки плотности}

\begin{figure}[H]
	\centering \includegraphics[width=\myPictWidth]{norm_kde20.jpg}
	\caption{Нормальное распределение (20 элементов в выборке)}
	\label{img:norm_kde20}
\end{figure}
\begin{figure}[H]
	\centering \includegraphics[width=\myPictWidth]{norm_kde60.jpg}
	\caption{Нормальное распределение (60 элементов в выборке)}
	\label{img:norm_kde60}
\end{figure}
\begin{figure}[H]
	\centering \includegraphics[width=\myPictWidth]{norm_kde100.jpg}
	\caption{Нормальное распределение (100 элементов в выборке)}
	\label{img:norm_kde100}
\end{figure}

\begin{figure}[H]
	\centering \includegraphics[width=\myPictWidth]{cauchy_kde20.jpg}
	\caption{Распределение Коши (20 элементов в выборке)}
	\label{img:cauchy_kde20}
\end{figure}
\begin{figure}[H]
	\centering \includegraphics[width=\myPictWidth]{cauchy_kde60.jpg}
	\caption{Распределение Коши (60 элементов в выборке)}
	\label{img:cauchy_kde60}
\end{figure}
\begin{figure}[H]
	\centering \includegraphics[width=\myPictWidth]{cauchy_kde100.jpg}
	\caption{Распределение Коши (100 элементов в выборке)}
	\label{img:cauchy_kde100}
\end{figure}

\begin{figure}[H]
	\centering \includegraphics[width=\myPictWidth]{laplace_kde20.jpg}
	\caption{Распределение Лапласа (20 элементов в выборке)}
	\label{img:laplace_kde20}
\end{figure}
\begin{figure}[H]
	\centering \includegraphics[width=\myPictWidth]{laplace_kde60.jpg}
	\caption{Распределение Лапласа (60 элементов в выборке)}
	\label{img:laplace_kde60}
\end{figure}
\begin{figure}[H]
	\centering \includegraphics[width=\myPictWidth]{laplace_kde100.jpg}
	\caption{Распределение Лапласа (100 элементов в выборке)}
	\label{img:laplace_kde100}
\end{figure}

\begin{figure}[H]
	\centering \includegraphics[width=\myPictWidth]{uniform_kde20.jpg}
	\caption{Равномерное распределение (20 элементов в выборке)}
	\label{img:uniform_kde20}
\end{figure}
\begin{figure}[H]
	\centering \includegraphics[width=\myPictWidth]{uniform_kde60.jpg}
	\caption{Равномерное распределение (60 элементов в выборке)}
	\label{img:uniform_kde60}
\end{figure}
\begin{figure}[H]
	\centering \includegraphics[width=\myPictWidth]{uniform_kde100.jpg}
	\caption{Равномерное распределение (100 элементов в выборке)}
	\label{img:uniform_kde100}
\end{figure}

\begin{figure}[H]
	%\centering \includegraphics[width=\myPictWidth]{poisson_cdf.jpg}
	\caption{Распределение Пуассона}
	\label{img:poisson}
\end{figure}
	
\end{document}
\section{ Выводы }
%По графикам времени работы, изображённым на рис. \ref{fig:lanczos_benchmark3} и рис. \ref{fig:lanczos_benchmark1}-\ref{fig:lanczos_benchmark4}, видно, что оптимизированная версия алгоритма Ланшоца работает быстрее детерминированного метода в исследованных диапазонах параметров.
Следовательно, её можно применять на практике для ускорения вычислений, но нужно подобрать оптимальное значение $ k $ и $ p $.

Как видно из рис. \ref{fig:trdev_lanc_rel}, при $ p \in \{ 5; 10 \} $ и собственных числах матрицы до $ 30 $ по модулю среднее относительное отклонение не превышает $ 0.15 $, т.о. при желаемом относительном отклонении $ \le 0.15 $ можно ограничиться $ p=10 $ при $ k=5 $; этот вывод, впрочем, требует дополнительных проверок на матрицах больших размеров.

Значение $ k $, как видно из графиков, изображённых на рис. \ref{fig:lanczos_benchmark3} и \ref{fig:lanczos_benchmark1}-\ref{fig:lanczos_benchmark4}, необходимо выбирать больше двух; при $ k=3 $ или $ k=4 $ средняя относительная ошибка с ростом $ n $ убывает и достигает $ 0.2 $-$0.4$.
Предпочтительно использовать $ k=4 $ (точность заметно возрастает, но время работы увеличивается лишь ненамного).

\section{ Заключение }
Программный пакет для моделирования структурными уравнениями \texttt{Semopy} предоставляет пользователям широкий арсенал средств для построения, оценки, визуализации и инспекции линейных моделей с латентными переменными и причинно-следственными связями.
Однако ещё существует пространство для усовершенствования этого пакета, внедрения новых классов моделей и прочих функциональных элементов, исправления недочётов.

В настоящей работе изучены несколько вопросов, возникших в ходе эксплуатации пакета: недетерминированность результатов, возможность и корректность вычисления статистик для класса моделей со случайным эффектом \texttt{ModelEffects}, вычисление широко употребимой статистики $ R^2 $.
Удалось написать код, решающий поставленные задачи; но, однако, вычисление статистик для \texttt{ModelEffects} ещё следует подкорректировать.

Во время выполнения работы автор отчёта стал соавтором в публикации \cite{lanczos}.


\newpage
\addcontentsline{toc}{section}{\protect\numberline{}Список использованных источников}
\printbibliography[title={Список использованных источников}]

\newpage
\section{ Приложения }
%\subfile{parts/appendix}

\end{document}
