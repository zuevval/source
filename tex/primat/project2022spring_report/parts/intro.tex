Лён (\textit{Linum usitatissimum L.}) -- растение, производящее ценные зёрна, которые служат источником масла.
Из стеблей льна производится волокно, служащее основой тканей.
В настоящее время интерес к льну возрастает в связи с появлением всё новых применений компонентам этого растения в индустрии.

Одна из наиболее серьёзных проблем при выращивании льна -- паразиты-фузариозы, из которых \textit{Fusarium oxysporum f. sp. lini} -- самый опасный в силу своей высокой специфичности ко льну.
Он проникает в растение через корни и затем распространяется по стеблю, убивая как ростки, так и взрослые экземпляры.
Потери урожая из-за этой болезни оцениваются в $ 20 \% $ \cite{planchon2021}.

Большая часть атак патогенных организмов отражается с помощью PAMP-рецепторов, расположенных на эндоплазматических мембранах клеток растений \cite{boba2018} (PAMP -- pathogen-associated molecular patterns -- ассоциированные с патогенами молекулярные паттерны).

Но некоторые специально приспособленные патогены умеют преодолевать первый защитный барьер.
С ними работает второй эшелон защиты -- иммунный ответ, вызываемый эффекторами (ETI, effector-triggered immunity) \cite{tyagi2021}.
Этот механизм обеспечивается специальными рецепторами, которые кодируются генами устойчивости к болезни растений (R-генами).
Рецепторы распознают эффекторы, пришедшие вместе с патогеном, и запускают цепочки иммунного ответа.

Надо отметить, что помимо R-генов в иммунном ответе участвуют многие другие.
Изучение работы этих генов позволяет улучшить понимание устройства иммунитета растений.
В последние годы был проведён ряд исследований, направленных на выяснение роли различных генов в резистентности к специфичным патогенам путём анализа транскриптома.
В частности, в работе \cite{boba2018} сравниваются профили экспрессии генов в корнях проростков льна сортов Nike (более устойчивого к \textit{F. Oxysporum}) и Regina.

В данной работе проводятся манипуляции с данными, полученными в \cite{boba2018}.
