% !TeX spellcheck = en_US
% !TeX program = xelatex

\documentclass[a4paper,12pt]{article}
\usepackage[utf8]{inputenc} % russian, do not change
\usepackage[T2A, T1]{fontenc} % russian, do not change
\usepackage[english, russian]{babel} % russian, do not change

% fonts
\usepackage{fontspec} % different fonts
\setmainfont{Times New Roman}
\usepackage{setspace,amsmath}
\usepackage{amssymb} %common math symbols
\usepackage{dsfont}

% utilities
\usepackage{systeme} % systems of equations
\usepackage{mathtools} % xRightarrow, xrightleftharpoons, etc
\usepackage{array} % utils for tables
\usepackage{makecell} % multirow for tables
\usepackage{subfiles}
\usepackage{hyperref}
\hypersetup{pdfstartview=FitH,  linkcolor=blue, urlcolor=blue, colorlinks=true}
\usepackage{framed} % advanced frames, boxes
\usepackage{graphicx}
\usepackage{caption}
\usepackage{subcaption} % captions for subfigures
\usepackage{color}
\usepackage{chngcntr} % change counters
%\counterwithout*{section}{chapter} % continue sections enumeration with chngcntr
% \usepackage{theorem}
\usepackage{amsthm} % theorems with proofs
\newtheorem{thrm}{Теорема}


% styling
\usepackage{float} % force pictures position
\floatstyle{plaintop} % force caption on top
\usepackage{enumitem} % itemize and enumerate with [noitemsep]
\setlength{\parindent}{0pt} % no indents!

% misc
\graphicspath{{./img/}}
\newcommand{\myPictWidth}{.95\textwidth}
\newcommand{\phm}{\phantom{-}}

\begin{document}

\title{ Статистический анализ динамики измерений наблюдаемой характеристики с течением времени }
\author{Валерий Зуев (вариант IV)}
\maketitle

\setcounter{section}{1}
\section{ Графическое изображение наблюдений }

\begin{leftbar}
Изобразить графически результаты наблюдений без учёта времени измерений исследуемой характеристики.
В предположении нормальности значений наблюдаемого признака провести двухфакторный дисперсионный анализ зависимости наблюдаемого признака от значений факторов А и В.
\end{leftbar}

Посмотрим на данные (верхние строки таблицы):

% latex table generated in R 4.1.1 by xtable 1.8-4 package
% Thu Jun  2 12:29:47 2022
\begin{table}[ht]
	\centering
	\begin{tabular}{rrrllr}
		\hline
		& ID & Y & A & B & Visit \\ 
		\hline
		1 &   1 & 8.91 & 1 & 1 &   0 \\ 
		2 &   1 & 1.60 & 1 & 3 &   1 \\ 
		3 &   1 & 10.63 & 1 & 1 &   2 \\ 
		4 &   2 & 4.21 & 0 & 1 &   0 \\ 
		5 &   2 & 5.30 & 0 & 3 &   1 \\ 
		6 &   2 & 8.34 & 0 & 3 &   2 \\ 
		\hline
	\end{tabular}
\end{table}

\begin{figure}[H]
	\label{img:data_no_time}
	\includegraphics[width=\textwidth]{data_no_time}
	\caption{ 
		Данные без учёта времени посещения врача.
		Сгруппированы по значению фактора $ B $ (красный -- $ B = 1 $, зелёный -- $ B = 2 $, синий -- $ B = 3 $ ).
		Слева направо: \textbf{верхний ряд:} сглаженное распределение ID; корреляции между Y и ID; boxplot ID при каждом значении A.
		\textbf{Средний ряд:} данные в координатах \{ID, Y\}; сглаженное распределение Y; boxplot Y при каждом значении A.
		\textbf{Нижний ряд:} гистограммы распределения  ID  и Y при каждом значении А; гистограмма распределения A.
	}
\end{figure}

Двухфакторный дисперсионный анализ: проверим модель среднего, модели с эффектами факторов А и В, модель с аддитивным влиянием факторов и с перекрёстными членами.

\[ 
\begin{array}{c|c}
	\text{ Формула } & \text{ Стандартное отклонение остатков } \\
	\hline
	Y \sim 1 & 3.302 \\
	Y \sim A & 3.262 \\
	Y \sim B & 3.304 \\
	Y \sim A + B & 3.264 \\
	Y \sim A \cdot B & 3.264 \\
\end{array}
\]


\section{ Траектории; корреляции значений наблюдаемого признака в разные моменты времени }
\label{section:trajectories}

\begin{leftbar}
Представить визуально динамику изменений наблюдаемого признака в виде траекторий и оценить наличие влияния постоянного фактора А на значения измеряемого признака.
Оценить корреляции значений наблюдаемого признака в различные моменты времени при каждом значении признака А без учёта и с учётом влияния признака В.
\end{leftbar}

\begin{figure}[H]
\label{img:traj_with_time}
\centering\includegraphics[width=.8\textwidth]{traj_with_time}
\caption{ Случайно выбранные 30 траекторий в осях (Visit, Y). Цвет -- значение фактора А (зелёный -- 1, синий -- 2).
Жирные зелёные и синие точки -- средние значения Y в каждой точке Visit (стратифицированы по значению фактора А).
Красная линия -- средние значения по всем значениям Y в каждой точке Visit, соединённые ломаной.
}
\end{figure}

Оценка корреляций: приведём данные к виду, при котором все измерения для данного ID собраны в одну строку таблицы.
Несколько строк такой таблицы:

% latex table generated in R 4.1.1 by xtable 1.8-4 package
% Thu Jun  2 23:34:22 2022
\begin{table}[H]
	\centering
	\begin{tabular}{rlrlrlrl}
		\hline
		ID & A & Y\_V0 & Y\_B0 & Y\_V1 & Y\_B1 & Y\_V2 & Y\_B2 \\ 
		\hline
		1 & 1 & 8.91 & 1 & 1.60 & 3 & 10.63 & 1 \\ 
		2 & 0 & 4.21 & 1 & 5.30 & 3 & 8.34 & 3 \\ 
		3 & 0 & 5.63 & 1 & 4.34 & 3 & 7.77 & 1 \\ 
		4 & 0 & 6.22 & 2 & 3.50 & 1 & 7.25 & 1 \\ 
		5 & 1 & 5.95 & 3 & 0.45 & 1 & 10.50 & 1 \\ 
		6 & 0 & 5.98 & 2 & 4.26 & 1 & 7.69 & 2 \\ 
		\hline
	\end{tabular}
\end{table}

Матрица ковариаций между  измерениями со значением $ A = 0 $: 

% latex table generated in R 4.1.1 by xtable 1.8-4 package
% Thu Jun  2 23:40:34 2022
\begin{table}[H]
	\centering
	\begin{tabular}{rrrr}
		\hline
		& Y\_V0 & Y\_V1 & Y\_V2 \\ 
		\hline
		Y\_V0 & 2.03 & -0.00 & -0.27 \\ 
		Y\_V1 & -0.00 & 1.58 & -0.12 \\ 
		Y\_V2 & -0.27 & -0.12 & 1.84 \\ 
		\hline
	\end{tabular}
\end{table}

Матрица ковариаций между измерениями со значением $ A = 1 $:

% latex table generated in R 4.1.1 by xtable 1.8-4 package
% Thu Jun  2 23:42:10 2022
\begin{table}[ht]
	\centering
	\begin{tabular}{rrrr}
		\hline
		& Y\_V0 & Y\_V1 & Y\_V2 \\ 
		\hline
		Y\_V0 & 1.85 & 0.22 & 0.06 \\ 
		Y\_V1 & 0.22 & 1.89 & 0.05 \\ 
		Y\_V2 & 0.06 & 0.05 & 1.72 \\ 
		\hline
	\end{tabular}
\end{table}

При $ A = 0, B = 1 $:

% latex table generated in R 4.1.1 by xtable 1.8-4 package
% Thu Jun  2 23:47:00 2022
\begin{table}[H]
	\centering
	\begin{tabular}{rrrr}
		\hline
		& Y\_V0 & Y\_V2 & Y\_V1 \\ 
		\hline
		Y\_V0 & 1.72 & 0.06 & -0.06 \\ 
		Y\_V2 & 0.06 & 1.72 & -0.25 \\ 
		Y\_V1 & -0.06 & -0.25 & 1.57 \\ 
		\hline
	\end{tabular}
\end{table}

При $ A = 0, B = 2 $:

% latex table generated in R 4.1.1 by xtable 1.8-4 package
% Thu Jun  2 23:48:07 2022
\begin{table}[H]
	\centering
	\begin{tabular}{rrrr}
		\hline
		& Y\_V0 & Y\_V2 & Y\_V1 \\ 
		\hline
		Y\_V0 & 2.03 & -0.15 & -0.44 \\ 
		Y\_V2 & -0.15 & 1.62 & -0.34 \\ 
		Y\_V1 & -0.44 & -0.34 & 1.54 \\ 
		\hline
	\end{tabular}
\end{table}

При $ A = 1, B = 1 $:

% latex table generated in R 4.1.1 by xtable 1.8-4 package
% Thu Jun  2 23:48:58 2022
\begin{table}[ht]
	\centering
	\begin{tabular}{rrrr}
		\hline
		& Y\_V0 & Y\_V2 & Y\_V1 \\ 
		\hline
		Y\_V0 & 2.30 & -0.39 & -0.19 \\ 
		Y\_V2 & -0.39 & 1.62 & 0.69 \\ 
		Y\_V1 & -0.19 & 0.69 & 1.96 \\ 
		\hline
	\end{tabular}
\end{table}

Вариограмма:

\begin{figure}[H]
	\centering \includegraphics[width=\myPictWidth]{variogram}
\end{figure}


\section{ Смешанные линейные модели c эффектом индивида }

\begin{leftbar}
Построить смешанную линейную модель с простым эффектом индивида, а также смешанную линейную модель, допускающую линейную зависимость эффекта индивида от времени.
Определить целесообразность введения коэффициента наклона случайного эффекта, используя критерий cAIC
\end{leftbar}

Общая формула смешанной линейной модели:

\[ E(Y) = X \beta + Z b + \varepsilon  \]

$ b \sim \mathcal N(0,  \Upsilon) $

Результаты применения моделей: \\

\begin{tabular}{rc}
	формула & cAIC \\
	$ Y \sim Visit :  \sim 1 | ID $ & 10717 \\
	$ Y \sim Visit : \sim Visit | ID $ & 10715 \\
\end{tabular}

\section{ Смешанные линейные модели с учётом факторов А и В }

\begin{leftbar}
С учётом результатов п. \ref{section:trajectories} построить смешанную модель зависимости наблюдаемого признака от значений признаков А и В с учётом времени наблюдения.
При каждом значении фактора А проверить гипотезы аддитивности влияния фактора В и времени наблюдения, а также гипотезы отсутствия влияния каждого из факторов (В и времени наблюдения).
Включить в модель фактор А и оценить влияние фактора А на значение наблюдаемой характеристики.
\end{leftbar}

Исследованные модели:

\begin{tabular}{ccc}
	Формула & Значение A & AIC \\
	$ Y \sim B + Visit : B | Visit $  & 0 & 3453 \\
	$ Y \sim B + Visit : B | Visit $  & 1 & 3547 \\
	$ Y \sim B \cdot Visit : B | Visit $ & 0 & 3760 \\
	$ Y \sim B \cdot Visit : B | Visit $ & 1 & 3763 \\
\end{tabular}

Значения AIC примерно одинаковые.

Включим фактор $ A $ в модель:

\begin{enumerate}[noitemsep]
	\item $ Y \sim  A + B + Visit: A + B | Visit $: $ AIC = 7282 $
	\item $ Y ~ A * Visit + A * B : A + B | Visit $: $ AIC = 7279 $
\end{enumerate}

\section{ Дисперсионный анализ }
\label{section:anova}

\begin{leftbar}
Провести дисперсионный анализ зависимости распределения наблюдаемого признака, используя GEE-модель с неструктурированной корреляционной структурой наблюдений каждого индивида
\end{leftbar}

Проведём анализ с моделью $ Y ~ A + B $ с независимой корреляционной структурой по ID.

% TODO what?

\section{ Смешанная модель ковариационного анализа }

\begin{leftbar}
Построить смешанную модель ковариационного анализа в предположении полиномиальной зависимости второго порядка наблюдаемого признака от времени.
Проверить гипотезу линейной зависмости среднего значения наблюдаемого признака от времени в присутствии факторов А и В.
\end{leftbar}

\begin{tabular}{cc}
	Модель & AIC \\
	$ Y \sim Visit + Visit2: A + B | Visit $ & 7275.6 \\
	$ Y \sim Visit: A + B | Visit) $ & 7284 \\
\end{tabular}

\section{ Выбор наилучшей модели для неслучайного эффекта }

\begin{leftbar}
С использованием критериев AIC, BIC выбрать наилучшую модель для неслучайного эффекта в рамках смешанной модели, выбранной в  \ref{section:trajectories}.
Провести исследование влияния факторов A, B и времени обследования на значение исследуемой характеристики в рамках данной модели.
\end{leftbar}

Наилучшая модель -- $ Y \sim B + Visit : B | Visit $  (аддитивное неслучайное влияние факторов А и В вместе со случайным эффектом, который даёт фактор В при условии заданного значения Visit).

\section{ Количественный анализ динамики изменений наблюдаемого признака }

\begin{leftbar}
В условиях наилучшей модели, выбранной в п. \ref{section:anova}, провести количественный анализ динамики изменений наблюдаемого признака с течением времени.
Построить частные и совместные доверительные интервалы для значений параметров модели, принимая во внимание влияние сопутствующих факторов, присутствующих в наилучшей модели.
\end{leftbar}
	
\end{document}
