% !TeX spellcheck = en_US
% !TeX program = xelatex

\documentclass[a4paper,12pt]{article}
\usepackage[utf8]{inputenc} % russian, do not change
\usepackage[T2A, T1]{fontenc} % russian, do not change
\usepackage[english, russian]{babel} % russian, do not change

% fonts
\usepackage{fontspec} % different fonts
\setmainfont{Times New Roman}
\usepackage{setspace,amsmath}
\usepackage{amssymb} %common math symbols
\usepackage{dsfont}

% utilities
\usepackage{systeme} % systems of equations
\usepackage{mathtools} % xRightarrow, xrightleftharpoons, etc
\usepackage{array} % utils for tables
\usepackage{makecell} % multirow for tables
\usepackage{subfiles}
\usepackage{hyperref}
\hypersetup{pdfstartview=FitH,  linkcolor=blue, urlcolor=blue, colorlinks=true}
\usepackage{framed} % advanced frames, boxes
\usepackage{graphicx}
\usepackage{caption}
\usepackage{subcaption} % captions for subfigures
\usepackage{color}
\usepackage{chngcntr} % change counters
%\counterwithout*{section}{chapter} % continue sections enumeration with chngcntr
% \usepackage{theorem}
\usepackage{amsthm} % theorems with proofs
\newtheorem{thrm}{Теорема}


% styling
\usepackage{float} % force pictures position
\floatstyle{plaintop} % force caption on top
\usepackage{enumitem} % itemize and enumerate with [noitemsep]
\setlength{\parindent}{0pt} % no indents!

% misc
\graphicspath{{./img/}}
\newcommand{\myPictWidth}{.95\textwidth}
\newcommand{\phm}{\phantom{-}}

\begin{document}

\title{ Статистический анализ динамики измерений наблюдаемой характеристики с течением времени }
\author{Валерий Зуев (вариант IV)}
\maketitle

\setcounter{section}{1}
\section{ Графическое изображение наблюдений }

\begin{leftbar}
Изобразить графически результаты наблюдений без учёта времени измерений исследуемой характеристики.
В предположении нормальности значений наблюдаемого признака провести двухфакторный дисперсионный анализ зависимости наблюдаемого признака от значений факторов А и В.
\end{leftbar}

Посмотрим на данные (верхние строки таблицы):

% latex table generated in R 4.1.1 by xtable 1.8-4 package
% Thu Jun  2 12:29:47 2022
\begin{table}[ht]
	\centering
	\begin{tabular}{rrrllr}
		\hline
		& ID & Y & A & B & Visit \\
		\hline
		1 &   1 & 8.91 & 1 & 1 &   0 \\
		2 &   1 & 1.60 & 1 & 3 &   1 \\
		3 &   1 & 10.63 & 1 & 1 &   2 \\
		4 &   2 & 4.21 & 0 & 1 &   0 \\
		5 &   2 & 5.30 & 0 & 3 &   1 \\
		6 &   2 & 8.34 & 0 & 3 &   2 \\
		\hline
	\end{tabular}
\end{table}

Построим распределение $ Y $.

\begin{figure}[H]
	\label{img:data_no_time}
	\includegraphics[width=\textwidth]{y_only}
	\caption{Распределение Y}
\end{figure}

\begin{figure}[H]
    \includegraphics[width=\textwidth]{y_and_a}
    \caption{Распределение Y (стратификация по A)}
\end{figure}
\begin{figure}[H]
    \includegraphics[width=\textwidth]{y_and_b}
    \caption{Распределение Y (стратификация по B)}
\end{figure}

Двухфакторный дисперсионный анализ: проверим с помощью критерия хи-квадрат, можно ли без серьёзного увеличения дисперсии редуцировать общую модель.

\[
\begin{array}{c|c}
	\text{ Сравниваемые модели } & \text{ p-value } \\
	\hline
    Y \sim A \cdot B  \text{ VS } Y \sim A + B & 0.4394 \\
    Y \sim A + B \text{ VS } Y \sim A & 0.9141 \\
	Y \sim A + B \text{ VS } Y \sim B & 5.174e-13 \\
	Y \sim A \text{ VS } Y \sim 1 & 4.261e-13 \\
\end{array}
\]


\section{ Траектории; корреляции значений наблюдаемого признака в разные моменты времени }
\label{section:trajectories}

\begin{leftbar}
Представить визуально динамику изменений наблюдаемого признака в виде траекторий и оценить наличие влияния постоянного фактора А на значения измеряемого признака.
Оценить корреляции значений наблюдаемого признака в различные моменты времени при каждом значении признака А без учёта и с учётом влияния признака В.
\end{leftbar}

\begin{figure}[H]
\label{img:traj_with_time}
\centering\includegraphics[width=.8\textwidth]{traj_with_time}
\caption{ Случайно выбранные 30 траекторий в осях (Visit, Y). Цвет -- значение фактора А (зелёный -- 1, синий -- 2).
Жирные зелёные и синие точки -- средние значения Y в каждой точке Visit (стратифицированы по значению фактора А).
Красная линия -- средние значения по всем значениям Y в каждой точке Visit, соединённые ломаной.
}
\end{figure}

Оценка корреляций: приведём данные к виду, при котором все измерения для данного ID собраны в одну строку таблицы.
Несколько строк такой таблицы:

% latex table generated in R 4.1.1 by xtable 1.8-4 package
% Thu Jun  2 23:34:22 2022
\begin{table}[H]
	\centering
	\begin{tabular}{rrrr}
        \hline
        & Y\_V0 & Y\_V1 & Y\_V2 \\
        \hline
        Y\_V0 & 1.00 & -0.00 & -0.14 \\
        Y\_V1 & -0.00 & 1.00 & -0.07 \\
        Y\_V2 & -0.14 & -0.07 & 1.00 \\
        \hline
    \end{tabular}
\end{table}

Матрица корреляций между  измерениями со значением $ A = 0 $:

% latex table generated in R 4.1.1 by xtable 1.8-4 package
% Thu Jun  2 23:40:34 2022
\begin{table}[H]
	\centering
	\begin{tabular}{rrrr}
        \hline
        & Y\_V0 & Y\_V1 & Y\_V2 \\
        \hline
        Y\_V0 & 1.00 & -0.00 & -0.14 \\
        Y\_V1 & -0.00 & 1.00 & -0.07 \\
        Y\_V2 & -0.14 & -0.07 & 1.00 \\
        \hline
    \end{tabular}
\end{table}

Матрица корреляций между измерениями со значением $ A = 1 $:

% latex table generated in R 4.1.1 by xtable 1.8-4 package
% Thu Jun  2 23:42:10 2022
\begin{table}[H]
	\centering
	\begin{tabular}{rrrr}
        \hline
        & Y\_V0 & Y\_V1 & Y\_V2 \\
        \hline
        Y\_V0 & 1.00 & 0.12 & 0.03 \\
        Y\_V1 & 0.12 & 1.00 & 0.03 \\
        Y\_V2 & 0.03 & 0.03 & 1.00 \\
        \hline
    \end{tabular}
\end{table}

При $ A = 0, B = 1 $:

% latex table generated in R 4.1.1 by xtable 1.8-4 package
% Thu Jun  2 23:47:00 2022
\begin{table}[H]
	\centering
	\begin{tabular}{rrrr}
        \hline
        & Y\_V0 & Y\_V2 & Y\_V1 \\
        \hline
        Y\_V0 & 1.00 & 0.03 & -0.04 \\
        Y\_V2 & 0.03 & 1.00 & -0.14 \\
        Y\_V1 & -0.04 & -0.14 & 1.00 \\
        \hline
    \end{tabular}
\end{table}

При $ A = 0, B = 2 $:

% latex table generated in R 4.1.1 by xtable 1.8-4 package
% Thu Jun  2 23:48:07 2022
\begin{table}[H]
	\centering
	\begin{tabular}{rrrr}
        \hline
        & Y\_V0 & Y\_V2 & Y\_V1 \\
        \hline
        Y\_V0 & 1.00 & -0.10 & -0.28 \\
        Y\_V2 & -0.10 & 1.00 & -0.17 \\
        Y\_V1 & -0.28 & -0.17 & 1.00 \\
        \hline
    \end{tabular}
\end{table}

При $ A = 1, B = 1 $:

% latex table generated in R 4.1.1 by xtable 1.8-4 package
% Thu Jun  2 23:48:58 2022
\begin{table}[H]
	\centering
	\begin{tabular}{rrrr}
        \hline
        & Y\_V0 & Y\_V2 & Y\_V1 \\
        \hline
        Y\_V0 & 1.00 & -0.19 & -0.08 \\
        Y\_V2 & -0.19 & 1.00 & 0.32 \\
        Y\_V1 & -0.08 & 0.32 & 1.00 \\
        \hline
    \end{tabular}
\end{table}

Ковариации по вариограмме:

\begin{figure}[H]
	\centering \includegraphics[width=\myPictWidth]{variogram}
\end{figure}

% latex table generated in R 4.0.5 by xtable 1.8-4 package
% Wed Jun 08 11:27:57 2022
\begin{table}[H]
    \centering
    \begin{tabular}{rrrr}
        \hline
        & Y\_V0 & Y\_V2 & Y\_V1 \\
        \hline
        Y\_V0 & 1.43 & -1.68 & 1.53 \\
         Y\_V1 & -1.68 & 1.26 & -1.59 \\
        Y\_V2 & 1.53 & -1.59 & 1.36 \\
        \hline
    \end{tabular}
\end{table}


\section{ Смешанные линейные модели c эффектом индивида }

\begin{leftbar}
Построить смешанную линейную модель с простым эффектом индивида, а также смешанную линейную модель, допускающую линейную зависимость эффекта индивида от времени.
Определить целесообразность введения коэффициента наклона случайного эффекта, используя критерий cAIC
\end{leftbar}

Общая формула смешанной линейной модели:

\[ E(Y) = X \beta + Z b + \varepsilon  \]

$ b \sim \mathcal N(0,  \Upsilon) $

Результаты применения моделей: \\

% latex table generated in R 4.0.5 by xtable 1.8-4 package
% Wed Jun 08 08:32:31 2022
\begin{table}[H]
    \centering
    \begin{tabular}{lrrrrr}
        \hline
        fixed.part & AIC & cAIC (1|ID) & cAIC (Visit|ID) & pv1 & pv2 \\
        \hline
        Y \~{} A * B * Visit & 10670.98 & 10669.77 & 10668.47 & 1.00 & 1.00 \\
        Y \~{} A * Visit + B * Visit + A * B & 10668.39 & 10667.18 & 10665.87 & 1.00 & 1.00 \\
        Y \~{} A * Visit + B * Visit & 10666.80 & 10665.58 & 10664.27 & 1.00 & 1.00 \\
        Y \~{} A * Visit + A * B & 10664.71 & 10663.50 & 10662.20 & 1.00 & 1.00 \\
        Y \~{} B * Visit + A * B & 10669.80 & 10668.59 & 10667.28 & 1.00 & 1.00 \\
        Y \~{} A + B + Visit & 10664.48 & 10663.27 & 10661.95 & 1.00 & 1.00 \\
        Y \~{} A + B & 10964.69 & 10963.71 & 10961.53 & 1.00 & 1.00 \\
        Y \~{} A + Visit & 10661.19 & 10659.97 & \textbf{10658.66} & 1.00 & 1.00 \\
        Y \~{} B + Visit & 10722.62 & 10720.44 & 10718.63 & 1.00 & 1.00 \\
        Y \~{} Visit & 10719.76 & 10717.55 & 10715.75 & 1.00 & 1.00 \\
        Y \~{} A & 10960.87 & 10959.88 & 10957.71 & 1.00 & 1.00 \\
        Y \~{} B & 11014.94 & 11013.26 & 11010.46 & 1.00 & 1.00 \\
        Y \~{} 1 & 11011.45 & 11009.75 & 11006.95 & 1.00 & 1.00 \\
        \hline
    \end{tabular}
\end{table}


\section{ Смешанные линейные модели с учётом факторов А и В }

\begin{leftbar}
С учётом результатов п. \ref{section:trajectories} построить смешанную модель зависимости наблюдаемого признака от значений признаков А и В с учётом времени наблюдения.
При каждом значении фактора А проверить гипотезы аддитивности влияния фактора В и времени наблюдения, а также гипотезы отсутствия влияния каждого из факторов (В и времени наблюдения).
Включить в модель фактор А и оценить влияние фактора А на значение наблюдаемой характеристики.
\end{leftbar}

Исследованные модели:
\begin{align*}
    & Y \sim B + Visit : Visit | ID \\
    & Y \sim B \cdot Visit : Visit | ID
\end{align*}

Значения $ p-value $ в тесте хи-квадрат при дисперсионном анализе остатков:
$ 0.82 $ при $ A = 0 $, $ 0.59 $ при $ A = 1 $.

Т. о. в каждом случае модель можно редуцировать.

Если же сравнить модели

\begin{align*}
    & Y \sim A + B + Visit : Visit | ID \\
    & Y \sim B  + Visit : Visit | ID
\end{align*}

то $ p-value < 10^{-14} $ (редуцировать модель нельзя).

\section{ Дисперсионный анализ }
\label{section:anova}

\begin{leftbar}
Провести дисперсионный анализ зависимости распределения наблюдаемого признака, используя GEE-модель с неструктурированной корреляционной структурой наблюдений каждого индивида
\end{leftbar}

Протестируем модель общего вида против всех остальных.

% latex table generated in R 4.0.5 by xtable 1.8-4 package
% Wed Jun 08 09:23:36 2022
\begin{table}[H]
    \centering
    \begin{tabular}{lr}
        \hline
        model & p.value \\
        \hline
        Y \~{} A * Visit + B * Visit + A * B & 0.37 \\
        Y \~{} A * Visit + B * Visit & 0.14 \\
        Y \~{} A * Visit + A * B & 0.65 \\
        Y \~{} B * Visit + A * B & 0.00 \\
        Y \~{} A + B + Visit & 0.01 \\
        Y \~{} A + B & 0.00 \\
        Y \~{} A + Visit & 0.01 \\
        Y \~{} B + Visit & 0.00 \\
        Y \~{} Visit & 0.00 \\
        Y \~{} A & 0.00 \\
        Y \~{} B & 0.00 \\
        Y \~{} 1 & 0.00 \\
        \hline
    \end{tabular}
\end{table}


\section{ Смешанная модель ковариационного анализа }

\begin{leftbar}
Построить смешанную модель ковариационного анализа в предположении полиномиальной зависимости второго порядка наблюдаемого признака от времени.
Проверить гипотезу линейной зависмости среднего значения наблюдаемого признака от времени в присутствии факторов А и В.
\end{leftbar}

При сравнении дисперсий остатков в моделях $ Y \sim Visit + Visit2 + ( Visit | ID ) $ и $ Y \sim Visit + ( Visit | ID ) $ получено $ P $-значение $ < 10^{-15} $, то есть квадратичная модель намного лучше объясняет дисперсию.


\section{ Выбор наилучшей модели для неслучайного эффекта }

\begin{leftbar}
С использованием критериев AIC, BIC выбрать наилучшую модель для неслучайного эффекта в рамках смешанной модели, выбранной в  \ref{section:trajectories}.
Провести исследование влияния факторов A, B и времени обследования на значение исследуемой характеристики в рамках данной модели.
\end{leftbar}

% latex table generated in R 4.0.5 by xtable 1.8-4 package
% Wed Jun 08 10:18:37 2022
\begin{table}[H]
    \centering
    \begin{tabular}{lrr}
        \hline
        model & cAIC & BIC \\
        \hline
        Y \~{} A * B * Visit & 10669.77 & 10752.11 \\
        Y \~{} A * Visit + B * Visit + A * B & 10667.18 & 10738.22 \\
        Y \~{} A * Visit + B * Visit & 10665.58 & 10725.32 \\
        Y \~{} A * Visit + A * B & 10663.50 & 10723.24 \\
        Y \~{} B * Visit + A * B & 10668.59 & 10733.98 \\
        Y \~{} A + B + Visit & 10663.27 & 10706.05 \\
        Y \~{} A + B & 10963.71 & 11000.60 \\
        Y \~{} A + Visit & \textbf{10659.97} & \textbf{10691.46} \\
        Y \~{} B + Visit & 10720.44 & 10758.54 \\
        Y \~{} Visit & 10717.55 & 10744.37 \\
        Y \~{} A & 10959.88 & 10985.48 \\
        Y \~{} B & 11013.26 & 11045.20 \\
        Y \~{} 1 & 11009.75 & 11030.41 \\
        \hline
    \end{tabular}
\end{table}

Наилучшая модель -- $ Y \sim A + Visit +(Visit | ID) $  (аддитивное влияние факторов А и Visit).
Её характеристики:

\begin{verbatim}
>> df.lme.1_id[[8]]

Linear mixed-effects model fit by maximum likelihood
Data: df
Log-likelihood: -5326.597
Fixed: Y ~ A + Visit
(Intercept)          A1       Visit
3.946173    1.037328    1.458898

Random effects:
Formula: ~1 | ID
(Intercept) Residual
StdDev: 9.915825e-05 3.035298

Number of Observations: 2106
Number of Groups: 702
\end{verbatim}

\section{ Количественный анализ динамики изменений наблюдаемого признака }

\begin{leftbar}
В условиях наилучшей модели, выбранной в п. \ref{section:anova}, провести количественный анализ динамики изменений наблюдаемого признака с течением времени.
Построить частные и совместные доверительные интервалы для значений параметров модели, принимая во внимание влияние сопутствующих факторов, присутствующих в наилучшей модели.
\end{leftbar}

\begin{verbatim}

> intervals(lm.Visit2, which = "fixed")
Approximate 95% confidence intervals

Fixed effects:
            lower      est.     upper
(Intercept)  6.201718  6.323118  6.444519
Visit       -9.911643 -9.602131 -9.292619
Visit2       5.381830  5.530515  5.679199
attr(,"label")

\end{verbatim}

\end{document}
