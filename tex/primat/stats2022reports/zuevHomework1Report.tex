% !TeX spellcheck = en_US
% !TeX program = xelatex

\documentclass[a4paper,12pt]{article}
\usepackage[utf8]{inputenc} % russian, do not change
\usepackage[T2A, T1]{fontenc} % russian, do not change
\usepackage[english, russian]{babel} % russian, do not change

% fonts
\usepackage{fontspec} % different fonts
\setmainfont{Times New Roman}
\usepackage{setspace,amsmath}
\usepackage{amssymb} %common math symbols
\usepackage{dsfont}

% utilities
\usepackage{systeme} % systems of equations
\usepackage{mathtools} % xRightarrow, xrightleftharpoons, etc
\usepackage{array} % utils for tables
\usepackage{makecell} % multirow for tables
\usepackage{subfiles}
\usepackage{hyperref}
\hypersetup{pdfstartview=FitH,  linkcolor=blue, urlcolor=blue, colorlinks=true}
\usepackage{framed} % advanced frames, boxes
\usepackage{graphicx}
\usepackage{caption}
\usepackage{subcaption} % captions for subfigures
\usepackage{color}
\usepackage{chngcntr} % change counters
%\counterwithout*{section}{chapter} % continue sections enumeration with chngcntr


% styling
\usepackage{float} % force pictures position
\floatstyle{plaintop} % force caption on top
\usepackage{enumitem} % itemize and enumerate with [noitemsep]
\setlength{\parindent}{0pt} % no indents!
\usepackage[left=30mm, top=20mm, right=30mm, bottom=20mm, nohead, footskip=10mm]{geometry}


% misc
\graphicspath{{./img/}}
\newcommand{\myPictWidth}{.9\textwidth}
\newcommand{\phm}{\phantom{-}}

\begin{document}

\title{Лабораторная работа 1. \\
     Анализ категориальных данных}
\author{Валерий Зуев}
\maketitle

\setcounter{section}{-1}

\section{Постановка задачи}

Исходные данные задачи представлены в таблице ниже.
Необходимо провести манипуляции, описанные в разделах \ref{section:barplot}-\ref{section:poisson}.

% latex table generated in R 4.0.5 by xtable 1.8-4 package
% Mon May 23 10:22:25 2022
\begin{table}[ht]
    \centering
    \begin{tabular}{r|rr|rr}
        \hline
        & \multicolumn{2}{c}{Admitted, male} & \multicolumn{2}{c}{Admitted, female} \\
        \hline
        Department & yes & no & yes & no \\
        \hline
        1 & 512 & 313 &  89 &  19 \\
        2 & 353 & 207 &  17 &   8 \\
        3 & 120 & 205 & 202 & 391 \\
        4 & 138 & 279 & 131 & 244 \\
        5 &  53 & 138 &  94 & 299 \\
        6 &  22 & 351 &  24 & 317 \\
        \hline
    \end{tabular}
    \label{tbl:task}
    \caption{Статистика поступления в UCB (1973 год)}
\end{table}

\section{Графическое изображение (barplot)}
\label{section:barplot}

\begin{leftbar}
    Изобразить графически частоты для двух групп абитуриентов, принятых и не принятых в высшие школы с классификацией по полу (barplot).
    Сравнить полученные результаты.
\end{leftbar}

Частота положительного результата в группе $ g \in \{ \text{male}, \text{female} \} $ в высшей школе $ d \in \{1;6 \} $
\[ p_{dg} = \frac{n_{d,g,yes}}{n_{d,g,yes} + n_{d,g,no}} \]

Эти частоты, а также суммарное число принятых/не принятых кандидатов изображены на столбчатых диаграммах ниже.

\begin{figure}[H]
    \center{\includegraphics[width=\myPictWidth]{admitted_stats}}
    \label{fig:frequencies}
    \caption{Доля принятых мужчин и женщин в каждой высшей школе}
\end{figure}

\begin{figure}[H]
    \center{\includegraphics[width=\myPictWidth]{total_counts}}
    \caption{Общее число принятых и не принятых мужчин и женщин}
\end{figure}

Шансы на поступление в каждом департаменте примерно одинаковые у мужчин и женщин, но мужчин поступает больше, а также доля неудачно поступивших в целом среди женщин выше (стоит отметить, что в первом департаменте число абитуриентов больше всех остальных, и в нём же доля поступивших мужчин среди поступавших наиболее значительно превышает долю поступивших женщин).

\section{Хи-квадрат, CMH-тест}

\begin{leftbar}
    Проверить наличие статистической зависимости между поступлением в школу и полом (Хи-квадрат) и гипотезу условной независимости (CMH-test)
\end{leftbar}

Тест хи-квадрат проводится с таблицей сопряжённости $ 2 \times 2 $ и проверяет гипотезу о наличии зависимости между значениями двух случайных величин (нулевая гипотеза -- шансы попадания в каждую клетку таблицы получены просто произведением шансов попадания в данную строку и шансов попадания в данный столбец).
Статистика теста при выполнении нулевой гипотезы имеет асимптотическое распределение хи-квадрат с одной степенью свободы. \\

Асимптотический обобщённый критерий Кочрана-Мантела-Хензела (CMH-тест) проводится с таблицей сопряжённости $ 2 \times 2 \times d $ и проверяет наличие зависимости при условии фиксированного департамента $d$. \\

Таблица, по которой проведены оба теста:

% latex table generated in R 4.0.5 by xtable 1.8-4 package
% Mon May 23 13:02:12 2022
\begin{table}[ht]
    \centering
    \begin{tabular}{llrl}
        \hline
        gender & admitted & count & department \\
        \hline
        male & yes & 512 & 1 \\
        male & yes & 353 & 2 \\
        male & yes & 120 & 3 \\
        male & yes & 138 & 4 \\
        male & yes &  53 & 5 \\
        male & yes &  22 & 6 \\
        male & no & 313 & 1 \\
        male & no & 207 & 2 \\
        male & no & 205 & 3 \\
        male & no & 279 & 4 \\
        male & no & 138 & 5 \\
        male & no & 351 & 6 \\
        female & yes &  89 & 1 \\
        female & yes &  17 & 2 \\
        female & yes & 202 & 3 \\
        female & yes & 131 & 4 \\
        female & yes &  94 & 5 \\
        female & yes &  24 & 6 \\
        female & no &  19 & 1 \\
        female & no &   8 & 2 \\
        female & no & 391 & 3 \\
        female & no & 244 & 4 \\
        female & no & 299 & 5 \\
        female & no & 317 & 6 \\
        \hline
    \end{tabular}
\end{table}

Таблица сопряжённости:

% latex table generated in R 4.0.5 by xtable 1.8-4 package
% Mon May 23 16:21:11 2022
\begin{table}[H]
    \centering
    \begin{tabular}{rrr}
        \hline
        & no & yes \\
        \hline
        female & 1278 & 557 \\
        male & 1493 & 1198 \\
        \hline
    \end{tabular}
\end{table}

Результаты тестов:

\begin{table}[H]
    \centering
    \begin{tabular}{rll}
        \hline
        тест & статистика & p-value \\
        \hline
        хи-квадрат & 96.61 & 2.2e-16 \\
        CMH & 1.52 & 0.22 \\
        \hline
    \end{tabular}
\end{table}

Согласно результатам тестов, в общем случае шансы поступления зависят от пола, но условные шансы не зависят.

\section{Подготовка к применению модели регрессии}

Воспользуемся функцией \texttt{uncount} из пакета \texttt{tidyr}.

Результат (приведены несколько избранных строк):

% latex table generated in R 4.0.5 by xtable 1.8-4 package
% Mon May 23 14:28:26 2022
\begin{table}[H]
    \centering
    \begin{tabular}{rlll}
        \hline
        & gender & admitted & department \\
        \hline
        1 & male & yes & 1 \\
        513 & male & yes & 2 \\
        866 & male & yes & 3 \\
        986 & male & yes & 4 \\
        \hline
    \end{tabular}
\end{table}


\section{Модель логистический регрессии}

\begin{leftbar}
    С использованием модели логистической регрессии описать зависимость шансов поступления в школу от пола и факультета.
    С использованием AIC выбрать оптимальную модель логистической регрессии для описания этих данных.
\end{leftbar}

Обобщённая линейная модель:

\[ g(E(y_a|\vec x)) = \beta_0 + \vec \beta \cdot \vec x \]

где $ y_a $ -- индикатор зачисления/незачисления, $ \vec x $ -- бинарный вектор, содержащий информацию о гендере и факультете в форме one-hot encoding (вклад гендера, вклад факультета, совместный вклад гендера и факультета).

В случае логистической регрессии $ g(y) = logit(y) $.

Сравниваемые модели отличаются компонентами вектора $ \beta $, которые приравниваются к нулю: в общем случае все могут быть ненулевые, в модели с независимым влиянием факторов нулевые коэффициенты при совместном вкладе и так далее.
Всего проверены пять моделей.

Соответствующие значения AIC:

% latex table generated in R 4.0.5 by xtable 1.8-4 package
% Mon May 23 15:26:54 2022
\begin{table}[ht]
    \centering
    \begin{tabular}{lr}
        \hline
        model & AIC \\
        \hline
        admitted \~{} 1 & 6046.34 \\
        admitted \~{} gender & 5954.89 \\
        admitted \~{} department & 5201.02 \\
        admitted \~{} gender + department & 5201.49 \\
        admitted \~{} gender * department & 5191.28 \\
        \hline
    \end{tabular}
\end{table}

Следовательно, наилучшая из всех -- модель среднего.

\section{Логарифмическая модель (Пуассона)}
\label{section:poisson}

\begin{leftbar}
    С использованием логарифмической модели (Пуассона) проверить гипотезы однородности зависимости шансов поступления в школу от пола и условной независимости шансов поступления от пола.
\end{leftbar}

В модели регрессии Пуассона $ g(y) = \ln(y) $.
Модель строится по исходной таблице с числом студентов (counts).
В качестве зависимой переменной $y$ выступает уже не бинарная величина, а число студентов; факторы -- пол, высшая школа (5 факторов) и приняты/не приняты.

Базовая, наиболее общая модель предполагает наличие совместных взаимодействий вплоть до третьего порядка (пол $ \times $ высшая школа $ \times $ приняты/не приняты):

\begin{verbatim}
count ~ gender * admitted * department
\end{verbatim}

В модели с однородной зависимостью перекрёстные члены только до второго порядка:

\begin{verbatim}
count ~ gender * admitted + gender * department + admitted * department
\end{verbatim}

В модели условной независимости шансов поступления от пола отсутствует перекрёстный член, связывающий фактор поступления/непоступления и фактор пола:


\begin{verbatim}
count ~ gender * department + admitted * department
\end{verbatim}

Для сравнения обеих моделей с базовой используется тест отношения правдоподобия.
Разность функций правдоподобия общей модели и тестируемой в условиях справедливости тестируемой гипотезы имеет распределение хи-квадрат с числом степеней свободы, равному числу переменных базовой модели, <<обнулённых>> в тестируемой.

Результаты приведены в таблице.

\begin{table}[H]
    \centering
    \begin{tabular}{rll}
        \hline
        модель & LLR & p-value \\
        \hline
        однородность & 20.205 & 0.0011 \\
        условная независимость & 21.736 & 0.0013 \\
        \hline
    \end{tabular}
\end{table}

Соответственно, на уровне значимости $ \alpha = 0.01 $ мы отвергаем модель однородности.
Как и следует ожидать с учётом этого факта, модель условной независимости также отвергается.


\section{Интерпретация результатов}

Результаты хи-квадрат и CMH-теста любопытные: вообще говоря, у юношей шансы поступления отличаются от шансов поступления у девушек; но стоит выбрать конкретную высшую школу, и шансы становятся примерно одинаковыми.

Модель логистической регрессии подсказывает, что для конкретного абитуриента лучше всего предсказать вероятность поступления как некую фиксированную величину, не зависящую ни от пола, ни от факультета.
Модель Пуассона, однако, показывает, что число студентов, попавших в конкретную клетку таблицы, должна быть описана сложной моделью, включающей все взаимодействия вплоть до третьего порядка, а однородной быть описана не может.

\end{document}
