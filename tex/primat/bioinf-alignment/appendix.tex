\documentclass[main.tex]{subfiles}
\begin{document}
	\section{Приложение: добавление пропусков в нуклеотидное мультивыравнивание}
	На рис. \ref{fig:1} приведены примеры исправления аминокислотного мультивыравнивания.
	\begin{figure}[H]
		\centering
		\begin{subfigure}{.5\textwidth}
			\centering
			\includegraphics[width=\myPictWidth]{defect1}
			\caption{До исправления}
			\label{fig:defect1}
		\end{subfigure}%
		\begin{subfigure}{.5\textwidth}
			\centering
			\includegraphics[width=\myPictWidth]{defect1fixed}
			\caption{После исправления}
			\label{fig:defect1fixed}
		\end{subfigure}
	
		\begin{subfigure}{.5\textwidth}
			\centering
			\includegraphics[width=\myPictWidth]{defect2}
			\caption{До исправления}
			\label{fig:defect2}
		\end{subfigure}%
		\begin{subfigure}{.5\textwidth}
			\centering
			\includegraphics[width=\myPictWidth]{defect2fixed}
			\caption{После исправления}
			\label{fig:defect2fixed}
		\end{subfigure}
		\caption{Участки аминокислотного мультивыравнивания (вверху каждого изображения) и транслированного аминокислотного (внизу). Строка из пропусков служит для визуального разделения}
		\label{fig:1}
	\end{figure}
\end{document}