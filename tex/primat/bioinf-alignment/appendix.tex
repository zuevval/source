\documentclass[main.tex]{subfiles}
\begin{document}
	\section{Приложения}
	\subsection{Добавление пропусков в нуклеотидное мультивыравнивание}
	На рис. \ref{fig:1} приведены примеры исправления нуклеотидного мультивыравнивания.
	\begin{figure}[H]
		\centering
		\begin{subfigure}{.5\textwidth}
			\centering
			\includegraphics[width=\myPictWidth]{defect1}
			\caption{До исправления}
			\label{fig:defect1}
		\end{subfigure}%
		\begin{subfigure}{.5\textwidth}
			\centering
			\includegraphics[width=\myPictWidth]{defect1fixed}
			\caption{После исправления}
			\label{fig:defect1fixed}
		\end{subfigure}
	
		\begin{subfigure}{.5\textwidth}
			\centering
			\includegraphics[width=\myPictWidth]{defect2}
			\caption{До исправления}
			\label{fig:defect2}
		\end{subfigure}%
		\begin{subfigure}{.5\textwidth}
			\centering
			\includegraphics[width=\myPictWidth]{defect2fixed}
			\caption{После исправления}
			\label{fig:defect2fixed}
		\end{subfigure}
		\caption{Участки аминокислотного мультивыравнивания (вверху каждого изображения) и транслированного нуклеотидного (внизу). Строка из пропусков служит для визуального разделения}
		\label{fig:1}
	\end{figure}

	\newpage
	\subsection{Танглограммы}
	Приведены построенные танглограммы: для нуклеотидных (рис. \ref{fig:tanglegram_dna1}, \ref{fig:tanglegram_dna2}, \ref{fig:tanglegram_dna3}) и белковых выравниваний (рис. \ref{fig:tanglegram_protein1}, \ref{fig:tanglegram_protein2}, \ref{fig:tanglegram_protein3})
	
	\newcommand{\mypicture}[2]{
		\begin{figure}[H]
			\centering \includegraphics[width=\myPictWidth]{#1}
			\caption{#2}
			\label{fig:#1}
		\end{figure}
	}

	\mypicture{tanglegram_dna1}{Танглограмма для нуклеотидных последовательностей. Слева: дерево, построенное по подвыравниванию длины 30 оснований, справа: дерево, построенное по полному выравниванию.}
	\mypicture{tanglegram_dna2}{Ещё одна танглограмма для нуклеотидных выравниваний}
	\mypicture{tanglegram_dna3}{Ещё одна танглограмма для нуклеотидных выравниваний}
	\mypicture{tanglegram_protein1}{Пример танглограммы для белковых последовательностей. Слева: дерево, построенное по подвыравниванию длины 30 аминокислотных остатков, справа: дерево, построенное по полному выравниванию.}
	\mypicture{tanglegram_protein2}{Ещё одна танглограмма для аминокислотных выравниваний}
	\mypicture{tanglegram_protein3}{Ещё одна танглограмма для аминокислотных выравниваний}
	

	
\end{document}