% compile with XeLaTeX or LuaLaTeX
\documentclass[a4paper,12pt]{article} %14pt - extarticle
\usepackage[utf8]{inputenc} % russian, do not change
\usepackage[T2A, T1]{fontenc} % russian, do not change
\usepackage[english, russian]{babel} % russian, do not change

% fonts
\usepackage{fontspec} % different fonts
\setmainfont{Times New Roman}
\usepackage{setspace,amsmath}
\usepackage{amssymb} %common math symbols
\usepackage{dsfont}

% utilities
\usepackage{subfiles}
\usepackage{hyperref}
\hypersetup{pdfstartview=FitH,  linkcolor=blue, urlcolor=blue, colorlinks=true}
\usepackage{graphicx}
\usepackage{caption}
\usepackage{subcaption} % captions for subfigures
\usepackage{float} % force pictures position
\floatstyle{plaintop} % force caption on top

\graphicspath{{./img/}}
\newcommand{\myPictWidth}{.95\textwidth}

\begin{document}
	\subfile{titlepage}
	
	\section{Выбор гена}
	Для исследования был выбран ген белка ATP13A2.
	\section{Результат обращения к GenBank}
	В результате поиска в базе данных GenBank выбран транскрипт мРНК \href{https://www.ncbi.nlm.nih.gov/nuccore/NM_022089.4}{NM\_022089.4} исследуемого гена и соответствующая аминокислотная последовательность \href{https://www.ncbi.nlm.nih.gov/protein/13435129}{NP\_071372.1}, полученная \textit{in silico}.
	\section{Поиск гомологов гена при помощи BLAST (tblastn)}
	По белковой последовательности проведены поиски гомологов для восьми организмов в базе данных последовательностей нуклеотидов с использованием веб-инструмента tblastn. Ниже приведена информация о видах и соответствующих транскриптах, которые были отобраны инструментом tblastn как наиболее похожие на белок человека:
	\begin{enumerate}
		\item Pan troglodytes -- \href{https://www.ncbi.nlm.nih.gov/nuccore/XM_016954962.2}{XM\_016954962.2} 
		\item Pan paniscus -- \href{https://www.ncbi.nlm.nih.gov/nuccore/XM_034956516.1}{XM\_034956516.1} 
		\item Macaca mulatta -- \href{https://www.ncbi.nlm.nih.gov/nuccore/XM_001087655.4}{XM\_001087655.4}
		\item Mus musculus -- \href{https://www.ncbi.nlm.nih.gov/nuccore/NM_029097.3}{NM\_029097.3}
		\item Rattus norvegicus -- \href{https://www.ncbi.nlm.nih.gov/nuccore/XM_006239246.3}{XM\_006239246.3}
		\item Canis lupus familiaris -- \href{https://www.ncbi.nlm.nih.gov/nuccore/XM_005617949.3}{XM\_005617949.3}
		\item Felis catus -- \href{https://www.ncbi.nlm.nih.gov/nuccore/XM_023258195.1}{XM\_023258195.1}
		\item Bos taurus -- \href{https://www.ncbi.nlm.nih.gov/nuccore/NM_001192271.3}{NM\_001192271.3}
	\end{enumerate}
	\section{Построение множественного выравнивания}
	При помощи утилиты CLUSTALW и графической оболочки MEGA составлено множественное выравнивание найденных девяти нуклеотидных последовательностей и соответствующих триплетов аминокислотных последовательностей.
	\section{Анализ консервативных участков и структурных вариаций}
	Выравнивание аминокислотных последовательностей, полученное при помощи CLUSTALW, содержит пробелы в середине, которых не было в аминокислотном выравнивании. При трансляции нуклеотидного выравнивания в белковое видно, что это приводит к очевидно сдвинутым последовательностям (примеры приведены в приложении, см. рисунки \ref{fig:defect1} и \ref{fig:defect2}). Это было исправлено добавлением вручную пробелов в аминокислотное выравнивание.
	
	\section{Вариабельные блоки}
	Написан скрипт для выбора из выравнивания участков заданной длины и в каждом выравнивании выбраны по три блока из 30 символов, содержащие полиморфизмы. Код \href{https://github.com/zuevval/source/blob/master/python/bioinf_practice/needles/random_subalignment.py}{доступен на GitHub}.\\
	Полученные участки последовательностей нуклеотидов и аминокислот, содержащие вариации, прилагаются к отчёту в .fas-файлах. Также к отчёту прилагаются полные выравнивания, как аминокислотное, так и исправленное нуклеотидное.
	\section{Описание исследуемого гена}
	Белок ATP13A2 (Probable cation-transporting ATPase), синтезируемый соответствующим геном, несёт транспортную функцию в клетках - переносит неорганические катионы (металлов) через клеточные мембраны; был обнаружен, к примеру, во внутриклеточных структурах нейронов \cite{omim}.
	\begin{thebibliography}{99}
		\bibitem{omim} ATPase, TYPE 13A2; ATP13A2 [Электронный ресурс] // Online Mendelian Inheritance in Man, 2020. URL: \url{https://www.omim.org/entry/610513}
	\end{thebibliography}

	\newpage
	\subfile{appendix}		
		
\end{document}
