\documentclass[main.tex]{subfiles}
\begin{document}

Мы отметили, что клетка может начать пролиферировать, выбрать старение, выбрать смерть (апоптоз).
Смерть может быть разная.
Сейчас принято выделять до десяти типов смертей.

Дифференцированные клетки, как правило, не делятся: они какое-то время живут, выполняют свои функции; потом стареют и погибают.

В большинстве случаев эти каскады и есть способы реакции клеток на внешнее воздействие.

Raf=MAPKKK -- киназа киназы мап-киназы.
Mek=MAPK -- киназа мап-киназы.

Одна из мишеней Map-киназного пути -- Myc.
Вообще, у самого этого гена 4000 генов-мишеней!
Среди них есть несколько главных (тех, в кото)
Как правило, онкологическое перерождение сопровождается оверэкспрессией гена Myc и, как следствие, его мишеней.
Основные мишени -- циклин-зависимые киназы.

Во многих случаях, когда мы сталкиваемся с раком, в этих клетках, помимо мутаций, изменён метаболизм и энергетические процессы отличаются от любых таковых в нормальных клетках.

Репрессор Myc работает в комплексе с Mad.
Напомним, многие белки для работы требуют посттрансляционные модификации, в частности, метилирование, ацетилирование гистонов.
Комплекс Myc-Mad привлекает гистоновую деацетилазу.

Оверэкспрессия, как правило, влияет не только на включение-выключение...
но и на метаболизм жиров, белков.
В итоге это изменяет клеточную энергетику, что способствует раковому перерождению.

За счёт чего может происходить...

Мини-хромосомы.
Пациент, у которого был рак крови... Его лечили какой-то химией, и в ответ на это у него появились мини-хромосомы, состоящие из многократно отреплицированных генов Myc.
Разные онкогены могут быть так многократно отреплицированы, 
Спонтанное фрагментирование хромосом: иногда фрагменты восстанавливаются, иногда утрачиваются из-за отсутствия центромеры, иногда сворачиваются в кольцевые структуры наподобно плазмид, которые постоянно реплицируются.
При делении они случайным образом расходятся по дочерним клеткам.

До сих пор нет чёткого понимания причин появления мини-хромосом.

Эти кольцевые последовательности могут быть интегрированы обратно в хромосомы.

Откуда берутся гибридные киназные белки BCR, BCL (когда возникают лимфомы)?

Система репарации делает это <<не со зла>>, её задача -- убрать дырки.
Конечно, это может привести к негомологичной репарации, но дырки страшнее, чем что-либо.

Но всё это гипотезы!
Чёткого подтверждения нет.
Видимо, и тот, и другой механизм иногда работает.

Клетки HELA: Henrietta Lacks (очень живучая клеточная линия, вызванная вирусом папилломы человека).
С неё в 50-х годах началось культивирование клеточных линий.

На клетках HELA тестировались различные лекарства.

Потом научились постепенно культивировать и другие.
В Институте цитологии банк клеточных культур, и все они раковые. \\

Напомним, мы говорили про ген P53, который отвечает за смерть в случае конфликтного сигналинга (когда извне поступают противоречивые сигналы).
Одно время казалось, что, если мы всё поймём про P53, сможем лечить любой рак.
Оказалось, что нет.

p53 -- сложный белок, работающий в тетрамерных комплексах.
Даже в гетерозиготном состоянии многие мутации, которые нарушают процесс работы p53,

p53 -- тоже транскрипционный фактор, у него тоже сотни мишеней.
У него есть участки (конкретные аминокислотные остатки), которые должны быть фосфорилирова

Его в клетке множество, но он быстро разлагается (у него на конце мишени для убиквитинирования), чтобы клетка могла быстро отреагировать на изменение ситуации.
Он может % TODO a bit

Активный тетрамер может взаимодействовать с конкретным тандемом, который он узнаёт в ДНК (такие последовательности находятся в промоторных участках).

Сайты фосфорилирования, ацетилирования; трансактивирующий домен. % TODO a bit

Оказалось, что у p53 очень много вариантов сплайсинга.
Белковых продуктов несколько, причём транскрипция может начинаться с альтернативного промотора (расположенного в некоем интроне), поэтому иногда трансактивирующий домен утрачивается.

У этого гена есть и паралоги (p63, ...), и они тоже имеют большое число изоформ.
Эти паралоги могут работать в комплексе с p53.

Некоторые изоформы мы можем выключать с помощью интерферирующих мРНК.

p53 называют <<хранителем генома>>: его основная функция -- реагировать на аномалии в геноме.
Оказывается, что некоторые изоформы важны для того, чтобы клетка выбирала путь старения, другие -- для дифференцировки или смерти.

Активация других онкогенов, как и нарушения в ДНК, а также появление активных форм кислорода ($ OH^-, O_2^-, ... $) в нормальной клетке являются поводом для самоубийства клетки.
В норме вначале клетка включает антиоксидантные системы, потом, если не справляется, убивает себя.

В раковых клетках p53, как оказалось, часто принимает прионоподобные формы.
Если в гене, кодирующем p53, мутация (даже в одном аллеле), прион будет вовлекать <<нормальный>> p53 в прионовые комплексы!
В этом и проявляется доминантный эффект <<нехороших>> мутаций.

Одно из направлений работы -- создание лекарств, которые препятствуют возникновению амилоидных тяжей.

\begin{leftbar}
Здесь видно, зачем нужны фундаментальные исследования.
Как бы мы могли искать мишени для терапии, не зная ничего о том, как работает клетка?
\end{leftbar}

Это не единственная проблема, с которой сталкиваются люди при работе с клетками.
Ещё одна проблема -- устойчивость клеток к ингибиторам.
Может быть связано с работой белков теплового шока (HSP или шепероны).

Повреждения в клетке при облучении или химическом воздействии захватывают не только ДНК, но и мембраны, белки...
Что может случиться?
Например, неправильно уложенные белки.
Плохо уложенные белки -- сигнал для HSP, которые пытаются их переуложить.
Если не удаётся, белки убиквитинируются и отправляются в протеасому.
В результате белки теплового шока способствуют возникновению устойчивости.

HSP40 -- шеперонины (помогают другим шеперонам).

$ \beta $-катенин -- белок, необходимый для клеточных контактов.
Если его мало, клетка начинает передвигаться.
Белки семейства HSP40 способствуют ингибированию комплексов, отвечающих за деградацию бета-катенина.

HSP70, HSP90: активация этих белков тоже способствует выживанию. \\

Итог: пока рак лечить не удаётся (потому что всё очень сложно, очень много систем, завязанных на клеточную жизнь -- не получается убить клетки просто, они будут сопротивляться), но иногда удаётся разными методами достичь стойкой ремиссии.

\end{document}