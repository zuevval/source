\documentclass[main.tex]{subfiles}
\begin{document}
\section{Сигнальная трансдукция. Продолжение}

Активация генов, которые отвечают в первую очередь за клеточный цикл.
Стимулируется пролиферация, метастазирование ()

Каждый сигнальный путь мы можем проследить (кто кого форфорилирует, кто кого дефосфорилирует).
Эти пути -- предполагаемые мишени для терапии.
Пример -- один из вариантов рецептора фактора роста HER2 (ErbB-2) (всего, напомню, их 4 и они отличаются наличием различных доменов).
Ещё один вариант -- слитый из двух белков белок, наличие которого в клетках приводит

Надо как-то запретить передачу сигнала.
Можно придумывать молекулы, которые будут запрещать связь 
Можно придумывать ингибиторы киназной активности. % TODO a bit
Лиганд (эпидермальный фактор роста или ...) % TODO a bit
Напомним, многие рецепторы не сами по себе передают сигнал, а через субстраты, лежащие в цитоплазме.
Ещё вариант -- ингибировать димеризацию (если нет димеризации, белки не могут собраться); токсин % TODO ask: ингибирование димеризации можно устроить для конкретного пути?

Хемотерапия: выживаемость клеток снижается, клетка становится более чувствительной к индукторам апоптоза, т.о. метастазирование снижается.

Самые распространённые препараты -- гефитиниб и эрлотиниб (они напрямую взаимодействуют с киназным доменом, ингибируя его).
Проблема в том, что клетки быстро привыкают к этим препаратов.
Бывает, что опухоль сразу устойчива к препаратам, а иногда после долгого воздействия.
Известны замены в ДНК, которые чётко предсказывают наличие устойчивости к препаратам.
Здесь применяется \emph{персонализирования медицина} -- перед лечением проверяются мутации в геноме с помощью секвенирования.

Также есть планшеты... % TODO a bit
Определяем статус конкретной опухоли.

Если активна тирозинкиназа BCR-ABL, нужно ингибировать её, и тогда вместо вышеназванных применяется препарат \textbf{иматиниб}. % TODO a bit

Опасно неспецифическое воздействие, при котором часть раковых клеток гибнет, часть нет (опухоли от этого бывает только лучше).

Сейчас в основном идёт поиск соединений

Что такое "предрасположенность"?
Пусть человек гетерозиготен, у него одна аллель нормальная, другая располагает к раку.
Тогда мы можем работать

Нельзя просто вырубить все тирозинкиназы.
Человек от этого сразу умрёт.
Поэтому идёт подбор веществ, которые...

При миеломах частно встречается вариант BCR-ABL, но бывают и другие.
Классический вариант -- "филадельфийская хромосома", но бывают и слияния 9-22, 10-22, ...
При этом часто применяется иматиниб.

Вообще белок BCR -- киназа; то, с чем он связвается -- тоже киназа. У BCR одна особенность: у него есть домен, отвечающий за димеризацию (вообще, все эти белки требуют димеризации для работы).
Один из подходов, который сейчас разрабатывается -- нивелировать активность этого домена: разрабатываются специальные последовательности -- \emph{пептидомиметики}, аминокислотные цепи, которые связываются с киназой.
Этот подход разрабатывался с надеждой на то, что не потребуется разрабатывать разные последовательности для разных киназ.

Ещё один подход -- конечно же, антитела.
MAb -- monoclonal antibody.
Моноклональность важна, поскольку лекарство должно быть специфично к определённому рецептору.
Отличие моноклональных антител от поликлональных: поликлональные антитела -- с той или иной склонностью к каждому  антигену.
Их аффинность не всегда одинакова к какому-то антигену.

Последовательность белка можно нарабатывать в бактериях, можно в дрожжах.
Затем вкалываем белок в животное (например, кролик), и в крови генерируются антитела.
Для лекарства подход не очень хороший: может оказаться, что эти антитела, произведённые чужим организмом, сами окажутся антигенами для человеческого организма; к тому же, антитела могут связываться с белком непрочно (в исслеованиях, где препарат иммобилизован, это не проблема, а в клетках проблема).
Поэтому используют моноклональные антитела.

Антитела связывают с токсином (например, разрушающим клеточную мембрану; не очень хорошо, поскольку это приводит к апоптозу и возникает воспаление, но из двух зол выбирают меньшее).

HER1, HER2, HER3, HER4
Pertuzumab is a modern fully-humanized antibody

Breast cancer in characterized by HER-2 hyperactivity.
TDM-1: anti-HER2 antibody Trastuzumab
+ microtubic cytotoxic agent: selectively binds to HER2 domains and is internalazed...
That cytotoxin in designed to not detach from Trastuzumab prematurely. \\

По поводу моноклональных антител: их получают, используя \emph{гибридомы} (гибриды клеток).
% TODO a bit
Сливают B-клетки с клетками миеломы (используя полиэтиленгликоль, который нарушает структуру мембраны).
Это соматические клетки с 4N-набором хромосом.
Сливаются в итоге и ядра.
Сливаются, конечно, и B-лимфоциты 

Как клетки миеломы, так и B-лимфоциты -- потомки стволовых клеток крови.
Путь биосинтеза нуклеотидов разный; основной % TODO

Клетки могут активировать
Добавляем в клетку ингибитор.
Основной путь блокируется,
Миеломные клетки сами по себе 

Миеломные клетки могут делиться (бесконтрольно), B-клетки делиться не могут.

\begin{leftbar}
	Вопрос от Насти: что же заставляет раковые клетки делиться бесконтрольно?
	
	Ответ: в разных случаях причины разные.
	Характерный рак груди, как мы только что смотрели -- на мембранах сверхпредставленны белки HER2, в результате чего клетка получает сигналы делиться и не умирать.
	
	Более половины опухолей имеют мутацию в гене белка P53 (нарушение в контроле клеточного цикла), но одного сбоя обычно недостаточно.
	
	Очень много теорий развития опухоли.
	Иногда предлагают термин <<стволовые раковые клетки>> (имеется в виду изначальная клетка, с которой начинается опухоль), но это глупость. % TODO a bit
	Клетки в опухоли, как правило, действительно, теряют свою маркировку -- к примеру, в клетках опухоли мозга мы часто не видим тканеспецифичные маркёры.
	Кажется, что они могут дать и нейрональную, и глиальную линию.
	Но кроме того, что клетки опухоли бессмертны и бесконтрольно делятся, они также неспособны к дифференцировке.
	В самом деле, терминальная дифференцировка -- фактически смерть: клетка некоторое число раз делится, потом всё.
	Если бы раковые клетки могли дифференцироваться, можно было бы накапать в опухоль фактор дифференцировки, и клетки бы вскоре погибли.
\end{leftbar}

Из гибридом мы можем выбрать индивидуальную (ту, которая содержит конкретные антитела).
В мультфильме говорили о \emph{гуманизированных} антителах.
Дополнительные генно-инженерные обработки антител могут приводить к тому, что одна цепь ДНК, кодирующая иммуноглобулин, получается

Нобелевская премия по физиологии и медицине 1984 года \\

Сейчас модно направление, основанное на использовании коротких олигонуклеотидных последовательностей.
Это называется \emph{антисенс-технологии} (антисмысловые последовательности, комплементарные РНК)

\end{document}