\documentclass[main.tex]{subfiles}
\begin{document}

\section{}
<<Уровень заболеваемости болезнью Альцгеймера, болезнью Паркинсона и ддругими нейродегенеративными заболеваниями может превысить в России эпидемиологический порог в $5\%$ к 2050 году>>, пугает чиновник.
Это, впрочем, в основном из-за увеличения продолжительности жизни.

Боковой аминотрофический склероз не имеет отношения к возрасту и может развиться когда угодно.

Всем бы хотелось избежать старческого слабоумия.

% TODO

Основная причина развития заболевания -- аггрегация синуклеина.
Агрегаты уже приводят к гибели нейронов.
Т. н. \emph{аутосомно-доминантный} тип наследования.

Есть ещё несколько генов, изменения в которых связывают с зависимостью заболевания Паркинсона.
(Один из них, между прочим, обнаружили в ПИЯФ - институте ядерной физики).

От чего отталкиваемся, когда ищем гены, отвечающие за предрасположенности?
GWAS: были проанализированы 11 геномов.
Повторы плохо собираются -- это проблема (особенно для современных методов секвенирования, которые умеют читать только короткие последовательности).

\subsubsection{Альфа-синуклеин в патогенезе болезни Паркинсона}

Механизм действия белка до сих пор до конца не изучен.
Известно, что какая-то мутация приводит к его аггрегации в \emph{тельца Леви} (основной компонент -- белок альфа-синуклеин, также есть другие белки, с которыми он взаимодействует).

Дофамин -- основной белок, с которым взаимодействует альфа-синуклеин.
Известна одна замена нуклеотида, которая препятствует связи с дофамином.
Ещё фосфолипаза-Д каким-то образом влияет на функции синуклеина (и не только его).

В норме альфа-синуклеин в мономерном состоянии.
Образование агрегатов начинается с т.н. \emph{протофибрилл}, которые затем становятся \emph{фибриллами}, а те -- \emph{тельцами Леви}.

Окислительный стресс, реакция на окислительный стресс -- эти процессы могут привести к тому, что альфа-синуклеин начнёт выпадать в осадок.

\subsubsection{}

В норме неправильно уложенные белки должны разбираться на части \emph{протеосомами}. Работа этого механизма связана с белками теплового шока Hps70.
Когда образуются фибриллы, 
% TODO a bit
Фосфорелирование

Ещё один путь, который может влиять на аггрегацию синуклеина -- фосфолидаза.

\subsubsection{Мутации в гене LRRK2}

С 2004 года ведутся активные исследования мутаций в гене LRRK2, являющихся наиболее частой причиной наследственных форм болезни Паркинсона.

ПИЯФ: разработан метод скрининга

\subsection{Болезнь Альцгеймера}

Амилоидное заболевание.
Белок, функция которого тоже до сих пор не изучена.

Причину болезни Альцгеймера выяснил ещё Рудольф Вирхов (врач-па\-то\-ло\-го\-ана\-том).
Анализировал структуру тканей мозга людей, страдающих этой паталогией.
Окрашивал ткани мозга; сначала решил, что в мозгу откладывается крахмал (отсюда название -- \textit{амилоиды}, т.е. сахара), но потом понял, что амилоидозы вызываются только белками.

Сейчас известно, что, если окрашивать специфическими красителями (конго-красный, тиофлавин C) ткани, содержащие амилоиды, то в рассеянном свете они светятся красным, а в поляризованном -- зелёным.

Амилоиды малорастворимы.
В нашей лаборатории разработан метод обнаружения агрегатов на основе детергента SDS (между прочим, это вещество входит практически во все моющие средства).
SDS сильный, но даже при кипячении в SDS агрегаты не разрушаются
(в этом и проблема: агрегаты крайне устойчивы, поэтому протеаза не может с ними справляться).

Вспомним коровье бешенство, вызываемое белком  PRP ("губчатая энцефалопатия").
От коровьего мяса с белком PRP может заразиться человек, причём неважно, насколько хорошо оно термически обработано.
Этот же белок вызывает болезнь "куру-куру" (<<Смеющаяся смерть>>), которой заражались при ритуальном каннибализме.
Механизм действия PRP тоже не изучен!

Если здоровой мыши ввести сыворотку нервной ткани мыши, поражённой болезнью Альцгеймера, она заболеет.
Но случаи заражения болезнью Альцгеймера людей неизвестны (видимо, как раз из-за отсутствия каннибализма, а через воздух эта болезнь не передаётся).

\subsubsection{beta- пептид}
$\beta$-APP: его функцию изучали с помощью нокаута в мышах.
Мыши, нокаутные по beta-пептиду, выживают, НО нарушается процесс клеточного транспорта.

\subsubsection{белок тау}

Этот как-то участвует в формировании болезни Паркинсона, а также Альцгеймера.
Функция его известна: он участвует в формировании цитоскелета (микротрубучек), напрямую взаимодействуя с тубулированием, а также в везикулярном транспорте.

Фосфорелирование этого белка % TODO a bit

Почему все эти болезни различают?
Назвали бы <<нейродегенерация>>, и всё.
Но эти белки поражают разные виды нейронов, и клиническое проявление разное.
Соответственно, разные подходы к лечению.

Сейчас начинаются споры: почему клетки гибнут, когда в них белковые агрегаты?
Возможно, мысль о суициде приходит к клетке ещё раньше, чем агрегат формируется.

\end{document}
