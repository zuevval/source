\documentclass[main.tex]{subfiles}
\begin{document}

% TODO much

% ~ 30 m from the beginning

Если неправильный сплайсинг в циклине-д, результат -- рак.
Нарушение происходит в разных тканях и лечить сложно.

Помимо антисмысловых последовательностей ещё микроРНК.
Сейчас уже понятно, что микроРНК имеют большое значение.
Довольно долго была проблема: секвенируют геном человека с раком и не находят никакие мутации.
Сейчас есть довольно обширные базы данных SNP и по ним можно проводить поиск;
но чётких гаплотипов пока нет, слишком много данных.

Часто оказывается, что это сбой в работе генов за счёт нарушения в работе микроРНК, и это те типы раков.

\subsection{Aberrant microRNA expression in human cancer}

В разных статьях можно увидеть разные данные по экспрессии микроРНК, потому что раки разные.
Но это область довольно скользкая.
Если берут одну клетку и смотрят, что в ней, оказывается, что даже между клетками рисунок микроРНК разный.
Неудивительно,

Многие микроРНК секретируются в окружающую ткань (в последнее время стало ясно, что для клетки важно окружение).
Раковые клетки влияют на окружающие ткани, которые тоже становятся подчинёнными жизни опухоли.

Изменение спектра микроРНК часто происходит в ответ на химиотерапию, что может вызвать устойчивость.
(напомним, действуем ингибитором киназы; в результате внутри опухоли происходит селекция клеток).
Поэтому применение ингибиторов тирозинкиназ -- терапия, которая требует замены примерно через полгода.

Воздействие на клетку в течение длительного времени с помощью микроРНК меняет весь оркестр процессов в клетке, и они включаются в жизнь опухоли.

У нас в клинике Отто
Химическим способом они синтезируют полипептиды, потом к полипептидам привешивают сигнальные молекулы.
На них РНК накручивается и в итоге получается RNP-комплекс.
Уже этими комплексами
Проблема в том, что там всегда надо подбирать соотношение с носителем, проверять токсичность носителя.

Например, боремся с какой-то гормон-зависимой опухолью; хотим, чтобы 
Как правило, сигнальные молекулы подбирают так, чтобы они узнавали конкретные рецепторы.

\begin{leftbar}
Нельзя сказать, что достигнуты большие успехи в этом деле.
Сейчас есть носители, производимые несколькими фирмами, но они не совсем
Люди находятся в поиске хороших защитных носителей.
Самые хорошие способы -- вирусные векторы (например, первое успешное лечение врождённых заболеваний было проведено с помощью инактивированных аденовирусов).
Но поведение вирусной ДНК мы не можем чётко контролировать.
К тому же это подходит для случаев, когда гены должны попасть в ядро, а нам необходимо работать в цитоплазме (как правило, чтобы убить клетку). В ядре и так есть всё, что нужно для смерти клеток, просто выключены механизмы смерти.
\end{leftbar}

Облучение нехорошо тем, что клетки гибнут не апоптозом, а некрозом.
А мы ищем способ убить клетку нежно.
Апоптоз хорош тем, что никаких следов после него не остаётся, воспаления нет, место планово зарастает новыми клетками.

\begin{leftbar}
Аналог апоптоза во многоклеточных, если так можно выразиться, есть у бамбука (старые стебли гибнут в основном апоптозом, чтобы освободить место новому поколению); у лосося после нереста (но личинка жемчужницы может блокировать этот механизм). Называется феноптоз.
\end{leftbar}

Есть ещё один современный метод, который активно развивается, основан на получении Т-клеток, которые узнают конкретные опухолевые клетки.
Эти подходы довольно активно разрабатываются.
У нас это активно делает \textbf{BioCAD}; они вообще довольно продвинутые в этом, одни из первых в мире делают лекарство против аутоиммунной болезни Бехтерева.

Подход хорош тем, что это не чужие Т-лимфоциты, а собственные Т-лимфоциты организма, модифицированные химерными рецепторами.

\subsection{G-белки}

Ещё один класс белков, которые участвуют в механизмах работы опухолей -- Ras (RAt Sarcoma), G-белки (малые ГТФ-азы).

% TODO a bit

...стимулируют клетку к делению и выживанию.

Надо как-то повлиять на активность ГТФ-азы.
Напомним, ещё в прошлом году мы говорили про белок P53; мутации в гене этого белка обнаруживаются в 50\% случаев. А в генах Ras -- в 20 \% случаев.

В клетках человека присутствуют 4 Ras-белка, которые отличаются хвостиками.
Хвостик называется форнизил (?).
Можно повлиять на процесс форнизилирования, чтобы он вообще не прикреплялся к мембране. \\

На каждом этапе можно пробовать искать какую-то мишень.
Есть довольно простой путь JAK-STAT, который влияет на янус-киназы.
Янус-киназа может проявлять как киназную, так и антикиназную активность.
Многие процессы в клетке запускаются путём активации JAK-пути, в том числе...
% TODO a bit

Внутри клетки тоже есть механизмы, отслеживающие качество ДНК в клетке и прочие показатели.
Перед репликацией нужно всегда проверить качество ДНК и репарировать при необходимости.

Основной путь, ответственный за пролиферацию -- MAPK-pathway, mitogen-activated protein kinase pathway.

% TODO a bit

Циклины -- белки, которые, будучи связанными с циклин-зависимыми киназами, активизируют  ...

MYC репрессирует факторы, выключающие деления.
Есть специальные промоторные последовательности, которые узнаются  MYCом.

\subsection{Клеточный цикл}

Это отдельная машинерия.

Основные регуляторы в этой области -- ретинобластомы и белок P53.
В своё время казалось, что стоит научиться управлять белком P53 (<<Белок года 1993>>), и все раки будут побеждены.
Потом оказалось, что это не панацея.

\end{document}