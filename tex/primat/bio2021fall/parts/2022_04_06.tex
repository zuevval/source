\documentclass[main.tex]{subfiles}
\begin{document}
	
\section{микроРНК}

Механизмы плохо изучены.

Длинные некодирующие РНК в целом направлены на ингибирование малых кольцевых РНК.

Некодирующая РНК NTT: специфичная, выявлена только в активных Т-клетках.
Для большинства таких некодирующих РНК есть какая-то узкая ниша, в которой они работают.

Способствует метастазированию: определённый класс микроРНК.

Одна из основных проблем в клетках опухоли -- гипоксия (конечно, опухоль прорастает сосудами, но на это требуется время).
Перспективное направление в биомедицине -- поиск некодирующих РНК, которые маркируют те или иные болезни.

SAF -  антисмысловая РНК к антиапоптозному гену FAS, которая связывается с РНК-транскриптами гена FAS.

\subsection{транскрипция РНК}

с РНК не так всё просто.
Длинные некодирующие РНК, кольцевые РНК погут ингибировать транскрипцию/трансляцию (в хроматине).

матричные РНК, транспортные РНК, рибосомные РНК транскрибируются разными полимеразами.

Транскрипция с

Энхарсеры, сайленсеры, инсуляторы, LCR (Locus control region) --

Инсуляторы -- не энхансеры и не сайленсеры, а последовательности, которые по-разному действуют на разные гены (мешают энхансерам и сайленсерам).

Чтобы транскрипция началась, одной РНК-полимеразы недостаточно.
Должны быть транскрипционные факторы.

"Главные факторы транскрипции": без них РНК синтезировать не будут.
\begin{leftbar}
	GC-богатые участки -- характерная черта промоторных областей.
	Намекают на то, что в данной области содержится 
\end{leftbar}

Транскрипция и сборка рибосом начинается в ядрышке.

После начала транскрипции есть "пауза", во время которой транскрипция может быть остановлена.

Репликация, транскрипция, трансляция -- три матричных процесса.
У каждого из них выделяют три стадии -- инициация, элонгация и терминация.

Белки-"писатели", белки-"стиратели", белки-"читатели".
Основные способы влияния на функции белков -- метилирование, ацетилирование, фосфорилирование.
Также есть сумоилирование и убиквитинирование.

Сами по себе зоны

Конфокальная микроскопия -- лазером возбуждаются различные красители.

В хромосомных территориях видны выпетливания.
Теломерные и центромерные последовательности: близко к ним почти ничего не происходит.

Локус-контролирующие регионы: петли, как правило, содержат гены; % TODO

Транскрипционно активный домен: как правило, он может содержать и % TODO

И "коровые", и специфичные ТФ 

ДНК-связывающие домены у всех ТФ тоже могут быть разные.
ДНК имеет большую бороздку и малую бороздку.
Последовательности большой бороздки 

ТФ соединяется с последователями водородными связами (водородные связи хорошо обратимы).

"Цинковые пальцы": $\alpha$-спираль и $\beta$-слой, которые содержат внутри ион цинка.
Ьакже "спираль-поворот-спираль".

"Элементы отклика" -- последовательности для специфических факторов транскрипции.

Гомеодомен-содержащие транскрипционные факторы.
Тонкая настройка транскрипци (потому экспрессия генов разная в разных клетках).

Напомним, мы говоили о том, как гормоны влияют на экспрессию генов.

ДНК-метилтрансфераза: после репликации узнаёт последовательности, где метильные группы есть только в одной цепи.

\subsection{Редактирование РНК}

Явление было открыто довольно давно, но вначале казалось, что это что-то экзотическое.

Трипаносомы -- внутриклеточные паразиты, переносимые мухой Цеце и вызывающие сонную болезнь.
Один вариант цитохромоксидазы работает в клетках мухи (3 субъединицы), другой у человека (2 субъединицы).

Механизм похож на CRISPR-CAS.
Эдитосома ("редактосома").

Механизм редактирования похож на механизм работы ДНК- и РНК-полимераз (что неудивительно, задача та же -- создание вставок и делеций). Правда, ферменты работают не в направлении от 5' к 3', а наоборот.

Все липиды нерастворимые, а основной растворитель в кишечнике -- вода.
Аполипопротеины помогают организму усваивать липиды.
В крови липиды передаются в фосфолипидных гранулах.

Замена аденина на инозин:
все нуклеотиды в РНК, как правило, модифицированные (в самом деле, организм должен как-то отличать зрелые тРНК от незрелых, от кусков РНК и так далее).

Итак, оказывается, что процессы включения/выключения генов довольно сложные, требуют сборки специальных комплексов; и это не просто энхансеры/сайленсеры, но и сложные механизмы вроде редактирования РНК.

\begin{leftbar}
	Почти всякую соматическую клетку можно превратить в стволовую путём редактирования всего лишь четырёх генов.
\end{leftbar}

\end{document}