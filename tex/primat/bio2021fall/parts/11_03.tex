\documentclass[main.tex]{subfiles}
\begin{document}
\section{Сигнальная трансдукция}
Oct 3, 2021

Мы говорили о сигнальной трансдукции и о том, что её нарушения / изменения

Онкогены, протоонкогены.
В основном гены, отвечающие за передачу сигнала (кодируют рецепторы, эффекторные гены) являются онкогенами.

Клетка всегда готова к апоптозу.
Чтобы она себя не убивала, ей нужны всё время "поддерживающие" сигналы извне.
Факторы роста: тромбоцитарные, интерфероны.

Фактор роста опухолей-$\beta$.

Описание факторов роста: Нобелевская премия 1986 года.

Рецепторы, которые принимают гормоны: могут быть в мембране, а могут быть цитозольные (ядерные).
% TODO a bit
Как правило, внутриклеточная часть мембранных рецепторов обладает ферментативной активностью, либо же 
G-белок... активирует находящиеся обычно внутри мембраны или рядом с мембраной рецепторы.
Как правило, эти белки умеют, к примеру, фосфорилировать белки или создавать циклические...

Могут меняться мишени.
При освобождении белк % TODO a bit
Тоже один из способов регуляции генов путём освобождения ТФ.

Вообще, фосфорилирование и дефосфорилирование -- один из наиболее распространённых способов посттрансляционной модификации белков и контроля его активности.
Естественно, это АТФ-зависимый процесс.

За открытие фосфорилирования и дефосфорилирования Фишер и Кребс получили Нобелевскую премию по медицине в 1992 году.
Учёные поняли, что фосфорилирование/дефосфорилирование в клетке происходит вследствие воздействия разных рецепторов.

Киназ в клетке много.
В первую очередь -- серин-трианиновые киназы (кодируют в белках серин и трианин).
Сейчас модно говорить про "-ом"ы; говорят и про "кином".

Активность серин-трианиновых киназ зависит от концентрации в клетках ионов кальция.
% TODO a bit

Большое поли-исследование посвящено тирозин-киназам.
Их много (почти 17\% от всех); они подразделяются на рецепторные и внутриклеточные, они же \emph{эндоплазматические}.
Эти белки могут различаться по числу и составу доменов.
Основной домен, по которому их относят к группе -- киназы.

% TODO a bit

То, как работают тирозин-киназы, сейчас хорошо изучено.
Известно, как работает \emph{каталитический центр}.

Специфичность тирозин-киназ (субстрат, с которым связывается киназа) определяется в первую оченедь локализацией (возме мембраны, в ЭПР, вблизи митохондрий...) % TODO a bit

Активный центр распознаёт субстрат (подходит ему этот белок или нет), и важны не только аминокислоты в распознающей петле, но могут быть важны и электростатические взаимодействия. \\

Помимо цитоплазматических (ядерных) киназ, есть ещё рецепторные.
Они делятся на 20 подсемейств согласно тем частям последовательности, которые торчат наружу из клетки.
Тирозиназный домен всегда находится в цитоплазме.

При связи с ростовым фактором или другим лигандом... происходит активация киназных моделей, \emph{автофосфорилирование} (белок фосфорилирует соседа), присоединяется адаптер и молекулы могут продвигаться дальше в клетку.

Тирозин-киназный домен у рецепторных тирозин-киназ выглядит примерно так же, как и у растворимых (мембранных). % TODO some

Разные типы рака возникают при сбоях в работе рецепторов эпидермального семейства.
EGFR, он же erbB1 -- рецептор эпидермального фактора роста; ещё один -- erbB4.
Это паралоги.

Гипотеза: при переходе земноводных на сушу происходили сложные перестройки и полиплоидизация; в частности, из-за этого, согласно гипотезе, многие гены присутствуют в большом числе копий.
Рецепторов эпидермального фактора роста всего четыре.

% TODO a bit
Такая сложная передача сигнала подразумевает, что сигнал может усилиться.
В итоге клетка отвечает на небольшую изначальную концентрацию сигнального фермента мощными изменениями.

Рецепторы не могут постоянно торчать.
В клетках происходит круговорот рецепторов, контроль их активности.
Часть убиквитинируются (это тоже посттрансляционная модификация). % TODO a bit

Способы передачи сигнала бывают сами разные.
Длинные расстояния -- эндокринный путь, близкие -- паракринный, а также клетка может поддерживать сама себя.
Есть специальные металлопротеазы, которые отщепляют кусок одного рецептора и прикрепляют к другому (аутокринная регуляция).

% TODO a bit
Это -- главное семейство рецепторных тирозин-киназ.

Чем больше опухоль прорастает сосудами, тем ей лучше (в любой опухоли начинается момент, когда ей требуется для роста больше ресурсов, чем есть, и она даёт сигнал для увеличения числа сосудов).

% TODO a bit

...В итоге фосфорилируются транскрипционные факторы, регулирующие те или иные гены.

\subsubsection{G-белки}

Обычно они находятся в цитоплазме,  % TODO a bit

Они утрачивают сродство с GDF (гуаназиндифосфат), GDF диссоциирует, на его место встаёт  GTF (гуаназинтрифосфат).


\subsubsection{Протоонкогены} 

...Когда одни сигналы говорят "делись", другие -- "не делись", клетка с хорошей защитой убивает себя.
Но раковая трансформация происходит после накопления нескольких мутаций.
Говорят, что нужно накопление как минимум трёх мутаций.

Какие же изменения могут вести к превращению в "плохих парней"?
К примеру, увеличение копийности (все паралоги работают, в клетке накапливается большое число продуктов-белков).
Клетка и её соседи получают сигналы делиться; делятся и, во-первых, при этом не успевает происходить репарация из-за быстрого деления, во-вторых, во время   % TODO ask: difference between amplification and copy number increase?

\begin{leftbar}
	Ферменты -- тоже люди, они могут ошибаться.
\end{leftbar}

Может быть причина в точковых мутациях.

% TODO some

Механизмы амплификации (увеличения числа копий) генов до сих пор непонятны.
Один из предполагаемых механизмов -- ошибка репликации, когда репликация началась, хромосома синтезировалась новая, и внутри этой вилки репликации вдруг начинается ещё одна репликация.
Но вообще вопрос довольно непонятный.

С увеличенным числом репликатов связаны многие типы рака.
См. таблицу на слайде -- "Amplification activated oncogenes in human cancers and their clinical significance".
Практически на каждой хромосоме есть участки, кодирующие протоонкогены, и их амплификация ведёт к появлению различных типов рака.

Транслокации иногда приводят к овер-экспрессии.
Пример: ген c-myc транслоцируется с 8 на 14 хромосому (в участок, где находятся группы иммуноглобулинов), и энхансер, расположенный в этом районе, ответственный за активный иммунитет.
Постоянно производятся продукты, которые способствуют пролиферации.
В итоге в клетке оказываются только лимфоциты.

"Филадельфийская хромосома"\hspace{0pt} -- результат слияния 22-й хромосомы с кусочком 14-й.
Это можно увидеть, посмотрев на препарат.
Можно предложить способ терапии: взять белок, который специфичен для онко-клеток, подобрать к нему антитела и, вводя их в организм, бороться с раком.

Не забываем про мобильные элементы.

Напомним, что перемещение ALU тоже может приводить к раку, поскольку, встраиваясь в регуляторные области генов, они могут влиять на транскрипционную активность.
Перемещаясь в разные места, ALU   вызывают разные типы рака.

\end{document}