\documentclass[main.tex]{subfiles}
\begin{document}
\section{Сигнальная трансдукция}
Oct 3, 2021 % changed Jan 09, 2021 using the video

Мы говорили о сигнальной трансдукции и о том, что её нарушения / изменения могут повлечь раковую трансформацию.

Онкогены, протоонкогены.
В основном гены, отвечающие за передачу сигнала (кодируют рецепторы, эффекторные гены) являются онкогенами. \\

В клеточном цикле есть важная точка $ G_0 $, в которой клетка выбирает, будет она дальше делиться, или входить в стадию терминальной дифференцировки (после которой выполняет свои функции ткани, но больше не делится), или запустить механизм клеточной смерти, или начать стареть.

Некоторые исследователи предлагают относиться к старению как к виду терминальной дифференцировки: клетка уже не делится и характеризуется специфическим фенотипом, как и дифференцированные клетки.

Основные сигнальные молекулы -- гормоны, нейромедиаторы, нейромодуляторы, цитокины, разные комплексы иммуноглобулинов (например, MHC -- комплекс гистоосовместимости, сигнальные молекулы, определяющие и отражающие клеточную физиологию)...

\textbf{Главную роль играют факторы роста}.
В клетке всегда присутствуют белки, способные её убить очень быстро.
Клетка всегда готова к апоптозу.
Чтобы она себя не убивала, ей нужны всё время "поддерживающие" сигналы извне.
Факторы роста: тромбоцитарные, интерфероны/интерлекины (стимулируют размножение клеток иммунной системы -- лимфоцитов), а также факторы роста опухолей.
Сбои в передаче сигналов часто приводят к раку.

Фактор роста опухолей-$\beta$ (TGF-$\beta$): сбои в работе этого белка приводят к раку.

За описание факторов роста -- Нобелевская премия 1986 года.


Рецепторы, которые принимают гормоны: могут быть в мембране, а могут быть \emph{цитозольные} (\emph{ядерные}) -- находятся в цитоплазме, не торчат наружу и тоже способны попадающую внутрь клетки сигнальную молекулу связывать и перемещать в ядро.

Как правило, цитозольные рецепторы связывают молекулу, из-за чего перестраивают свою конформацию и мигрируют в ядро, где взаимодействуют с промоторной областью какого-то гена.
Передача сигнала через мембранные рецепторы -- более длинный путь; как правило,  внутриклеточная часть мембранных рецепторов обладает ферментативной активностью, либо же они определяют проникновение в клетку или выход из клетки различных ионов (что само по себе является сигналом), либо же путь может быть связан с так называемыми  G-белками.

\href{https://youtu.be/zbtDvIQ53lA?t=487}{YouTube (time: 8:07)}

G-белок... активирует находящиеся обычно внутри мембраны или рядом с мембраной рецепторы.
Как правило, эти белки умеют, к примеру, фосфорилировать белки или создавать циклические...

Могут меняться мишени.
При освобождении белк % TODO a bit
Тоже один из способов регуляции генов путём освобождения ТФ.

Вообще, фосфорилирование и дефосфорилирование -- один из наиболее распространённых способов посттрансляционной модификации белков и контроля его активности.
Естественно, это АТФ-зависимый процесс.

За открытие фосфорилирования и дефосфорилирования Фишер и Кребс получили Нобелевскую премию по медицине в 1992 году.
Учёные поняли, что фосфорилирование/дефосфорилирование в клетке происходит вследствие воздействия разных рецепторов.

Киназ в клетке много.
В первую очередь -- серин-трианиновые киназы (кодируют в белках серин и трианин).
Сейчас модно говорить про "-ом"ы; говорят и про "кином".

Активность серин-трианиновых киназ зависит от концентрации в клетках ионов кальция.
% TODO a bit

Большое поли-исследование посвящено тирозин-киназам.
Их много (почти 17\% от всех); они подразделяются на рецепторные и внутриклеточные, они же \emph{эндоплазматические}.
Эти белки могут различаться по числу и составу доменов.
Основной домен, по которому их относят к группе -- киназы.

% TODO a bit

То, как работают тирозин-киназы, сейчас хорошо изучено.
Известно, как работает \emph{каталитический центр}.

Специфичность тирозин-киназ (субстрат, с которым связывается киназа) определяется в первую оченедь локализацией (возме мембраны, в ЭПР, вблизи митохондрий...) % TODO a bit

Активный центр распознаёт субстрат (подходит ему этот белок или нет), и важны не только аминокислоты в распознающей петле, но могут быть важны и электростатические взаимодействия. \\

Помимо цитоплазматических (ядерных) киназ, есть ещё рецепторные.
Они делятся на 20 подсемейств согласно тем частям последовательности, которые торчат наружу из клетки.
Тирозиназный домен всегда находится в цитоплазме.

При связи с ростовым фактором или другим лигандом... происходит активация киназных моделей, \emph{автофосфорилирование} (белок фосфорилирует соседа), присоединяется адаптер и молекулы могут продвигаться дальше в клетку.

Тирозин-киназный домен у рецепторных тирозин-киназ выглядит примерно так же, как и у растворимых (мембранных). % TODO some

Разные типы рака возникают при сбоях в работе рецепторов эпидермального семейства.
EGFR, он же erbB1 -- рецептор эпидермального фактора роста; ещё один -- erbB4.
Это паралоги.

Гипотеза: при переходе земноводных на сушу происходили сложные перестройки и полиплоидизация; в частности, из-за этого, согласно гипотезе, многие гены присутствуют в большом числе копий.
Рецепторов эпидермального фактора роста всего четыре.

% TODO a bit
Такая сложная передача сигнала подразумевает, что сигнал может усилиться.
В итоге клетка отвечает на небольшую изначальную концентрацию сигнального фермента мощными изменениями.

Рецепторы не могут постоянно торчать.
В клетках происходит круговорот рецепторов, контроль их активности.
Часть убиквитинируются (это тоже посттрансляционная модификация). % TODO a bit

Способы передачи сигнала бывают сами разные.
Длинные расстояния -- эндокринный путь, близкие -- паракринный, а также клетка может поддерживать сама себя.
Есть специальные металлопротеазы, которые отщепляют кусок одного рецептора и прикрепляют к другому (аутокринная регуляция).

% TODO a bit
Это -- главное семейство рецепторных тирозин-киназ.

Чем больше опухоль прорастает сосудами, тем ей лучше (в любой опухоли начинается момент, когда ей требуется для роста больше ресурсов, чем есть, и она даёт сигнал для увеличения числа сосудов).

% TODO a bit

...В итоге фосфорилируются транскрипционные факторы, регулирующие те или иные гены.

\subsubsection{G-белки}

Обычно они находятся в цитоплазме,  % TODO a bit

Они утрачивают сродство с GDF (гуаназиндифосфат), GDF диссоциирует, на его место встаёт  GTF (гуаназинтрифосфат).


\subsubsection{Протоонкогены} 

\href{https://youtu.be/zbtDvIQ53lA?t=3381}{YouTube, lecture 8 (time: 56:21)}

Протоонкогены -- гены, кодирующие компоненты сигнальных путей (гены рецепторов, гены самих сигнальных молекул -- ростовых факторов, гены вторичных мессенджеров, гены различных киназ).
От сигнальных молекул в первую очередь сигналы идут к генам, отвечающим, с одной стороны, за клеточную пролиферацию и с другой стороны за апоптоз.
Естественно, что сбои в регуляции этих процессов ведут к развитию рака.

Но надо отметить, что есть серьёзная система защиты от перерождений; в клетке всегда есть белки, способные её быстро убить (те же протеазные[каспазные] каскады: в клетке всегда присутствуют \emph{прокаспазы} -- неактивные каспазы).
В большинстве случаев клетка требует для жизни поддерживающих сигналов.

\href{https://youtu.be/zbtDvIQ53lA?t=3578}{YouTube, lecture 8 (time: 59:38)}

Но может возникнуть состояние внутреннего конфликта, если поступают противоречивые сигналы.
Когда одни сигналы говорят "делись", другие -- "не делись", клетка с хорошей защитой убивает себя (это нормально для клеток многоклеточного организма).
Но раковая трансформация происходит, как правило, после накопления нескольких мутаций.
Некоторые исследователи считают, что нужно накопление как минимум трёх мутаций (в протоонкогенах или генах, контролирующих клеточную гибель). \\

Какие изменения могут привести к превращению протоонкогенов в онкогены?

\begin{enumerate}[noitemsep]
	\item Увеличение копийности гена (увеличивается уровень его экспрессии; все паралоги работают, в клетке накапливается большое число продуктов-белков).
	
	Пример: увеличилось число копий эпидермального фактора роста.
	Клетка и её соседи получают сигналы делиться; увеличивается пролиферстивная активность и, во-первых, при этом не успевает происходить репарация из-за быстрого деления (вследствие чего увеличивается частота мутаций и происходит её накопление; может происходить изменение других протоонкогенов).
	
	\item Точковые мутации
	\item Изменения структуры промотора (как и увеличение копийности, может привести к накоплению белка, усилению его функции)
	\item Хромосомные перестройки, которые приводят к возникновению новых промоторных последовательностей или вообще новых белков.
\end{enumerate}
   % TODO ask: difference between amplification and copy number increase?

Ещё пример, связанный с увеличением числа копий: ген MYC -- ген транскрипционного фактора, запускающего гена, которые отвечают за синтез ДНК; увеличение его копийности может привести к лейкемии.

Нормальный ген MYC присутствует в двух копиях: на хромосоме 8 и на второй хромосоме 8. % 1:08:18
В примере на слайде увеличение копийности гена  MYC за счёт появления дополнительных отдельно находящихся кусочков хромосом; они даже обычно не имеют центромер, поэтому при клеточном делении могут теряться, число их может меняться.

Механизмы амплификации (увеличения числа копий) генов до сих пор непонятны.
Почему одни последовательности начинают реплицироваться гораздо активнее других?
(Бывает, что амплификаты выпадают из хромосом, а бывает, что остаются там, но занимают почти всю хромосому).

Один из предполагаемых механизмов -- ошибка репликации, когда репликация началась, хромосома синтезировалась новая, и внутри этой вилки репликации вдруг начинается ещё одна репликация (бесконтрольный синтез новой молекулы).
Но вообще вопрос довольно непонятный.

С увеличенным числом репликатов связаны многие типы рака.
Разные участки хромосом могут быть амплифицированы; всякие MYC-гены (MYC не один, у него есть паралоги).
Чрезмерная амплификация приводит к раку.
Притом для многих генов показано, что увеличение копийности характерно для агрессивных раковых перерождений.
См. таблицу на слайде -- "Amplification activated oncogenes in human cancers and their clinical significance".
Практически на каждой хромосоме есть участки, кодирующие протоонкогены, и их амплификация ведёт к появлению различных типов рака.

Хромосомные перестройки также могут вызывать рак.
Могут возникнуть новые белковые продукты, или регуляторные последовательности сближаются с геном и начинается овер-экспрессия.

Итак, транслокации иногда приводят к овер-экспрессии.
Пример: ген c-myc транслоцируется с 8 на 14 хромосому (в участок, где находятся группы иммуноглобулинов), и энхансер, расположенный в этом районе, ответственный за активный иммунитет (энхансер нужен для постоянного производства иммуноглобулинов, в клетках иммунитета это важно).
Постоянно производятся продукты, которые способствуют пролиферации.
В итоге в крови оказываются только лимфоциты, причём недифференцированные (т. н. лимфома Беркетта).

Пример транслокации, приводящей к появлению нового белка -- с 9-й на 22-ю хромосому.
Гены BCR и ABL сливаются; ABL -- это киназа, но она получает новые функции благодаря участку BCR.
Увеличивается пролиферация.
Появляется "Филадельфийская хромосома"\hspace{0pt} -- результат слияния 22-й хромосомы с кусочком 9-й.
Это можно увидеть, посмотрев на препарат.
Можно предложить способ терапии: взять новый слитый белок, который специфичен для онко-клеток, подобрать к нему специфичные антитела и, вводя их в организм, бороться с раком.

Не забываем про мобильные элементы.

Кстати говоря, перемещение ALU (напомним, наш геном обогащён такими) тоже может приводить к раку, поскольку, встраиваясь в регуляторные области генов, они могут влиять на транскрипционную активность.
Перемещаясь в разные места, ALU  вызывают разные типы рака.

\end{document}