\documentclass[main.tex]{subfiles}
\begin{document}
\section{ Инфекционные амилоиды -- прионы  }

13 октября 2021 г.

PRP -- Prion Protein

Protein Infection $ \rightarrow $ ProIn $ \rightarrow $ Prion

Врач Гайдушек доказал, что PRP -- инфекционное заболевание; отнёс к группе "медленных" инфекций (ведь это может развиваться лет 20, в отличие от вирусных и бактериальных инфекций).
Гайдушек фильтровал вытяжки из тканей умерших... показал, что там нет нуклеиновых кислот (т.е. нет РНК/ДНК бактерий или вирусов) и в итоге получил Нобелевскую премию.
Впоследствии за механизм прионизации была дана ещё одна Нобелевская премия (Прузинер).

Основой для слипания белковых агрегатов являются бета-слои.
Существует ген (PRNP) этого белка; сейчас обнаружили несколько мутаций, которые вызывают предрасположенность к прионизации.

Тот же белок вызывает скрепи у овец.

PRP$^\text{C} $ (PRP-cellular) -- нормальная растворимая форма.
Замены аминокислот могут порождать разные формы укладки.
% TODO a bit

Полная защита может быть только у гомозиготы по глутаминному аллелю. \\

Фатальная бессонница (вызывается самой редкой мутацией в PRP; передаётся по наследству): сперва сбивается ритм сна, затем возникает бессонница, которая снотворными не лечится (гибнут нейроны, отвечающие за контроль сна).
Довольно быстро нарастают проблемы, человек не может нормально питаться и двигаться, возникают галлюцинации; через 4-5 месяцев -- смерть.

Почему прионные заболевания, как и Альцгеймер/Паркинсон, развиваются лишь с возрастом?
Считается, что старение организма является катализатором прионизации.

\begin{leftbar}
	Антитела против Альцгеймера/Паркинсона есть и прошли доклинические испытания.
	Но проблема в том, что эти болезни, как правило, начинают лечить уже на поздней стадии, а добиться улучшения когнитивной функции пока не получается.
	Много усилий направлено на раннюю диагностику.
\end{leftbar}

Шванновские клетки -- те, которые поддерживают функционирование аксонов.
Они тоже поражаются PRP.
% TODO a bit

\begin{leftbar}
"Нервные клетки не восстанавливаются"\hspace{0pt} -- миф.
"Спящие"\hspace{0pt} глиальные клетки в мозгу при резком повреждении (при травмах) могут активироваться.

Но пересадка нервной ткани, наверное, дело далёкого будущего.
Пересаженным нейронам ещё нужно научиться общаться с остальными.
\end{leftbar}

\subsection{Стресс-гранулы}

<<Стресс-гранулы>> -- рибонулкеиновые комплексы, которые в ответ на стресс запрещают синтез всего, кроме защитных белков (шеперонов, шеперонинов...) путём остановки трансляции РНК в белки.
Этот механизм используется в терапии рака.

Стресс-гранулы имеют прионоподобный домен.
% TODO ? a bit

Формирование стресс-гранул:
Напомним, механизм долговременной памяти (РНК) был впервые экспериментально показан на улитках.
CPEB -- регулятор трансляции.
У человека, у дрозофилы и так далее есть ортологи CPEB.

Прионоподобные агрегаты используются в механизмах защиты от вирусов; также \emph{меланосомы} (органеллы клеток кожи, защищающие нас от ультрафиолета) вырабатывают пигмент, похожий на амилоидные тяжи (и это устойчивые образования).
Недавно у мужских половых клеток обнаружили органеллы  \emph{акросомы}, которые содержат агрессивные вещества.
Паутина целиком состоит из белка, обогащённого бета-складками и уложенного в амилоидные тяжи (между прочим, синтетическую паутину сейчас пытаются использовать как основу для посадки стволовых клеток при лечении ожогов кожи).

Есть амилоиды и у прокариот.
Некоторые бактерии, напомним, образуют \emph{биоплёнки}.

Самые изученные прионы -- у грибов.
Когда они размножаются, вместе с цитоплазмой дочерним грибам передаются и прионы.
Причём эти белки часто жизненно важные: транскрипционные факторы, регуляторы обмена...

% TODO a bit
Теперь мы знаем, что амилоидизация широко распространена в природе.
У бобовых, к примеру, большая часть запасов в зародыше сохраняется с помощью амилоидов.
Есть даже гипотеза о том, что абсолютно все белки могут образовывать агрегаты, но она спорная.

\end{document}
