\documentclass[main.tex]{subfiles}
\begin{document}
	
\section{Болезнь Альцгеймера. Продолжение}
Напомним, кроме бета-пептида, второй белок, участвующий в формировании БА -- белок тау.
Белок тау теряет свою функцию к организации цитоскелета и организует агрегаты.

До сих пор идут споры о причине заболевания.
Первая гипотеза: причина -- агрегация амилоидов.
Вторая гипотеза: агрегация -- лишь следствие.

Мутации в гене, кодирующем тау-белок, приводят к похожему нейродегенеративному заболеванию.

Если больной мыши впрыснуть иммунные тела против бета-амилоида в мозг, число агрегатов тау-белка тоже уменьшается.

\subsubsection{Исследование метаболической активности клеток}

В нормальных, живых клетках работает фермент \emph{митохондриальная редуктаза}, которая при связи с некоторым реагентом окрашивается синим цветом.

\subsubsection{ Гамма-секретазы }

Напомним, расщеплением бета-пептида, фрагменты которого образуют амилоидные структуры, заведуют специфические протеазы \emph{пресинилины}, которые находятся внутри мембраны и умеют прямо в толще мембраны работать (белок трансмембранный).

Мутации в генах пресинилинов -- тоже перспективная мишень для лечения болезни Альцгеймера.

% TODO some

\subsection{ Патогенез }

При болезни Альцгеймера нет конкретных локаций, в которых сосредоточена больная ткань.

Мутации: гены пресинилинов PSN-1 и PSN-2.

Аполипопротеин Е (основная функция -- обмен липидов, отвечает за транспорт холестирина к клеткам, в которых он должен работать).
Холестирин входит в состав всех клеток.
В мозге он синтезируется астроцитами и глиальными клетками; нейроны его не синтезируют, но у них есть специальные рецепторы, через которые холестирин попадает внутрь.

Есть гипотеза: нарушение липидного обмена -- причина того, что клетки начинают "плохо себя вести", вследствие чего образуются агрегаты белков.
Мутации в генах аполипопротеинов Е, как выяснили недавние исследования, влияют на предрасположенность / отсутствие предрасположенности к болезни Альцгеймера. \\

До сих пор нет строгого видения и понимания того, с чего начинается болезнь Альцгеймера.
Хотя и производятся попытки разработать схемы лечения.

Долго считалось, что тау-белковые отложения более опасны, чем отложения бета-пептида (поскольку нарушения в работе тау-белка приводят к разрушению клеток).

Сейчас говорят, что есть каскады нарушений: нарушение фосфорилирования $\rightarrow$ ...
Старение как фактор.
Напомним, активность бета-галактазидазы всегда маркирует стареющие клетки.

Очень много ещё непонятного, неизвестного.

Недавно обнаружили у дельфинов, которые выбрасываются на берег, агрегаты белков.

Конечно, нейродегенеративных болезней много (не только Альцгеймера). \\

У нас есть механизмы защиты от белковых отложений, например, шепероны.
Шепероны начинают взаимодействовать с белком, как только он вышел из рибосомы, пытаясь его уложить.
Если шепероны не работают, включается автофагия (поедание собственных клеточных структур лизосомами).
Если и это не работает, включаются протеасомы (убиквитинирование $ \Rightarrow $ уничтожение).
Вот если всё это не работает, образуются отложения.

Интересно, что есть болезнь -- гипофизарная карликовость: мутация в гене гормона роста, которая встречается довольно часто.
В 50-е годы придумали способ лечения: больным детям давали вытяжку из гипофиза (умерших людей).
Их перестали использовать в 85-м году, а до того лечили >30 000 человек, и при этом часть заболели болезнью Кройцфельда-Якоба (белок PRP).

Глиальные клетки являются аналогами макрофагов в нервной ткани.
	
\end{document}
