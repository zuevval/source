\documentclass[main.tex]{subfiles}
\begin{document}
	\section{Онкология}
	1 мужчина из 4 и 1 женщина из 5 встречает в своей жизни рак (не всегда это приводит к летальному исходу).
	ВОС отчитывается о росте числа больных раком с каждым следующим годом.
	
	Фирма AbCam производит антитела для лабораторной практики.
	Можно выявлять конкретные этапы заболевания раком.
	
	Большая часть раковых мутаций, как правило, происходят в генах, контролирующих сигнальные пути, клеточную смерть.
	Но не обязательно.
	
	Опухоль (= неоплазия, новообразование).
	Что отличает её от нормальной?
	Происходит некий кризис (какие-то перестройки, мутации), которые приводят к \emph{иммортализации}.
	Напомним, нормальная клетка подчиняется пределу Хейфика (какое-то число делений происходит, затем наступает клеточная смерть).
	
	Первичные опухолевые линии моноклональные (т.е. происходят из одной клетки), что легко доказать по женским клеткам (в таких линиях инактивирована одна и та же X-хромосома).
	
	Первичная клетка делится неограниченно, но это всё ещё не совсем рак: они ещё требуют субстрат, на котором должны расти, способны какое-то время чувствовать контактное ингибирование.
	Но постепенно идёт сдвиг; постепенно они приобретают возможность отрываться от подложки, возможность к метастазированию и потоком крови разносятся по организму. образуя новые опухоли.
	
	Удаление метастазов не приводит к излечению.
	С метастазами борются с помощью жёстких метастатиков; пытаются убрать защиту белков-шеперонов; это жёсткая терапия со многими побочными эффектами.
	Пытаются найти первичную опухоль
	
	Сейчас обычно делаются чипы на несколько типов рака сразу (самые распространённые: колоректальный, простаты или шейки матки / молочной железы).
	
	Напомним, голые землекопы живут долго (примерно 30 лет), но раком не страдают.
	Считается, что у них более тонко устроена регуляция контактным ингибированием.
	Экстраклеточный матрикс, через которые клетки общаются: % TODO a bit
	Мембрана не просто голая торчит в пространство!
	
Пример онкогена: потеря функции белка E-cadherin (Е-кадгерин).
Клетки, находящиеся в глубине, начинают голодать.
Голодание есть стресс, что есть дополнительный толчок к онкогенезу.

Более 50\% раковых клеток имеют мутацию в гене P53.

\subsection{Вирусный рак}

Есть виды рака, вызываемые вирусом: вирус папилломы человека может вызывать рак прямой кишки и многие другие.
Вирус Т-клеточного лейкоза способствует перепроизводству Т-клеток, и, как следствие, Т-клеточный лейкоз.

Вирусы разные, но есть нечно общее: в заражённых вирусом клетках часто экспрессируются особые вирусные белки, которые препятствуют смерти клетки.
Есть ретровирусы, которые прямо встраиваются в наш геном.
Противовирусными препаратами их не вылечить.

\begin{leftbar}
	Противовирусная терапия -- скорее миф.
	Перспективными технологиями в борьбе с вирусами считаются генные технологии (микроРНК и так далее).
	Но здесь нужно научиться доставлять эти РНК в клетку.
	
	В лаборатории перинатальной диагностики в клинике Отто есть разработки различных систем доставки (на базе некоторой давно разработанной платформы).
	
	<<Спутник-5>>: давно зарекомендовавшая себя платформа для доставки в клетки участков аденовируса, применённая к COVID-19.
	
\end{leftbar}

Мутации могут быть разные.
SMN-1, SMN-2.

Раньше, когда не было генно-инженерных технологий, использовалась кислота, которая мешает сплайсингу (впрочем, она мешает сплайсингу не только того гена, который хотим заглушить).
Сейчас разрабатываются различные системы доставки РНК в клетку.

Вакцинация против вируса папилломы (одноразовая), как показали учёные из Швеции, по, снижает вероятность (Arch Gynecol Obstet. 2021).
Вирус свободно ходит в популяции; лучше вакцинироваться до 17 лет, иначе, если вы с ним встречались до вакцинации, эффективность вакцины снижается (вероятность заболеть выше).

\subsection{Сигнальная трансдукция и прото-онкогены}

% TODO a bit

Гены рецепторов -- это прото-онкогены.
Факторы транскрипции, которые участвуют в работе генов пролиферации -- тоже прото-онкогены.

Напомним, в прошлом году мы говорили про пигментную ксиродерму.
Таким людям нельзя выходить на солнце, но, как бы они ни защищались, всё равно рано или поздно они заболевают раком.

В каждом случае есть дополнительные "факторы риска".

Все гены, участвующие в механизмах репарации -- тоже прото-онкогены.

% TODO a bit

Сама по себе сигнальная трансдукция -- не какая-то
Наиболее распространённая регуляция -- эндокринная.
Паракринный сигналинг -- соседние клетки; автокринный сигналинг -- клетка может регулировать сама себя.

<<Поцелуй смерти>>: Т-киллер узнаёт мутантную клетку (или клетку с вирусом), прикасается к ней фас-рецептором (фас-лигандом); Т-киллер идёт дальше, а в клетках запускается процесс программируемой гибели.

\begin{leftbar}
Стресс есть неспецифическая опосредованная реакция кортикостеоридов (как правило, воспалительный процесс в органах и тканях).
Если мы шли и нам отдавили ногу, это ещё не стресс (если не произошла активация кортикостероидного комплекса).

Чрезмерное голодание приводит к стрессу.

Бактериям, которые живут в кишечнике, может не понравиться питание, и они будут давать стрессовый сигнал.

Как передаются бактерии-симбионты от матери к ребёнку?
Сейчас считается, что при проходе через родовые пути.
Поэтому при кесаревом сечении нужна дополнительная 

Не все бактерии выживают.
Есть гипотеза о том, что наш характер частично определяется тем, какие бактерии в нашем кишечнике живут.

Пробиотики вроде кефиров -- по большей части ерунда.
Есть споры в порошке в кишечнорастворимых капсулах, это работает.

<<Витамины не нужны, надо восстанавливать эндосимбионтов>>!
Нужно правильно питаться (кефиры, в меру жирного) -- этому бактерии рады.
Главное -- разнообразие.
Однообразное питание всегда плохо.

\end{leftbar}

Действие сигнала всегда можно прекратить, поскольку у нас есть фосфатазы.

\end{document}