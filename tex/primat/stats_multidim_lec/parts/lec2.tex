\documentclass[main.tex]{subfiles}
\begin{document}

\section{Лекция 2. Проверка гипотезы однородности}

9 февраля 2021 года

Проверка однородности используется, к примеру, при оценке качества кластеризации.

Пусть проведено $ k $ серий независимых наблюдений, каждая серия представлена выборкой, $ n_j $ -- объём выборки $j$, $ x_{ij} $ -- наблюдение $i$ в выборке $j$; $F_j(t)$, $\hat \mu_j$.

\emph{Однородность} выборки $ \rightleftarrows $ все выборки в серии взяты из одного распределения.

В многомерном случае критерии однородности выработаны только для нормального распределения, поэтому говорят % TODO в случае нормального или в остальных случаях

\subsubsection{Критерий однородности Смирнова}

Является продолжением идеи критерия Колмогорова.
Предположения: функции распределения двух выборок непрерывны.

Воспользуемся статистикой $ D_{n,m}(X,Y) = \sup_t|F_{1n}(t)-F_{2m}(t)|, t \in \mathds{R}^1 $, $ F_{1n}(t) $ -- э. ф. р., построенная по выборке $ X $, $ F_{2m} $ -- э. ф. р., построенная по выборке $ Y $.

Если объёмы выборки стремятся к бесконечности, значение случайной величины $ t = ... $ % TODO
аппроксимируется распределением Колмогорова.

\textbf{Замечание:} в прошлый раз функция распределения Колмогорова была написана не совсем верно. На самом деле $ $ % TODO

Вероятность ошибки первого рода (попадание в критическую область при справедливости нулевой гипотезы) аппроксимируется значением $ 1 - K(t_\alpha) \Rightarrow $ критическая область равна $ t_{1-\alpha} $ -- $ \alpha $-квантиль распределения Колмогорова.

\subsubsection{Критерий однородности Якобсона}

Основан на критерии Манна-Уитни.
В этом критерии тоже требуется непрерывность, \textbf{но} есть более сложные критерии, основанные на той же статистике Манна-Уитни, которые обходятся без непрерывности.

Итак, проверяем основную гипотезу: $ F = G $.
Проверим против двусторонней альтернативы: $ F > G or F < G $

Будем считать, что в реализациях выборок нет повторяющихся элементов.

Согласно центральной предельной теореме $ U_Z = \frac{U-\frac{nm}{2}}{\sqrt{\frac{nm(n+m+1)}{12}}} | H_0 \xrightarrow{n \to \infty} \mathrm{N}(0,1) $ (то есть функция распределения есть функция Лапласа $ \Phi $)

Критическая область: поскольку критерий двусторонний, $ \mathcal{\Gamma} = \{\} $ % TODO

\subsubsection{Критерий Краскела-Уоллиса}

Предназначен для проверки гипотезы против \emph{произвольной} альтернативы (как и критерий Смирнова).
Есть модификации критерия Краскела-Уолииса для проверки односторонней альтернативой, но мы их не будем рассматривать.

Это ранговый критерий $ \to $ чувствителен к матожиданиям. Если матожидания совпадают, критерий может оказаться нечувствительным к неоднородности.
Таким образом, лучше использовать при проверке однородности несколько критериев. % TODO

\begin{leftbar}
	
\end{leftbar}

\subsubsection{t-критерий Стьюдента}

Это критерий устойчивый, но % TODO

Предполагается, что дисперсии неизвестны, но равны между собой.
В этом случае Стьюдентом доказано, что случайная величина подчиняется закону Стьюдента $ S(n+m-2) $. Здесь $ S(X), S(Y) $ -- среднеквадратичные отклонения

Напомним, распределение Стьюдента -- симметричное относительно нуля; нас интересует отклонение от нуля в обе стороны, т. о. критерий двусторонний.
Критическое значение $ t_\alpha = t_{1 - \alpha/2, S(n+m-2)} $ -- $ \alpha $-квантиль распределения Стьюдента.

Обобщение критерия Стьюдента на случай $ k > 2 $.
Пусть есть несколько выборок, пришедших из распределений с одинаковыми дисперсиями, и нужно проверить их принадлежность одной \textbf{нормальной} генеральной совокупности.

Статистика критерия есть отношение суммы внутригрупповых выборочных дисперсий к межгрупповой выборочной дисперсии.

Показано, что в условиях справедливости нулевой гипотезы статистика распределена по закону Фишера (тот же закон называют иногда распределением Снодекора и обозначают его функцию вероятности, как и Стьюдента, $S$, но это двухпараметрический закон, не путать); $n$ здесь -- суммарное число элементов в объёдинённой выборке.

\subsubsection{Критерий дисперсионного отношения (F-отношение)}

Статистика -- т. н. \emph{отношение Фишера}, оно же F-отношение. Число выборок $ k = 2 $.

$ S $ -- средневадратичное отклонение, $ S^2 $ -- дисперсия.

Критическая область -- и большие, и малые значения статистики.

\subsubsection{критерий Бартлетта}

Идеологическое продолжение F-отношения. Позволяет проверить равенство дисперсий при числе выборок $ k > 2 $.

Для обоснованного применения критерия требуется, чтобы минимальный объём выборки в семействе $ \min_j n_j > 3 $.

\subsection{Критерии однородности многомерных нормальных совокупностей}

Говорим о многомерном \emph{нормальном} распределении. $ H_0: F_1(t)=F_2(t)=...F_k(t), t \in \mathds{R}^p $.

Многомерное нормальное распределение полностью характеризуется матожиданием $\vec \mu_j$ и матрицей ковариации $ \Sigma_j $ ($ \in \{1,k\}$ -- номер выборки). Их выборочные оценки обозначим $ \hat \mu_j, \hat \Sigma_j = S^{(j)} $

\begin{leftbar}
	жирными символами в тексте ЛВ обозначены векторы, обычными -- скаляры.
\end{leftbar}

Пусть $ x_i^{(j)} \in \mathds{R}^p $ -- $ i $-й элемент в $ j $-й выборке. Будем обозначать $ x_i^{(j)} = \{x_{i1}^{(j)}, ..., x_{ip}^{(j)}\}^T $

\subsubsection{Проверка гипотезы равенства матожиданий двух выборок}

Необходимо условие равенства ковариационных матриц.
Проверяется равенство матожиданий.

Общую ковариационую матрицу обозначим $ \hat \Sigma = S, S^{(j)} $ -- ковариационные матрицы для каждой выборки.

Статистика $ T^2 $ определяется через выборочные характеристики.
Для проверки гипотезы используется статистика $t$, основанная на $ T^2 $; доказано, что в условиях справедливости нулевой гипотезы последняя имеет распределение Фишера.

В пользу справедливости $H_0$ говорят малые значения статистики $ t $, критические значения -- большие.

\subsubsection{Лямбда-статистика Уилкса}

Оцениваем равенство матожиданий при условии одинаковых ковариаций в выборках.

Если значение статистики  $ \Lambda \ll 1 $, гипотеза отвергается.

Оказывается, что распределение статистики $ \Lambda $ очень сложное, и на практике пользуются функциями от критерия $ \Lambda $.

\subsubsection{Статистика, предложенная Бартлеттом}

$$ B(\Lambda) = ... $$ % TODO

Показано, что эта статистика аппроксимируется распределением хи-квадрат с $ p $ степенями свободы.

Критические значения -- большие.

\subsubsection{статистика, предложенная Рао (при малых объёмах выборки)}

Не очень интересна в биоинформатике, где обычно большие объёмы данных.

$ \tilde n, n_\Lambda $ не обязательно целые.

Для статистики Бартлетта и Рао критическая область статистики -- большие значения.
Но как неправдоподобно большие, так и малые значения статистик могут возникать в эксперименте из-за нарушения условия равенства ковариационных матриц.
Поэтому нужна также статистика для проверки гипотезы о равенстве ковариационных матриц.

\subsubsection{Лямбда-статистика}

Проверяем гипотезу совпадения ковариационных матриц: $ H_0: \Sigma_j = \Sigma, j \in \{1;k\} $

Матожидания при этом не обязаны совпадать.

\end{document}