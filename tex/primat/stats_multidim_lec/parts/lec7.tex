\documentclass[main.tex]{subfiles}
\begin{document}

\section{Лекция 7}
23 марта 2021 г.

\subsection{Замечание к прошлой лекции}

Относительно критерия отношения правдоподобия для линейной регрессии:
Он применим для регрессии с
Для  k=1 это эффективный (т. е. равномерно более мощный несмещённый) критерий.
Также доказано, что, если проверяется общая гипотеза ($ k >= 1 $), равномерно более мощный критерий вовсе не существует.
Но есть оптимальные в узком смысле критерии.

\subsection{Доверительное оценивание в линейной регрессии}

\subsubsection{Доверительное оценивание. Введение}

Построение доверительных интервалов и областей тесно связано с проверками простых гипотез.
Даже есть теорема, связывающая проверку гипотез с построением доверительных оценок.

Обозначим за $ D(z) $ доверительное множество (в частном случае доверительный интервал), оценивающее

Пусть $ \alpha $ -- истинное значение параметра, $ \beta $ -- ошибочное.
Введём $ P_\alpha \{ \beta \in D(z) \} $ -- вероятность ошибочного оценивания.

Говорят, что $ D(z) $ -- \emph{несмещённое доверительное множество}, если для любых $ \alpha, \beta \in \Theta $ вероятность ошибочного оценивания не превосходит вероятности $ P_\alpha \{ \alpha \in D(z) \} $.

Доверительное множество $ D(z) $ \emph{точнее} другого доверительного множества $ S(z) $, если

\subsubsection{Построение индивидуальных доверительных интервалов для параметров регрессии}

Теорема 2 (выделена серым цветом в тексте лекции) была на второй лекции по.

Из этой теоремы следует, что центрированные и нормированные коэффициенты линейной регрессии распределены по стандартному нормальному закону.

% TODO a bit
... величина распределена по закону хи-квадрат $ \Rightarrow $ можно построить статистику Стьюдента $ \Rightarrow $ можно построить доверительный интервал для параметра $ \alpha $!

Статистика попадает в заданный интервал с вероятностью $ \gamma $; здесь $ t_{\frac{1-\gamma}{2}, S(n-m)} $ и $ t_{\frac{1+\gamma}{2}, S(n-m)} $ есть $ \frac{1-\gamma}{2} $ и $ \frac{1+\gamma}{2} $-квантили распределения Стьюдента.

Построение доверительного интервала $ D_i $ тесно связанного с проверкой простой гипотезы $ H_0: \alpha_i = \beta $.
В самом деле, область принятия гипотезы $ \mathcal{T} = \{ t: t = \frac{|\beta - a_i|}{s_i} \} $

Вероятность принятия гипотезы есть вероятность попадания $ \beta $ в доверительный интервал.

$ \gamma $-доверительный интервал для параметра $ \sigma $: строим примерно похожим образом, но надо учитывать, что $ \chi^2 $-распределение не является симметричным, как распределение Стьюдента.
\textbf{Пожалуйста, сделайте сами}.


\subsubsection{} % TODO title

Совместная доверительная область для вектора $ \vec \alpha $ -- эллипсоид.

\begin{theorem}
    Равномерно более мощный несмещённый критерий проверки простой гипотезы с уровнем значимости $ \alpha $ приводит к тому, что доверительное множество $ D(z) $ с доверительный вероятностью $ 1 - \alpha $ является равномерно наиболее точным, и наоборот.
\end{theorem}

\subsubsection{Построение совместной доверительной области в форме обобщённого прямоугольного параллелепипеда}

Матрица $ X^TX $, как правило, плохо обусловлена, и эллипсоид получается вытянутым
На практике вместо построения совместной доверительной области в виде эллипсоида пользуются \textbf{принципом Тьюки} и строят область в виде прямоугольного параллелепипеда.

\begin{theorem}
\textbf{Тьюки (Turkey)}: индивидуальные доверительные интервалы для параметров линейной регрессии $ \alpha_i $ с доверительными вероятностями $ 1 - \frac{\alpha}{m} $ будут совместными с вероятностью не менее $ 1 - \alpha $.

\end{theorem}

% TODO ещё одна подсекция, замечание: выбрать tau в своей работе ЛР3!

\subsection{Прогноз по регрессии}

Пусть есть регрессия $ y_t = \alpha_1 x_{t1} + \alpha_2 x_{t2} + ... $.

Пусть на вход поступает новая величина $ x_\tau = (x_{1\tau}, x_{2\tau}, ...)6T $.
Хотим выдать значение $ y_\tau $.
Поскольку $ y_\tau $ -- случайная величина, говорят не об оценке, а о \emph{прогнозе} величины.

В качестве прогноза мы берём безусловное матожидание
$$ \hat y_\tau: M \hat y_\tau = M y_\tau, My_\tau = \sum $$

\begin{theorem}
    Прогноз $ \hat y_\tau $ является несмещённым и эффективным в классе несмещённых прогнозов, причём его дисперсия

    $$ \sigma_\tau^2 = \sigma^2[x_\tau(X^TX)^{-1}x_\tau + 1] $$
\end{theorem}

Настоящая дисперсия -- матожидание квадрата отклонения от матожидания случайной величины.

"Дисперсия" прогноза $ \hat y_\tau $ определяется как
$$ \sigma_tau^2 = M[(\tilde y_\tau - y_\tau)^2] $$
На самом деле это дисперсия $ \tilde y $, поэтому $ \tilde y $ должна быть несмещённой оценкой.

Находим $ c $, построив лагранжиан по целевой функции и ограничениям...

Оптимальный линейный по $ y $ несмещённый прогноз $ \tilde y_\tau^{opt} $ совпадает с МНК-прогнозом $ \hat y_\tau $.

\subsubsection{Замечания}

\begin{enumerate}[noitemsep]
    \item Структура модели не меняется
    \item Если говорить не о дисперсии прогноза, а о дисперсии случайной величины $ \tilde y_\tau $, дисперсия изменится: не будет единицы.
    \item Формулы для дисперсии не очень пригодны для практики, т. к. содержат неизвестный параметр $ \sigma^2 $.
    На практике используют несмещённую оценку $ s_\tau^2 = \hat \sigma_\tau^2 $.
    \item Разумно делать не точечный, а интервальный закон.
\end{enumerate}

\subsection{Замечание к работе №3 и №2}

В третьем задании, может быть, модель можно редуцировать (упростить: отбросить какие-то признаки).
Но после этого нужно оценить дисперсию: не будет ли она сильно возрастать?

Во второй работе:
\begin{itemize}[noitemsep]
    \item нужно работать \textbf{с оценками} параметров распределения, а не с исходными параметрами!
    \item В отчётах очень скудно представлена теоретическая часть.
    На основании чего вы отбираете главные компоненты?
    Какую долю дисперсии воспроизводят эти компоненты?
    \item С какими данными Вы работаете? Вы их центрируете, нормируете? (если нормировать, то метод главных компонент может и не работать хорошо).
    Приведите эмпирические вероятности ошибочной классификации (по таблице сопряжённости), например, обозначив их $ \hat p $ (оценки).
\end{itemize}

\end{document}
