% compile with XeLaTeX or LuaLaTeX

\documentclass[a4paper,12pt]{report}
\usepackage[utf8]{inputenc} % russian, do not change
\usepackage[T2A, T1]{fontenc} % russian, do not change
\usepackage[english, russian]{babel} % russian, do not change

% fonts
\usepackage{fontspec} % different fonts
\setmainfont{Times New Roman}
\usepackage{setspace,amsmath}
\usepackage{amssymb} %common math symbols
\usepackage{dsfont}
\usepackage{ mathrsfs } % curved letters with e. g. \mathscr{F}

% utilities
\usepackage{systeme} % systems of equations
\usepackage{mathtools} % xRightarrow, xrightleftharpoons, etc
\usepackage{array} % utils for tables
\usepackage{makecell} % multirow for tables
\usepackage{subfiles}
\usepackage{hyperref}
\hypersetup{pdfstartview=FitH,  linkcolor=blue, urlcolor=blue, colorlinks=true}
\usepackage{framed} % advanced frames, boxes
\usepackage{graphicx}
\usepackage{caption}
\usepackage{subcaption} % captions for subfigures
\usepackage{color}
\usepackage{chngcntr} % change counters
\counterwithout*{section}{chapter} % continue sections enumeration with chngcntr


% styling
\usepackage{float} % force pictures position
\floatstyle{plaintop} % force caption on top
\usepackage{enumitem} % itemize and enumerate with [noitemsep]
\setlength{\parindent}{0pt} % no indents!

% misc
\graphicspath{{./img/}}
\newcommand{\myPictWidth}{.95\textwidth}
\newcommand{\phm}{\phantom{-}}

\begin{document}
	\subfile{titlepage}
    \tableofcontents
    \newpage

    Павлова Людмила Владимировна (\href{mailto:lyu0510@gmail.com}{lyu0510@gmail.com})

    Зачёт с оценкой + экзамен. План лекций можно рассматривать практически как список вопросов.

    Отчёты по лабораторным: сдавать не больше трёх раз!
    Лучше с первого раза, но после третьей итерации уже всё, больше сдавать нельзя.
    Язык программирования: любой.
    ЛВ работает в MATLAB, можно Python/R.

    Форма не очень строгая: постановка задачи, описание алгоритма, (какой-то пункт о том, что и как мы делаем - как готовим данные, как их обрабатываем), результаты, анализ результатов; в приложении обязательно код программ.
    Код можно вставлять не целиком, а только важные кусочки.

    Как нужно делать отчёты: ЛВ пришлёт файл в Teams.
    В первой работе лучше перебрать несколько уровней значимости, посмотреть на $ p-value $ .
    Вторая работа объёмная: первые данные у всех одни (о пчёлах), вторые (попроще) можно найти самим или взять \texttt{wine} -- там три класса.
    Сначала работаем с исходными данными, задем переходим к главным компонентам и работаем там.
    Третья работа: данные будут одни, а шум у всех разный.

    Первую работу можно начать делать сразу.

    Отчёты делать в Word\texttrademark.
    Можно (если очень хочется) окончательный вариант делать в \LaTeX.

    \textbf{Литература}: мы придерживаемся фишеровской статистики (в отличие от Георгия Леонидовича).
    \begin{enumerate}[noitemsep]
        \item Ивченко, Медведев: Математическая статистика (любое издание). \emph{Материал этой книжки пройдём на первой паре лекций}.
        \item Тюрин, Микиров: Анализ данных на компьютере (2016).
        \item Фундаментальный труд: Айвазян, Енюков, Мешалкин: Прикладная статистика.
        \emph{Мы пользуемся в основном Ч. 1., Основы моделирования и первичная обработка данных.}
        \item Айвазян, Мхитарян: Прикладная статистика и основы эконометрики (2001) \emph{Если хочется, посмотрите книжечку (тоже фундаментальный труд).} \label{item:aivazyan_mhitaryan}
        \item Афифи, Эйзен: Статистический анализ. Подход с использованием ЭВМ (1982). \emph{Устаревшая книжка.}
        \item Линейная и нелинейная регрессия (любое издание) \emph{-- хорошо описана регрессия}.
    \end{enumerate}

    \href{https://drive.google.com/drive/folders/18WM37ML5G8PERAv0esU2HV9sqEgyejj0?usp=sharing}{Литература на Google Drive} (кроме Тюрина и Афифи, а также \ref{item:aivazyan_mhitaryan} нашёл только в издании 1989 года).

    \chapter{Статистические гипотезы}
    \subfile{parts/lec1}
    \subfile{parts/lec2}
\end{document}
