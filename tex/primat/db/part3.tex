\documentclass[zuevDbReport.tex]{subfiles}
\begin{document}
\newpage
\section{Выводы по значениям времени исполнения запросов}
\begin{table}[H]
\begin{center}
\caption{\label{resTab}Зависимость усреднённого времени выполнения запросов (мсек) от размера таблицы}
%не решена проблема с отступом по вертикали в ячейках. Решение из школьного отчёта по практике не работает (из-за XeLaTeX?)
\begin{tabular}{| p{.35\textwidth} |p{.2\textwidth}|p{.2\textwidth}|p{.2\textwidth}|}
\hline
\textbf{Запрос}&\textbf{T($\mathbf{1000}$ строк)}& \textbf{T($\mathbf{10,000}$ строк)}&\textbf{T($\mathbf{100,000}$ строк)}\\
\hline
Поиск по ключевому полю&$0.07547$&$0.07561$&$0.07990$\\
\hline
Поиск по не ключевому полю&$0.07570$&$0.07978$&$0.13953$\\
\hline
Поиск по маске&$0.07702$&$0.08116$&$0.13593$\\
\hline
Добавление записи&$0.17165$&$0.16466$&$0.16726$\\
\hline
Добавление группы записей&$0.19083$&$0.19797$&$0.19645$\\
\hline
Изменение записи (определение записи по ключевому полю)&$0.19886$&$0.20015$&$0.21307$\\
\hline
Изменение записи (определение изменяемой записи не по ключевому полю)&$0.19655$&$0.23831$&$0.22266$\\
\hline
Удаление записи (определение удаляемой записи по ключевому полю)&$0.19604$&$0.17950$&$0.18609$\\
\hline
Удаление записи (определение удаляемой записи не по ключевому полю)&$0.19705$&$0.19907$&$0.22236$\\
\hline
Удаление группы записей&$0.19987$&$0.19896$&$0.21387$\\
\hline
Сжатие базы данных после удаления 200 строк&$0.02205$&$0.01844$&$0.01910$\\
\hline
Сжатие базы данных после удаления, в результате которого остаётся 200 строк&$0.02245$&$0.02032$&$0.01741$\\
\hline
\end{tabular}
\end{center}
\end{table}
В табл. \ref{resTab} приведены значения, усреднённые по 100 запросам (за исключением операции оптимизации таблицы). Пример программы расчёта времени приведён в приложении (\ref{QTimeCode})\\
\newpage
По полученным данным сделаны два вывода:
\begin{enumerate}
\item{}Время поиска по ключевому полю (и изменения записи с таким поиском) медленнее растёт в зависимости от числа записей, чем при использовании поиска по неключевому полю. Объяснение таково: значения ключевого поля в памяти базы упорядочены в виде двоичного дерева, поэтому поиск по дереву в хорошем случае: $T(n) \in O(log(n))$. Поиск по неключевому полю: $T(n) \in O(n)$.
\item{}В большинстве операций время выполнения растёт с увеличением числа записей в таблице. В некоторых же (удаление записи по ключевому полю, сжатие базы данных) быстрее обрабатываются большие таблицы. Можно предположить, что иногда на полученный результат влияют процессы, параллельно происходящие в системе, поскольку разница во времени выполнения операций мала и может быть чувствительна к условиям. Также при сжатии базы данных, возможно, требуется временное выделение дополнительной памяти; если сжимается большая таблица, по предположению используя <<мусорная>> память, уже выделенная для этой таблицы, и не тратится время на аллокацию.
\end{enumerate}

\newpage
\section{Приложения}
\subsection{Пример кода программы, рассчитывающей усреднённое время запроса}
\label{QTimeCode}
Приведён пример расчёта времени запроса выборки элементов с поиском по маске. Используется предварительно написанная функция dbQuery(query\_string), исполняющая заданный запрос.\\
Функция maskQuery(lastDigits, n) ищет в таблице с заданным числом тысяч строк (1, 10 или 100) преподавателей, чьи фамилии оканчиваются на lastDigits.
Функция measure(n) в цикле от 0 до 100 запрашивает данные об учителях с разными фамилиями и выводит на экран усреднённое время, вычисленное с помощью time.perf\_counter.\\
\lstset{language=Python,keywordstyle=\color{blue},tabsize=2,breaklines=true}
\begin{lstlisting}
from dbfunctions import dbQuery;
import time;

def maskQuery(lastDigits, n):
    query_string = '{}{}{}{}{}'.format("\
SELECT * \
FROM teachers",n,"k.teachers \
WHERE t_surn LIKE '%",lastDigits,"';")
    query_rows_list = dbQuery(query_string);
    return query_rows_list;

def measure (n):
    maskQueryTime = 0;
    for i in range(100):
        start = time.perf_counter()
        maskQuery(i*n, n);
        stop = time.perf_counter()
        maskQueryTime += stop - start;
    print ("not key field select query with mask: time = ")
    print( maskQueryTime/100);
\end{lstlisting}
\subsection{Код программы, исполняющей запрос к базе данных}
\label{qCode}
Использована библиотека MySQL Connector Python для соединения с базой данных и библиотека configparser для чтения реквизитов базы данных.\\
Функция read\_db\_config принимает на вход два необязательных аргрумента - имя конфигурационного файла и тип базы данных (по умолчанию config.ini и mysql соответственно) и возвращает список параметров, содержащихся в файле.
Функция dbQuery принимает на вход строку (в строке может быть несколько команд) и соединяется с базой данных. Создаётся курсор, выполняется запрос; после этого в цикле из курсора извлекаются и выводятся на экран результаты запроса. Обработка ошибок для краткости опущена.\\
\begin{lstlisting}
import mysql.connector
from mysql.connector import MySQLConnection, Error
from configparser import ConfigParser

def read_db_config(filename='config.ini', section='mysql'):
    parser = ConfigParser()
    parser.read(filename)
    db = {}
    if parser.has_section(section):
        items = parser.items(section)
        for item in items:
            db[item[0]] = item[1]
    return db

def dbQuery(query_string):
    query_rows = [];
    try:
        dbconfig = read_db_config()
        conn = MySQLConnection(**dbconfig)
        cursor = conn.cursor(buffered=True)
        results = cursor.execute(query_string, multi=True)
        for cur in results:
            print('cursor:', cur)
            if cur.with_rows:
                print('result:', cur.fetchall())
        conn.commit()

    finally:
        cursor.close()
        conn.close()
\end{lstlisting}
\end{document}