% compile with XeLaTeX or LuaLaTeX

\documentclass[a4paper,12pt]{article} %14pt - extarticle
\usepackage[utf8]{inputenc} % russian, do not change
\usepackage[T2A, T1]{fontenc} % russian, do not change
\usepackage[english, russian]{babel} % russian, do not change

% fonts
\usepackage{fontspec} % different fonts
\setmainfont{Times New Roman}
\usepackage{setspace,amsmath}
\usepackage{amssymb} %common math symbols
\usepackage{dsfont}

\begin{document}

\section{63. Алгоритм Витерби для парных HMM}

Начало: \\
$v^M(0,0)=1$. Все остальные $v^\centerdot(i,0), v^\centerdot(0,j)$ полагаются равными $0$. \\
Рекурсия: $i=1, ..., n, j=1, ..., m$;

\begin{multline*}
	\cr	v^M(i,j) = p_{x_i,y_j} \max \left\{
	\begin{array}{ll}
		(1 - 2 \delta - \tau) \cdot v^M(i-1,j-1)\\
		(1 - \varepsilon - \tau) \cdot v^X(i-1,j-1)\\
		(1 - \varepsilon - \tau) \cdot v^Y(i-1,j-1)
	\end{array}
	\right. \\
	\cr	v^X(i,j) = q_{x_i} \max \left\{
	\begin{array}{ll}
		\delta \cdot v^M(i-1,j)\\
		\varepsilon \cdot v^X(i-1,j)
	\end{array}
	\right. \\
	\cr	v^Y(i,j) = q_{y_j} \max \left\{
	\begin{array}{ll}
		\delta \cdot v^M(i,j-1)\\
		\varepsilon \cdot v^Y(i,j-1)
	\end{array}
	\right. \\
\end{multline*}

Завершение:
$$v^\varepsilon = \tau \cdot max(v^M(n,m), v^X(n,m), v^Y(n,m))$$
\end{document}