\documentclass[main.tex]{sufbfiles}
\begin{document}
\section{Анализ данных типа времени жизни}
Sept 02, 2022

Список литературы: на слайде \\

Лекции: \href{https://sites.google.com/site/malovsergvas/%D0%BC%D0%B5%D1%82%D0%BE%D0%B4%D0%B8%D1%87%D0%B5%D1%81%D0%BA%D0%B8%D0%B5-%D0%BC%D0%B0%D1%82%D0%B5%D1%80%D0%B8%D0%B0%D0%BB%D1%8B/%D0%B8%D0%B7%D0%B1%D1%80%D0%B0%D0%BD%D0%BD%D1%8B%D0%B5-%D0%BB%D0%B5%D0%BA%D1%86%D0%B8%D0%B8}{здесь}

Каждое наблюдение -- измерение времени от некой стартовой точки до некоего события.

Время жизни $ T $  конечно, $ T \in \mathds R_+ = [0; + \infty) $.
Считается, что событие рано или поздно непременно произойдёт.

Пример исследования: начальная точка -- момент обнаружения рака в 1 стадии, событие -- переход во 2 стадию.

Довольно часто искомого события не удаётся дождаться.
Такие наблюдения нельзя выкинуть из рассмотрения (возникает сдвиг оценок).
Наиболее распространённая модель в таком случае -- модель с цензурированием справа.

Распределение
\[ F(t) = \mathds P (T \le t), t > 0 \]
\[ p(t) = F'(t) \]

\emph{Интенсивность отказа} (здесь $ F(x_-) $ -- предел слева; напомним, в нашем определении $ F(t) $ непрерывна справа и имеет предел слева).

\[ \lambda(t) = \frac{p(t)}{ 1 - F(t_-) } = \frac{p(t)}{\mathds P (T \ge x)}  \]

\emph{Функция отказа} $ S(t) = 1 - F(t) $

\emph{Интенсивность отказа} величины $ T $ по отношению к доминирующей мере $ \mu $

\[ TODO \]

\emph{Накопленная интенсивность}

\[ \Lambda(t) = \int_0^t \lambda(u) d \mu(u), t \in \mathds R \]

\subsubsection{ Некоторые связи между введёнными характеристиками }

В случае абсолютно непрерывного распределения $ T $ в качестве меры $ \mu $ используется мера Лебега, поэтому

\[ \lambda(t) = - \frac{d \ln S(t)}{dt} \Rightarrow S(t) = \exp (- \Lambda t) \]

\subsection{Типы неполных данных типа времени жизни}

\subsubsection{Модели неполных данных}

В середине 70х предложены описания моделей неполных данных с использованием тех или иных преобразований.

Статистический эксперимент -- множество результатов (или статистическая информация -- НЕ информация Фишера).

Пропущенные наблюдения \emph{потеряны совершенно случайно}, если распределение набора пропущенных значений равномерное на множестве всех подмножеств индексов наблюдений и не зависит от значения и распределения случайной величины.

Наблюдения \emph{ потеряны случайно }, если те же условия выполнены БЕЗ требования независимости от распределения.

\subsubsection{Модель усечения}

Усечение -- потеря части данных ввиду нарушения специфических условий наблюдаемости.
При этом сам факт наличия эксперимента остаётся неизвестным.

Усечение в этом курсе мы как следует не рассматриваем.
Информация скорее дана для общего развития.

Восстановление -- ключевой момент.

Усечение может быть случайным, поэтому в модель эксперимента с усечением кроме исходного пространства эксперимента добавлены условия усечения.

\subsection{ Модель цензурирования }

Цензурирующее отображение -- это измеримое отображение $ \Theta :  (\mathfrak X, \mathfrak F) \to (\mathcal Y, \mathcal C) $ 

\begin{leftbar}
	Напомним, $ \mathfrak X $ -- множество возможных исходов эксперимента, $ \mathfrak F $ -- множество результатов наблюдений (множество множеств полученных данных)
\end{leftbar}

Модель цензурирования справа

Более сложное, обобщённое -- двухстороннее цензурирование (но оно практически не используется).
В этом случае подразумевается наличие минимального времени отказа $ L $ и максимального $ U $.
Данные -- либо $ T, T \in [L; U] $; либо известно, что $ T \le L $; либо известно, что $ T > U $.

Есть цензурирование слева (но С.В. такое вообще не включал).

Есть альтернативный вариант двухстороннего цензурирования: % TODO не очень понятно

\subsubsection{ Интервальное цензурирование }

Информация о времени отказа никогда не бывает точной.
Подразумевается, что мы знаем лишь интервалы, в которые событие произошло или не произошло.
Это, конечно, серьёзно обедняет модель, но зато часто более соответствует данным.

Состоятельной оценки непараметрической функции распределения при фиксированных временных интервалах не существует.
Если времена случайны, то существует.

Более простой с теоретической точки зрения случай -- когда есть всего одна точка наблюдения случайной величины (Case 1 interval censoring or Current status data).

Есть модели усечения слева и цензурирования справа (кто-то не дошёл до обследования, но из тех, кто дошёл, не у всех дождались, когда наступило время отказа).

\end{document}