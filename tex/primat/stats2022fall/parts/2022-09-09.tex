\documentclass[main.tex]{subfiles}
\begin{document}

\section{ Общие подходы к цензурированным данным }

Sept 09, 2022 \\

Напомним, распространённые модели -- цензурирование справа, интервальное цензурирование.
Также говорили об усечении справа и общих моделях цензурирования. \\

Общий метод при работе с цензурированными данными -- ММП.
Вопрос -- как будет выглядеть функция правдоподобия?

Используем классический подход: семейство распределений доминировано некоторой мерой, поэтому по теореме Радона-Никодима при всяком значении параметра определена плотность $ p_{\Theta} = \frac{d P_{\Theta}}{d \mu} $

Оценки МП позволяют строить асимптотические методы (но точные методы редко удаётся построить). \\

Оценка МП $ \hat \Theta(X)  $, соответствующая функции правдоподобия $ L(X, \Theta) $ определяется так: $ L(X, \hat \Theta(X)) \ge L(X, \tilde \Theta(X)) \forall X, \tilde \Theta (X) $.

\begin{itemize}[noitemsep]
	\item $ \hat \Theta $ -- ОМП $ \Rightarrow g(\hat \Theta) $ -- ОМП для $ g(\Theta) $.
	\item Если $ X_1, \dots, X_n $ -- выборка из распределения, то $ L(\Theta, X) $ распадается в произведение. 
\end{itemize}

Находим ОМП, как обычно, взяв логарифм и приравняв его производную к нулю (если функция дифференцируема).
Но в большинстве случаев семейство распределений выбирается так, чтобы ф-я была дифференцируема.

Впрочем, есть обобщение ММП, позволяющее получать оценки даже в случае невыполнения условия доминирования.

\subsection{ Обобщённые оценки максимального правдоподобия }

% TODO a bit

Дискретные распределения имеют приоритет в смысле обобщённого правдоподобия (согласно Киеферу и Вольфовицу).

Если есть два сингулярных распределения и мы представляем нашу в виде их выпуклой комбинации,

% TODO some

% slide 29
В непараметрическом случае: выбираем не функцию правдоподобия, а относительное попарное правдоподобие.

$ \mathbb P_\Theta^r (A) $ -- сужение на множество наблюдаемых событий.

Можно искать такую оценку как оценку МП исходя из событий или из наблюдений $ y $ (второе, конечно, удобнее).


\end{document}