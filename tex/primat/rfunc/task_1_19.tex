% compile with XeLaTeX or LuaLaTeX

\documentclass[a4paper,14pt]{extarticle}
\usepackage[utf8]{inputenc} % russian, do not change
\usepackage[T2A, T1]{fontenc} % russian, do not change
\usepackage[english, russian]{babel} % russian, do not change

% fonts
\usepackage{fontspec} % different fonts
\setmainfont{Times New Roman}
\usepackage{setspace,amsmath}
\usepackage{amssymb} %common math symbols
\usepackage{dsfont}

% utilities
\usepackage{pdfpages} % insert pdf pages, e.g. scans
\usepackage{systeme} % systems of equations
\usepackage{mathtools} % xRightarrow, xrightleftharpoons, etc
\usepackage{hyperref}
\hypersetup{pdfstartview=FitH,  linkcolor=blue, urlcolor=blue, colorlinks=true}
\usepackage{framed} % advanced frames, boxes
\usepackage{mdframed}
\usepackage{graphicx}
\usepackage{color}

% styling
\usepackage{float} % force pictures position
\floatstyle{plaintop} % force caption on top
\usepackage{enumitem} % itemize and enumerate with [noitemsep]
\setlength{\parindent}{0pt} % no indents!
\usepackage[left=10mm, top=20mm, right=10mm, bottom=20mm, nohead, footskip=10mm]{geometry}

% misc
\graphicspath{{./img/}}
\newcommand{\myPictWidth}{.7\textwidth}
\newcommand{\phm}{\phantom{-}}
\newcommand{\defeq}{\overset{def}=}

\begin{document}

\title{Задача I.19}
\author{Валерий Зуев}
\maketitle

\section{Постановка}

Пусть $U(t)$ -- процесс Уленбека-Орнштейна, т. е. стационарный нормальный центрированный с корреляционной функцией
\[ K_u(\tau) = \sigma^2 e^{-\alpha |\tau|} \]
Рассмотрим независимые процессы $X$, $Y$, определяемые уравнениями
\[ \begin{cases}
    \dot X + \beta X = U, X(0) = X_0 \sim \mathcal{N}(0, \sigma) \\
    \dot Y + \gamma Y = U, Y(0) = Y_0 \sim \mathcal{N}(0, \sigma)
\end{cases} \]
Найти аналитически среднеквадратическое отклонение
\[ \Delta(t) = \sqrt{M\left[ (X(t)-Y(t))^2 \right]} \]
и построить график при $ \sigma=1 $, $ \alpha=1 $, $ \beta=1 $, $ \gamma=2 $, $ t \in [0;10] $.

\section{Аналитическое решение}

\begin{align*}
    M\left[ (X(t)-Y(t))^2 \right] & = M\left[ X^2(t) + 2X(t)Y(t) + Y^2(t) \right] \\
    &= M[X^2] - \bar x^2 + \bar x^2 - 2M[XY] + 2\bar x \bar y - 2\bar x \bar y + M[Y^2] - \bar y^2 + \bar y^2 \\
    &= D[X] + \bar x^2 + R_{xy}(t,t) - 2\bar x \bar y + D[Y] + \bar y^2 \\
\end{align*}

Найдём $ D[X] $, $ D[Y] $, $ R_{xy}(t,t) $, $\bar x$ и $ \bar y $ в стационарном режиме, используя спектральную теорию.
\[ \begin{cases}
    \bar x = \frac{\bar u}{\beta} = 0 \\
    \bar y = \frac{\bar u}{\gamma} = 0 \\
    D[X] = K_x(0) = \int_{-\infty}^\infty S_x(\omega) d\omega \\
    D[Y] = K_y(0) = \int_{-\infty}^\infty S_y(\omega) d\omega \\
    R_{xy}(t,t) = R_{xy}(0,0) = \int_{-\infty}^\infty S_{xy}(\omega) d\omega
\end{cases}\]

Запишем систему, заменяя дифференцирование умножением на $i\omega$:
\[ \begin{cases}
    i\omega X + \beta X = U \\
    i\omega Y + \gamma Y = U
\end{cases} \]
Определитель этой системы и его миноры
\[\begin{cases}
    \Delta(\omega) = \begin{vmatrix}
        i\omega + \beta & 0 \\
        0 & i\omega + \gamma
    \end{vmatrix} = \beta \gamma - \omega^2 + i\omega(\beta + \gamma) \\
    A_{11} = i\omega + \gamma \\
    A_{22} = i\omega + \beta \\
    A_{12} = A_{21} = 0
\end{cases}\]
\begin{gather*}
    \begin{cases}
        S_x(\omega) = \frac{S_u(\omega)}{|i\omega+\beta|^2} \\
        S_y(\omega) = \frac{S_u(\omega)}{|i\omega+\gamma|^2} \\
        S_{xy}(\omega) = \frac{A_{11}^* A_{12} S_{u}(\omega) + A_{11}^* A_{22} S_{u,u}(\omega) + A_{21}^* A_{12} S_{u,u}(\omega) + A_{21}^* A_{22} S_{u}(\omega)}{|\Delta(\omega)|^2} \overset{A_{12}=A_{21}=0}= \frac{A_{11}^* A_{22} S_{u,u}(\omega)}{|\Delta(\omega)|^2}
    \end{cases} \\
    |\Delta(\omega)|^2 = (\omega^2 + \beta^2)(\omega^2 + \gamma^2) \\
    \begin{aligned}
        S_u(\omega) &= \frac{1}{2\pi} \int_{-\infty}^\infty e^{-i\omega \tau} K_u(\tau) d\tau = \frac{\sigma^2}{2\pi} \int_{-\infty}^\infty e^{-i\omega\tau} e^{-\alpha|\tau|} d\tau \\
        &= \frac{\sigma^2}{2\pi} \left[ -\left.\frac{e^{-(\alpha+i\omega)\tau}}{\alpha+i\omega}\right|_0^\infty + \left.\frac{e^{(\alpha-i\omega)\tau}}{\alpha-i\omega}\right|_{-\infty}^0 \right] = \frac{\sigma^2}{\pi} \frac{\alpha}{\omega^2 + \alpha^2}
    \end{aligned}
\end{gather*}
Притом, очевидно, $R_{u,u}(\tau) = K_u(\tau) \Rightarrow S_{u,u}(\omega) = S_u(\omega)$.
Используя выведенные формулы, можно записать:
\begin{gather*}
    S_{xy}(\omega) = \frac{(\gamma-i\omega)(\beta+i\omega)S_{u}(\omega)}{(\omega^2+\beta^2)(\omega^2+\gamma^2)} = \frac{\sigma^2}{\pi} \frac{\alpha}{(\beta-i\omega)(\gamma+i\omega)(\omega^2+\alpha^2)} \\
    R_{xy}(0) = \frac{\sigma^2\alpha}{\pi} \underbrace{ \int_{-\infty}^\infty \frac{d\omega}{(\beta-i\omega)(\gamma+i\omega)(\omega^2+\alpha^2)}}_{\text{обозначим } J_1} \\
    K_x(0) = \frac{\sigma^2\alpha}{\pi} \underbrace{ \int_{-\infty}^\infty \frac{d\omega}{(\omega^2+\alpha^2)(\omega^2+\beta^2)}}_{\text{обозначим } J_2}
\end{gather*}

Воспользуемся теорией вычетов: если $f(x)$ -- дробно-рациональное выражение, степень числителя не менее чем на 2 меньше степени знаменателя и $\omega_k$ -- полюсы $f$ -- не лежат на вещественной оси, то
\[ \int_{-\infty}^\infty f(x)dx = 2\pi i \sum_{\Im \omega_k > 0} Res_{\omega=\omega_k}(f(\omega)) \]

\begin{gather*}
    J_1 = \int_{-\infty}^\infty \frac{d\omega}{(\omega + i\beta)(\omega-i\gamma)(\omega^2+\alpha^2)} = 2\pi i (res_{11} + res_{12}) \\
    \begin{cases}
        \omega_{11} = \gamma i, \text{ кратность } 1 \\
        \omega_{12} = \alpha i, \text{ кратность } 1
    \end{cases} \\
    res_{11} = \left.\frac{1}{(\omega + \beta i)(\omega^2+\alpha^2)}\right|_{\omega=\gamma i} = \frac{1}{i(\beta+\gamma)(\alpha^2-\gamma^2)} \\
    \begin{aligned}
        res_{12} &= \left.\frac{1}{(\omega+\beta i)(\omega-\gamma i)(\omega+\alpha i)}\right|_{\omega=\alpha i} = \frac{1}{i^2 (\alpha+\beta)(\alpha-\gamma) \cdot 2 i \alpha} \\
        &= -\frac{1}{2i\alpha(\alpha+\beta)(\alpha-\gamma)}
    \end{aligned} \\
    \begin{aligned}
        J_1 &= \frac{2\pi i}{i(\alpha-\gamma)} \cdot \left( \frac{1}{(\beta + \gamma)(\gamma+\alpha)} - \frac{1}{2\alpha(\alpha+\beta)} \right) = \frac{\pi}{\alpha-\gamma}\frac{2\alpha^2+\alpha\beta - \alpha\gamma-\beta\gamma-\gamma^2}{\alpha(\alpha+\beta)(\alpha+\gamma)(\beta+\gamma)} \\
        &= \frac{\pi(2\alpha + \beta + \gamma)}{\alpha(\alpha+\beta)(\alpha+\gamma)(\beta+\gamma)}
    \end{aligned}
\end{gather*}
Аналогично вычисляется второй интеграл:
\begin{gather*}
    J_2 = \int_{-\infty}^\infty \frac{d\omega}{(\omega^2 + \alpha^2)(\omega^2+\beta^2)} = 2\pi i (res_{21} + res_{22}) \\
    \begin{cases}
        \omega_{21} = \alpha i, \text{ кратность } 1 \\
        \omega_{22} = \beta i, \text{ кратность } 1
    \end{cases} \\
    res_{21} = \left.\frac{1}{(\omega + \alpha i)(\omega^2+\beta^2)}\right|_{\omega=\alpha i} = \frac{1}{2\alpha i (\beta^2-\alpha^2)} \\
    res_{22} = \left.\frac{1}{(\omega+\beta i)(\omega^2+\alpha^2)}\right|_{\omega=\beta i} = \frac{1}{2\beta i (\alpha^2-\beta^2)} \\
    J_2 = 2\pi i \left( \frac{1}{2\alpha i} - \frac{1}{2\beta i} \right) \cdot \frac{1}{\beta^2-\alpha^2} = \pi \frac{\beta-\alpha}{\alpha\beta} \cdot \frac{1}{\beta^2-\alpha^2} = \frac{\pi}{\alpha\beta(\alpha+\beta)}
\end{gather*}

Следовательно,
\begin{gather*}
    R_{xy}(0) = \frac{\sigma^2\alpha}{\pi} J_1 = \frac{\sigma^2(2\alpha + \beta + \gamma)}{(\alpha+\beta)(\alpha+\gamma)(\beta+\gamma)} \\
    D[X] = K_x(0) = \frac{\sigma^2\alpha}{\pi} J_2 = \frac{\sigma^2}{\beta(\alpha+\beta)} \\
    \shortintertext{и, симметрично формуле для корреляционной функции X,}
    D[Y] = K_y(0) = \frac{\sigma^2\alpha}{\pi} \int_{-\infty}^\infty \frac{d\omega}{(\omega^2+\alpha^2)(\omega^2+\gamma^2)} = \frac{\sigma^2}{\gamma(\alpha+\gamma)}
\end{gather*}

откуда окончательно получаем
\begin{align*}
    \Delta(t) &= \sqrt{M\left[ (X(t)-Y(t))^2 \right]} = \sqrt{D[X]+D[Y]+R_{xy}(0)} \\
    &= \sigma \sqrt{ \frac{\alpha(\alpha+\beta)(\beta+\gamma)+\gamma(\alpha+\gamma)(\beta+\gamma)+\alpha\gamma(2\alpha+\beta+\gamma)}{\alpha\gamma(\alpha+\beta)(\alpha+\gamma)(\beta+\gamma)} } \\
    &= \sigma \sqrt{ \frac{\alpha^2\beta+\alpha\beta^2+\alpha^2\gamma+\alpha\beta\gamma + \alpha\beta\gamma+\beta\gamma^2+\alpha\gamma^2+\gamma^3 + 2\alpha^2\gamma+\alpha\beta\gamma+\alpha\gamma^2 }{\alpha\gamma(\alpha+\beta)(\alpha+\gamma)(\beta+\gamma)} } \\
    &= \sigma \sqrt{ \frac{\alpha^2\beta+\alpha\beta^2+3\alpha^2\gamma+3\alpha\beta\gamma + \beta\gamma^2+2\alpha\gamma^2+\gamma^3 }{\alpha\gamma(\alpha+\beta)(\alpha+\gamma)(\beta+\gamma)} }
\end{align*}

\section{Подстановка численных значений}
Положим $ \sigma=1 $, $ \alpha=1 $, $ \beta=1 $, $ \gamma=2 $ $\Rightarrow$
\[ \Delta(t) = \sqrt{\frac{1+1+3\cdot2+3\cdot2+4+2\cdot4+8}{2\cdot2\cdot3\cdot3}} = 1 \]

Очевидно, это константа (рис. \ref{fig:dev})

\begin{figure}[H]
    \centering
    \includegraphics[width=\myPictWidth]{deviation}
    \caption{Среднеквадратическое отклонение $\Delta(t), t \in [0;10]$}
    \label{fig:dev}
\end{figure}

\end{document}