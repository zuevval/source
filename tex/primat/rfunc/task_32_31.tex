% compile with XeLaTeX or LuaLaTeX

\documentclass[a4paper,12pt]{article}
\usepackage[utf8]{inputenc} % russian, do not change
\usepackage[T2A, T1]{fontenc} % russian, do not change
\usepackage[english, russian]{babel} % russian, do not change

% fonts
\usepackage{fontspec} % different fonts
\setmainfont{Times New Roman}
\usepackage{setspace,amsmath}
\usepackage{amssymb} %common math symbols
\usepackage{dsfont}

% utilities
\usepackage{pdfpages} % insert pdf pages, e.g. scans
\usepackage{systeme} % systems of equations
\usepackage{mathtools} % xRightarrow, xrightleftharpoons, etc
\usepackage{hyperref}
\hypersetup{pdfstartview=FitH,  linkcolor=blue, urlcolor=blue, colorlinks=true}
\usepackage{framed} % advanced frames, boxes
\usepackage{mdframed}
\usepackage{graphicx}
\usepackage{color}

% styling
\usepackage{float} % force pictures position
\floatstyle{plaintop} % force caption on top
\usepackage{enumitem} % itemize and enumerate with [noitemsep]
\setlength{\parindent}{0pt} % no indents!
\usepackage[left=20mm, top=20mm, right=20mm, bottom=20mm, nohead, footskip=10mm]{geometry}

% misc
\graphicspath{{./img/}}
\newcommand{\myPictWidth}{.99\textwidth}
\newcommand{\phm}{\phantom{-}}
\newcommand{\defeq}{\overset{def}=}

\begin{document}

\title{Задача 32.31}
\author{Валерий Зуев}
\maketitle

\section{Постановка}
\[ \dot Z(t) + a^2 [1 + Y(t)]Z(t) = X(t) \]
$X$, $Y$ независимые, стационарные, нормальные; $\bar x = \bar y = 0$;
известны $K_x(\tau)$, $K_y(\tau)$; $Z(0)\equiv0$.

Найти $D[z(t)]$.

\begin{leftbar}
    Ответ:
    \begin{equation} \label{eq:ans}
        D[Z] = \int_{0}^{t} \int_{0}^{t} \exp\left( a^2(\tau_1+\tau_2) + \frac{a^2}{4} [ 3 \phi(\tau_1) + 3 \phi(\tau_2) - \phi(\tau_2-\tau_1) ] \right) K_x(\tau_2 - \tau_1) d\tau_1 d\tau_2
    \end{equation}
    где $ \phi(\tau) = 2 \int_{0}^{\tau} (\tau - \tau_1) K_y(\tau_1) d \tau_1 $
\end{leftbar}

\section{Решение}

\[\begin{cases}
    \dot Z + a^2(1-Y) Z = X \\
    Z(0)=0
\end{cases}\]
Представим решение в виде суммы общего решения однородного уравнения и частного решения неоднородного с однородными начальными условиями:
 \[Z=Z_0 + Z_*\]
Ищем общее решение $Z_0$.
\begin{gather*}
    \dot Z_0 + a^2(1-Y)Z_0 = 0 \\
    \dot Z_0 = -a^2(1-Y(t))Z_0 \\
\end{gather*}
Это линейная система первого порядка, следовательно, $ Z_0 = c_1 z_1 $, $ Z_1 = k \cdot (...) $, $k$ выбирается из условия $ W = Z_1(0) = 1 $.
Выражение вида $ z_1 = k \exp\left(- a^2 \int_{0}^{t} (1+Y(s)) ds \right) $ удовлетворяет решению:
\[ \dot Z_0 = c_1 k \exp \left(- a^2 \int_{0}^{t} (1+Y(s)) ds \right) \cdot \left(- a^2 \frac{d}{dt} \left( \int_{0}^{t} (1+Y(s)) ds \right) \right) = c_1 k (...) \cdot (- a^2(1+Y(t))) \]
\[ \begin{cases}
    Z_0(0)=0 \\
    Z_1(0)=1
\end{cases} \Rightarrow
\begin{cases}
    k = 1 \\
    c_1 = 0
\end{cases} \Rightarrow
\begin{cases}
    Z_1 = \exp \left( - a^2 \int_{0}^{t} (1+Y(s)) ds \right) \\
    Z_0 \equiv 0
\end{cases} \]
Т. о. $ Z = Z_* $.
Частное решение $Z_*$ ищем в виде интеграла:
\begin{gather*}
    \begin{cases}
        Z = \int_{0}^{t} p(t, \tau) X(\tau) d \tau \\
        p(t,\tau) = \frac{Z_1(t)}{Z_1(\tau)} = \exp \left( - a^2 \int_{\tau}^{t} (1+Y(s)) ds \right)
    \end{cases} \\
    \boxed{ Z = \int_{0}^{t} \left[ \exp \left( - a^2 \int_{\tau}^{t} (1+Y(s)) ds \right) X(\tau) \right] d \tau } \\
    D[Z] = \int_{0}^{t} \int_{0}^{t} \left[
        \exp \left(- a^2 \int_{\tau_1}^{t} (1+Y(s)) ds \right) \cdot
        \exp \left(- a^2 \int_{\tau_2}^{t} (1+Y(s)) ds \right) K_x(\tau_2-\tau_2) \right] d\tau_1 d\tau_2 \\
\end{gather*}
\begin{equation}\label{eq:var}
    D[Z] = \int_{0}^{t} \int_{0}^{t} \left[
    \exp \left(- a^2 \left (\int_{\tau_1}^{t} (1+Y(s)) ds + \int_{\tau_2}^{t} (1+Y(s)) ds \right) \right) \cdot
    K_x(\tau_2-\tau_2) \right] d\tau_1 d\tau_2
\end{equation}

\section{Сомнения}

Сравним показатели экспонент в \eqref{eq:ans} и \eqref{eq:var}.
Обозначим их $ p_{ans} $ и $p$ соответственно.
Подставим $\phi(\tau)$ в \eqref{eq:ans}:
\[ \frac{p_{ans}}{a^2} = \tau_1 + \tau_2 + \frac{1}{2} \left[
    3 \int_{0}^{\tau_1} (\tau_1 - \tau) K_y(\tau) d\tau +
    3 \int_{0}^{\tau_2} (\tau_2 - \tau) K_y(\tau) d\tau -
    \int_{0}^{\tau_2-\tau_1} ((\tau_2-\tau_1) - \tau) K_y(\tau) d\tau \right] \]
В то же время
\begin{align*}
    \frac{p}{a^2} & = - \left[ \int_{\tau_1}^{t} ds + \int_{\tau_2}^{t} ds + 2 \int_{0}^{t} Y(s) ds - \left( \int_{0}^{\tau_1} Y(s) ds + \int_{0}^{\tau_2} Y(s) ds \right) \right] \\
    & = -2t + \tau_1 + \tau_2 - 2 \int_0^t Y(s) ds + \left( + \int_{0}^{\tau_1} Y(s) ds + \int_{0}^{\tau_2} Y(s) ds \right)
\end{align*}
\[  \]
Сравниваем два выражения:
\begin{gather*}
    -2 \left(t+\int_{0}^{t} Y(s) ds \right) + \int_{0}^{\tau_1} Y(s) ds + \int_{0}^{\tau_2} Y(s) ds \overset{?}= \\
    \frac{1}{2} \left[
        3 \int_{0}^{\tau_1} (\tau_1 - \tau) K_y(\tau) d\tau +
        3 \int_{0}^{\tau_2} (\tau_2 - \tau) K_y(\tau) d\tau -
        \int_{0}^{\tau_2-\tau_1} ((\tau_2-\tau_1) - \tau) K_y(\tau) d\tau \right]
\end{gather*}

Левая часть зависит от $t$, в то время как правая не зависит.
Например, в предельном случае при $ Y \equiv 0 $ $ K_y \equiv 0 $, и мы имеем
\[ -2t \overset{?}= 0 \]
что, очевидно, при $t \ne 0$ не выполняется.

\end{document}