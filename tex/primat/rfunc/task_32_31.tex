% compile with XeLaTeX or LuaLaTeX

\documentclass[a4paper,12pt]{article}
\usepackage[utf8]{inputenc} % russian, do not change
\usepackage[T2A, T1]{fontenc} % russian, do not change
\usepackage[english, russian]{babel} % russian, do not change

% fonts
\usepackage{fontspec} % different fonts
\setmainfont{Times New Roman}
\usepackage{setspace,amsmath}
\usepackage{amssymb} %common math symbols
\usepackage{dsfont}

% utilities
\usepackage{pdfpages} % insert pdf pages, e.g. scans
\usepackage{systeme} % systems of equations
\usepackage{mathtools} % xRightarrow, xrightleftharpoons, etc
\usepackage{hyperref}
\hypersetup{pdfstartview=FitH,  linkcolor=blue, urlcolor=blue, colorlinks=true}
\usepackage{framed} % advanced frames, boxes
\usepackage{mdframed}
\usepackage{graphicx}
\usepackage{color}

% styling
\usepackage{float} % force pictures position
\floatstyle{plaintop} % force caption on top
\usepackage{enumitem} % itemize and enumerate with [noitemsep]
\setlength{\parindent}{0pt} % no indents!
\usepackage[left=10mm, top=5mm, right=10mm, bottom=20mm, nohead, footskip=10mm]{geometry}

% misc
\graphicspath{{./img/}}
\newcommand{\myPictWidth}{.99\textwidth}
\newcommand{\phm}{\phantom{-}}
\newcommand{\defeq}{\overset{def}=}

\begin{document}

\title{Задача 32.31}
\author{Валерий Зуев}
\maketitle

\section{Постановка}
\[ \dot Z(t) + a^2 [1 + Y]Z(t) = X(t) \]
$X$ -- стационарный нормальный процесс, $\bar x = 0$; $Y$ -- случайная величина, не зависящая от $X$; $Y \sim \mathcal{N}(0,\sigma)$; известна корреляционная функция $K_x(\tau)$; $Z(0)\equiv0$.

Найти $D[Z(t)]$.

\section{Решение}

\[\begin{cases}
    \dot Z + a^2(1+Y) Z = X \\
    Z(0)=0
\end{cases}\]
Система имеет порядок $1$, следовательно, ф. с. р. однородного уравнения -- единственный вектор  $z_1(t) = e^{-a^2(1+Y)t}$.
Поскольку начальные условия однородные, решение представимо в виде интеграла:

\begin{gather*}
    \begin{cases}
        Z = \int_{0}^{t} p(t, \tau) X(\tau) d \tau \\
        p(t,\tau) = \frac{Z_1(t)}{Z_1(\tau)} = e^{-a^2(1+Y)(t-\tau)}
    \end{cases} \\
    D[Z(t)] = K_z(t,t) = \iint_{0,0}^{t,t} \underbrace{M\left[ e^{-a^2(1+Y)(t-\tau_1)} e^{-a^2(1+Y)(t-\tau_2)} \right]}_{\text{обозн. }\eta} K_x(\tau_2-\tau_1) d\tau_1 d\tau_2 \\
    \eta = e^{-a^2(2t-\tau_1-\tau_2)} \underbrace{M\left[ e^{-a^2(2t-\tau_1-\tau_2)Y} \right]}_{\text{обозн. } \mu}
\end{gather*}

Обозначим показатель экспоненты за $\frac{\Theta}{i}$, т. е. $ \mu = M[\exp(i\Theta)] $.
Тогда характеристическая функция
\begin{gather*}
    \begin{cases}
        E_\Theta(u) = M[e^{i\Theta u}] \\
        \Theta \text{ нормальная, центрированная}
    \end{cases} \\
    \mu = E_\Theta(u)|_{u=1} = e^{-\frac{1}{2}\sigma_{\theta}^2 u^2}|_{u=1} = e^{-\frac{1}{2}\sigma_{\theta}^2}
\end{gather*}

Найдём дисперсию $ \sigma_\theta^2 = D[\Theta] $:

\begin{gather*}
    \frac{\Theta}{i} = -a^2(2t-\tau_1 -\tau_2)Y \Rightarrow \Theta = -i a^2(2t-\tau_1 -\tau_2)Y \\
    \sigma_\theta^2 = M[\mathring \Theta^2] = -a^4(2t-\tau_1-\tau_2)^2 M[\mathring Y^2] = -a^4(2t-\tau_1-\tau_2)^2\sigma^2\\
    \mu=e^{-\frac{\sigma_\theta^2}{2}}=\exp \left(\frac{a^4}{2}(2t-\tau_1-\tau_2)^2\sigma^2\right) \Rightarrow \eta = \exp \left( a^2(2t-\tau_1-\tau_2) \left( \frac{a^2\sigma^2(2t-\tau_1-\tau_2)}{2} - 1 \right) \right)
\end{gather*}

Окончательно получаем:
\begin{align*}
    D[Z(t)] = \iint_{0,0}^{t,t} \exp \left( a^2(2t-\tau_1-\tau_2) \left( \frac{a^2\sigma^2(2t-\tau_1-\tau_2)}{2} - 1 \right) \right)  K_x(\tau_2-\tau_1) d\tau_1 d\tau_2 \thickspace \square
\end{align*}

\end{document}