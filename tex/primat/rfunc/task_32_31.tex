% compile with XeLaTeX or LuaLaTeX

\documentclass[a4paper,12pt]{article}
\usepackage[utf8]{inputenc} % russian, do not change
\usepackage[T2A, T1]{fontenc} % russian, do not change
\usepackage[english, russian]{babel} % russian, do not change

% fonts
\usepackage{fontspec} % different fonts
\setmainfont{Times New Roman}
\usepackage{setspace,amsmath}
\usepackage{amssymb} %common math symbols
\usepackage{dsfont}

% utilities
\usepackage{pdfpages} % insert pdf pages, e.g. scans
\usepackage{systeme} % systems of equations
\usepackage{mathtools} % xRightarrow, xrightleftharpoons, etc
\usepackage{hyperref}
\hypersetup{pdfstartview=FitH,  linkcolor=blue, urlcolor=blue, colorlinks=true}
\usepackage{framed} % advanced frames, boxes
\usepackage{mdframed}
\usepackage{graphicx}
\usepackage{color}

% styling
\usepackage{float} % force pictures position
\floatstyle{plaintop} % force caption on top
\usepackage{enumitem} % itemize and enumerate with [noitemsep]
\setlength{\parindent}{0pt} % no indents!
\usepackage[left=10mm, top=20mm, right=10mm, bottom=20mm, nohead, footskip=10mm]{geometry}

% misc
\graphicspath{{./img/}}
\newcommand{\myPictWidth}{.99\textwidth}
\newcommand{\phm}{\phantom{-}}
\newcommand{\defeq}{\overset{def}=}

\begin{document}

\title{Задача 32.31}
\author{Валерий Зуев}
\maketitle

\section{Постановка}
\[ \dot Z(t) + a^2 [1 + Y(t)]Z(t) = X(t) \]
$X$, $Y$ независимые, стационарные, нормальные; $\bar x = \bar y = 0$;
известны $K_x(\tau)$, $K_y(\tau)$; $Z(0)\equiv0$.

Найти $D[z(t)]$.

\begin{leftbar}
    Ответ:
    \begin{equation} \label{eq:ans}
        D[Z] = \int_{0}^{t} \int_{0}^{t} \exp\left( a^2(\tau_1+\tau_2) + \frac{a^4}{4} [ 3 \phi(\tau_1) + 3 \phi(\tau_2) - \phi(\tau_2-\tau_1) ] \right) K_x(\tau_2 - \tau_1) d\tau_1 d\tau_2
    \end{equation}
    где $ \phi(s) = 2 \int_0^s (s-\tau) K_y(\tau) d \tau $
\end{leftbar}

\section{Решение}

\[\begin{cases}
    \dot Z + a^2(1-Y) Z = X \\
    Z(0)=0
\end{cases}\]
Представим решение в виде суммы общего решения однородного уравнения и частного решения неоднородного с однородными начальными условиями:
 \[Z=Z_0 + Z_*\]
Ищем общее решение $Z_0$.
\begin{gather*}
    \dot Z_0 + a^2(1-Y)Z_0 = 0 \\
    \dot Z_0 = -a^2(1-Y(t))Z_0 \\
\end{gather*}
Это линейная система первого порядка, следовательно, $ Z_0 = c_1 z_1 $, $ Z_1 = k \cdot (...) $, $k$ выбирается из условия $ W = Z_1(0) = 1 $.
Выражение вида $ z_1 = k \exp\left(- a^2 \int_{0}^{t} (1+Y(s)) ds \right) $ удовлетворяет решению:
\[ \dot Z_0 = c_1 k \exp \left(- a^2 \int_{0}^{t} (1+Y(s)) ds \right) \cdot \left(- a^2 \frac{d}{dt} \left( \int_{0}^{t} (1+Y(s)) ds \right) \right) = c_1 k (...) \cdot (- a^2(1+Y(t))) \]
\[ \begin{cases}
    Z_0(0)=0 \\
    Z_1(0)=1
\end{cases} \Rightarrow
\begin{cases}
    k = 1 \\
    c_1 = 0
\end{cases} \Rightarrow
\begin{cases}
    Z_1 = \exp \left( - a^2 \int_{0}^{t} (1+Y(s)) ds \right) \\
    Z_0 \equiv 0
\end{cases} \]
Т. о. $ Z = Z_* $.
Частное решение $Z_*$ ищем в виде интеграла:
\begin{gather*}
    \begin{cases}
        Z = \int_{0}^{t} p(t, \tau) X(\tau) d \tau \\
        p(t,\tau) = \frac{Z_1(t)}{Z_1(\tau)} = \exp \left( - a^2 \int_{\tau}^{t} (1+Y(s)) ds \right)
    \end{cases} \\
    \boxed{ Z = \int_{0}^{t} \left[ \exp \left( - a^2 \int_{\tau}^{t} (1+Y(s)) ds \right) X(\tau) \right] d \tau } \\
    D[Z(t)] = K_z(t,t) \\
    K_z(t_1,t_2) = \iint_{0,0}^{t_1,t_2} M\left[
        \exp \left(- a^2 \int_{\tau_1}^{t_1} (1+Y(s)) ds \right) \cdot
        \exp \left(- a^2 \int_{\tau_2}^{t_2} (1+Y(s)) ds \right)\right] K_x(\tau_2-\tau_2) d\tau_1 d\tau_2 \\
\end{gather*}
\begin{equation}\label{eq:var}
    K_z(t_1, t_2) = e^{-a^2(t_1+t_2)} \int_0^{t_1} \int_0^{t_2} e^{a^2(\tau_1+\tau_2)} M\left[ \exp \left(- a^2 \left(\int_{\tau_1}^{t_1}Y(s)ds+\int_{\tau_2}^{t_2}Y(s)ds\right)\right)\right] K_x(\tau_2-\tau_2) d\tau_1 d\tau_2
\end{equation}
Теперь необходимо найти величину
\[ \mu = M\left[ \exp \left(- a^2 \left(\int_{\tau_1}^{t_1}Y(s)ds+\int_{\tau_2}^{t_2}Y(s)ds\right)\right)\right] \]
Обозначим показатель экспоненты за $\frac{\Theta}{i}$, т. е. $ \mu = M[\exp(i\Theta)] $.
Тогда
\begin{gather*}
    \begin{cases}
        E_\Theta(u) = M[e^{i\Theta u}] \\
        \Theta \text{ нормальная, центрированная}
    \end{cases} \\
    \mu = E_\Theta(u)|_{u=1} = e^{-\frac{1}{2}\sigma_{\theta}^2 u^2}|_{u=1} = e^{-\frac{1}{2}\sigma_{\theta}^2}
\end{gather*}

Найдём величину $ \sigma_\theta^2 = D[\Theta] $ ($ \Theta $, заметим, есть случайная величина, зависящая от $t_1, t_2, \tau_1$ и $\tau_2$).

\begin{gather*}
    i\Theta = -a^2 \left(\int_{\tau_1}^{t_1}Y(s)ds+\int_{\tau_2}^{t_2}Y(s)ds\right) \\
    \Theta = ia^2 \left(\int_{\tau_1}^{t_1}Y(s)ds+\int_{\tau_2}^{t_2}Y(s)ds\right) \\
    \sigma_\Theta^2 = -a^4 M\left[ \left(\int_{\tau_1}^{t_1}Y(s)ds+\int_{\tau_2}^{t_2}Y(s)ds\right)^2 \right]
\end{gather*}
Заметим, что в силу стационарности $Y(t)$ $\int_{\tau_k}^{t_k}Y(s)ds = \int_0^{t_k-\tau_k}Y(s)ds$.
Обозначим $\psi(t):= \int_0^{t}Y(s)ds$.
Поскольку $\bar y(t) \equiv 0$, то и $ \bar \psi(t) \equiv 0 $.
Тогда
\begin{align*}
    -\frac{\sigma_\Theta^2}{a^4} &= M\left[ (\psi(t_1-\tau_1)+\psi(t_2-\tau_2))^2 \right] = M[\psi^2(t_1-\tau_1)] + M[\psi^2(t_2-\tau_2)] + 2M[\psi(t_1-\tau_1)\psi(t_2-\tau_2)] \\
    &= K_\psi(t_1-\tau_1, t_1-\tau_1) + K_\psi(t_2-\tau_2, t_2-\tau_2) + 2K_\psi(t_1-\tau_1, t_2-\tau_2)
\end{align*}

Притом по формуле для корреляционной функции интеграла от стационарной с. ф.
\[ K_\psi(s_1,s_2) = \int_0^{s_1}(s_1-\tau)K_y(\tau)d\tau + \int_0^{s_2}(s_2-\tau)K_y(\tau)d\tau - \int_0^{|s_2-s_1|}(|s_2-s_1|-\tau)K_y(\tau)d\tau \]
Обозначив $ \phi(s) := 2 \int_0^s (s - \tau) K_y(\tau) d \tau $, получим
\[ K_\psi(s_1,s_2) = \frac{1}{2}(\phi(s_1)+\phi(s_2) - \phi(|s_2-s_1|)) \]
откуда
\[ -\frac{\sigma_\Theta^2}{a^4} = \frac{1}{2}(2\phi(t_1-\tau_1) + 0) + \frac{1}{2}(2\phi(t_2-\tau_2) + 0) + 2 \cdot \frac{1}{2}(\phi(t_1-\tau_1)+\phi(t_2-\tau_2)-\phi(|t_1-\tau_1-t_2+\tau_2|)) \]

\[ \mu = \exp\left(\frac{a^4}{2} \left( 2\phi(t_1-\tau_1) + 2\phi(t_2-\tau_2) - \phi(|t_1-\tau_1-t_2+\tau_2|) \right) \right)  \]

Окончательно получаем:
\begin{align*}
    D[Z(t)] &= K_z(t,t) = e^{-2a^2 t} \iint_{0,0}^{t,t} e^{a^2(\tau_1+\tau_2)} \mu K_x(\tau_2-\tau_1) d\tau_1 d\tau_2 \\
    &= e^{-2a^2 t} \iint_{0,0}^{t,t} \exp\left(a^2(\tau_1+\tau_2) + \frac{a^4}{2} \left( 2\phi(t-\tau_1) + 2\phi(t-\tau_2) - \phi(|\tau_2-\tau_1|) \right) \right)  K_x(\tau_2-\tau_1) d\tau_1 d\tau_2
\end{align*}

\end{document}