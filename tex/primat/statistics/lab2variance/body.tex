\documentclass[report1.tex]{subfiles}
\begin{document}

\section{Постановка задачи}
Необходимо исследовать характеристики положения и рассеяния перечисленных ниже распределений:
\begin{enumerate}
	\item равномерное
	\item Гаусса (нормальное)
	\item Коши
	\item Лапласа
	\item биномиальное.
\end{enumerate}
Для каждого закона проделать следующие вычисления:
\begin{enumerate}
	\item Тысячу раз сгенерировать выборки по 10, 100 и 1000 элементов, распределённых по данному закону.
	\item Вычислить среднее значение $\overline{x}$, медиану $med(x)$, полусумму экстремальных элементов $z_R(x)$, полусумму квартилей $z_Q$ и усечённое среднее $z_{tr}$ для каждой выборки $x$ из трёх (снова тысячу раз).
	\item Для каждой из порядковых статистик $z$, перечисленных в предыдущем пункте, рассчитать среднее значение $\overline{z}$ и дисперсию $D(z)$ по тысяче испытаний.
\end{enumerate}
Таким образом, в результате должно получиться 150 значений (5 распределений $\times$ 3 выборки $\times$ 5 характеристик $\times$ 2 -- среднее и дисперсия)
\newpage
\section{Теория}
Формулы плотностей вероятности для каждого из непрерывных законов распределения и ряд распределения Пуассона приведены в отчёте по лабораторной работе 1.\\
Порядковые статистики вычислены по формулам, приведённым в \cite{maksimov}.\\
Выборочное среднее значение массива $x$ из $n \in \mathds{N}$ элементов
\begin{equation} \label{eq:mean}
\overline{x} = \frac{1}{n} \sum_{k=1}^{n} x_k
\end{equation}
Медиана\\
\begin{equation} \label{eq:med}
med(x) = 
\begin{cases}
x_{(n-1)/2}, n \hspace{2pt} !\vdots \hspace{2pt} 2\\
\frac{x_{n/2} + x_{n/2 + 1}}{2}, n \hspace{2pt} \vdots \hspace{2pt} 2
\end{cases}
\end{equation}
где $x_(k)$ -- $k$-я порядковая статистика, т. е. элемент, стоящий на $k$-ой позиции в выборке, если её упорядочить по неубыванию.\\
Полусумма экстремальных значений\\
\begin{equation} \label{eq:zr}
z_R = \frac{x_{(1)} + x_{n}}{2}
\end{equation}
Полусумма квартилей\\
\begin{equation} \label{eq:zq}
z_Q = \frac{z_{1/4}+z_{3/4}}{2}
\end{equation}
где $z_p = x_{\lceil np \rceil}$ - порядковая статистика с номером, полученном округлением вверх вещественного числа $np$.\\
Усечённое среднее\\
\begin{equation} \label{eq:ztr}
z_{tr} = \frac{1}{n-2s} \sum_{k=s+1}^{n-s} x_{(k)}
\end{equation}
где $s$, как правило, от $0.05$ до $0.25$ (в данной работе используется значение $s=0.25$ -- вычисляется среднее по половине исходных данных).

\newpage
\section{Реализация}
Выборки сгенерированы средствами пакета MATLAB (подробнее о механизме их генерации -- в первом отчёте). Для удобства вычисления порядковых статистик каждая выборка была отсортирована по неубыванию встроенным в пакет методом \emph{sort}.\\
Для полученных массивов данных вычислены средние (\ref{eq:mean}) и медианные (\ref{eq:med}) значения встроенными методами \emph{mean} и \emph{median} соответственно. Для вычисления статистик (\ref{eq:zr}), (\ref{eq:zq}), (\ref{eq:ztr}) были написаны отдельные функции. Код программы по состоянию на 5 мая 2020 года доступен по сслыке: \\ \href{https://github.com/zuevval/source/tree/master/matlab/statistics2020/lab2variance}{github.com/zuevval/source/tree/master/matlab/statistics2020/lab2variance}.

\newpage
\section{Результаты}
Результаты сведены в таблицы ($n$ - объём выборки).
\begin{table}[H]\label{table:norm}
\centering
\caption{Нормальное распределение}
\begin{tabular}{c*6r}
\toprule
{} &         
$n$  & 
\centering $z_R$ & 
\centering $z_Q$ &
\centering $\overline{x}$& 
\centering $med(x)$ &
\centering $z_{tr}$ \tabularnewline
\midrule
\multirow{3}{*}{$\overline{z}$}
& $10$    & $ 0.003$  & $ 0.011$  & $ 0.008$  & $ 0.008$  & $ 0.010$ \\
& $100$   & $ 0.016$  & $-0.014$  & $ 0.000$  & $-0.005$  & $-0.002$ \\
& $1000$  & $-0.015$  & $-0.002$  & $-0.001$  & $-0.002$  & $-0.001$ \\
\midrule
\multirow{3}{*}{$D(z)$}
& $10$    & $0.183$  & $0.134$  & $0.104$  & $0.136$  & $0.121$ \\
& $100$   & $0.099$  & $0.012$  & $0.010$  & $0.015$  & $0.012$ \\
& $1000$  & $0.060$  & $0.001$  & $0.001$  & $0.002$  & $0.001$ \\              
\bottomrule
\end{tabular}
\end{table}

\begin{table}[H]\label{table:cauchy}
\centering
\caption{Распределение Коши}
\begin{tabular}{c*6r}
\toprule
{} &         
$n$  & 
\centering $z_R$ & 
\centering $z_Q$ &
\centering $\overline{x}$& 
\centering $med(x)$ &
\centering $z_{tr}$ \tabularnewline
\midrule
\multirow{3}{*}{$\overline{z}$}
& $10$    & $-0.445$     & $-0.048$  & $-0.106$  & $-0.010$  & $-0.023$ \\
& $100$   & $ 6.393$     & $-0.034$  & $ 0.100$  & $ 0.001$  & $ 0.000$ \\
& $1000$  & $-1744.909$  & $ 0.000$  & $-3.498$  & $ 0.001$  & $ 0.001$ \\
\midrule
\multirow{3}{*}{$D(z)$}
& $10$    & $6799.695$        & $0.975$  & $277.544$   & $0.307$  & $0.440$ \\
& $100$   & $204803.510$      & $0.050$  & $86.421$    & $0.025$  & $0.026$ \\
& $1000$  & $2142451078.924$  & $0.005$  & $8598.876$  & $0.002$  & $0.002$ \\
\bottomrule
\end{tabular}
\end{table}

\begin{table}[H]\label{table:laplace}
\centering
\caption{Распределение Лапласа}
\begin{tabular}{c*6r}
\toprule
{} &         
$n$ & 
\centering $z_R$ & 
\centering $z_Q$ &
\centering $\overline{x}$& 
\centering $med(x)$ &
\centering $z_{tr}$ \tabularnewline
\midrule
\multirow{3}{*}{$\overline{z}$}
& $10$    & $ 0.010$  & $-0.016$  & $-0.008$  & $-0.007$  & $-0.012$ \\
& $100$   & $ 0.026$  & $-0.012$  & $ 0.001$  & $ 0.000$  & $ 0.000$ \\
& $1000$  & $-0.031$  & $-0.002$  & $-0.001$  & $ 0.000$  & $ 0.000$ \\
\midrule
\multirow{3}{*}{$D(z)$}
& $10$    & $0.388$  & $0.096$  & $0.095$  & $0.068$  & $0.069$ \\
& $100$   & $0.413$  & $0.009$  & $0.010$  & $0.006$  & $0.006$ \\
& $1000$  & $0.402$  & $0.001$  & $0.001$  & $0.000$  & $0.001$ \\
\bottomrule
\end{tabular}
\end{table}

\begin{table}[H]\label{table:poisson}
\centering
\caption{Распределение Пуассона}
\begin{tabular}{c*6r}
\toprule
{} &         
$n$ & 
\centering $z_R$ & 
\centering $z_Q$ &
\centering $\overline{x}$& 
\centering $med(x)$ &
\centering $z_{tr}$ \tabularnewline
\midrule
\multirow{3}{*}{$\overline{z}$}
& $10$    & $ 0.003$  & $ 0.011$  & $ 0.008$  & $ 0.008$  & $ 0.010$ \\
& $100$   & $ 0.016$  & $-0.014$  & $ 0.000$  & $-0.005$  & $-0.002$ \\
& $1000$  & $-0.015$  & $-0.002$  & $-0.001$  & $-0.002$  & $-0.001$ \\
\midrule
\multirow{3}{*}{$D(z)$}
& $10$    & $0.183$  & $0.134$  & $0.104$  & $0.136$  & $0.121$ \\
& $100$   & $0.099$  & $0.012$  & $0.010$  & $0.015$  & $0.012$ \\
& $1000$  & $0.060$  & $0.001$  & $0.001$  & $0.002$  & $0.001$ \\
\bottomrule
\end{tabular}
\end{table}

\begin{table}[H]\label{table:uniform}
\centering
\caption{Равномерное распределение}
\begin{tabular}{c*6r}
\toprule
{} &         
$n$ & 
\centering $z_R$ & 
\centering $z_Q$ &
\centering $\overline{x}$& 
\centering $med(x)$ &
\centering $z_{tr}$ \tabularnewline
\midrule
\multirow{3}{*}{$\overline{z}$}
& $10$    & $ 0.003$  & $ 0.017$  & $ 0.012$  & $ 0.013$  & $ 0.016$ \\
& $100$   & $ 0.000$  & $-0.018$  & $-0.001$  & $-0.001$  & $ 0.000$ \\
& $1000$  & $ 0.000$  & $-0.001$  & $ 0.000$  & $-0.001$  & $ 0.000$ \\
\midrule
\multirow{3}{*}{$D(z)$}
& $10$    & $0.047$  & $0.139$  & $0.099$  & $0.224$  & $0.161$ \\
& $100$   & $0.001$  & $0.015$  & $0.010$  & $0.029$  & $0.020$ \\
& $1000$  & $0.000$  & $0.002$  & $0.001$  & $0.003$  & $0.002$ \\
\bottomrule
\end{tabular}
\end{table}


\newpage
\section{Обсуждение}
Можно выделить некоторые общие закономерности в результатах для всех распределений, \textbf{кроме Коши}:
\begin{enumerate}
	\item Среднее значение среднего значения $\overline{x}$ по модулю находится в пределах $0.008$ для выборок размера $10$,с ростом числа выборки убывает или, по меньшей мере, не возрастает (т. о. стремится к медиане распределения: у каждого распределения медиана $0$). Выборочная дисперсия среднего значения убывает всякий раз примерно в десять раз: можно сказать, что средняя плотность значений $\overline{x}$ пропорциональна размеру выборки.
	\item Среднее значение усечённого среднего обычно по модулю превышает среднее значение среднего значения. Это согласуется с тем, что модуль среднего значения в вероятностном смысле убывает с ростом числа элементов, ведь при расчёте усечённого среднего используется информация не о всех точках. Дисперсия усечённого среднего с увеличением размеров выборки вдесятеро тоже убывает всякий раз примерно в десять раз, причём, хотя модуль дисперсии больше аналогичных модулей дисперсии для обычных средних значений, видно, что отношение значения модуля предыдущего с следующему ближе к $10$: не учитывая выбросы, мы получаем более предсказуемое поведение среднего значения в зависимости от размера выборки. 
	\item Похожие значения у среднего и дисперсии выборочной медианы: среднее значение убывает с ростом числа элементах в выборке, но медленнее чем линейно, дисперсия обратно пропорциональна числу элементов в выборках. Иначе говоря, чем больше элементов в выборке, тем выше вероятность встретить значение вблизи матожидания распределения.
	\item Полусумма экстремальных значений $Z_R$ - одна из наименее стабильных характеристика среднего значения (зато, пожалуй, её вычисление обходится дешевле всех остальных). Интуитивно это можно объяснить тем, что с увеличением числа элементов в выборке для распределения Гаусса, Лапласа и Пуассона возрастает вероятность встретить значения сколь угодно далеко от матожидания.  У равномерного распределения матожидание и дисперсия $Z_R$, впрочем, при большом числе элементов в выборке оказалось равным нулю с точностью до третьего знака после запятой. Это объясняется ограниченностью возможных значений случайной величины: с ростом числа элементов в выборке крайние значения всё ближе к $-\sqrt{3}$ и $\sqrt{3}$ соответственно.
	\item Полусумма квартилей отличается от полусуммы экстремальных значений примерно так же, как усечённое среднее от обычного выборочного среднего: значения разбросаны в более широком диапазоне, но зато этот диапазон лучше коррелирует с размером выборки.
\end{enumerate}
Особняком стоят результаты для распределения Коши, которое не имеет дисперсии; случайная величина, распределённая по этому закону, не подчиняется центральной предельной теореме. Видно, что среднее и дисперсия среднего значения и полусуммы экстремальных значений возрастает с ростом числа выборки. Среднее значение медианы, полусуммы квартилей и усечённого среднего, впрочем,  с ростом числа элементов уменьшается (и дисперсия этих статистик обратно пропорциональна числу элементов). Можно дать следующую интерпретацию: доля выбросов с увеличением числа элементов в выборке падает, но зато эти выбросы появляются настолько далеко от матожидания распределения, что характеристики, не отбрасывающие крайние значения, плохо оценивают положение матожидания.


\end{document}