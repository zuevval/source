\documentclass[zuev_report2.tex]{subfiles}
\begin{document}

\section{Постановка задачи}
Необходимо исследовать характеристики положения и рассеяния перечисленных ниже распределений:
\begin{enumerate}
	\item равномерное
	\item Гаусса (нормальное)
	\item Коши
	\item Лапласа
	\item биномиальное.
\end{enumerate}
\subsection{Лабораторная \textnumero 2: характеристики положения и рассеяния}
Для каждого закона проделать следующие вычисления:
\begin{enumerate}
	\item Тысячу раз сгенерировать выборки по 10, 100 и 1000 элементов, распределённых по данному закону.
	\item Вычислить среднее значение $\overline{x}$, медиану $med(x)$, полусумму экстремальных элементов $z_R(x)$, полусумму квартилей $z_Q$ и усечённое среднее $z_{tr}$ для каждой выборки $x$ из трёх (снова тысячу раз).
	\item Для каждой из порядковых статистик $z$, перечисленных в предыдущем пункте, рассчитать среднее значение $\overline{z}$ и дисперсию $D(z)$ по тысяче испытаний.
\end{enumerate}
Таким образом, в результате должно получиться 150 значений (5 распределений $\times$ 3 выборки $\times$ 5 характеристик $\times$ 2 -- среднее и дисперсия)

\subsection{Лабораторная \textnumero 3: боксплот Тьюки и выбросы}
\begin{enumerate}
	\item Для каждого закона сгенерировать выборки по 20 и 100 элементов и построить по каждому из массивов значений случайной величины \textit{боксплот Тьюки}
	\item Сгенерировать такие же выборки ещё 1000 раз, сосчитать для каждого массива долю выбросов и рассчитать среднее по 1000 пробам.
	\item Вычислить для каждого распределения ожидаемую долю выбросов из теоретических соображений и сравнить с полученными экспериментально значениями.
\end{enumerate}
\newpage
\section{Теория}
\subsection{Характеристики положения и рассеяния}
Формулы плотностей вероятности для каждого из непрерывных законов распределения и ряд распределения Пуассона приведены в отчёте по лабораторной работе 1.\\
Порядковые статистики вычислены по формулам, приведённым в \cite{maksimov}.\\
Выборочное среднее значение массива $x$ из $n \in \mathds{N}$ элементов
\begin{equation} \label{eq:mean}
\overline{x} = \frac{1}{n} \sum_{k=1}^{n} x_k
\end{equation}
Медиана\\
\begin{equation} \label{eq:med}
med(x) = 
\begin{cases}
x_{(n-1)/2}, n \hspace{2pt} !\vdots \hspace{2pt} 2\\
\frac{x_{n/2} + x_{n/2 + 1}}{2}, n \hspace{2pt} \vdots \hspace{2pt} 2
\end{cases}
\end{equation}
где $x_(k)$ -- $k$-я порядковая статистика, т. е. элемент, стоящий на $k$-ой позиции в выборке, если её упорядочить по неубыванию.\\
Полусумма экстремальных значений\\
\begin{equation} \label{eq:zr}
z_R = \frac{x_{(1)} + x_{n}}{2}
\end{equation}
Полусумма квартилей\\
\begin{equation} \label{eq:zq}
z_Q = \frac{z_{1/4}+z_{3/4}}{2}
\end{equation}
где $z_p = x_{\lceil np \rceil}$ - порядковая статистика с номером, полученном округлением вверх вещественного числа $np$.\\
Усечённое среднее\\
\begin{equation} \label{eq:ztr}
z_{tr} = \frac{1}{n-2s} \sum_{k=s+1}^{n-s} x_{(k)}
\end{equation}
где $s$, как правило, от $0.05$ до $0.25$ (в данной работе используется значение $s=0.25$ -- вычисляется среднее по половине исходных данных).

\subsection{Боксплот Тьюки и выбросы}
Боксплот Тьюки, или <<ящик с усами>> -- диаграмма, позволяющая по числовой выборке наглядно представить положение областей, в которых сосредоточена статистически значимая доля выборки. Для его построения используются три значения:
\begin{itemize}
	\item Медиана выборки $m$
	\item Первый квартиль $z_{0.25}$
	\item Третий квартиль $z_{0.75}$
\end{itemize}
По первому и третьему квартилю вычисяются оценки границ диапазона статистически значимой доли выборки: левая граница $\overline{X_1} := z_{0.25} - \frac{3}{2}(\Delta z_{(0.25;0.75)}) = \frac{5z_{0.25}-3z_{0.75}}{2}$ и правая $\overline{X_2} := z_{0.25} + \frac{3}{2}(\Delta z_{(0.25;0.75)}) = \frac{5z_{0.75}-3z_{0.25}}{2}$
На диаграмме обозначается медиана, вокруг неё строится <<ящик>>, границы которого -- квартили, и от ящика отходят <<усы>> с границами в точках $\overline{X_1}$ и $\overline{X_2}$.\\

При увеличении числа элементов в выбрке значения квартилей (и, соответственно, границы диапазона статистически значимой части выборки) будут стремиться к значениям, полученным из теории, т. е. определяться функцией распределения случайной величины:
\begin{gather*}
Q_1 : P\{\xi \ge Q_1\} = \frac{1}{4} \Rightarrow Q_1 = F^{-1}\left(\frac{1}{4}\right)\\
Q_2 : P\{\xi \le Q_2\} = \frac{3}{4} \Rightarrow Q_2 = F^{-1}\left(\frac{3}{4}\right)\\
X_1 := Q_1 - \frac{3}{2}(\Delta Q_{31}) = \frac{5Q_1-3Q_3}{2}\\
X_2 := Q_3 + \frac{3}{2}(\Delta Q_{31}) = \frac{5Q_3-3Q_1}{2}\\
\end{gather*}
где $\xi$ - наблюдаемая случайная величина, распределённая по закону с функцией распределения $F$ \cite{sevastianov}.\\
Вероятность выбросов для равномерного распределения:
\begin{gather} 
\label{eq:outliers_probab1}
P\{x<X_1 \wedge x>X_2\} = P\{x<X_1\}+P\{x>X_2\}\\
\label{eq:outliers_probab2}
P\{x<X_1\} = P\{x \le X_1\} = F(X_1)\\
\label{eq:outliers_probab3}
P\{x>X_2\} = 1 - P\{x\le X_1\} = 1 - F(X_2)
\end{gather}
Для дискретного распределения Пуассона можно применить аналогичные формулы с той лишь разницей, что 
\begin{equation}\label{eq:poisson_outliers}
P\{x<X_1\} = P\{x \le X_1\} - P\{x=X_1\} = F(X_1)-P(x=X_1)
\end{equation}
Доля выбросов в тестовой выборке оценим как долю элементов, находящихся за пределами <<усов>> боксплота:
\begin{equation}\label{eq:outliers_fraction}
D := \frac{n(\{x | x<\overline{X_1} \wedge x>\overline{X_2}\})}{n}
\end{equation}
где $n$ -- мощность выборки.


\newpage
\section{Реализация}
\subsection{Характеристики положения и рассеяния}
Выборки сгенерированы средствами пакета MATLAB (подробнее о механизме их генерации -- в первом отчёте). Для удобства вычисления порядковых статистик каждая выборка была отсортирована по неубыванию встроенным в пакет методом \emph{sort}.\\
Для полученных массивов данных вычислены средние (\ref{eq:mean}) и медианные (\ref{eq:med}) значения встроенными методами \emph{mean} и \emph{median} соответственно. Для вычисления статистик (\ref{eq:zr}), (\ref{eq:zq}), (\ref{eq:ztr}) были написаны отдельные функции. Код программы по состоянию на 5 мая 2020 года доступен по сслыке [\ref{link:lab2}].\\
\subsection{Боксплот Тьюки и выбросы}
<<Ящики с усами>> построены при помощи поставляемой с пакетом MATLAB фукнции \textit{boxplot(...)}.\\
Вероятности выбросов оценены по формулам (\ref{eq:outliers_probab1}), (\ref{eq:outliers_probab2}) и (\ref{eq:outliers_probab3}), для чего написана функция \textit{outliers\_probab(inv\_cdf, cdf)}, принимающая на вход функцию распределения и обратную к ней. Отдельно с использованием (\ref{eq:poisson_outliers}) была рассчитана вероятность выбросов в распределении Пуассона.\\
Доля выбросов в выборках была рассчитана по формуле (\ref{eq:outliers_fraction}).\\
Текст программы, осуществляющей построение боксплотов и вычисление долей выбросов, доступен по ссылке [\ref{link:lab3}].


\newpage
\section{Результаты}
\subsection{Характеристики положения и рассеяния}
Результаты сведены в таблицы ($n$ - объём выборки).
\begin{table}[H]
\centering
\caption{Нормальное распределение}
\begin{tabular}{c*6r}
\toprule
{} &         
$n$  & 
\centering $z_R$ & 
\centering $z_Q$ &
\centering $\overline{x}$& 
\centering $med(x)$ &
\centering $z_{tr}$ \tabularnewline
\midrule
\multirow{3}{*}{$\overline{z}$}
& $10$    & $ 0.0$  & $ 0.0$  & $ 0.0$  & $ 0.0$  & $ 0.0$ \\
& $100$   & $ 0.0$  & $-0.01$  & $ 0.00$  & $-0.00$  & $-0.00$ \\
& $1000$  & $-0.01$  & $-0.002$  & $-0.001$  & $-0.002$  & $-0.001$ \\
\midrule
\multirow{3}{*}{$D(z)$}
& $10$    & $0.2$  & $0.2$  & $0.1$  & $0.2$  & $0.2$ \\
& $100$   & $0.1$  & $0.02$  & $0.01$  & $0.02$  & $0.02$ \\
& $1000$  & $0.06$  & $0.001$  & $0.001$  & $0.002$  & $0.001$ \\              
\bottomrule
\end{tabular}
\label{table:norm}
\end{table}

\begin{table}[H]
\centering
\caption{Распределение Коши}
\begin{tabular}{c*6r}
\toprule
{} &         
$n$  & 
\centering $z_R$ & 
\centering $z_Q$ &
\centering $\overline{x}$& 
\centering $med(x)$ &
\centering $z_{tr}$ \tabularnewline
\midrule
\multirow{3}{*}{$\overline{z}$}
& $10$    & $0$     & $0$  & $0$  & $0.0$  & $0.0$ \\
& $100$   & $ 6$     & $-0.03$  & $0$  & $ 0.00$  & $ 0.00$ \\
& $1000$  & $-2000$  & $ 0.000$  & $-3$  & $ 0.001$  & $ 0.001$ \\
\midrule
\multirow{3}{*}{$D(z)$}
& $10$    & $7000$        & $1$  & $300$   & $0.3$  & $0.5$ \\
& $100$   & $200000$      & $0.05$  & $90$    & $0.03$  & $0.03$ \\
& $1000$  & $2000000000$  & $0.005$  & $9000$  & $0.002$  & $0.002$ \\
\bottomrule
\end{tabular}
\label{table:cauchy}
\end{table}

\begin{table}[H]
\centering
\caption{Распределение Лапласа}
\begin{tabular}{c*6r}
\toprule
{} &         
$n$ & 
\centering $z_R$ & 
\centering $z_Q$ &
\centering $\overline{x}$& 
\centering $med(x)$ &
\centering $z_{tr}$ \tabularnewline
\midrule
\multirow{3}{*}{$\overline{z}$}
& $10$    & $ 0.0$  & $0.0$  & $-0.01$  & $-0.01$  & $-0.01$ \\
& $100$   & $ 0.0$  & $-0.01$  & $ 0.001$  & $ 0.000$  & $ 0.000$ \\
& $1000$  & $0.0$  & $-0.002$  & $-0.001$  & $ 0.000$  & $ 0.000$ \\
\midrule
\multirow{3}{*}{$D(z)$}
& $10$    & $0.4$  & $0.1$  & $0.1$  & $0.07$  & $0.07$ \\
& $100$   & $0.4$  & $0.01$  & $0.01$  & $0.006$  & $0.006$ \\
& $1000$  & $0.4$  & $0.001$  & $0.001$  & $0.001$  & $0.001$ \\
\bottomrule
\end{tabular}
\label{table:laplace}
\end{table}

\begin{table}[H]
\centering
\caption{Распределение Пуассона}
\begin{tabular}{c*6r}
\toprule
{} &         
$n$ & 
\centering $z_R$ & 
\centering $z_Q$ &
\centering $\overline{x}$& 
\centering $med(x)$ &
\centering $z_{tr}$ \tabularnewline
\midrule
\multirow{3}{*}{$\overline{z}$}
& $10$    & $ 0.0$  & $ 0.0$  & $ 0.0$  & $ 0.0$  & $ 0.0$ \\
& $100$   & $ 0.0$  & $-0.02$  & $ 0.00$  & $-0.01$  & $-0.00$ \\
& $1000$  & $-0.02$  & $-0.002$  & $-0.001$  & $-0.002$  & $-0.001$ \\
\midrule
\multirow{3}{*}{$D(z)$}
& $10$    & $0.2$  & $0.2$  & $0.1$  & $0.2$  & $0.2$ \\
& $100$   & $0.1$  & $0.02$  & $0.01$  & $0.02$  & $0.02$ \\
& $1000$  & $0.06$  & $0.001$  & $0.001$  & $0.002$  & $0.002$ \\
\bottomrule
\end{tabular}
\label{table:poisson}
\end{table}

\begin{table}[H]
\centering
\caption{Равномерное распределение}
\begin{tabular}{c*6r}
\toprule
{} &         
$n$ & 
\centering $z_R$ & 
\centering $z_Q$ &
\centering $\overline{x}$& 
\centering $med(x)$ &
\centering $z_{tr}$ \tabularnewline
\midrule
\multirow{3}{*}{$\overline{z}$}
& $10$    & $ 0.00$  & $ 0.0$  & $ 0.0$  & $ 0.0$  & $ 0.0$ \\
& $100$   & $ 0.000$  & $-0.02$  & $-0.001$  & $0.00$  & $ 0.000$ \\
& $1000$  & $ 0.000$  & $-0.001$  & $ 0.000$  & $-0.001$  & $ 0.000$ \\
\midrule
\multirow{3}{*}{$D(z)$}
& $10$    & $0.04$  & $0.2$  & $0.1$  & $0.3$  & $0.2$ \\
& $100$   & $0.001$  & $0.02$  & $0.01$  & $0.03$  & $0.02$ \\
& $1000$  & $0.001$  & $0.002$  & $0.001$  & $0.003$  & $0.002$ \\
\bottomrule
\end{tabular}
\label{table:uniform}
\end{table}
\newpage
\subsection{Боксплот Тьюки и выбросы}
Ниже приведены диаграммы для выборок мощности 20 элементов (слева) и 100 (справа). 
\begin{figure}[H]
	\centering \includegraphics[width=\myPictWidth]{norm.jpg}
	\caption{Нормальное распределение}
	\label{img:norm}
\end{figure}
\begin{figure}[H]
	\centering \includegraphics[width=\myPictWidth]{cauchy.jpg}
	\caption{Распределение Коши}
	\label{img:cauchy}
\end{figure}
\begin{figure}[H]
	\centering \includegraphics[width=\myPictWidth]{laplace.jpg}
	\caption{Распределение Лапласа}
	\label{img:laplace}
\end{figure}
\begin{figure}[H]
	\centering \includegraphics[width=\myPictWidth]{uniform.jpg}
	\caption{Равномерное распределение}
	\label{img:uniform}
\end{figure}
\begin{figure}[H]
	\centering \includegraphics[width=\myPictWidth]{poisson.jpg}
	\caption{Распределение Пуассона}
	\label{img:poisson}
\end{figure}
\newpage

Результаты теоретических расчётов и экспериментов сведены в таблицу ниже.
\begin{table}[H]
\centering
\caption{Доля выбросов $\eta$}
\begin{tabular}{*6r}
	\toprule
	\multirow{3}{*}{Распределение}&
	\multirow{3}{*}{Вер. выброса}&
	\multicolumn{4}{r}{Средняя доля выбросов}\tabularnewline
	& & \multicolumn{2}{r}{20 эл-тов} & \multicolumn{2}{r}{100 эл-тов}\\
	& & $\overline{\eta}$ & $D\eta$ & $\overline{\eta}$ & $D\eta$ \\
	\midrule
	Нормальное  & $0.0070$ & $0.018$ & $0.002$ & $0.009$ & $0.002$ \\
	Коши        & $0.1560$ & $0.142$ & $0.005$  & $0.153$ & $0.001$ \\
	Лапласа     & $0.0625$ & $0.066$ & $0.004$ & $0.064$ & $0.001$ \\
	Равномерное & $0$ & $0.0020$ & $0.0001$ & $0$  & $0$\\
	Пуассона    & $0.0077$ & $0.020$ & $0.001$ & $0.0120$ & $0.0003$ \\
	\bottomrule
\end{tabular}
\label{table:outliers}
\end{table}

\newpage
\section{Обсуждение}
\subsection{Характеристики положения и рассеяния}
Можно выделить некоторые общие закономерности в результатах для всех распределений, \textbf{кроме Коши} (табл. \ref{table:cauchy}):
\begin{enumerate}
	\item Среднее значение среднего значения $\overline{x}$ по модулю находится в пределах $0.008$ для выборок размера $10$,с ростом числа выборки убывает или, по меньшей мере, не возрастает (т. о. стремится к медиане распределения: у каждого распределения медиана $0$). Выборочная дисперсия среднего значения убывает всякий раз примерно в десять раз: можно сказать, что средняя плотность значений $\overline{x}$ пропорциональна размеру выборки.
	\item Среднее значение усечённого среднего обычно по модулю превышает среднее значение среднего значения. Это согласуется с тем, что модуль среднего значения в вероятностном смысле убывает с ростом числа элементов, ведь при расчёте усечённого среднего используется информация не о всех точках. Дисперсия усечённого среднего с увеличением размеров выборки вдесятеро тоже убывает всякий раз примерно в десять раз, причём, хотя модуль дисперсии больше аналогичных модулей дисперсии для обычных средних значений, видно, что отношение значения модуля предыдущего с следующему ближе к $10$: не учитывая выбросы, мы получаем более предсказуемое поведение среднего значения в зависимости от размера выборки. 
	\item Похожие значения у среднего и дисперсии выборочной медианы: среднее значение убывает с ростом числа элементах в выборке, но медленнее чем линейно, дисперсия обратно пропорциональна числу элементов в выборках. Иначе говоря, чем больше элементов в выборке, тем выше вероятность встретить значение вблизи матожидания распределения.
	\item Полусумма экстремальных значений $Z_R$ - одна из наименее стабильных характеристика среднего значения (зато, пожалуй, её вычисление обходится дешевле всех остальных). Интуитивно это можно объяснить тем, что с увеличением числа элементов в выборке для распределения Гаусса (табл. \ref{table:norm}), Лапласа (табл. \ref{table:laplace}) и Пуассона (табл. \ref{table:poisson}) возрастает вероятность встретить значения сколь угодно далеко от матожидания.  У равномерного распределения (табл. \ref{table:uniform}) матожидание и дисперсия $Z_R$, впрочем, при большом числе элементов в выборке оказалось равным нулю с точностью до третьего знака после запятой. Это объясняется ограниченностью возможных значений случайной величины: с ростом числа элементов в выборке крайние значения всё ближе к $-\sqrt{3}$ и $\sqrt{3}$ соответственно.
	\item Полусумма квартилей отличается от полусуммы экстремальных значений примерно так же, как усечённое среднее от обычного выборочного среднего: значения разбросаны в более широком диапазоне, но зато этот диапазон лучше коррелирует с размером выборки.
\end{enumerate}
Особняком стоят результаты для распределения Коши, которое не имеет дисперсии; случайная величина, распределённая по этому закону, не подчиняется центральной предельной теореме. Видно, что среднее и дисперсия среднего значения и полусуммы экстремальных значений возрастает с ростом числа выборки. Среднее значение медианы, полусуммы квартилей и усечённого среднего, впрочем,  с ростом числа элементов уменьшается (и дисперсия этих статистик обратно пропорциональна числу элементов). Можно дать следующую интерпретацию: доля выбросов с увеличением числа элементов в выборке падает, но зато эти выбросы появляются настолько далеко от матожидания распределения, что характеристики, не отбрасывающие крайние значения, плохо оценивают положение матожидания.

\subsection{Боксплот Тьюки и выбросы}
Почти все полученные диаграммы (для распределения Гаусса (\ref{img:norm}), Лапласа (\ref{img:laplace}) и Пуассона (\ref{img:poisson})) демонстрируют похожие результаты, характерные для унимодальных распределений, подчиняющихся закону больших чисел: границы <<ящика>> примерно одинаковые для малой и большой выборки, они <<плавают>> даже меньше, чем положение медианы; вместе с тем растёт и число выбросов, а также их удалённость от центра. Таким образом, диаграмма позволяет хорошо оценить и визуализировать свойства распределения даже по относительно малой выборке.  Видно, что распределение Коши (\ref{img:cauchy}), не имеющее конечной дисперсии, даёт выбросы, наиболее удалённые от статистически значимой части.\\
Похожие результаты наблюдаются для равномерного распределения (\ref{img:uniform}), за тем лишь исключением, что число выбросов с увеличением размера выборки снижается. Это объясняется ограниченностью множества допустимых значений: в теории при равномерном распределении границы <<усов>> должны совпадать с границами отрезка (при объёме выборки $n \rightarrow \infty$).\\
Результаты расчёта усреднённой вероятности выбросов таковы: согласно таблице (\ref{table:outliers}), экспериментально полученная доля выбросов с ростом числа элементов в выборке стремится к теоретической для всех распределений без исключения, но для распределения Коши снизу, а для остальных сверху (видно, что разница между теоретическим значением и измеренным систематическая и не покрывается дисперсией). По-видимому, при малом числе элементов характер распределения ещё выражен неярко, т. е. меньше информации о том, где должна находиться медиана и основная доля элементов. Поэтому <<ящик>> более узкий, чем в теории, для распределений с конечной дисперсией, и более широкий для распредления Коши.\\  

\newpage
\section{Ссылки}
\begin{enumerate}
	\item \href{https://github.com/zuevval/source/tree/master/matlab/statistics2020/lab2variance}{github.com/zuevval/source/tree/master/matlab/statistics2020/lab2variance} \label{link:lab2}
	
	\item \href{https://github.com/zuevval/source/tree/master/matlab/statistics2020/lab3boxplot}{github.com/zuevval/source/tree/master/matlab/statistics2020/lab3boxplot} \label{link:lab3}
\end{enumerate}
\end{document}