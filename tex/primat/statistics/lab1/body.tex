\documentclass[report1.tex]{subfiles}
\begin{document}

\section{Постановка задачи}
Требуется исследовать следующие распределения:
\begin{enumerate}
	\item равномерное
	\item Гаусса (нормальное)
	\item Коши
	\item Лапласа
	\item биномиальное.
\end{enumerate}
Для каждого распределения сгенерировать выборки из 10, 50 и 1000 значений случайных величин, распределённых согласно соответствующему закону. Построить гистограммы распределения и на изображения гистограмм наложить графики плотностей вероятности соответствующих распределений.
\newpage
\section{Теория}
Ниже приведены математические описания распределений.\\
\emph{Равномерное} -- непрерывное распределение с параметрами $a=-\sqrt{3}$, $b=\sqrt{3}$ и плотностью вероятности
\begin{equation}\label{eq:uniform}
f(x)=\frac{1}{b-a} = \frac{1}{2\sqrt{3}}
\end{equation}
\emph{Нормальное} -- непрерывное с параметрами $\mu=0$, $\sigma=1$ и плотностью
\begin{equation}\label{eq:norm}
f(\xi=x)=\frac{e^{-\frac{x^2}{2\sigma^2}}}{2\pi\sigma} = \frac{e^{-\frac{x^2}{2}}}{2\pi}
\end{equation}
\emph{Коши} -- непрерывное с параметрами $\alpha=0$, $\lambda=1$ и плотностью
\begin{equation}\label{eq:cauchy}
f(\xi=x)=\frac{1}{\pi} \frac{\lambda}{\lambda^2+(x-\alpha)}=\frac{1}{\pi(x+1)}
\end{equation}
\emph{Лапласа} -- также непрерывное с параметрами $\alpha=0$, $\lambda=\frac{1}{\sqrt{2}}$ и плотностью
\begin{equation}\label{eq:laplace}
f(\xi=x)=\frac{\lambda}{2} e^{-\lambda |x-\alpha|}=\frac{1}{2\sqrt{2}} e^{-\sqrt{2}x}
\end{equation}
\emph{Пуассона} -- целочисленное с параметром $\lambda=10$, соотсветственно, с рядом распределения
\begin{equation}\label{eq:poisson}
P(\xi=k)=\frac{\lambda^k}{k!} e^{-\lambda}=\frac{10^k}{k!} e^{-10}
\end{equation}
Все непрерывные распределения определены на вещественной оси $x\in \mathds{R}$, распределение Пуассона - на множестве целых чисел $k\in\mathds{N}$.\\


\newpage
\section{Реализация}
Программа, генерирующая наборы псевдослучайных величин с заданными распределениями, составлена на языке MATLAB\textsuperscript{\textregistered}. Были использованы встроенные функции генерации случайных чисел \texttt{normrnd} (для нормального распределения), \texttt{trnd} (распределение Коши), \texttt{rand} (равномерное распределение).\\
Для остальных распределений генерируется псевдослучайная выборка, распределённая согласно равномерному закону, после чего к полученному массиву чисел поэлементно применяется обратная функция распределения. Получаемый в результате набор значений распределён по тому закону, обратную функцию которого применили к равномерно распределённым числам.\\
Гистограммы, графики плотностей распределения и график ряда распределения Пуассона были также составлены средствами MATLAB\textsuperscript{\textregistered} Исходный код доступен по ссылке \url{https://github.com/zuevval/source/blob/master/matlab/statistics2020/lab1dists/src/lab1distributions.m} (дата обращения: 09 марта 2020 года).

\newpage
\section{Результаты}
Ниже приведены иллюстрации к исследуемым распределениям: плотность вероятности (масштаб по оси ординат отмечен на шкале справа) и гистограмма, построенная по сгенерированной выборке (шкала слева). Каждому распределению сопоставлены три иллюстрации с размерами выборки 10, 50 и 1000 псевдослучайных значений соответственно.\\
В случае распределения Пуассона (рис. \ref{poissonPict}) вместо плотности вероятности показаны значения ряда распределения (ф-ла \ref{eq:poisson}).
\begin{figure}[H]
	\centering \includegraphics[width=\myPictWidth]{uniform.jpg}
	\caption{Равномерное распределение (см. формулу \ref{eq:uniform})}
	\label{uniformDistPict}
\end{figure}
\begin{figure}[H]
	\centering \includegraphics[width=\myPictWidth]{norm.jpg}
	\caption{Распределение Гаусса  (см. формулу \ref{eq:norm})}
	\label{gaussPict}
\end{figure}
\begin{figure}[H]
	\centering \includegraphics[width=\myPictWidth]{cauchy.jpg}
	\caption{Распределение Коши (см. формулу \ref{eq:cauchy})}
	\label{cauchyPict}
\end{figure}
\begin{figure}[H]
	\centering \includegraphics[width=\myPictWidth]{laplace.jpg}
	\caption{Распределение Лапласа (см. формулу \ref{eq:laplace})}
	\label{laplacePict}
\end{figure}
\begin{figure}[H]
	\centering \includegraphics[width=\myPictWidth]{poisson.jpg}
	\caption{Распределение Пуассона (см. формулу \ref{eq:poisson})}
	\label{poissonPict}
\end{figure}

\newpage
\section{Обсуждение}
Встроенный в программную среду генератор псевдослучайных чисел работает так, что в полученных выборках каждый элемент можно считать значением случайной величины в серии независимых одинаково распределённых величин. По следствию из теоремы Чебышёва такая последовательность подчиняется закону больших чисел, если дисперсия случайной величины конечна \cite{sevastianov}. Все законы распределения, кроме Коши, таковы, что соответствующие дисперсии определены и конечны \cite{koroluk}. Говоря интуитивным языком, во всех случаях, кроме распределения Коши, с ростом числа элементов вероятность отклонения среднего значения от математического ожидания стремится к нулю На графиках, соответствующих распределениям с конечной дисперсией, видно, что данные располагаются в той области вещественной оси, где плотность <<велика>> (или, в случае распределения Пуассона, там, где <<велика>> вероятность появления). Но в случае распределения Коши даже в большой выборке среднее значение может существенно отличаться от математического ожидания. На графиках распределения Коши (\ref{cauchyPict}) видно, что гистограмма с ростом объёма выборки по форме приближается к кривой плотности распределения, но вместе с тем появляются значительные краевые выбросы.
\end{document}