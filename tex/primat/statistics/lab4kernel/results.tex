\documentclass[main.tex]{subfiles}
\begin{document}

\subsection{Эмпирическая функция распределения}
Результаты представлены на графиках.
\begin{figure}[H]
	\centering \includegraphics[width=\myPictWidth]{norm_cdf.jpg}
	\caption{Нормальное распределение. Эмпирическая функция распределения}
	\label{img:norm_cdf}
\end{figure}
\begin{figure}[H]
	\centering \includegraphics[width=\myPictWidth]{cauchy_cdf.jpg}
	\caption{Распределение Коши. Эмпирическая функция распределения}
	\label{img:cauchy_cdf}
\end{figure}
\begin{figure}[H]
	\centering \includegraphics[width=\myPictWidth]{laplace_cdf.jpg}
	\caption{Распределение Лапласа. Эмпирическая функция распределения}
	\label{img:laplace_cdf}
\end{figure}
\begin{figure}[H]
	\centering \includegraphics[width=\myPictWidth]{uniform_cdf.jpg}
	\caption{Равномерное распределение. Эмпирическая функция распределения}
	\label{img:uniform_cdf}
\end{figure}
\begin{figure}[H]
	%\centering \includegraphics[width=\myPictWidth]{poisson_cdf.jpg}
	\caption{Распределение Пуассона. Эмпирическая функция распределения}
	\label{img:poisson_cdf}
\end{figure}

\subsection{Ядерные оценки плотности}

\begin{figure}[H]
	\centering \includegraphics[width=\myPictWidth]{norm_kde20.jpg}
	\caption{Нормальное распределение (20 элементов в выборке)}
	\label{img:norm_kde20}
\end{figure}
\begin{figure}[H]
	\centering \includegraphics[width=\myPictWidth]{norm_kde60.jpg}
	\caption{Нормальное распределение (60 элементов в выборке)}
	\label{img:norm_kde60}
\end{figure}
\begin{figure}[H]
	\centering \includegraphics[width=\myPictWidth]{norm_kde100.jpg}
	\caption{Нормальное распределение (100 элементов в выборке)}
	\label{img:norm_kde100}
\end{figure}

\begin{figure}[H]
	\centering \includegraphics[width=\myPictWidth]{cauchy_kde20.jpg}
	\caption{Распределение Коши (20 элементов в выборке)}
	\label{img:cauchy_kde20}
\end{figure}
\begin{figure}[H]
	\centering \includegraphics[width=\myPictWidth]{cauchy_kde60.jpg}
	\caption{Распределение Коши (60 элементов в выборке)}
	\label{img:cauchy_kde60}
\end{figure}
\begin{figure}[H]
	\centering \includegraphics[width=\myPictWidth]{cauchy_kde100.jpg}
	\caption{Распределение Коши (100 элементов в выборке)}
	\label{img:cauchy_kde100}
\end{figure}

\begin{figure}[H]
	\centering \includegraphics[width=\myPictWidth]{laplace_kde20.jpg}
	\caption{Распределение Лапласа (20 элементов в выборке)}
	\label{img:laplace_kde20}
\end{figure}
\begin{figure}[H]
	\centering \includegraphics[width=\myPictWidth]{laplace_kde60.jpg}
	\caption{Распределение Лапласа (60 элементов в выборке)}
	\label{img:laplace_kde60}
\end{figure}
\begin{figure}[H]
	\centering \includegraphics[width=\myPictWidth]{laplace_kde100.jpg}
	\caption{Распределение Лапласа (100 элементов в выборке)}
	\label{img:laplace_kde100}
\end{figure}

\begin{figure}[H]
	\centering \includegraphics[width=\myPictWidth]{uniform_kde20.jpg}
	\caption{Равномерное распределение (20 элементов в выборке)}
	\label{img:uniform_kde20}
\end{figure}
\begin{figure}[H]
	\centering \includegraphics[width=\myPictWidth]{uniform_kde60.jpg}
	\caption{Равномерное распределение (60 элементов в выборке)}
	\label{img:uniform_kde60}
\end{figure}
\begin{figure}[H]
	\centering \includegraphics[width=\myPictWidth]{uniform_kde100.jpg}
	\caption{Равномерное распределение (100 элементов в выборке)}
	\label{img:uniform_kde100}
\end{figure}

\begin{figure}[H]
	%\centering \includegraphics[width=\myPictWidth]{poisson_cdf.jpg}
	\caption{Распределение Пуассона}
	\label{img:poisson}
\end{figure}
	
\end{document}