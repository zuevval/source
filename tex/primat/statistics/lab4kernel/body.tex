\documentclass[main.tex]{subfiles}
\begin{document}
\section{Постановка задачи}
Необходимо исследовать перечисленные ниже распределения с параметрами, аналогичными таковым в лабораторной работе \textnumero 1:
\begin{enumerate}
	\item равномерное
	\item Гаусса (нормальное)
	\item Коши
	\item Лапласа
	\item биномиальное.
\end{enumerate}
Для каждого из них сгенерировать выборки размера $20$, $60$ и $100$ элементов, после чего для каждой выборки
\begin{itemize}
	\item Построить в одной координатной плоскости изображения графиков эмпирической функции распределения и теоретической функции распределения (в интервале $[-4;4]$ для непрерывных распределений и $[4;16]$ для распределения Пуассона).
	\item В одних и тех же осях построить изображение ядерной оценки плотности распределения по выборке с гауссовым ядром с различными значениями параметра сглаживания $h_n$. 
\end{itemize}

\newpage
\section{Теория}
\subsection{Эмпирическая функция распределения}
Эмпирическая фукнция распределения, полученная по данной выборке
\begin{equation}
	F^*(x|{x_i}_{i=1}^{n}) = P^*(X < x| \{x_i\}_{i=1}^n)
\end{equation}
где $P^*$ -- относительная частота события $X < x$. Тогда, если $\{z_i\}_{i=1}^n$ -- статистический ряд и $\{n_i\}$ -- относительные частоты появления соответствующих событий, можно вычислить $F^*(x)$ по формуле, выведенной в \cite{maksimov_book}:
\begin{equation}
	F^*(x|{x_i}_{i=1}^{n}) = \frac{1}{n} \sum_{z_i < x} n_i
\end{equation}
где $n$ -- мощность выборки.
\subsection{Ядерные оценки плотности}
Оценка плотности вероятности -- это функция $\hat{f}(x) \approx f(x)$, построенная на основе выборки. Например, оценкой плотности может служить ступенчатая линия, построенная по гистограмме распределения.\\
Оценку плотности вероятности, записанную в виде
\begin{equation}\label{eq:kernel}
	\hat{f}(x) = \frac{1}{nh_n} \sum_{i=1}^{n} K\left(\frac{x-x_i}{h_n}\right)
\end{equation}
называют \emph{ядерной} оценкой плотности. Здесь $K$ -- непрерывная функция, именуемая \emph{ядром}, $h_n$ -- параметр сглаживания.\\
В данной работе используется так называемое гауссово ядро:
\begin{equation}\label{eq:gauss-kernel}
K(u) = \frac{1}{\sqrt{2\pi}} \exp(-\frac{u^2}{2})
\end{equation}
Для ядра (\ref{eq:gauss-kernel}) известно рекомендуемое значение параметра сглаживания $h_n$ \cite{maksimov_book}:
\begin{equation}\label{eq:bandwidth}
	h_n^{opt} = \left(\frac{4 \hat{\sigma}^5}{3n}\right)^{\frac{1}{5}}
\end{equation}
где $\hat{\sigma}$ -- стандартное отклонение элементов выборки.

\newpage
\section{Реализация}
Исходный код программ (см. ссылку [\ref*{link:lab4}]) для построения эмпирической функции распределения, ядерных оценок и создания иллюстраций написан на языке MATLAB. Эмпирическая функция распределения найдена с помощью функции \textit{ecdf}, доступную в пакете MATLAB. Оптимальный параметр сглаживания \eqref{eq:bandwidth} $h_n^{opt}$ найден с использованием стандартной функции \textit{std(x)}, вычисляющей стандартное отклонение. Построены ядерные оценки с параметрами сглаживания $0.5\cdot h_n^{opt}, h_n^{opt}, 2 \cdot h_n^{opt}$ по формулам \eqref{eq:gauss-kernel}, \eqref{eq:kernel}.

\newpage
\section{Результаты}
\subfile{results}

\newpage
\section{Обсуждение}
На изображениях эмпирических функций распределения видно, что для каждого распределения ломаная приближается к кривой функции распределения, находясь попеременно сверху и снизу. С увеличением числа элементов близость выборочной функции распределения к теоретической возрастает.\\
Ядерные оценки плотности распределения также тем лучше согласуются с плотностью, чем больше в выборке элементов. Когда параметр $h_n$ меньше оптимального, видно, что оценка плотности учитывает отдельные точки и плохо отражает  поведение распределения. Когда параметр $h_n$ больше оптимального, кривая плотности становится слишком гладкой, что особенно заметно в случае равномерного распределения. Впрочем, для распределения Пуассона оценка с параметром $2\cdot h_n^{opt}$ выглядит более правдоподобно, чем $h_n^{opt}$, что, по всей видимости, связано с тем, что значение $h_n^{opt}$ выведено для равномерных распределений, где точки при большом числе элементов в выборке отстоят недалеко друг от друга.\\
Надо отметить важность построения ядерных оценок не по всей выборке, а по подмножеству с исключёнными крайними точками (лежащими за пределами интервала $[-4;4]$ для равномерных распределений и $[6;14]$ для распределения Пуассона). Если делать построение по всем значениям в выборке, то, к примеру, оценка плотности распределения Коши, где выбросы встречаются далеко от центра, будет с ростом числа элементов стремиться не к теоретической плотности, а к линии $x=0$ из-за очень большого значения параметра $h_n$.

\newpage
\section{Ссылки}
\begin{enumerate}
	\item \href{https://github.com/zuevval/source/tree/master/matlab/statistics2020/lab4kernel}{github.com/zuevval/source/tree/master/matlab/statistics2020/lab4kernel} \label{link:lab4}
\end{enumerate}

\end{document}