\documentclass[main.tex]{subfiles}
\begin{document}
\section{Постановка задачи}
Сгенерировать двумерные выборки размерами 20, 60, 100 для нормального двумерного распределения $N(x, y, 0, 0, 1, 1, \rho)$, т. е.$\mu_x = \mu_y = 0$, $\sigma_x =\sigma_y = 0$.
Коэффициент корреляции $\rho$ взять равным последовательно $0, 0.5, 0.9$.\\
Каждую выборку сгенерировать 1000 раз, всякий раз вычислить коэффициенты корреляции Пирсона, Спирмена, а также квадрантный коэффициент корреляции. Для получившихся 1000 значений каждого параметра рассчитать:
\begin{enumerate}
	\item Среднее значение
	\item Среднее значение квадрата величины
	\item Дисперсию.
\end{enumerate}
То же проделать для распределения с плотностью
\begin{equation}\label{eq:mix}
f(x,y)= 0.9 f_N(x,y,0,0,1,1,0.9) + 0.1 f_N(x,y,0,0,10,10,-0.9)
\end{equation}

Один раз из тысячи сгеренированные точки изобразить на плоскости и обозначить эллипс рассеяния.

\newpage
\section{Теория}
Двумерное нормальное распределение \cite{sevastianov} -- непрерывное распределение с параметрами $m_x$, $m_y$, $\sigma_x$, $\sigma_y$, $\rho$ и с плотностью \eqref{eq:pdf_general}:
\begin{equation}\label{eq:pdf_general}
\begin{split}
f_N(x,y) = &: \frac{1}{2\pi\sigma_x \sigma_y \sqrt{1-\rho^2}} \times \\
& \times \exp \left( - \frac{1}{2(1-\rho^2)} \left[\frac{(x-m_x)^2}{\sigma_x^2} - 2 \rho \frac{(x-m_x)(y-m_y)}{\sigma_x \sigma_y} + \frac{(y-m_y)^2}{\sigma_y^2}\right]\right)
\end{split}
\end{equation}
В частном случае при $m_x=0$, $m_y=0$, $\sigma_x=1$ и $\sigma_y=1$ плотность принимает вид \eqref{eq:pdf_our}
\begin{equation}\label{eq:pdf_our}
	f_N(x,y) = \frac{1}{2\pi \sqrt{1-\rho^2}} \exp \left( - \frac{1}{2(1-\rho^2)} \left[x^2 - 2 \rho xy + y^2\right]\right)
\end{equation}

\emph{Выборочный коэффициент корреляции} (коэффициент корреляции Пирсона) \cite{sevastianov} -- показатель, характеризующий степень линейной зависимости между двумя выборками данных (аналог коэффициента корреляции двух случайных величин). Он вычисляется по формуле
\begin{equation}\label{eq:pearson}
	r_{XY} = \frac{\frac{1}{n} \sum (x_i- E_x)(y_i-E_y)
	}{\sqrt{\frac{1}{n^2} \sum (x_i-E_x)^2 \sum (y_i-E_y)^2}}
\end{equation}
где $E_x$, $E_y$ -- соответствующие матожидания, $n$ -- мощность выборки.\\

Есть и другие способы оценки взаимосвязи между случайными величинами \cite{sevastianov}, например \emph{выборочный квадрантный коэффициент корреляции}
\begin{equation}\label{eq:quad_coeff}
	r_Q = \frac{1}{n} \sum_{i=1}^n sign(x_i - med(x)) sign(y_i - med(y))
\end{equation}
и \emph{выборочный коэффициент ранговой корреляции Спирмена}
\begin{equation}\label{eq:spearman}
	r_S = 1 - \frac{6 \sum (x_i-y_i)^2}{n(n^2-1)}
\end{equation}

\emph{Эллипсами рассеяния} случайной величины называют линии уровня функции $f_N(x,y)$. Очевидно, их уравнения при указанных выше значениях параметров имеют вид
\begin{equation}
	x^2 - 2 \rho xy + y^2 = const
\end{equation}
что и объясняет слово <<эллипс>> в названии. Используя аппарат аналитической геометрии, можно убедиться в том, что оси эллипса повёрнуты на $\frac{\pi}{4}$ относительно координатных осей, а отношение их длин составляет $\sqrt{\frac{1+\rho}{1-\rho}}$.

\newpage
\section{Реализация}
Программа, генерирующая данные, рассчитывающая коэффициенты корреляции и осуществляющая построение графиков, написана на языке MATLAB.  Коэффициент корреляции Пирсона \eqref{eq:pearson}  и Спирмана \eqref{eq:spearman} вычислены встроенными функциями \textit{corrcoef} и \textit{corr}, для квадрантного коэффициента корреляции написан метод \textit{quadrant\_corrcoef}.\\ 

Выборки, элементы которых распределены по нормальному закону, сгенерированы встроенной функцией \textit{mvrnd}. Выборка, элементы которой распределены по закону \eqref{eq:mix}, смоделирована так: сгенерированы выборки с плотностями $f_N^{1}(x,y,0,0,1,1,0.9)$ и \linebreak $f_N^2(x,y,0,0,10,10,-0.9)$, после чего в результирующую выборку  с вероятностью $0.9$ включается элемент из первой, $0.1$ -- из второй. \\
Код программы доступен по ссылке [\ref{link:lab5}].


\newpage
\section{Результаты}
\subfile{results}

\newpage
\section{Обсуждение}
Квадрантный коэффициент корреляции, как может показаться, должен быь более грубой оценкой зависимости между случайными величинами, чем коэффициенты Пирсона или Спирмена: <<пятно>>, образованное точками выборки, может принимать самые разнообразные симметричные и асимметричные формы при одном и том же квадрантном коэффициенте корреляции. Тем не менее, по данным из таблицы \ref{table:quadrant} видно, что квадрантный коэффициент даёт правдоподобные оценки зависимости величин: более <<вытянутые>> выборки соответствуют большему среднему значению (и квадрату среднего значения) коэффициента, дисперсия мала (иногда меньше соответствующей дисперсии коэффициента корреляции Пирсона и Спирмана) и уменьшается с ростом числа элементов в выборке. По-видимому, дело в том, что нормальное распределение относительно простое и не имеет асимметричных <<хвостов>> в разных квадрантах.\\
Дисперсии всех трёх коэффициентов относительно высоки в случае <<смешанного>> распределения с плотностью \eqref{eq:mix}. Видимо, причина в больших значениях коэффициентов $\sigma_x$, $\sigma_y$ одного из двух слагаемых в плотности вероятности.

\end{document}