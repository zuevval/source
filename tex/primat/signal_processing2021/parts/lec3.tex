\documentclass[main.tex]{subfiles}
\begin{document}
	
\section{Лекция 3. Идентификация объектов. Визуальные признаки, особые точки, дескрипторы}

18 февраля 2021 года.

\textbf{Важная и сложная лекция}.
В ней рассказывается основное, что нужно для большой лабораторной.
\subsection{Сравнение по наборам признаков}
\subsubsection{Сравнение фрагментов сигнала}

Исходно считаем, что решаем задачу: содержит ли сигнал какой-то объект (фрагмент сигнала)?

Нужно учесть повороты, искажения, возможные перекрытия и т. д.

Можно искать края (лекция 1), объекты по шаблону (лекция 2).
Но для разных поворотов и искажений это может выглядеть по-разному.
Конечно, шаблоны/геометрические примитивы можно искать в разных масштабах, с разным поворотом и так далее, но это, скорее, перебор (bruteforce).

Хочется выделить \emph{признаки} изображения, которых не так много, но которые хорошо идентифицируют ответ.

\subsubsection{Визуальные признаки}

\emph{Визуальные признаки} - то, что определяется цветом изображения.

Пример.
На фрагменте изображения, где присутствует крокодил, должно быть много зелёного, также, возможно, должно быть что-то белое и красное, содержание белого и красного подчинено какому-то статистическому закону (например, нормально распределено).

Какие характеристики распределений можно использовать для их идентификации на изображении?
Выборочные оценки матожидания и последующих моментов в распределении цветов: не очень хорошо работает, меняются от крокодила к крокодилу.

\emph{Моменты Ху (Hou)}:
$$ \nu_{ij} = \frac{\mu_{ij}}{\mu_{00}^{\left(1 + \frac{i+j}{2}\right)}} $$
где 
$$\mu_{pq} = \sum_{m, p} \sum_{n, q} 
\begin{pmatrix} p \\ m \end{pmatrix}
\begin{pmatrix} q \\ n \end{pmatrix}
(-\bar x)^{(p-m)}(-\bar y)^{(q-n)} M_{mn}, \thickspace
M_{ij} = \sum_x\sum_y x^i y^j I(x,y) $$
Долго разрабатывались.
Считается, что они инварианты к параллельному переносу, сдвигу и масштабированию.
Но тоже не очень хорошо работают и сложны.

А что работает?
Работают цветовые гистограммы (и их вычислять проще моментов).
Цвета обычно дискретизированы целыми числами от 0 до 255.
Считаем, сколько пикселей имеют цвет 0, 1, ..., отображаем столбцы пропорциональной значению высоты.
Если гистограмма распределения интенсивности цвета похожа на гистограмму шаблона, считаем, что шаблон присутствует на картинке.

Как сравнить две гистограммы?
Ведь изображения одного предмета с разными освещённостями могут содержать совсем разные цвета.
\begin{enumerate}[noitemsep]
	\item Нормализация: сделать так, чтобы площадь под графиком была одинаковая
	\item Квантуем цвета (если 256 цветов на картинке, то неразумно сравнивать 256 гистограмм, поэтому объединяем соседние цвета, например, для)
	\item Критерии, которые работают:
	\begin{enumerate}[noitemsep]
		\item пересечение гистограмм 
		$$ histint(h_i, h_j) = 1 - \sum_{m=1}^K \min (h_i(m), h_j(m)) $$
		\item критерий на основе хи-квадрат
		$$ \chi^2(h_i,h_j) = \frac{1}{2} \sum_{m=1}^{K} \frac{[h_j(m)-h_j(m)]^2}{h_i(m)+h_j(m)} $$
	\end{enumerate}
	
\end{enumerate}

При квантовании может помочь кластеризация, чтобы сгруппировать цвета по близости.
Например, можно задать сетку, которая в первом приближении поделит гамму на группы; если в групе мало цветов, объединим её с соседней.
(Но это очень приближённый алгоритм, подробнее поговорим в лекции 6).

\subsubsection{Недостатки цветовых гистограмм}

\textbf{Не учитывается подобие цветов:} на картинке со слайда (где $H_2$ -- смещённая гистограмма $H_1$)  для любой из двух предложенных выше метрик $d(H_1, H_2) > d(H_1, H_3)$

Проблема решается введением \emph{матрицы подобия} $A$.
Её элемент $a_{ij}$ -- то, насколько цвет $i$ подобен цвету $j$.
Тогда расстояние 
$$ d(H_1, H_2) = \sqrt{(H_1 - H_2) \cdot A \cdot (H_1 - H_2)^T} $$

\textbf{Гистограммы разных объектов могут совпадать:} например, у объектов с разной геометрией, но одинаковой площадью фигур.

Что можно сделать?
Разбить объекты на регионы, сравнивать гистограммы по регионам и надеяться, что разбивка решит проблему.

\subsubsection{Цветовая гистограмма с информацией о пространственном расположении цветов}

Интересный инструмент.
Идея: храним тройки значений <<координата центра масс пикселей с некоторым цветом; количество пикселей этого цвета; сам цвет>>.
Получаются точки, распределённые в двумерном пространстве.
Есть метрика, которая успешно сравнивает два таких множества точек.

\subsubsection{Визуальные признаки. Текстуры}

\emph{Текстура} отражает что-то повторяющееся на нашей картинке, как правило, какое-то свойство поверхности: зернистость, контрастность, направленность, регулярность...
Хотим все эти характеристики перевести в числовые признаки.

Есть алгоритм GLM -- Grey Level Co-Occurence Matrices:

\begin{enumerate}[noitemsep]
    \item Кластеризуем цвета, получаем множество усреднённых цветов -- кластеров $ G $.
    Стоит делать совсем немного, 3-5 групп.
    \item Для каждой пары цветов из $ G $ считаем частоту встречи этой пары в нескольких заранее определённых последовательностях $I(p, q), I(p + \Delta x, q + \Delta y)$.
    Последовательности надо подбирать в зависимости от случая, по-хорошему их надо обучать.
    Например, правый-левый пиксель, или правый верхний-левый нижний... (Соответствует значениям $\Delta x, \Delta y$).
    
    \item Вычислив частоту встречи признака на картинке, считаем матрицы, которые характеризуют текстуры:
    $$ C(i,j) = \sum_{p=1}^{N} \sum_{q=1}^{M} \begin{cases}
    1, \text{ если} I(p, q)=i, I(p + \Delta x, q + \Delta y)=j \\
    0 \text{ иначе}
    \end{cases} $$
    Матрица имеет размерность $ |G| \times |G| $.
    Её элементы - число переходов из кластера в кластер для выбранного паттерна перехода из пикселя в пиксель.
\end{enumerate}
Иногда по полученным матрицам вычисляют новые характеристики для упрощения сравнения.

\subsubsection{Проблемы наборов признаков}

На текущий момент мы умеем по геометрическим и цветовым характеристикам попробовать понять, что изображено на картинке: кусок здания, кусок неба...

Проблемы:
\begin{enumerate}[noitemsep]
	\item Изменение масштаба (можно решать перебором масштабов), изменение точки съёмки
	\item Изменение освещения
	\item Перекрытия: серьёзная проблема для наборов признаков, с текстурами лучше.
\end{enumerate}

\subsection{Особые точки. Детектор Харриса и детектор блобов}
\subsubsection{Особые точки}

\emph{Особая точка} -- небольшой объект, отличающийся от всех других локаций в окрестности, позволяющий идентифицировать объект (сравнить два изображения).
Небольшой -- чтобы, например, не бояться перекрытий.

Требования:

\begin{enumerate}[noitemsep]
	\item Повторяемость. Пример: фотографируем дом; правый верхний угол дома всегда должен быть справа вверху сцены, вне зависимости от освещения, времени года, ракурса.
	\item Значимость
	\item Компактность и эффектовность (нужно, чтобы особых точек было немного, гораздо меньше, чем пикселей изображения)
	\item Локальность (занимает небольшую часть изображения).
	К примеру, если перед домом выросла ель, которая не закрыла полностью правый верхний угол дома, то мы должны продолжить распознавать это правый верхний угол.
\end{enumerate}

Для сравнения необходимо, чтобы хоть часть особых точек одного фрагмента была обнаружена на другом сравниваемом фрагменте.

Если берём точку на прямой линии, или там, где нет перепадов интенсивности, мы можем немного сместиться и получить такую же точку $ \Rightarrow $ требования не выполнены, так нельзя.
Угол (два доминирующих изображения) -- хорошо.

\subsubsection{Детектор Харриса. Идея}
Рассмотрим фото пирамиды.
Будем считать распределение значений градиента в выделенных областях.

Начнём с левого верхнего прямоугольника.
Видим, что цвет везде почти один (небо).
Если возьмём квадрат, содержащий боковую грань пирамиды, увидим, что градиенты распределены вдоль одного направления.
Если же возьмём квадрат, содержащий вершину пирамиды, получим, что градиент распределены как минимум в двух направлениях (это нам подходит в качестве особой точки).

\subsubsection{Детектор Харриса. Реализация}
Среднее изменение яркости при сдвиге на $(u,v)$
$$ E(u,v) = \sum_{x,y} w(x,y)[I(x+u,y+v)-I(x,y)]^2 $$
здесь $w(x,y)$ -- веса: может быть ступенька (учитываем пиксели рядом с весом 1, остальные с весом 0) или колокол.

Для небольших сдвигов $ (u,v) $ можно разложить цвет в ряд Тейлора:
$$ I(u+x,v+y) \approx I(x,y) + I_x(x,y)u + I_y(x,y)v $$
Отсюда
$$ E(u,v) \approx \sum_{x,y} w(x,y) (I_x(x,y)u + I_y(x,y)v)^2 = (u v) M \begin{pmatrix} u \\ v \end{pmatrix} $$

где $M = \sum_{x,y} w(x,y) \begin{pmatrix}
I_x^2 & I_x I_y \\ I_x I_y I_y^2\end{pmatrix}\}$

Таким образом, можем посчитать разностные производные $I_x, I_y$ и на их основе -- матрицу $M$.

На основе матрицы $M$ можно понять, сколько в рассматриваемой области доминирующих направлений градиента.
Как? 

\subsubsection{Интерпретация собственных чисел матрицы М}

Рассмотрим вначале простой случай: градиенты есть только вертикальные или горизонтальные.
Тогда матрица будет диагональной.
Нужно найти угол (место, где оба диагональных элемента велики).

В произвольном случае можно представить $ M $ как диагональную матрицу, преобразованную с помощью поворота (поскольку $M$ симметричная): $ M = R^{-1} \begin{pmatrix} \lambda_1 & 0 \\ 0 & \lambda_2 \end{pmatrix} R $

Найденная точка будет углом тогда и только тогда, когда оба собственных числа не будут близки к нулю.

Почему?
Интуитивное объяснение: можем вспомнить из линейной алгебры, что уравнение эллипса --
 $ \begin{pmatrix} u & v \end{pmatrix} M \begin{pmatrix} u \\ v \end{pmatrix} = const $

\subsubsection{Детектор Харриса. Критерий}
\begin{enumerate}[noitemsep]
	\item $\lambda_1, \lambda_2 $  малы $ \Rightarrow $ однородная область
	\item $\lambda_1 \gg \lambda_2 $ или $\lambda_1 \ll \lambda_2 \Rightarrow $ край
	\item $\lambda_1, \lambda_2$ велики $ \Rightarrow $ угол (то, что нам нужно) \label{item:harris:lambda12great}
\end{enumerate}

Были выработаны два критерия, по которым можно распознать \ref{item:harris:lambda12great}:

\begin{enumerate}[noitemsep]
	\item Мера отклика угла по Харрису $ R = det M - k(trace M)^2 $
	\item Мера отклика по Фёрстнеру: $ R = \frac{det M}{trace M} $
\end{enumerate}

Они были придуманы независимо, но довольно похожи.

\subsubsection{Детектор Харриса. Алгоритм}

\begin{enumerate}[noitemsep]
	\item Вычислить градиент изображения в каждом пикселе с использованием сглаживания по Гауссу
	\item Вычислить матрицу $M$ по окну вокруг каждого пикселя (выбор размера окна -- отдельный вопрос, общий совет -- не нужно брать слишком большим)
	\item Вычислить меру отклика $R$ для каждого пикселя
	\item Отрезать $R$ по порогу
	\item Найти в каждой области, где значение $R$ велико, локальный максимум
	\item Выбрать $ N $ самых сильных локальных максимумов.
\end{enumerate}

\subsubsection{Оценка детектора Харриса}

Что может изменяться на изображениях одного и того же объекта?

\begin{enumerate}[noitemsep]
	\item Аффинные преобразования координат. К ним детектор Харриса устойчив.
	\item Фотометрические изменения яркости. Здесь детектор работает хорошо при аффинных преобразованиях яркости (в большинстве случаев на практике получается приемлимое качество)
	\item Масштаб: здесь не так хорошо, например, вблизи скруглённый угол не будет детектироваться.
	Можно попробовать поискать, изменяя масштаб картинки (bruteforce).
\end{enumerate}

Чтобы понять, какой масштаб (размер окрестности) следует взять, чтобы угол выражал себя как угол, можно применить детектор блобов.

\subsubsection{Поиск блобов}

\emph{Блоб} -- характерный фрагмент изображения с круглыми фрагментами границы.

Круг или что-то похожее = перепады интенсивности во всех направлениях на примерно одном расстоянии от одной точки (центра).

Если начать изменять масштаб картинки, где присутствует блоб, то найдётся один масштаб, когда его центр будет занимать весь участок изменяющейся с наибольшим ускорением интенсивности (при увеличении масштаба будет два максимума второй производной).

\subsubsection{Реализация}

Напомним, при поиске границ мы сворачивали изображение с производной сглаженного гауссиана (искали максимальную скорость изменений).

Максимальное ускорение изменений ищется фильтром -- второй производной гауссиана, то есть лапласианом.

\subsubsection{Максимумы ускорений при разных размерах блоба}

Край блоба - всплеск второй производной (всплеск отклика лапласиана).

С уменьшением расстояния между краями расстояние между пиками изменяется, и рано или поздно произойдёт <<резонансное чудо>>: два отклика сливаются в один.
Этот масштаб и определяет характеристический масштаб нашего блоба.

\subsubsection{Нормализация лапласиана}

Во второй производной гауссиана -- лапласиане -- в знаменателе присутствует параметр $ \sigma^2 $, и из-за этого при уменьшении масштаба картинки отклик затухает.
Проблема решается домножением лапласиана на $ \sigma^2 $.

\begin{leftbar}
	Несложно вывести, что $r = \sqrt 2 \sigma $ -- радиус блоба.
\end{leftbar}

\subsubsection{Поиск блобов. Особенности}

Считаем пиксель центром некого блоба тогда и только тогда, когда максимум нормированного лапласиана \textbf{один} и \textbf{ярко выражен}, а не размазан.

\subsubsection{Полученный алгоритм поиска блобов -- LoG}
\begin{enumerate}
	\item Для каждой точки:
	\begin{enumerate}[noitemsep]
		\item Для каждого из масштабов изображения (уменьшающих разрешение) ищем максимум лапласианов \label{item:LoG:max}
		\item Итоговый масштаб и радиус возможного блоба в этой точке соответствуют максимальному из найденных в пункте \ref{item:LoG:max} максимумов.
	\end{enumerate}
	\item Считаем, что точки, в которых в локальной окрестности достигается максимум -- центры блобов (и, соответственно, особые точки)
\end{enumerate}

\subsubsection{Алгоритм DoG}

Авторы DoG (Difference of Gaussian) заметили, что лапласиан есть разность двух гауссианов.
Поэтому можно чуть ускорить алгоритм.

DoG является сейчас основным для поиска особых точек (наравне с детектором Харриса).

Есть большой простор для творчества при комбинации детектора Харриса с DoG или LoG.

\subsubsection{Выбор точек}

Чтобы особые точки действительно распознавались в ключевых местах картинки, рекомендуется обращать внимание на те найденные особые точки, в окрестности которых нет или не очень много прочих особых точек.

\subsubsection{Прочие детекторы. Детектор областей}
\begin{enumerate}
	\item \textbf{IBR}: Берём какую-то яркую точку на изображении (экстремум яркости), выбираем несколько лучей, идущих из этой точки, и на каждом луче ищем точку границы (перепад интенсивности).
	\item \textbf{MSER}: похожая идея -- задаём порог яркости, проводим сегментацию и расширяем область.
\end{enumerate}

\subsubsection{Сравнение объектов по найденным точкам}

Требования: особые точки должны быть

\begin{enumerate}[noitemsep]
	\item Специфичны (нужен способ отличать разные особые точки друг от друга)
	\item Локальны (их распознавание должно зависеть только от небольшой окрестности)
	\item Инвариантны (к искажениям и к изменению освещения)
	\item 
\end{enumerate}
Какие есть алгоритмы сравнения?

\subsubsection{Дескриптор SIFT (Scale-Invariant Feature Transform)}

\begin{enumerate}[noitemsep]
	\item Определяем ориентацию точки, находим доминантное направление в ориентации градиентов (для этого смотрим направление в каждом маленьком кусочке).
	Возможно, квантуем направление градиента.
	\item Определяем окрестность точки, приводим к стандартному размеру
	\item Вычисляем векторы признаков, которые характеризуют окрестность точки (прямоугольную окрестность, как правило, делим на сетку размера $ 4 \times 4 $, в каждой из которых строим гистограммы направлений градиентов с шагом $\pi/4$).
\end{enumerate}

Т. о. для каждой точки получаем направление градиента и по четыре гистограммы, всего $ 32+1=33 $ признака.

Существуют различные модификации SIFT: использовать цвет (HSV), уменьшить число компонент с помощью PCA...
Надо помнить однако, что не всегда модификации, предлагаемые в статьях, работают так хорошо, как описано в самой статье. \\

Итак, нашли точки.
Как по особым точкам идентифицировать объект?

\subsubsection{Объединяем особые точки / блобы / области в объекты}

Можно сравнивать с шаблоном, но это не очень хорошая идея.

Вопрос: как объединять наборы особых точек в объекты?

\begin{enumerate}[noitemsep]
	\item $ M $-оценки (разностные методы)
	\item Схемы голосования (считаем не погрешность, а смотрим те точки, в которых мы работаем хорошо, даём им проголосовать за объект).
	Пример -- алгоритм Хафа.
\end{enumerate}

Если имеем дело с временными рядами, разумно уменьшать веса старых точек.

\subsubsection{Поиск объектов на изображении}

Есть входной датасет, в нём ищем особые фрагменты (композиции особых точек / блобов / областей).

Распознавание: смотрим, сколько из отобранных нами особых фрагментов нашлось на входном изображении и с какой вероятностью.
Дескрипторы могут быть обучены на исходном датасете.

\subsection{Организационные моменты. Замечания к большой лабораторной}

Алгоритмы, обсуждённые в этой лекции, хотелось бы обсудить.
Просьба делать это в ВК.

В присланных на почту В.И. данных есть не все возможные случаи!
Нужны объекты под разными углами, с разных ракурсов, разных расстояний.

Каков должен быть процесс решения?
Предполагается примерно следующее: попробовать один подход (прямые, преобразования Хафа, особые точки...), посмотреть его проблемы, попробовать альтернативу и посмотреть её недостатки.

\end{document}
