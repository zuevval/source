\documentclass[main.tex]{subfiles}
\begin{document}
\setcounter{section}{-1}
\section{Лекция 0. Краткое введение, руководство к выбору практических задач} % TODO title

Кацман Виктор Игоревич \href{mailto:vikto9494@gmail.com}{vikto9494@gmail.com} -- аналитик-разработчик машинного обучения в Yandex, аспирант кафедры ПМ, а также веду свой небольшой проект; до этого работал в OpenWay (тоже рекомендую, очень хорошая компания).

В этом курсе: не совсем то, что в названии.

\href{https://drive.google.com/drive/u/0/folders/1gItRk7-yscR5AXBk26G6O33JU7StZNkU}{Материалы прошлых лет}

Будет большая лабораторная (почти курсовой проект) и мини-лабораторные (с ними будет помогать Андрей). Лабораторные делаем дома. Практика будет, скорее всего, плавно вытекать из лекции (м. б. там будет продолжение лекций).

Правила отчётности сложные. Курс состоит из двух частей: теоретическая и практическая.

Практическая часть: без практической части курс вообще не сдать. Вначале -- маленькие лабораторные.
Основная идея -- чтобы мы поняли, какие вообще бывают библиотеки, как с ними работать.
Ожидается, что мы начнём их делать сразу же и за месяц мы их закончим.

Второй элемент в практической части -- большая лабораторная.
Даётся задача по умолчанию (см. \href{https://docs.google.com/spreadsheets/d/1CrtpI9oEBldscjZSrAinBPnJN01S5UECVl5VxgtM_ic/edit#gid=506672535}{таблицу}), если в рамках бакалаврской или ещё чего-то делаете свою задачу с семантическим разрывом, можно договориться и взять её.
Есть минимальный вариант этой задачи для допуска к экзамену; можно модифицировать, см. таблицу.

Теоретическая часть: те, кто работают в семестре, могут получить автомат; на экзамене, если не хватает 1 балла до автомата, можно добрать его несложной беседой по определениям.

За что даются бонусы по теоретической части:

\begin{enumerate}[noitemsep]
    \item Если Вам кажется, что курс Вам не нужен (например, Вы считаете, что курс не нужен), можно побеседовать со мной в течение первого месяца.
    \item Можно отправлять конспекты по почте и получать за это небольшие баллы.
    Я читать конспекты почти не буду, разве что чтобы посмотреть, что нет заимствований.
    Конспекты в основном для себя.
    Можно письменные / печатные.
    Присылать можно в течение дня или двух с лекции (скажем, дедлайн -- вечер пятницы).
    \item ...
\end{enumerate}

\textbf{Литература}: см. ссылку.
Впрочем, не очень рекомендую читать много сложных книг, скорее гуглить, искать, смотреть курсы / видеолекции.
Особенно много мне нравится курс Антона Конушина (оттуда, в частности, взял термин <<семантический разрыв>>) % TODO

Из классики: <<Глубокое обучение, погружение...>>. В той же папочке материалы прошлых лет.

\textbf{Идея курса}: работающие алгоритмы машинного обучения $ \ne $ магия!

Ключевое понятие, которое нам потребуется -- \emph{алгоритм}.
Алгоритм принимает вход и выдаёт выход; алгоритм -- чёткая конечная последовательность действий, направленная на достижение результата.

Под задачей мы в этом курсе будем понимать придумывание алгоритма.

Какие бывают задачи?

\begin{enumerate}[noitemsep]
    \item Транслировать код на языке Python в язык C++ \emph{-- вполне понятная задача; язык входа однозначно транслируется в язык выхода.}
    \item Посчитать сумму двух чисел \emph{ -- наверное, самая простая задача}
    \item Определить, подходит ли конкретному человеку конкретная одежда \emph{-- не совсем понятно, что на входе; что значит <<хорошо смотрится>>, <<слишком длинные рукава>>?}
    \item Найти производную \emph{ -- вполне понятная задача, хотя возможны нюансы. }
    \item Установить, есть ли на изображении крокодил \emph{-- примерно понятно, что на вход: можно пойти к биологам и спросить определение. Понятно, что на выходе: <<да/нет>>. Проблема в сопоставлении определения от биологов и матрицы пикселей \textrightarrow начинаются эвристики: например, наличие на картинке чего-то, похожего на зубы, плотность зелёных пикселей и т. д. (семантический разрыв). }
    \item Выбрать оптимальный путь между двумя точками в городе
\end{enumerate}

Виды задач:

\begin{enumerate}[noitemsep]
    \item Задача алгоритмически однозначна
    \item Язык входа однозначно транслируется в язык входа
    \item Между языками входа и выхода существует семантический разрыв \label{item:semantic}
\end{enumerate}

Задачи типа \ref{item:semantic}, как правило, невозможно решить до конца качественно.

\subsubsection{Классификация задач обработки сигналов}

\begin{itemize}[noitemsep]
    \item Существует и известен детерминированный алгоритм
    \begin{itemize}[noitemsep]
        \item можно ДОКАЗАТЬ, что алогоритм ВСЕГДА приводит к желаемому результату
    \end{itemize}
    \item Детерминированный алгоритм неизвестен % TODO
    \begin{itemize}[noitemsep]
        \item присутствует СЕМАНТИЧЕСКИЙ РАЗРЫВ между входным сигналом и требуемым представлением
        \item предлагаемые алгоритмы работают только на части входных данных
    \end{itemize}
\end{itemize}

\subsubsection{Семантическая информация}
Мы видим изображение (\emph{зрение -- источник 70-80\% информации}).
\begin{itemize}[noitemsep]
    \item Первый вариант определения семантической информации: можем понять, что это город (вне дома), что это Пекин, площадь Тяньанмэнь, где обычно проходят военные парады, а в 1998 году были подавлены протесты.
    \item Второй вариант -- сегментация: выделить области, где находится здание, стена, автомобиль, ...
    \item Третий вариант -- локализация объектов: обвести рамками автомобиль, автобус, портрет человека...
    \item Четвёртый вариант -- качественная информация: крыша наклонная, небо голубое, ветер дует справа налево, автобус жёсткий...
\end{itemize}


\subsubsection{Задача с семантическим разрывом}
Между входными и выходными данными есть \emph{семантический разрыв}.
Нужно определить выходные данные через входные.

Определить -- означает придумать \emph{алгоритм}, т. е. инструкции, которые должны исполняться однозначно.

\subsubsection{Прикладные задачи}
\begin{itemize}[noitemsep]
    \item Медицина: моделирование организмов и их взаимодействия для выявления аномалий, диагностики заболеваний и выбора курса лечения.
    \item Обучение: оценка качества методов обучения и выбор оптимального на её основе
    \item % TODO ...
    \item Примеров очень много...
\end{itemize}

\subsubsection{Почему сложно устранить семантический разрыв?}
Почему сложно понять, что на изображении крокодил?

Очень много физических тонкостей процесса / недостатков модели, которые
\begin{itemize}[noitemsep]
    \item тяжело заметить, осознать
    \begin{itemize}[noitemsep]
        \item Отсутствуют на имеющихся датасетах (пример: при некотором освещении пингвины чёрные, а не белые, чего может не быть) % TODO
        \item компенсируются другими недостатками
        \item незаметны на фоне других недостатков, вносящих большую погрешность
    \end{itemize}
    Пример: крокодил может как-то деформирован, перекрыт другими предметами, может быстро двигаться и быть смазанным на снимке; может быть замаскирован; может быть только из контекста понятно, что это крокодил, а не что-то иное.
    \item Пусть собрали очень хороший набор данных. Тогда всё равно сложно учесть тонкости:
       \begin{itemize}[noitemsep]
           \item тяжело понять, как именно и в каких случаях они имеют значение
       \end{itemize}
   \item
\end{itemize}

\subsubsection{Первый очень общий метод устранения семантического разрыва}

\begin{enumerate}[noitemsep]
    \item Понимаем, решается ли задача детерминированно, если да -- решаем.
    \item Строим произвольный алгоритм, приближённо решающий задачу.
    \item До тех пор, пока алгоритм работает недостаточно хорошо (например, в распознавании поребриков некоторые заборы распознаются как поребрики):
    \begin{itemize}[noitemsep]
        \item Находим несколько физических причин, по которым алгоритм работает недостаточно хорошо (допущенные нами пренебрежения)
        \item Выбираем самые значимые, уточняем модель и алгоритм
    \end{itemize}
    \item Используем разработанный алгоритм
\end{enumerate}

\subsubsection{Зачем этот курс?}

\begin{enumerate}[noitemsep]
    \item Получить удовольствие от <<я понимаю, как все это устроено и могу сам>>
    \item Осознать
    \begin{enumerate}[noitemsep]
        \item базовую теорию обработки сигналов
        \item что делать и <<в какую сторону думат>> при необходимости решить задачу на обработку сигналов с семантическим разрывом
        \begin{enumerate}[noitemsep]
            \item что не следует делать в таких задачах
        \end{enumerate}
        \item начать получать опыт, помогающий ориентироваться в мире задач с семантическими разрывами
        \item представить, что можно ожидать от аналитиков-разработчиков ML и им подобных
    \end{enumerate}
\end{enumerate}

Из опыта: мы с менеджерами в компаниях часто говорили на разных языках.
Если не станете разработчиками-аналитиками, то, возможно, вы будете менеджерами и вам легче будет понимать разработчиков!

\subsubsection{План курса}

\begin{enumerate}[noitemsep]
    \item Введение
    \begin{enumerate}[noitemsep]
        \item сегодня, общий метод
    \end{enumerate}
    \item Основные простейшие методы
    \begin{enumerate}[noitemsep]
        \item Основы поиска объектов. Алгоритм Canny. Морфологический анализ.
        \item Частотный анализ. Преобразование Фурье. Вейвлеты. Частотные фильтры. Пространство Хафа
        \item Локальные и глобальные признаки. Поиск сопоставимых точек, углов. Метрики
    \end{enumerate}
    \item Более сложные общие более современные методы
    \begin{enumerate}[noitemsep]
        \item Основы ML: марковские модели, нейронные сети, деревья решений, … (мб 2 лекции)
        \item Кластеризация, сравнение фрагментов сигналов
    \end{enumerate}
    \item Разбор прикладных задач (с применением разобранных методов)
    \begin{enumerate}[noitemsep]
        \item распознавание границ, анализ видеопотока
        \item распознавание лиц
        \item построение модели стопы по фотографиям, виртуальная примерка обуви
        \item распознавание звука, алгоритм Витерби
        \item ...
    \end{enumerate}
\end{enumerate}

\href{https://docs.google.com/spreadsheets/d/1CrtpI9oEBldscjZSrAinBPnJN01S5UECVl5VxgtM_ic/edit#gid=2040984582}{Правила отчётности} -- довольно сложные.
Этот файл лежит в <<Обработка сигнала / Весна 20201>>.

Три балла можно получить только за практику.

\subsubsection{Большая лабораторная}

Вначале надо придумать план и прислать на почту.
Это очень важный пункт.

Плюс BitBucket: в приватном проекте может быть до 5 ревьюеров, в GitHub -- кажется, до трёх.
Можно и GitHub.

После формулировки задачи надо написать код, коммитить его в репозиторий и потом друг друга ревьюить (по несколько ревью на один проект).


Подводные камни плана:
\begin{enumerate}[noitemsep]
    \item Скорее всего, кто-то не доделает свою задачу к концу
    \item Скорее всего, не все смогут оставить адекватный review
\end{enumerate}

Ревьюеров назначаю я.

Есть <<неиспользованный лимит дедлайна>> -- 120 часов % TODO

Язык программирования: рекомендуется Python + Jupyter

Основная информация будет в группе ВКонтакте.

\end{document}
