\documentclass[main.tex]{subfiles}
\begin{document}

\section{Лекция 1. Простейшие методы анализа сигналов. Поиск и обработка перепадов интенсивности}

\emph{Сигнал} -- некий носитель информации, использующийся для передачи сообщений в системе связи.

По сути: сигнал -- некий источник информации; он записан в виде битов (например, пикселей).
Иногда надо учитывать, что сигнал может быть повреждён по дороге.

\subsubsection{Задачи обработки сигналов}

Три типа таких задач:

\begin{enumerate}[noitemsep]
	\item Классификация сигналов
	\item Интерпретация: определение того, что сигнал представляет собой; вычисление характеристик сигнала.
	Например, если в классификации мы лишь определяем наличие или отсутствие крокодила, то в рамках данной задачи можем определять, где у него глаза, зубы, какой длины вход...
	\item Предобработка сигнала для последующего анализа -- во многом часть предыдущих задач, точнее, их первый этап.
	Особенность: хуже представляем, как мы определяем качество результата.
\end{enumerate}

\subsubsection{Как мы можем получить информацию о сигнале}

\begin{enumerate}[noitemsep]
	\item Наверняка есть характерные границы/точки у больших объектов
	\item Возможно, есть маленькие, <<атомарно выделяемые>> объекты
\end{enumerate}

Дальше вопрос: как именно мы определяем большие предметы?

\emph{Граница} -- резкий перепад интенсивности, то есть изменение количественных показателей сигнала (например, на картинке -- изменение цвета поверхности, перепад освещённости, изменение нормали к поверхности и так далее).


\subsubsection{Поиск перепадов интенсивности}

Края соответствуют экстремумам первой производной.

Как искать в многомерном случае?

\subsubsection{Направление и сила перепадов интенсивности}

Вводим градиент изображения: $ \nabla f = \left[ \frac{\partial f}{\partial x}, \frac{\partial f}{\partial y} \right] $

Градиент направлен в сторону наибольшего возрастания интенсивности.

\emph{Сила} края задаётся как норма градиента: $ || \nabla f || = \sqrt{\frac{\partial f}{\partial x}^2 + \frac{\partial f}{\partial y}^2} $

Но как искать в дискретном случае?

\subsubsection{Отступление. Свёртка}

Вообще \emph{свёртка} двух функций -- интеграл от их произведения по промежутку, но при численном интегрировании это сумма произведений.

Пусть $ g $ -- ядро свёртки, $ f $ -- функция, над которой производим операцию $ \to $ одномерная свёртка:

$$ (f * g) = f(1)g(x-1) + f(2)g(x-2) + ... $$

Двумерная свёртка вводится похожим образом.

Осторожно, есть похожая операция, когда элементы операнда и ядра перемножаются так: $ f(1)g(1) + f(2)g(2) + ... $

\subsubsection{Производная и свёртка}

Производная линейно инвариантная к переносам.

$ \frac{\partial f}{\partial x} = \lim_{\varepsilon \to 0} \left( \frac{f(x + \varepsilon, y)}{\varepsilon} - \frac{f(x - \varepsilon, y)}{\varepsilon} \right) $

Разностная производная (разность близких значений) -- уже почти свёртка:

$ \frac{\partial f}{\partial x} \approx \frac{f(x_{n+1},y) - f(x_n,y)}{\Delta x} $

Интуитивно: белый пиксель, рядом чёрный $ \to $ производная велика $\to $

Разные производные по направлениям можно аппроксимировать свёртками с разными ядрами (=фильтрами).

\begin{enumerate}[noitemsep]
	\item Фильтры Робертса -- производные по диагонали
	\item Фильтры Превитта --производные вдоль осей
	\item Фильтры Собеля: модифицированные фильтры Превитта. Если пиксели в углах, у них должен быть меньше вес.
\end{enumerate}

Что делать в крайних пикселях изображения? Чем дополнять пиксели, которые за границей?

Дополнять чёрным / серым / белым: не очень хорошо, т. к. семантически нагружаем изображение дополнительной информацией, которой ранее не было на картинке.

\subsubsection{Влияние шумов. Фильтр Гаусса}

Шумы могут мешать детектированию краёв.
Что делать?

Сглаживание краёв: делаем пиксели похожими на соседей. Тогда ступенька с шумом превратится в сглаженную ступеньку.
Возможно, в усреднении берём ближайшие пиксели с бОльшим весом, чем более далёкие $ \to $ приходим к идее сглаживания с гауссовым ядром (фильтр Гаусса).

Т. о. два раза делаем свёртку: сначала для фильтрации, затем для разностной производной.
Можно использовать свойство свёртки: делать вместо двух операций одну.

Чем больше сглаживаем (параметр фильтра $ \sigma $), тем больше размываются границы.

\subsubsection{Фильтр Гаусса для повышения резкости}

Пусть есть исходное изображение и сглаженное гауссовским фильтром, скажем, $ 5 \times 5 $.
Можно вычесть сглаженное из исходного, получить какие-то детали изображения и добавить это к исходному.
Интерпретация: единичный фильтр - гауссиан $ \approx $ лапласиан гауссиана.

\subsubsection{Двойная пороговая фильтрация по гистерезису}

Как бороться с тем, что полученные границы -- жирные линии?
Можно обрезать пиксели на картинке с границами по некоторому порогу.

Усложнение этой идеи: использовать двойную пороговую фильтрацию.
На графике интенсивности вводим два порога: порог начала границы и конца границы (второй порог меньше первого).

\subsubsection{Уточнение границ}

Разумно у каждой границы теперь взять только один, средний пиксель

\subsubsection{Детектор Canny}

Долгое время этот алгоритм был классическим для распознавания границ.

В предыдущих пунктах лекции мы прошли множество разных приёмов и фильтров, у многих были параметры; какие из них, в какой последовательности и с какими параметрами их применять?

Canny -- параметры как-то разумно подобраны.

\begin{enumerate}[noitemsep]
	\item Сглаживание фильтром Гаусса
	\item Применение фильтров для поиска границ и градиентов (это, как правило, объединяется с предыдущим пунктом)
	\item Подавление не-максимумов (уточнение границ)
	\item Связывание краёв и обрезание по двойному порогу (по гистерезису)
\end{enumerate}

В итоге получаем тонкие линии.

У Canny есть необязательные параметры, которые можно подобрать индивидуально в некоторых специальных случаях, для которых плохо работают значения по умолчанию.

\subsubsection{Проблемы Canny}

Canny определяет границы по перепадам интенсивности.
Глазом мы определяем границы часто по контексту, например, между двумя листьями -- разрыв рисунка.

\subsubsection{Что делать после поиска границ?}

Первая идея: допустим, мы ищем прямоугольники (или монеты).
Возьмём \emph{шаблон} (прямоугольник/кружок) и будем сравнивать найденные границы с шаблоны.
Можно одному объекту сопоставить несколько шаблонов (разные ракурсы, масштабы).

Как сравнивать шаблоны?
Самая плохая идея -- по пикселям.
Идеи лучше, как оказывается, придумать не удаётся.

Позже будем изучать сравнение не с шаблонами.

\subsubsection{Сопоставление с шаблоном}

Нужно придумать метрику схожести распознанных границ с границами шаблона.
Их много, у каждой свои плюсы и минусы.
\begin{enumerate}
	\item Попиксельные метрики:
	\begin{enumerate}[noitemsep]
		\item SAD -- Sum of absolute difference:
		\item SSD -- Sum of squared difference
		\item CC -- Cross-correlation: $ \sum_X \sum_Y I_1(X,Y)I_2(X,Y) $
	\end{enumerate}
	
	Но эти, кажется, никто не использует.
	
	\item Метрики между краями (эти имеют больше смысла; лучше борятся с плохой фильтрацией, когда внутри объекта могут остаться ложные участки границ)
	
	\begin{enumerate}[noitemsep]
		\item Champfer distance: $ ChDist(A,B)=\sum_{a \in A} \min_{b \in B} ||a-b|| $ (ищет ближайшую точку границы объекта к данной точке границы шаблона; складываем все расстояния между этими парами точек).
		\item HausDorf: вместо суммы берём максимум
		\item HausDorf с медианой вместо максимума
	\end{enumerate}
\end{enumerate}

Как эти метрики быстро считать?
DistanceTransform -- специальное преобразование: для каждого пикселя изображения вычисляется расстояние до ближайшего пикселя края (по сути, поиск в ширину).
DistanceTransform, помимо того, помогает найти наиболее удалённые от границ точки (скелет объекта).

\subsubsection{Уже можем как-то искать объекты на изображении}

Первый, примитивный прототип распознавателя:

сделали фотографии стульев $ \to $ из них делаем шаблоны $ \to $ обрабатываем новые изображения детектором Canny и ищем на них шаблоны с помощью сравнения по метрике.

Иногда такой алгоритм работает.
Пример: съёмка самолётов сверху -- такой детектор подходит.
Но если самолёт на новых фотографиях будет расположен в нескольких ракурсах, может не сработать.

\subsubsection{Поиск больших инвариантных к условиям съемки (или в специальных условиях) объектов}

Приведены примеры инвариантных к условиям съёмки объектов: медицина (клетки крови)...

Для контрастных объектов на фоне есть иной алгоритм: бинаризуем изображение $ \to $ устраняем шумы $ \to $ выделяем объекты и проверяем соответствие их форм заданным критериям.

\subsubsection{Бинаризация}

Как бинаризовать?
Первая лобовая идея -- бинаризация по порогу: всё, что выше порога по интенсивности -- белое, ниже -- чёрное.
Этот порог можно брать общий для всей картинки или разный в разных областях.

Подбор порога:
\begin{enumerate}[noitemsep]
	\item извне
	\item из гистограммы (если снимки сделаны в лабораторных условиях, там, как правило, будет какой-то ярко выделяющийся пик, )
	\begin{enumerate}[noitemsep]
		\item бинаризация Оцу
	\end{enumerate}
\end{enumerate}

Бинаризация Оцу -- способ разбить гистограмму на два класса.
Идея: шумы (яркости точек) распределены нормально; минимизируем дисперсию внутри класса.

$$ - \sigma_\omega^2(t) = \omega_1(t)\sigma_1^2(t) + \omega_2(t) \sigma_2^2(t) \to \min $$

Например, перебираем все варианты порогов и смотрим, какой из них даёт наилучший результат в смысле минимизации дисперсии.

\subsubsection{Адаптивная бинаризация}

Освещение объекта может быть неравномерное, поэтому часто порог подбирается не для всего изображения, а для фрагмента в некотором радиусе от рассматриваемого пикселя.

Например, можно использовать для поиска порога фильтр Гаусса (\emph{адаптивная Гауссова бинаризация}):

$$ \frac{1}{| \Omega (x, y) |} \sum_{(x,y) \in \Omega} F(x,y)G(x,y) $$

Важно, что адаптивная бинаризация тоже работает не всегда.
Например, на картинке с лошадью области внутри коня будут слишком контрастные, т. к. пиксели будут сравниваться только с пикселями внутри лошади.

Общий совет: распознаём маленькие объекты $ \to $ обычно используем адаптивную бинаризацию, иначе -- с порогом по всей картинке.

\subsubsection{Шум на бинарных изображениях. Морфологические операции}

Для удаления шума используется математическая морфология:

\begin{enumerate}[noitemsep]
	\item \emph{Расширение} (дилатация, dilatation) -- добавляем к изображению пиксели элементов вокруг
	\item \emph{Сужение} (эрозия, erosion) -- убираем пиксели изображения.
\end{enumerate}

Для изображения, как правило, можно подобрать такой радиус эрозии, что все шумы при этом уйдут.

\subsubsection{Морфологические операции. Открытие и закрытие}

\emph{Раскрытие} -- сначала сужаем, затем расширяем пиксели.

\emph{Закрытие} -- сначала расширяем, затем сужаем.

Закрытие позволяет бороться с шумами внутри объекта (когда что-то внутреннее распозналось как фон), раскрытие - с шумами на фоне.

\subsubsection{Морфология. Что ещё можно?}

\begin{enumerate}[noitemsep]
	\item  Преобразование <<градиент>>: изображение после эрозии вычитается из изображения после дилатации
	\item Преобразование <<Top Hat>> вычитает разомкнутое изображение из исходного.
\end{enumerate}

Морфология не идеальна. Есть модификации с более робастными фильтрами и т. д.


\subsubsection{Анализ соответствия выделенных областей критериям объектов. Итог}

При распознавании объектов, в отличие от распознавания краёв, можно сравнивать обнаруженный объект с шаблоном.
Например, считать у каждого объекта на бинаризованной картинке геометрические характеристики (эксцентриситет, отношение площади к периметру, компактность...) и сравнить с эталоном.
Если близки, то это тот объект, который ищем.
Также можно посмотреть на исходное, небинаризованное изображение в том месте, где нашли объект: какая там средняя яркость, средняя освещённость... (т. н. \emph{фотометрические} признаки).

Вышеуказанные техники работают в основном только в лабораторных условиях.
В лаборатории геометрические признаки обычно более значимы, чем фотометрические.

\textbf{Замечание}. В последовательности шагов алгоритма Canny поиск градиентов и подавление немаксимумов примерно соответствуют получению градиента при бинаризации; сглаживание -- открытиям / закрытиям; обрезание по гистерезису -- бинаризации.

\subsubsection{Связь этого урока с большой лабораторной работой}

После этой лекции должно было стать намного понятнее, с чего начинать в первой лабораторной.

Например, если у нас задача о столе и стуле, выбриаем фотографии, на каждой из которых есть стол и стул; пытаемся делать фильтрацию, искать границы, применять детектор Canny...
Можем обучать шаблоны, параметры критериев; выводим в итоге какой-то разумный алгоритм нахождения стула и стола, который потом будем улучшать.

\end{document}
