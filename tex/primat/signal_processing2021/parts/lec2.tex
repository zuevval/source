\documentclass[main.tex]{subfiles}
\begin{document}
	
\section{Лекция 2. Представление сигнала в различных пространствах}
	
11 февраля 2021.

Сегодняшняя лекция -- теоретическая; она имела бы и серьёзный практический смысл, если бы не было современных алгоритмов наподобие нейронных сетей.

\subsubsection{О чём курс | сигнал}

Сигнал, напомним, бывает двух видов:

\begin{enumerate}[noitemsep]
	\item Дискретный (биты, число студентов в классе...)
	\item Непрерывный. Большинство сигналов (в частности, изображения) именно такие, но компьютер понимает только дискретные сигналы, поэтому представляем изображение в виде матрицы пикселей с некоторым уровнем квантования.
	
	\textbf{Теорема Котельникова:} если частота дискретизации не меньше, чем максимальная удвоенная частота в спектре сигнала, то по дискретизованному сигналу можно однозначно восстановить исходный.
	Иначе,возможно, нельзя.
	\emph{(б/док-ва)}
	
	Пример: для синусоиды достаточно по две точки на период, чтобы однозначно её восстановить.
\end{enumerate}

\subsubsection{Представление дискретных сигналов}

Как можно представлять дискретные сигналы в компьютере?

\begin{enumerate}[noitemsep]
	\item Геометрия в $n$-мерном пространстве (есть координаты) или граф (вершины и рёбра между ними; у вершин могут быть веса, у рёбер тоже какие-то нагруженности).
	\begin{enumerate}[noitemsep]
		\item Общее: есть точки и связи между ними
		\item Различия: по графу мы не всегда можем восстановить геометрию, а по геометрии (с заданным расстоянием) можно построить полный граф с весами рёбер.
	\end{enumerate}
	\item Конечный / бесконечный (конечное или бесконечное количество сигналов; бесконечное -- например, описание последовательности)
	\item Явно / неявно заданный (звук, изображение -- явно заданные; неявно заданные, к примеру, -- решение дифференциального уравнения)
	\item Нагруженные характеристиками вершины (как в графе, так и при использовании геометрии)
	\item Нагруженные характеристиками и весами связи (в графе и при использовании геометрии)
	\item ...
\end{enumerate}

\subsubsection{Сегодня}

Работаем с сигналами, представленными в n-мерном пространстве.
Грубо говоря, сигнал -- функция из n-мерного в m-мерное пространство (например, изображение -- функция из двухмерного пространства в трёхмерное, если оно цветное).
	
\subsubsection{Разложение сигнала по составляющим. Общая идея}
Вопрос: мы храним изображение в матрице пикселей.
Если возьмём срез по прямой, получим одномерный сигнал.
Действительно ли нужно все эти, к примеру, 250 пикселей на линии хранить по отдельности, нельзя ли сэкономить место?

Жан-Батист Фурье: \textbf{преобразование Фурье}.

Общая идея: любую функцию можно разложить в сумму гармонических функций (синусов и косинусов) с разными частотами.

Таким образом мы можем хотеть отделить высокочастотный шум от каких-то объектов, которые крупнее (пол, потолок, балки...)

Проблема: что-то с более высокой частотой считаем шумами, но вместе с тем это что-то может содержать полезную информацию.
Человеческий глаз с разного расстояния воспринимает разные частоты на изображении (пример: портрет Авраама Линкольна, наложенный на фотографию пейзажа).

Теоретически, когда мы раскладываем сигнал в ряд Фурье, мы, можно сказать, умножаем его на матрицу базисных функций ($\sin, \cos $).

Общая формула:
$$ f(x) = \frac{a_0}{2} + \sum_{k=1}^{+\infty} A_k \cos\left(k \frac{2\pi}{r}x + \vartheta_k\right)  $$

\subsubsection{Представление непериодической функции}

Берём первую гармонику и говорим, что прямоугольный импульс можно приближённо описать функцией этого вида.

Если хотим более точное приближение, добавляем следующие гармоники...

Даже сигнал-ступеньку можно бесконечно точно приблизить суммой

$$ A\sum_{k=1}^{\infty} \frac{1}{k} sin(2\pi kt) $$

Верхние частоты можно отбросить, считая их шумом.


\emph{Ringing} -- важное явление, появление артефактов на границах изображения при дискретизации / сглаживании фотографии: большие пиксели как-то неаккуратно аппроксимируем (особенно если они на границе картинки).
Это дефект, с ним борются.

\subsubsection{Примеры спектров частот}

\emph{Спектр} -- гистограмма вкладов частот.

Для одной гармонической функции будут две частоты, для прямоугольника -- симметричное изображение с пиком в нуле.

Замечание. На практике чаще используется косинус-преобразование Фурье, т. к. $ \cos $ -- чётная функция.

\subsubsection{Преобразование Фурье для изображений}

Примерно то же, что одномерное преобразование Фурье.
Превращаем матрицу пикселей (а не просто линейный сигнал, как в одномерном преобразовании) в матрицу частот.

\subsubsection{Спектральный анализ изображений}
Как отделять шум от не-шума, выделять объекты?

Видно, что на примере с акведуком две диагональные прямые соответствуют аркам акведука.
Как это находить автоматически? На помощь приходят быстрое преобразование Фурье (которое позволяет быстро получить разложение в сумму синусов и косинусов) и теорема о свёртке.

\subsubsection{Теорема о свёртке} 

Теорема: преобразование Фурье от свёртки двух функций -- то же, что произведение преобразований Фурье от каждой функции; преобразование Фурье от произведения можно представить как свёртку двух обратных функций.

\begin{align*}
	& F[g*h] = F[g]F[h] \\
	& F^{-1}[gh] = F^{-1}[g] F^{-1}[h]\\
\end{align*}

На прошлой лекции мы смотрели, как можно применять разные фильтры.
Фильтр $ 3 \times 3 $ требует $ 9 $ операций умножения для каждого пикселя. Если фильтр $ k \times k $, изображение $ m \times m $, свёртка требует $ O(m^2 \cdot k^2) $ времени.

Пользуясь теоремой о свёртке, можно один раз перейти из пространства пикселей в пространство частот, после чего вместо свёртки выполнять операции произведения.

\textbf{Пример выделения границ:} на изображение акведука, переведённое в пространстве частот, накладываем фильтр, тоже переведённый в частотное пространство (тёмный кружок в центре), после чего остаётся что-то, похожее на границы.

Конечно же, всё это разумно, когда нужно применить много фильтров, потому что потом нужно делать обратное преобразование. 

Всё это широко использовалось, пока не появились нейронные сети и не стало всё совсем по-другому...
В нейронных сетях подобное разложение тоже используется, но в неявном виде.

\subsubsection{Составляющие сигнала. Фильтры Габора}

Нельзя ли раскладывать сигнал по другим функциям, кроме $ \sin / \cos $, и сразу же получать что-то ценное -- границы и т. д.?

Работа пошла в сторону усложнения фильтров. Появились, например, фильтры Габора.

Функция $ g(x, y, \lambda, \vartheta, \psi, \omega, \gamma ) = \exp \left( - \frac{x'^2 + \gamma^2 y'^2}{2 \sigma^2}\right) \cos \left(2\pi \frac{x'}{\lambda} + \psi \right) $,
использующаяся в преобразовании Габора -- ядро Гауссиана, умноженное на синусоиду.

В приведённой выше формуле

\begin{align*}
& x' = x \cos(\vartheta) + y \sin(\vartheta) \\
& y' = -x \sin(\vartheta) + y \cos(\vartheta) \\
& \vartheta \text{ -- ориентация} \\
& \lambda \text{ -- длина волны} \\
& \sigma \text{ -- сигма гауссиана} \\
& \gamma \text{ -- aspect ratio} \\
& \psi \text{ -- сдвиг фазы}
\end{align*}

Как можно прийти к этой идее?

Напомним, как мы делали на прошлой лекции, берём сначала сглаживающий фильтр (чтобы убрать шумы), затем гармонический фильтр, который находит границы; это и делает фильтр Габора в один приём.

Нахождение границ: между минимумом и максимумом синусоиды есть расстояние, которое позволяет аппроксимировать разностные производные.

\textbf{Вопрос от Кости:} что есть система функций, какие из параметров её формируют по аналогии с фильтром Фурье?

Ответ: берём некий набор фильтров с разными параметрами (возможно, случайно).
Возможно, это не ортогональная система функций.
С таким количеством параметров сложно поверить, что система ортогональная; зачем столько параметров? -- они имеют физический смысл. 

Фильтры Габора похожи на форму рецептивных полей простых клеток в визуальной коре мозга человека.
Этот факт вдохновил многих на исследование таких фильтров.

\subsubsection{Составляющие сигнала. Вейвлеты}

Помимо функцией, порождённых тригонометрией, есть такая штука, как \emph{вейвлеты}.

Вейвлет -- функция, которая в хвостах ноль, а в центре -- сигнал (всплеск).
Она непериодическая; можно придумать множество вейвлетов (даже фильтр Гаусса можно назвать вейвлетом), далее раскладывать по набору масштабированных и смещённых копий нашего одного вейвлета (вейвлет-разложение).

На основе вейвлет-разложения придуман алгоритм сжатия jpeg 2000.
Алгоритм даёт лучшее качество, чем при быстром преобразовании Фурье, но работает дольше.

Многое зависит от того, какую вейвлет-функцию мы используем.

\subsubsection{Разложение сигнала по составляющим}

$ y \in \mathcal{R}^m $ -- сигнал, $ x \in \mathcal{R}^n $ -- представление $ \Rightarrow $ разложение по составляющим есть преобразование $ D \in \mathcal{R}^{n \times m}, y = Dx $

Разложения могут переводить сигнали в пространство той же размерности или иной.

\begin{enumerate}[noitemsep]
	\item $ n = m $ -- это мы рассмотрели (преобразование Фурье, фильтры Габора, вейвлеты, ...)
	\item $ n \gg m  $
	\begin{enumerate}[noitemsep]
		\item Сигнал представляем как комбинацию небольшого подмножества векторов из $ D $
		\item $ D $ -- словарь, строка $ D $ -- <<атом>>
	\end{enumerate}
	\item $ n < m $ -- сокращение размерности
\end{enumerate}

\subsubsection{Пространство значений параметров фрагмента сигнала}

Можем сопоставлять объекты по шаблонам с параметрами.
В частности, если объект несложный (например, прямая), можно использовать алгоритм для подбора параметров.

\subsubsection{Преобразование Хафа для поиска прямых. Идеи}

Пусть прямая задаётся уравнением $ y = kx + b $, через каждую точку проходит бесконечно множество прямых.

Пусть у нас есть изображение, мы применили к нему какое-то преобразование (например, Canny), и теперь хотим среди границ найти прямые.
Есть методы, основанные на минимизации (метод наименьших квадратов, линейная регрессия и так далее).

Принципиально иной подход -- Хафа: <<голосование>> каждой точки за прямые.

Каждая точка <<голосует>> за одну или несколько прямых.
Те прямые, за которые проголосовали много точек, будем считать реальными.

\subsubsection{Преобразование Хафа}

Пространство как-то дискретизируем (конечное число точек и прямых).

\begin{align*}
	& y = \left(- \frac{\cos \vartheta}{\sin \vartheta}\right)x + \left(\frac{\rho}{\sin \vartheta}\right) \\
	& x \cos \vartheta + y \sin \vartheta = \rho \\
\end{align*}

Точка голосует за те прямые, которые через неё проходят, если у неё, например, большая интенсивность (или если изображение бинарное и точка белого цвета).
При этом используем не коэффициенты $ k $ и $ b $, а $ \rho, \vartheta $ (в полярной системе координат получается более равномерная сетка прямых).

Свойства преобразования Хафа:

\begin{enumerate}[noitemsep]
	\item ++: он устойчивый, робастный к шумам (выбросы обычно голосуют за разные прямые), а также инвариантен к сдвигам и масштабированиям.
	\item ++: множество прямых могут быть найдены за один проход.
	\item --: сложность алгоритма растёт с ростом числа параметров
	\item --: сложно подбирать правильный шаг сетки (но зато, если мы что-то знаем про прямые, которые ожидаем, можем сузить пространство поиска, например, искать только прямые под углами от 20 до 30 градусов).
\end{enumerate}

\subsubsection{Преобразование Хафа. Вопросы реализации}

Для поиска максимумов можно сглаживать значения в аккумуляторе (когда в сетке было слишком много ячеек, появляется много максимумов для прямых рядом $ \Rightarrow $ можно выбрать одну, доминирующую прямую).

\subsubsection{преобразование Радона}

Делает примерно то же самое.
Его не рассматриваем (недостаточно времени).

\subsubsection{обобщённое преобразование Хафа}

Можем применять преобразование не только для прямых, но и для любых форм, у которых не слишком много параметров (круги, треугольники и так далее) и не слишком большое пространство значений этих параметров.

\subsubsection{Ссылки и использованные материалы}

Антон Конушин подробно рассказывает про фильтры Габора, а также про разложение по словарю, которое в лекции мы не рассматриваемаем.

\subsection{К вопросу о большой лабораторной}

Описание задачи надо будет записать в README, там же описать алгоритм.
Тот же Readme должен содержать описание хода решения задачи: что пробовали, какие результаты получили, какие трудности встречались и как преодолевались.

\textbf{Замечание:} датасет должен состоять ориентировочно из нескольких десятков, может быть, сотни изображений.
В данных должны встречаться все возможные случаи.

\end{document}
