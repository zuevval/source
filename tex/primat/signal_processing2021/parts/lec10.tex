\documentclass[main.tex]{subfiles}
\begin{document}

\section{ Лекция 10. Завершающая }
8 апреля 2021 г.

\subsubsection{Подходы к Feature Selection}

Нужно сравнивать, как ведут себя модели  с признаком и без него.
Сравнивать: по критериям качества (функция потерь, метрики); АБ-эксперименты.

Проблема: бывают зависимые признаки, причём по отдельности они тоже влияют на результат, но вместе лучше этот результат характеризуют.
Зависимость не обязательно линейная: может быть квадратичная, экспоненциальная...

Поэтому, если признаки как-то коррелируют с результатом (не обязательно линейно), можно считать корреляцию между признаком и результатом (или комбинациями признаков и результатом).
Обычно имеет смысл вначале обучить модель, используя все разумные признаки, а на следующих итерациях попробовать комбинации признаков и посчитать корреляции.

\subsection{ Задачи, для решения которых необходимы всевозможные виды сигналов }

Это задачи, которые сводятся к индексированию чего-то.

\subsubsection{ Поиск похожего сигнала }

Дана фотография; в каком месте планеты она сделана?

Эта задача решается без трудоёмкого машинного обучения.
Просто берём большое число картинок, обучаем дескрипторы особых точек и что-то находим.

\subsubsection{ Замена фрагмента сигнала }

Например, хотим убрать здание с изображения.

Выделяем здание; находим среди базы данных картинок такую, где есть участок, цветовые границы в котором наиболее похожи на цветовые границы той области, которую мы вырезаем.

Как быстро искать такую область на множестве картинок из базы -- нетривиальный вопрос, но, впрочем, такую задачу вряд ли нужно решать в реальном времени.

Как всегда, у нас вряд ли есть большое количество размеченных данных (где показано, какой результат мы хотим получить после замены).

Если хотим обучить дерево решений: имеет смысл посчитать какие-то простые, быстро вычислимые признаки объектов и <<целиться>> в них.
Например, как в алгоритме Виола-Джонс, интегральные признаки; или, как при распознавании видео, обходить пиксели в определённом порядке и что-то считать.

\subsection{ Организационные моменты}

\subsubsection{ Теорема о несуществовании демократии }

Теорема: если есть более двух альтернатив и более одного мнения, то выбрать из них справедливую невозможно.

Пример.
Пусть есть три темы: 1, 2, 3 и 2 студента: Вася и Петя.
Можно попросить каждого ранжировать темы по предпочтительности.

Оказывается, что людям в опросе может быть невыгодно высказывать свою реальную точку зрения.

\end{document}
