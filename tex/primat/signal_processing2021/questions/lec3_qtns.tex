% compile with XeLaTeX or LuaLaTeX

\documentclass[a4paper,12pt]{article}
\usepackage[utf8]{inputenc} % russian, do not change
\usepackage[T2A, T1]{fontenc} % russian, do not change
\usepackage[english, russian]{babel} % russian, do not change

% fonts
\usepackage{fontspec} % different fonts
\setmainfont{Times New Roman}
\usepackage{setspace,amsmath}
\usepackage{amssymb} %common math symbols
\usepackage{dsfont}

% utilities
\usepackage{amsthm} % theorems with proofs
\usepackage{systeme} % systems of equations
\usepackage{mathtools} % xRightarrow, xrightleftharpoons, etc
\usepackage{array} % utils for tables
\usepackage{makecell} % multirow for tables
\usepackage{subfiles}
\usepackage{hyperref}
\hypersetup{pdfstartview=FitH,  linkcolor=blue, urlcolor=blue, colorlinks=true}
\usepackage{framed} % advanced frames, boxes
\usepackage{graphicx}
\usepackage{caption}
\usepackage{subcaption} % captions for subfigures
\usepackage{color}
\usepackage{listings} % code chunks
\definecolor{dkgreen}{rgb}{0,0.6,0}
\lstset{language=Python,
    basicstyle=\small\ttfamily,
    stringstyle=\color{dkgreen},
    keywordstyle=\color{blue},
    commentstyle=\color{dkgreen},
    tabsize=2,
    breaklines=true,
    columns=fullflexible
}

% styling
\usepackage{float} % force pictures position
\floatstyle{plaintop} % force caption on top
\usepackage{enumitem} % itemize and enumerate with [noitemsep]
\setlength{\parindent}{0pt} % no indents!

% misc
\graphicspath{{./img/}}
\newcommand{\myPictWidth}{.95\textwidth}
\newcommand{\phm}{\phantom{-}}
\newtheorem{problem}{Задача}

\begin{document}

\section{Вопрос 7}

\textbf{Q:} На каких изображениях детектор Харриса найдет угол вероятнее, чем детектор Ферстнера?
А наоборот?

\textbf{A:} Детектор Фёрстнера неустойчив к шуму.
На рис. \ref{fig:xperia_foerstner} видно, что он находит углы практически везде.

В то же время, на картинке без помех (например, на созданной в редакторе), детектор Фёрстнера может обнаруживать некоторые скруглённые углы, которые детектор Харриса не будет распознавать или ошибочно считать множеством углов (рис. \ref{fig:drawing_harris}, \ref{fig:drawing_foerstner}).

\begin{figure}[H]
    \centering
    \begin{subfigure}{.5\textwidth}
        \centering
        \includegraphics[width=\myPictWidth]{xperia_harris}
        \caption{детектор Харриса}
        \label{fig:xperia_harris}
    \end{subfigure}%
    \begin{subfigure}{.5\textwidth}
        \centering
        \includegraphics[width=\myPictWidth]{xperia_foerstner}
        \caption{детектор Фёрстнера}
        \label{fig:xperia_foerstner}
    \end{subfigure}

    \begin{subfigure}{.5\textwidth}
        \centering
        \includegraphics[width=\myPictWidth]{drawing_harris}
        \caption{детектор Харриса}
        \label{fig:drawing_harris}
    \end{subfigure}%
    \begin{subfigure}{.5\textwidth}
        \centering
        \includegraphics[width=\myPictWidth]{drawing_foerstner}
        \caption{детектор Фёрстнера}
        \label{fig:drawing_foerstner}
    \end{subfigure}
    \caption{Нахождение углов детектором Харриса и детектором Фёрстнера}
    \label{fig:1}
\end{figure}

Исходный код (Python 3):

\begin{lstlisting}
import cv2
import numpy as np
from skimage.feature import corner_foerstner

for input_name in ("xperia", "drawing"):
    img = cv2.imread(input_name + ".jpg")
    gray = np.float32(cv2.cvtColor(img, cv2.COLOR_BGR2GRAY))
    dst = cv2.cornerHarris(gray, 2, 3, 0.04)

    img_harris = np.copy(img)
    img_harris[dst > 0.01 * dst.max()] = [0, 0, 255]
    cv2.imwrite(input_name + "_harris.jpg", img_harris)

    w, q = corner_foerstner(gray)
    accuracy_thresh = 1.8
    roundness_thresh = 0.99
    foerstner = (q > roundness_thresh) * (w > accuracy_thresh) * w

    img_foerstner = np.copy(img)
    img_foerstner[foerstner > .5] = [0, 0, 255]
    cv2.imwrite(input_name + "_foerstner.jpg", img_foerstner)
\end{lstlisting}

\section{Вопрос 8}


\begin{problem}
Дано:
\begin{align}
    \label{eq:laplacian}
    & L(x,y; \sigma) = (x^2 + y^2 - 2 \sigma^2) e^{-\frac{x^2+y^2}{2 \sigma^2}} \\
    \label{eq:f}
    & f(x,y) = \begin{cases}
        1, x^2 + y^2 \le r^2 \\
        0 \text{ иначе}
    \end{cases} \\
   \label{eq:convolution}
   & g(\sigma) = f * L
\end{align}
Доказать, что $ \sigma_0 = \frac{r}{\sqrt 2} $ -- экстремум функции $g$ \eqref{eq:convolution}.
\end{problem}

\begin{proof}
\[ g(\sigma) = f*L = \iint_D L(x,y) dxdy, \thickspace D := \{(x, y) | f(x, y) = 1\} \\ \]

Перейдём к полярным координатам $ (\rho, \phi)$:
\[ \begin{cases}
    x = \rho \cos(\phi) \\
    y = \rho \sin(\phi) \\
\end{cases}, \thickspace |J| = \rho \]
\begin{align*}
    g(\sigma) & = \iint_D L(x(\rho, \phi), y(\rho, \phi)) |J| d\rho d\phi  \overset{\eqref{eq:laplacian}}= \int_0^{2 \pi} d\phi \int_0^r d\rho \left[ (\rho^2 - 2 \sigma^2) e^{-\frac{\rho^2}{2 \sigma^2}} \cdot \rho \right] \\
    & = 2 \pi \int_{\rho=0}^{\rho=r} \sigma^2 \left(\left(\frac{\rho}{\sigma}\right)^2 - 2 \right) e^{- \frac{1}{2} \left(\frac{\rho}{\sigma}\right)^2} d \left[ \frac{\sigma^2}{2} \left(\frac{\rho}{\sigma}\right)^2\right]
\end{align*}
Замена: $ t = \frac{\rho^2}{\sigma^2} $

\[g(\sigma) = \pi \sigma^4 \int_{t=0}^{t=\frac{r^2}{\sigma^2}} (t-2) e^{-\frac{t}{2}} dt = [...] = \pi \sigma^4 \left(-2 \frac{r^2}{\sigma^2} e^{-\frac{r^2}{2 \sigma^2}}\right) \]
\begin{equation}\label{eq:g}
    \boxed{ g(\sigma) = -2 r^2 \sigma^2 e^{-\frac{r^2}{2 \sigma^2}} }
\end{equation}

Введём функцию $ h(\sigma) := \frac{g(\sigma)}{2 r^2} $. Решение сводится к доказательству того, что $ \sigma_0 $ есть экстремум $ h(\sigma) $.

\begin{align*}
    & h(\sigma) \overset{\eqref{eq:g}}= -\sigma^2 e^{-\frac{r^2}{2 \sigma^2}} \\
    & h'(\sigma) = \left( \frac{r^2}{\sigma} - \sigma \right) \cdot e^{-\frac{r^2}{2 \sigma^2}} \\
    & h'(\sigma_0) = 0 \\
    & h''(\sigma) = \frac{1}{\sigma} e^{-\frac{r^2}{2 \sigma^2}} (r^4 - 2 r^2 \sigma^2 - \sigma^4) \\
    & sign(h''(\sigma_0)) = sign(r^4 - 2 r^2 \sigma_0^2 - \sigma_0^4) = sign \left(\frac{r^4}{2} - r^4 \right) = -1 \\
\end{align*}

Т. о. первая производная $h$ обращается в ноль в точке $\sigma_0$, вторая производная отрицательна.

\end{proof}



\end{document}