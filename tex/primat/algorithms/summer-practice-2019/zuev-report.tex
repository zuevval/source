%XeLaTeX+makeIndex+BibTeX
\documentclass[a4paper,12pt]{article} %14pt - extarticle
\usepackage[utf8]{inputenc} %русский язык, не менять
\usepackage[T2A, T1]{fontenc} %русский язык, не менять
\usepackage[english, russian]{babel} %русский язык, не менять
\usepackage{fontspec} %различные шрифты
\setmainfont{Times New Roman}
\defaultfontfeatures{Ligatures={TeX},Renderer=Basic}
\usepackage{hyperref} %гиперссылки
\hypersetup{pdfstartview=FitH,  linkcolor=blue, urlcolor=blue, colorlinks=true} %гиперссылки
\usepackage{subfiles}%включение тех-текста
\usepackage{graphicx} %изображения
\usepackage{float}%картинки где угодно
\usepackage{textcomp}
\usepackage{fancyvrb} %fancy verbatim for \VerbatimInput

%\usepackage[dvips]{color} %попытка добавить цвета типа OliveGreen. Пропадают все картинки

\usepackage{dsfont}%мат. символы
\newcommand{\ovr}[1]{\overrightarrow{#1}}
\usepackage{listings} %code formatting
\lstset{language=c,
	%keywordstyle=\color{blue},
	%commentstyle=\color{},
	%stringstyle=\color{red},
	tabsize=1,
	breaklines=true,
	columns=fullflexible,
	%numbers=left,
	escapechar=@,
	morekeywords={UINT_MAX, TEST, EXPECT_EQ, EXPECT_TRUE, EXPECT_FALSE}}


\begin{document}
\subfile{titlepage}
\setcounter{page}{2}
\tableofcontents
\newpage
\subfile{part1}
\newpage
\section{Исходный код программы}
\subsection{файл <<main.c>>}
main.c - контроллер программы.
\lstinputlisting{code/main.c}
\subsection{файл <<graph.h>>}
Файлы graph.c, graph.h содержат функции для работы с графом как множеством рёбер.
\lstinputlisting{code/graph.h}
\subsection{файл <<graph.c>>}
\lstinputlisting{code/graph.c}
\subsection{файл <<dsf.h>>}
Файлы dsf.h, dsf.c содержат функции для работы с системой непересекающихся множеств (Disjoint set forest, DSF).
\lstinputlisting{code/dsf.h}
\newpage
\subsection{файл <<dsf.c>>}
\lstinputlisting{code/dsf.c}
\section{Описание тестирования}
Для тестирования применён фреймворк Google Test Framework. Раздельно протестированы функции \lstinline|unite(node_t *, node_t *), | Также проведён контроль отсутствия утечек памяти с помощью утилиты Valgrind Memcheck. Ниже приведён код тестирующей программы.\\
\lstinputlisting{code/tests.cpp}
\newpage
\section{Результаты тестирования}
Ниже приведён вывод тетирующей программы: тесты пройдены, утечек памяти нет.\\
\VerbatimInput{code/test.txt}
\newpage
\section{Выводы}
Использование системы непересекающихся множеств, реализованной на списках, позволяет достичь алгоритмической сложности $O(n + m)$ при объединении двух множеств, где $n, m$ -- мощности объединяемых множеств. Следовательно, алгоритм вычисления множеств вершин, находящихся в одной компоненте связности, работает за $O(p \cdot (q + 1))$, $p$ -- множество вершин, $q$ - множество рёбер. На графах с малым количеством рёбер это выгоднее по времени, чем использование алгоритма Флойда вычисления транзитивного замыкания, сложность которого -- $O(p^3)$.\\
Код программ, написанных для данной работы, доступен по адресу \url{https://github.com/zuevval/source/tree/master/c/disjoint_set_forest}.

%\subfile{part2}
\begin{thebibliography}{100}
	\bibitem{novikov} Новиков Ф. А. Дискретная математика для программистов: Учебник для вузов. 3-е изд. -- СПб: Питер, 2008. -- 384 с.: ил. -- (Серия <<Учебник для вузов>>).
	\bibitem{ifmo} СНМ (наивные реализации) [Электронный ресурс] // Викиконспекты ИТМО, 2018. URL: \href{https://neerc.ifmo.ru/wiki/index.php?title=\%D0\%A1\%D0\%9D\%D0\%9C_(\%D0\%BD\%D0\%B0\%D0\%B8\%D0\%B2\%D0\%BD\%D1\%8B\%D0\%B5_\%D1\%80\%D0\%B5\%D0\%B0\%D0\%BB\%D0\%B8\%D0\%B7\%D0\%B0\%D1\%86\%D0\%B8\%D0\%B8)}{https://neerc.ifmo.ru/wiki/index.php?title=СНМ\_(наивные\_реализации)}  (Дата обращения: 10.07.2019)
\end{thebibliography}
\end{document}