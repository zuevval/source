Моделирование структурными уравнениями (Structural Equation Modelling, SEM \cite{hoyle2021sem}) -- комплекс методик, позволяющих описать рассматриваемую систему набором линейных уравнений, оперируя понятиями \emph{скрытых переменных} (latent variables), иначе называемых \emph{факторами}, и \emph{наблюдаемых переменных} (observed variables).
Если исследователь располагает набором данных о своей системе (например, измерением генотипов и фенотипов множества растений), он может построить структурную модель; для этого нужно:
\begin{enumerate}
    \item Возможно, ввести какие-либо переменные, значения которых измерить не удаётся, но которые, по мнению исследователя, могут быть интерпретированы в терминах предметной области (например, размер растения как обобщение длины, ширины, веса)
    \item Указать, какие величины, по мнению исследователя, связаны отношением линейной регрессии
\end{enumerate}

Процедура SEM заключается в нахождении коэффициентов регрессии и соответствующих P-значений, а также дисперсий зависимых переменных (\emph{зависимой} в SEM называют переменную, которая зависит от хотя бы одной другой переменной).
Дисперсии независимых переменных считаются фиксированными и принимаются равными выборочным дисперсиям.

Существуют вариации SEM, которые отличаются, помимо прочего, используемыми при оценке параметров функциями цели.
В 2019-2021 годах на кафедре Прикладной математики Политехнического университета был разработан программный пакет Semopy \cite{semopy, semopy2}, реализующий различные версии SEM; он позволяет оптимизировать функцию максимального правдоподобия Уишарта (Maximum Likelihood, ML), для данных с большим количеством пропущенных значений -- функцию максимального правдоподобия с полной информацией (Full Information Maximum Likelihood, FIML); и другие.

В процессе оптимизации требуется многократно вычислять значение функции цели и её градиента.
Значительные вычислительные ресурсы требуются для расчёта выражений вида $ tr(f(A)), A \in \mathds{R}^{n \times n}, n \in \mathds N $, входящие в целевую функцию, в результате чего процедура в отдельных случаях занимает длительное время даже на современных вычислительных устройствах.
Применение приближённых алгоритмов для расчёта следа от матричной функции может дать выигрыш во времени.
