По графикам времени работы, изображённым на рис. \ref{fig:lanczos_benchmark3} и рис. \ref{fig:lanczos_benchmark1}-\ref{fig:lanczos_benchmark4}, видно, что оптимизированная версия алгоритма Ланшоца работает быстрее детерминированного метода в исследованных диапазонах параметров.
Следовательно, её можно применять на практике для ускорения вычислений, но нужно подобрать оптимальное значение $ k $ и $ p $.

Как видно из рис. \ref{fig:trdev_lanc_rel}, при $ p \in \{ 5; 10 \} $ и собственных числах матрицы до $ 30 $ по модулю среднее относительное отклонение не превышает $ 0.15 $, т.о. при желаемом относительном отклонении $ \le 0.15 $ можно ограничиться $ p=10 $ при $ k=5 $; этот вывод, впрочем, требует дополнительных проверок на матрицах больших размеров.

Значение $ k $, как видно из графиков, изображённых на рис. \ref{fig:lanczos_benchmark3} и \ref{fig:lanczos_benchmark1}-\ref{fig:lanczos_benchmark4}, необходимо выбирать больше двух; при $ k=3 $ или $ k=4 $ средняя относительная ошибка с ростом $ n $ убывает и достигает $ 0.2 $-$0.4$.
Предпочтительно использовать $ k=4 $ (точность заметно возрастает, но время работы увеличивается лишь ненамного).
