На данный момент алгоритм программно реализован в двух версиях: первая (<<наивная>>) использует непосредственно формулы \eqref{eq:lanczos}, вторая более оптимизирована по памяти за счёт отказа от хранения всех векторов $ q_k $ (на каждой итерации используется только два предыдущих).

Поставлен ряд вычислительных экспериментов с целью оценки точности и скорости работы алгоритмов:

\begin{enumerate}
    \item Для функции
    \begin{equation}\label{eq:matrixFun}
        f(A) = A^{-1}, A \in \mathds{R}^{n \times n}
    \end{equation}
    проведены измерения точности оценки следа $tr(f(A))$ <<наивным>> алгоритмом (для размера матрицы от $ 2 \times 2 $ до $ 45 \times 45 $).
    При каждом значении $ n \in \{ 2;45 \} $ сгенерированы $ 30 $ матриц $ n \times n $ с собственными числами, равномерно распределёнными в диапазоне $ [1; 5] $.
    Матрицы получены с помощью генерации собственных чисел (указанным выше образом), а также собственных векторов (с помощью функции \texttt{ortho\_groups.rvs} \cite{mezzadri2007generate} из пакета \texttt{scipy}) и распределены по закону Уишарта.
    Число итераций алгоритма Ланшоца положено $ k = \lfloor \frac{n}{2} \rfloor + 1 $, число итераций процедуры Хатчинсона $ p = 10 $.
    Этот эксперимент не может дать представления о поведении алгоритма на практике, поскольку в реальности для достижения выигрыша по времени необходимо брать $ k \ll n $ итераций, но служит для проверки корректности работы алгоритма в области малых значений $ n $. % TODO добавить ещё с другим числом итераций Ланшоца (на графике другим цветом, должно быть при большем числе итераций строго ниже)

    \item Для матриц $ 10 \times 10 $ измерена ошибка вычисления следа с помощью метода Ланшоца при $ k=2 $, а также с помощью алгоритма Хатчинсона при использовании детерминированного алгоритма вычисления квадратичной формы (что равносильно $k=n=10$ шагам алгоритма Ланшоца).
    Измерения проведены при генерации матриц с собственными значениями, сэмплированными из различных диапазонов, а также при различных значениях $ k $ в интервале $ \{ 1;9 \} $

    \item Проведено сравнение времени работы алгоритмов (<<наивного>>, оптимизированного и детерминированного алгоритма \texttt{trace} из пакета ) и относительного отклонения результата вычисления выражения \eqref{eq:matrixFun} стохастическим способом от результата, полученного применением детерминированного алгоритма.
    Сравнение проводилось для матриц размера $ 10 \times 10 $ при $ p = 10 $ итерациях процедуры Хатчинсона в диапазоне значений числа итераций алгоритма Ланшоца $ k \in \{1; 4\} $.

\end{enumerate}

При реализации алгоритмов использована программная среда \texttt{Python3.8}.
Для вычисления собственных значений трёхдиагональной матрицы $ T_k $ используется функция \texttt{linalg.eigh\_tridiagonal} из пакета \texttt{scipy}, которая обращается к процедуре \texttt{DSTEMR} из пакета \texttt{LAPACK} \cite{dhillon2003orthogonal}.
