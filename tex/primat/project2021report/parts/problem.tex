Одна из моделей, реализованных в пакете Semopy \cite{semopy2}, -- модель со случайными эффектами -- описывается уравнениями \eqref{eq:modelEffects}:

\begin{equation}\label{eq:modelEffects}
    \begin{cases}
        \begin{pmatrix}
            H \\ X^{(1)}
        \end{pmatrix} = W = \Gamma_1 X^{(2)} + BW + E, \quad E \sim \mathcal{MN}(0, \Psi, I_n) \\
    \begin{pmatrix}
        Y \\ X^{(1)}
    \end{pmatrix} = Z = \Gamma_2 X^{(2)} + \Lambda W + \Delta + U, \quad \Delta \sim \mathcal{MN}(0, \Theta, I_n), U \sim \mathcal{MN}(0, D, K)

    \end{cases}
\end{equation}

Здесь $ H \in \mathds{R}^{n_\eta \times n} $ -- матрица скрытых переменных, $ X^{(1)} \in \mathds{R}^{n_{x1} \times n} $, $ X^{(2)} \in \mathds{R}^{n_{x2} \times n} $ -- матрицы эндогенных (т.е. зависящих от чего-либо в нашей модели) и экзогенных (ни от чего не зависящих) переменных соответственно; $n$ -- число наблюдений.
Матрицы $ D, K $ известны; матрицы $ \Gamma_1, \Gamma_2, B, \Lambda, \Theta, \Psi $ некоторым образом параметризованы, и требуется оценить значения параметров. \\

Как показано в \cite{semopy2}, величина $ Z $ также распределена по матричному нормальному закону: $ Z \sim \mathcal{MN}(M(\theta), \hat L(\theta), \hat T(\theta)) $, где $ L = cov(Z), \hat L = \frac{L}{tr(\hat T)}; T = cov(Z^T), \hat T = \frac{T}{tr(\hat L)} = \frac{tr(\hat T)}{tr(L)} T $; и негативная функция максимального правдоподобия для векторного параметра $ \theta $ даётся выражением \eqref{eq:modelEffectsML}:

\begin{equation}\label{eq:modelEffectsML}
    f(\theta | Z) = tr \left\{ \hat T^{-1}(Z-M(\theta))^T \hat L^{-1} (Z-M(\theta)) \right\} + n \ln |\hat L| + n_z \ln |\hat T|
\end{equation}

В процессе оптимизации при каждом вычислении функции правдоподобия требуется делать пересчёт $ \hat T ^{-1} $ и $ \ln |\hat T| $.
Заметим, что $ \ln |\hat T| $ -- скаляр, следовательно, представив его как матрицу $ 1 \times 1 $, можно записать:
\begin{equation} \label{eq:scalarTraceTrick}
    \ln |\hat T| = tr(\ln |\hat T|)
\end{equation}

Нахождение \eqref{eq:scalarTraceTrick} есть частный случай задачи о нахождении следа от функции от матрицы, которая может быть приближённо решена методом Гаусса-Ланшоца \cite{golub2010matMomentsQuadr, golub2013matcomput, ubaru2017fast}.
Алгоритм итерационный, и его преимущество заключается в том, что, варьируя число шагов, можно балансировать между точностью и скоростью.
Задача работы -- реализовать алгоритм для оценивания \eqref{eq:scalarTraceTrick}. \\

Отметим, что область возможного применения процедуры Гаусса-Ланшоца в пакете Semopy не ограничивается нахождением параметров модели \eqref{eq:modelEffects} с данной целевой функцией \eqref{eq:modelEffectsML}; они приведены лишь в качестве примера.