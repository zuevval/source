Описание результатов экспериментов идёт в том же порядке, что и описание условий экспериментов в разделе \ref{section:experiments}.

\begin{enumerate}
    \item Результаты изображены на рис. \ref{fig:rel_deviation}. Видно, что ошибка в среднем $ \le 0.06 $ и постепенно убывает; это демонстрирует особенность метода, заключающуюся в повышении точности при увеличении $ n $ за счёт уменьшения дисперсии оценки методом Хатчинсона.
    При $ n = 2 $ ошибка мала из-за того, что $ \lfloor \frac{n}{2} \rfloor + 1 = 2 = n $, соответственно, алгоритм Ланшоца находит точное (в пределах погрешности арифметических операций) значение билинейной формы.

    \item При увеличении $ p $ вычислительная ошибка уменьшается, но, по-видимому, сублинейно.
    Когда собственные значения $ A $ лежат в более широком диапазоне, ошибка возрастает.

    Основной вклад в ошибку вносит приближение алгоритмом Хатчинсона.
    Алгоритм Ланшоца, несмотря на то, что проводится лишь $ k=2 $ шага, вычисляет билинейную форму почти точно. % TODO а почему в пункте 3 не очень хорошие результаты для $ k = 2 $?

    Иллюстрацию можно найти в приложении (раздел \ref{appendix:lanczosAccuracy}).

    \item Результаты изображены на рис. \ref{fig:lanczos_benchmark3} для $ k=3 $; остальные приведены в приложении (раздел \ref{appendix:benchmark}).
    <<Наивная>> и оптимизированная версия дают один и тот же ответ, соответственно, при одинаковых начальных условиях для генерации псевдослучайных чисел при созданиии исходных матриц и векторов дают одинаковую ошибку.
    Вследствие этого ломаные относительных ошибок полностью совпадают.

    При $ k=1 $ средняя относительная ошибка с ростом $ n $ не убывает (в исследованном диапазоне значений) и всегда более единицы (примерно $ 5.8 $).
    При $ k = 2 $ ошибка также не убывает и остаётся на уровне примерно $ 1.5 $.

    Время работы <<наивной>> реализации при $ n > 350 $ (приблизительно) превышает время работы детерминированного алгоритма. Время работы оптимизированной реализации всегда меньше времени работы точного алгоритма и <<наивного>> алгоритма Ланшоца.

\end{enumerate}

\begin{figure}[h]
    \centering\includegraphics[width=.7\linewidth]{rel_deviation}
    \caption{Зависимость относительной ошибки метода при запуске на случайно сгенерированных матрицах от числа элементов в матрице. Ошибка усредняется по 30 запускам}
    \label{fig:rel_deviation}
\end{figure}

\newcommand{\lanczosBenchmarkFig}[1]{
    \begin{figure}[H]
        \includegraphics[width=\myPictWidth]{benchmark/lanczos_iter#1}
        \caption{ Сравнение алгоритмов по времени и точности (при вычислении трёхдиагональной матрицы #1$\times$#1) }
        \label{fig:lanczos_benchmark#1}
    \end{figure}
}

\lanczosBenchmarkFig{3}

Графики для других значений числа итераций метода Ланшоца приведены в приложении (\ref{appendix:benchmark}).
