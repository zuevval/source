\documentclass[main.tex]{subfiles}
\begin{document}
\section{Lecture 3.  Phylogenetics and molecular evolution}
Oct 25, 2021

In the second lecture we discussed annotation.
Today we will focus mostly on protein coding genes.

\subsection{Phylogenetic tree}

These three trees on the picture are completely identical.
Branches can be rotated.

\emph{External node}: each external node is connected to a single (internal) node.
A \emph{clade} is a group in the tree.

"E" is a \emph{"basal group"}.
"E" is not necessairly more ancient than other populations, it is just the most distant from other.

Which state is "ancestral"?
For convenience, many biologists use a single outgroup.
The correct way to build the tree is to add multiple outgroups.
% TODO  ask: what's the problem with the outgroup? Why adding multiple G helps us to reconstruct a correct ancestral state?

Trees may be \emph{rooted} or \emph{unrooted}.
Unrooted trees do not give us direction of evolution, only relationship: which are closer to each other.

How do we find a root?
We add an outgroup!
Without outgroup, we should not draw a root of the tree.

How to make a phylogenetic tree?
The most simple way is to compare DNA sequences directly (measure similarity).

Recall that there are many gene duplications in the genome.
After gene duplication, subfunctionalization, neofunctialization, or degeneration may occur.

Remember that two genes derived from a single gene are \emph{homologs}.
Out of them, \emph{orthologs} are those diverged during the \emph{speciation} (species creation) and \emph{paralogs} are those 
Duplications sometimes distort the way we can reconstruct the tree.
This is why many evolution biologists emphasize that they use only single-copy genes to calculate similarities.

BLAST search genes:
we can search sequences similar to our input sequence.
The most common is \texttt{blastn} (search a nucleotide sequence in the database).

\subsection{DNA sequence alignment}

Because of indels, we cannot calculate similarities between sequences directly: we first need to construct the \emph{alignment}.

If we are dealing with protein coding sequences, we
\begin{enumerate}[noitemsep]
	\item Translate to protein an align with protein, then untranslate back
	\item Align using codons as units 
\end{enumerate}

When we do a next-gen sequencing (Illumina), we "map" short reads to the reference genome (this is a kind of alignment).

Software: MEGA (everybody use it, you just click buttons!)
We can translate genes to proteins, align sequences.

Frame shift mutations: indels of length $ l : l\%3 \ne 0 $ (very bad).

CLSUSTAL-W / MUSCLE: different algorithms for alignment.
Nevertheless, manual adjustment is better (and it is easy to manually adjust alignments using MEGA GUI).

\subsection{Building a phylogenetic tree}

Distance-based methods: we calculate the distance matrix and build a tree from it.

Maximum Parsimony tree (MP): computer calculates all possible trees and picks the one with minimum evolutionary changes.

Why biologists prefer MP over distance-based methods?

There is another problem with reverse mutations.
Distance-based methods look

How do we model the reverse mutation?
Biologists borrowed concepts from mathematics. % YouTube 48:34
C, T (Pyrimidines) interchange more often than A, G (Purines); C interchanges much more often with T (transition) than with G (transversion).
MP considers all mutations to be equally likely; Maximum Likelihood (ML) method addresses this problem (estimates the mutation rate).

When someone is talking about likelihood, there is a Bayes theorem somewhere near.
\emph{Bayesian methods to derive phylogenetic trees are becoming more popular}.

% YouTube 52:41 Bootstrap
To find confidences for branches, we apply \emph{bootstrap} (resampling $ \rightarrow $ the percentage of trees that support this branch).
Then we may "collapse" a low-support branch.

% YouTube 55:07 Back to MEGA: phylogeny
In MEGA, we may place a root on a branch (using the GUI).

% YouTube 58:46
\subsubsection{Food for thought}

Do we need complicated methods (parsimony, likelihood) for population genomics within a species?
In traditional phylogenetics, there is only distant relationship (among mammals, all plants, etc); even using only single-copy genes, we cannot use simple distant-based methods because they cannot handle reverse mutations.
Within one species, the evolutionary time is small and we may do the whole genome comparison.

The big misconception in the field is that it 

\subsection{Basics of molecular evolution}

% TODO a bit

The mutation rate is different from substitution (fixation) rate.
Mutations occur often, but not all are fixed.

Mutations that are good will be fixed; but there are very few mutations that are good.
Most mutations that are fixed are neutral.

Neutral mutation rate $ \mu_0 $: in one generation, the expected number of neutral mutations is $ 2 N \cdot \mu_0 $.

% TODO a bit

A \emph{saturation} of DNA difference (of molecular clock): at some point, mutations start happening at the same sites and species do not diverge so quickly.

A \emph{shadow gene} is a broken gene that is no longer working (and has a pretty high evolution rate).
Intron mutations have pretty high rate, too.
3'-UTR and 5'-cap have low substitution rate.

Evolutionists designed special metrics to estimate the substitution rate: $ K_A / K_S $ % TODO
We find synonymous sites ( $ K_S $) and scale total mutated sites .
$ K_A / K_S = 1 \Rightarrow $ a neutral mutation.
$ K_A / K_S > 1 $: positive selection

Most genes have low $ K_A / K_S $.

\subsection{The McDonald-Kreitman test (coding sequence)}

The length of the branch ~ the evolution time (how many mutation has actually happened).
% TODO some

Presume that our gene is completely neutral (branch lengths totally reflect the mutation rate).

If our gene is fixing many amino acids differences between species (positive selection), the number of non-synonymous divergent site will be significantly larger than 

\end{document}