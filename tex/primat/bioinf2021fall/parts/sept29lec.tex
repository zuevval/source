\documentclass[main.tex]{subfiles}
\begin{document}

\section{РНК-секвенирование}

С помощью РНК-секвенирования можно изучать изоформы.

После получения данных нужно провести нормализацию.
Чем лучше ген транскрибировался, тем больше фрагментов ему соответствует.
Количество прочтений может зависеть от исходной РНК; GC-островки амплифицируются иначе и т.д.
Таким образом, % TODO

Гены домашнего хозяйства ведут себя одинаково, и можно на них нормализовать.

Нормализация на размер библиотеки: можно ввести скалирующий фактор

\end{document}
