\documentclass[main.tex]{subfiles}

\begin{document}
\section{HI-C. Продолжение}
20 октября 2021 года \\

Хроматин разрезаем рестриктазами (этим способом: примерно знаем, где разрезы).
Может оказаться так, что два фрагмента ДНК, принадлежащие разным участкам хромосомы или разным хромосомам, оказываются сшитыми.
Надо найти эти участки.
Делаем с помощью легирования.
Часто при легировании удваивается рестрикционный сайт.
При легировании добавляем фрагменты, меченые биотином.
После этого просто убираем белки % TODO что значит "убираем белки"
Также в какой-то момент разрезаем ДНК ультразвуком (на более короткие фрагменты для Illumina).

Один вариант разреза -- сшиты два кусочка, связанные контактом.
Возможны другие варианты: не легированные фрагменты, self-ligation.

% TODO a bit

Когда paired reads, прочитываем с одного конца в одной цепи, с другого в комплементарной.
Какие цепи легировались, мы не знаем.

Рассматриваются т.н. \emph{расстояния} $ l_s $, $ l_r $, $ l_u $.
 $ l_u $ только для фильтрации очень маленьких фрагментов: если найдены фрагменты меньше $ l_u $, они выбрасываются.
 $ l_s $ есть расстояние между соседними рестрикционными сайтами; позволяет отфильтровать фрагменты, полученные с помощью self-ligation.
 Расстояние $ l_r $ % TODO on my own
 
 \subsubsection{Theoretical model}
 
 Итак, в этой молекуле мы рассматриваем молекулу ДНК как последовательность мономеров.
 Формула, которая показывает вероятность двух мономеров, находящихся на некотором расстоянии $ L $, % TODO some
 
 \subsubsection{Contact maps}
 
 Как вообще эти результаты подаются?
 В виде \emph{контактных карт}.
 
 Грубая карта затем модифицируется.
 Они очень шумные, поэтому карты с разрешением в один нуклеотид никто не делает.
 Могут быть участки длины 1 Мбаза, или 10-100 КБаз...
 
 Можно волюнтаристски задавать ширину окна, а можно исходить из длины рестрикционных фрагментов.
 
 \subsubsection{Какие могут быть ошибки}
 
 \begin{enumerate}[noitemsep]
 	\item Гэпы
 	\item Участки гомологии (дополнительных контактов)
 	\item Гетерогенное распределение длин фрагментов
 	\item Одни хромосомы имеют более открытый хроматин, на других сидит больше белков.
 \end{enumerate}

\subsubsection{Нормализация и коррекция}

Можно пытаться сразу делать коррекцию всех ошибок, считая, что контакты распределены примерно одинаково по районам.

ICE -- iterative correction and eigenvector decomposition

Считаем, что ожидаемая частота контактов... 

По рядам и колонкам всё должно суммироваться к единице.
Сырая матрица контактов
\[ O^{DS}_{ij} = B_i B_j T_{ij}^{DS} \]

Equal Visibility позволяет увидеть подробности \\ % TODO в конце лекции -- разбор алгоритма

Sequential Component Normalization: контакт между районами $ i $ и $ j $ делится на сумму контактов, в которых участвуют районы $ i, j $

\subsubsection{Reproducibility and control}

Считается корреляция -- Пирсона, Спирмена...

Посчитали Log Ratio между двумя матрицами (по пикселям).
Видно, что ничего особенно не нашли...

Также сделали PCA.
Выделяли хроматин во время анафазы, метафазы и т.д.; оказалось, что PCA очень хорошо разделяет эти данные.

\subsubsection{Зачем же нужны карты контактов?}

\textbf{Chromatin compartments}

Можно по модели рассчитать, как меняется вероятность контакта с расстоянием, и сравнить с экспериментально полученной картой контактов.

% TODO a bit

Тут же... такой же островок, если сравниваем хромосомы 14 и 16...

Сделали  PCA; оказалось, что есть районы с положительным, а есть с отрицательным значением <...>

Обработали районы ДНКазой-1

Лизин в 27 положении совпадает с метками А-районов.
Т. о., действительно, есть контакты эу- и гетерохроматина.

Сделали FISH-гибридизацию (\textit{in sito} гибридизация, в клетке).
Показали, что третий район расположен ближе ко второму, чем к первому (хотя, конечно, в развёрнутой ДНК ближе второй к первому).
Но с расстоянием контакты уменьшаются.
\begin{leftbar}
FISH-гибридизация: есть хромосомы; создаём условия для того, чтобы ДНК расплеталась, и тогда можно добавить олиги для окраски.
\end{leftbar}

 
Ещё один принципиальный результат: существуют TADы (topologically associated domains).
TADы включают участки, которые совместно регулируются и совместно экспрессируются.
На границе  TADов есть сайты присоединения CTCF-белка (транскрипционный фактор сайленсинга), а также сайт присоединения белка, который называется кохесин (\emph{cohesin}).
В результате образуется петля от регуляторного участка к белок-кодирующему.
 
Как находятся TADы?
Компьютерные программы.
Берём окна размером 40 килобаз и на расстоянии в 2 мегабазы смотрим частоты контактов.
Считаем % TODO a bit
 
На самом деле, TADы хорошо видны прямо по графикам у растений и дрозофилы, а у человека это не так (уже нужны статистические методы).

\subsubsection{Пример исследования с картами контактов}
 
Гены рибосомальной ДНК расположены в геноме тандемно.
Есть некая органелла, называемая \emph{нуклеосома}, где синтезируется и процессируется рибосомальная РНК.
Есть также некий хроматин, ассоциированный с этой нуклеосомой.
"Штука немного мутная": там много неассоциированного гетерохроматина, но именно внутри него

Перед делением клетки нуклеосома разрушается.
После деления она строится вновь.

В принципе, в клетке присутствует $ n \approx \in \{200; 2000\} $ копий генов рибосомальной ДНК.
И если число копий уменьшать, стабильность генома ухудшается.
Целью работы было выяснить причину нарушения стабильности.

Ген рибосомальной ДНК находится на 2 хромосоме (неактивный) и 4 хромосоме (активный).
Оказалось, что у мутантов с уменьшенным числом копий ($ 20\% $ от исходного числа) меняется организация хроматина, формирующего нуклеосому.
Причём в ходе последующих делений полученных мутантов число тандемных повторов вновь увеличивается!

Карты контактов: решили посмотреть, есть ли разница в организации хроматина.
Вычли одну 

У растений домены хроматина % TODO a bit
Сделан PCA-анализ и показано, что, действительно, есть связь между 
% TODO 

\subsubsection{Другие применения HI-C}

HI-C ещё можно использовать в сборке генома (соотнести скаффолды, понять, что где).

Пример исследования, где качество сборки улучшают с помощью. % TODO

Каждый контиг поделили на половинки H и T (сестринские половинки).
Считают число контактов между несестринскими половинками (density graph).
Потом переходят к некоторому Unfiltered confidence graph, где вес ребра -- не число контсктов, а число контактов, отнесённое к следующему по величине числу контактов.
\begin{leftbar}
Этот алгоритм будет разобран более подробно на Journal Club
\end{leftbar}

Можно использовать HI-C для поиска геномных перестроек.

Взяли карты контактов (17 хромосома).
В норме 2 TADа, в них попадают какие-то гены...
У пациента с аномалией мы с помощью HI-C видим, что появился новый TAD (тандемная дупликация большого размера).
В результате два гена кальциевых каналов оказались рядом с чужим регуляторным районом (а кальциевые/натриевые насосы очень нужны!).
В итоге неправильно развиваются фаланги пальцев.

Это мы увидели с помощью HI-C дупликацию.
А можно также увидеть инверсии и транслокации!

% TODO a bit

Смотрим на 17 хромосому.
У каждой хромосомы есть p- и q-плечо.
Кусочек хромосомы был перенесён из  p в q, после чего ещё произошла дупликация.
В результате произошло образование ещё одного TADа. \\

На самом деле, хромосомные перестройки ещё можно использовать CGH (на основе ДНК-чипов), но не позволяет детектировать инверсии, транслокации.


\end{document}