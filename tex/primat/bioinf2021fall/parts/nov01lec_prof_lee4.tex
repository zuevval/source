\documentclass[main.tex]{subfiles}
\begin{document}
\section{Lecture 4.  Population genetics 1: Population structure}
Nov 1, 2021

The next week: Natural Selection

% TODO 

1859: "On the origin of species", Charles Darwin.
1866: Gregor Mendel published his work.
1900: re-discovery of Mendel's work.

We only work with \emph{biallelic SNPs}.
We can simply change SNP genotypes into numbers: 0, 1, 2.
It is not important which allele we call 0 and which 2; we care only about differences, discan.
The convention is like this: the \emph{reference} allele is 0, the \emph{alternative} is 2.

We may calculate the pairwise distance matrix and then do UPGMA, neighbor-joining, multidimensional scaling, ...

There are more sophisticated methods, e.g. based on machine learning -- 

PCA: there are \emph{scree plots}.
PC 1 explains the most of the variation (by definition).

There is nothing fancy about population genetics software: we just use traditional statistical methods.

PCoA (principal coordinates analysis, or \emph{multidimensional scaling}): a bit different.

In genomics, we deal with hundreds of thousands or sometimes with millions of SNPs ($ p \gg n $).
We can get pairwise distance matrix and build PCoA.

A genetic PCA map of Europeans surprisingly resembles the European geographic map.

\subsubsection{Problems related with PCA: Uneven sampling}

Population

Methods using the K-clustering "idea":

\begin{itemize}[noitemsep]
	\item STRUCTURE, fineSTRUCTURE, ADMIXTURE
	\item First set a fixed K value representing K genetic groups
	\item ... % TODO
\end{itemize}

\begin{leftbar}

Non-negative matrix factorization:

TODO a very good example!! % TODO

But many people still use STRUCTURE or ADMIXTURE software.

\end{leftbar}

What is the best $ k $ value in biology?
There is no best $ k $ value!

"why is K-means so common in popgen, rather than hierarchical clustering, even when we don't have a priori expectations for K"?
- "There is no need to be obsessed with one best K value. It makes more sense to compare different K. That is similar to a hierarchical clustering".

Puerto RIco and other islands to the south of the USA were the first stop on the way of slave transportation from Africa to the USA.

Isolation by distance: two population are closer to each other when they live in the same area.

Northern Americans show weaker patterns of isolation by distance.

\subsubsection{When does isolation by distance NOT work?}

Arabidopsis study: in South Europe there is the native population.

Isolation by distance is the null hypothesis of spatial genetic vatiation! \\

\subsection{Genetic drift}

Why two population become different when they are separated?

There are mutations, but they do not contribute much to the genetic variation.
What 

Genotype frequencey of the next generation depends only on allele frequency (p,q), \textbf{not} original genotype frequency (X, Y, Z).

HWE tells us that as long as some assumption are met, the next generation's frequncies are (p, q) (no evolution!).

Assumptions: random mating, no natural selection, no mutations between A \& a, very large population = no genetic drift, no dramatic population size changes, etc...
 
Genetic drift is essentially a process of random sampling of genotypes.
It's like bootstrapping.

If we give the system enough time and the population is small enough, there will be a single allele.
Mitochondrial Eve.

% TODO mitochondrial Eve

Drift is determined by population size.
It's a key factor in popgen.

$F_{ST}$: assuming we have a species with 5 populations, each with 100 individuals.
Expected heterozygosity $ H_T = 2pq $.
We may compare $H_T$ and observed heterozygosity $H_S = 2pq() $ % TODO foruma
\[ F_{ST}  = \frac{H_T - H_S}{H_T}\]

Heterozygosity: assume two population with strong gene flow between them.

Population 1: $ A $ allele: $ p = 0.55, q = 0.45 $;

% TODO 

In a model with two populations with weak gene flow: 

Population 1: $ p=0.8, q=0.2 $; population 2: $ p=0.2, q=0.8 $ \\

When there is a species with 2000 individuals, then in three models (1 population of 2000, 10 populations of 200, 100 populations) results will be very different.
This is the essence of $ F_{ST} $.

\subsubsection{Effective population size}

In previous cases, despite the population size is small, we assume that the population is ideal.

The \emph{census population size} is simply a head count.

Imagine we have a population of 100 individuals.
Suppose we can estimate the amount of drift after many generation.
In the real world the amount of genetic drift will be stronger than simulated.

$ N_e $ -- the effective population size.
E.g. in walrus herds there are only a few males.

Drift and low $ N_e $ is bad.
"Pure blood" dogs have

What's the human effective population size?
Census size: $ > 7.8 \times 10^9 $.

Effective population size: about 3000 individual.

\end{document}