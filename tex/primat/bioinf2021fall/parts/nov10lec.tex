\documentclass[main.tex]{subfiles}
\begin{document}
\section{Single Cell RNA Sequencing. Продолжение}

Напомним, основная технология -- droplets.

При PCR РНК реверс-транслируются в К-ДНК.

В отдельную лунку планшета попадает одна бусина.
В лунке идёт реакция амплификации, поскольку материала мало и без амплификации ничего не сможем детектировать.

После прочтения сигнала -- Quality control.
Убираем адаптеры.
Получаем статистику прочтений.

Затем картируем на геном и смотрим, сколько куда картируется.
Принимают во внимание те прочтения, которые приходятся на экзоны (ещё могут быть <...> junctions, а могут иногда быть интроны, т.к. в ядре есть и несплайсированная РНК).

Далее нормализация.
Она всегда заключается в том, что находим какой-то скалирующий фактор и на него делим.
Классическая нормализация -- на размер библиотеки.
Если нормализовать сначала на размер гена, а потом на размер библиотеки, получим данные, которые можно сравнивать.
Но это плохие методы нормализации; есть Trimmed mean of m-values,  а сейчас де-факто стандарт -- relative log expression (Dseq).
Напомним, новая программа \textbf{Kallisto} работает быстро, поскольку использует псевдовыравнивания. Строим индекс -- цветной граф де Брёйна, хэш-таблицу... % TODO ? a bit

Особенность РНК -- разреженные матрицы.
 Хитрость -- \emph{деконволюция} (Pull of cells): вроде bootstrap -- случайно выбираем с возвращением несколько (например, 10) клеток
 
 % TODO a bit
 
 Можем переписать формулу для матожидания с помощью скалирующего фактора.
 Матожидание для пула можем записать так:
 \[ \mathds E \] % TODO formula
 Находим матожидание пула, отнесённое к матожиданию всего датасета.
 \[ \mathds E (R_{ik}) =  \] % TODO formula
 Это линейное уравнение.
 Можем найти такие уравнения для большого числа выборок и получить переопределённую систему.
 Затем методом наименьших квадратов находим аппроксимацию. \\
 
 Ещё один способ нормализации: добавить известную RNA.
 Тогда весь привнесённый шум будет техническим. \\
 
После нормализации -- лог-трансформация.

После лог-трансформации -- Feature Selection.
Данные многомерные и гетерогенные.
Мы исследуем профили экспрессии (данные для отдельного гена).
Нужно выбрать те признаки, которые соответствуют гетерогенности между типами клеток, а не 
Transcription Burst: ген экспрессируется импульсами (существуют периоды покоя, сменяющиеся периодами активности).

Наша задача -- в имеющейся популяции клеток выделить разные типы.
Типы клетки характеризуются различной экспрессией генов; нужно выбрать самые вариабельные гены.

Напомним, всегда есть связь между уровнем экспрессии и вариабельностью (если данные не логарифмированные, то чем больше значение, тем больше вараибельность).
Можно аппроксимировать значение % TODO a bit
и найти те гены, у которых 

Известно, что у отдельных генов (например, клеточного цикла -- киназы) всегда высокий уровень экспрессии; они могут мешать опр

Если в пробы добавлена spike-in RNA, можно строить аппроксимацию зави

Если spike-in DNA нет: когда у нас есть технические репликаты, можно считать, что зависимость дисперсии от среднего описывается законом Пуассона.

Confounding factors: <<мешающие факторы>> -- то, что мешает выяснить зависимость.
Batch effect -- влияние партии: допустим, у нас есть много планшетов; планшеты обрабатываются немного по-разному, что даёт дополнительную техническую вариабельность.
Эти две вещи убираются с помощью линейных моделей... % TODO a bit

Обычно выбираем $ 5\% $ генов с самой высокой вариабельностью и исследуем их.
Для визуализации можно сделать dimensionality reduction.
Делаем кластеризацию: иногда на исходных данных, иногда на данных с уменьшенной размерностью.
В Single-Cell RNA берут обычно много компонент (до 50), но важно отметить, что гены действуют координированно.
t-sne: параметр -- perplexity. t-sne увеличивает компактность <<размазанных>> кластеров и уменьшает 

t-sne: простой; umap -- сложный.
Параметры UMAP: n\_neighbours, min\_dist (минимальное расстояние между двумя соседями).

t-sne и UMAP чаще используются для визуализации.
Некоторые этапы анализа проводятся на PCA-данных.

Кластеры -- это proxy для того, чтобы увидеть разные типы клеток.
Но мы никогда не можем сказать, правильно ли кластеризовали и правильное ли число кластеров; обычно результат кластеризации оценивают по биологии.

Есть ещё иерархическая кластеризация; она распространена в данных RNA-Seq, но здесь -- нет (разве что на данных после PCA), поскольку сложность резко возрастает с размерностью задачи, а в single-cell множество образцов-клеток.
% TODO some
Первый метод кластеризации, который используется -- graph-based.
Ещё один метод -- k means. Недостаток k-means -- параметр (число кластеров). Обычно

\begin{leftbar}
	То, что в этой презентации, описано в книжке о том, как делать обработку данных single cell RNA Seq  с помощью Bioconductor.
\end{leftbar}

Есть двухшаговая кластеризация, \\

Следующий этап обработки -- нахождение маркерных генов (тех, которые определяют дифференциальную экспрессию).
Рассматриваем гены вутри кластеров и считаем различные статистики:

То, что отличает single cell RNA sequncing от bulk RNA sequence: cohen's D \& AUC.
Эти статистики были введены в медицине: они сказали <<зачем нам считать p-value различий между кластерами? Когда мы создавали кластеры, мы уже задали их так, чтобы точки в разных кластерах были разными>>.

Cohen's D есть разность между средними, отнесённая к... примерно то же, что и...
$ AUC = \Phi\left( \frac{d}{\sqrt{2}} \right) $

mean, median, minimum
Посчитав все эти статистики, располагаем гены по значению статистики и присваиваем генам ранги, которые затем используем.

Далее -- аннотация генов и траектории.

Результат Single Cell RNA Sequencing, как правило -- классификация клеток, к какому типу они относятся.

\end{document}