\documentclass[main.tex]{subfiles}
\begin{document}
\section{   }
Ковалентная связь.
Первичная структура белка; структуры более высоких порядков.
Каждая аминокислота имеет свой код (Leu, Ser).
Всего 64 кодона и 20 аминокислот.
3 кодона бессмысленные -- stop-кодоны.

Бывает, что одной аминокислоте соответствует 6 аминокислот.

Физико-химические свойства аминокислот определяют структуру.
Есть, к примеру, разделение на гидрофильные и гидрофобные аминокислоты; часть ни гидрофильные, ни гидрофобные.

Атом, связанный с карбоксильной группой, называет \emph{альфа-атомом} углерода.

Существуют заряженные аминокислоты (которые в воде образуют ионы) - His, Lys, Arg (отрицательно заряженные) и Asp, Glu ().
Есть ароматические аминокислоты.
Ароматическая группа большая по размеру.

Все эти свойства влияют на структуру белка и на белок-белковые взаимодействия.

В основном белки, взаимодействующие с аминокислотами, заряжены положительно, и на их поверхности находятся 

Белок, находящийся в составе мембраны, на своей поверхности должен содержать гидрофобные аминокислоты.

Очень важный аминокислотный остаток -- цистеин, поскольку он образует дисульфидные связи (дисульфидные мостики), которые важны для образования связей более высокого порядка.

Есть полярные и неполярные аминокислоты.

\subsection{Вторичная структура}

Белок начинает сворачиваться уже во время трансляции.
В образовании вторичной структуры участвуют атомы остова полипептидной цепи.

$\alpha$-спирали и $\beta$-листы -- основные элементы вторичной структуры.

Альфа-спирали: аминокислоты соединяются водородными связями через 4.

Бета-листы могут быть параллельными или антипараллельными.
Если лист начинается с атома углерода и оканчивается атомом азота, это параллельный лист, иначе антипараллельный.

Некоторые аминокислоты (например, пралин) не любят находиться во вторичной структуре.

\subsubsection{Способы визуализации вторичной структуры белка}

\begin{enumerate}[noitemsep]
	\item С помощью визуализации отдельных молекул
	\item Спирали, листы (стрелка указывает направление: параллельный или антипараллельный) + петли
\end{enumerate}

\subsection{Прочие структуры}

\subsubsection{Третичная структура}

Вторичная структура образуется с помощью водородных связей остова.
За образование третичной структуры отвечают остатки, и там в основном нековалентные связи (ионные, водородные; полярные (ван-дер-ваальсовские) взаимодействия).

\subsubsection{Четвертичная структура}

Четвертичная структура -- белок из нескольких субъединиц.
Взаимодействия -- те же, что в формировании третичной структуры.


Примеры -- гемоглобин; иммуноглобулины (антитела), которые состоят из четырёх цепей, удерживаемых вместе дисульфидными мостиками.

\subsection{Посттрансляционные модификации белков}

Практически все аминокислоты модифицируются после трансляции (фосфорилирование, гликозилирование, метилирование, ацетилирование, убиквитинирование (чтобы потом, после выполнения своей функции, белок был разрушен), сумоилирование (добавление групп  SUMO)).

Добавление метильных групп часто ведёт к инактивации белка.

\begin{leftbar}
	Представим, что белок синтезировался на рибосоме.
	Попадает в какие-то везикулы, путешествует по аппарату Гольджи.
	Белок перемещается, встраивается в состав мембраны.
	Всё это делают разные белковые комплексы.
\end{leftbar}

Также часть полипептидной цепи может вырезаться.

Пример -- инсулин.
Этот гормон регулирует обмен белков, жиров и углеводов.
Он синтезируется в поджелудочной железе и транспортируется в печень.

\begin{leftbar}
	Чтобы белок куда-то транспортировался, у него должен быть какой-то кусочек, который служит пропуском -- к нему присоединяется сигнальный пептид.
\end{leftbar}

Из инсулина 

\subsection{Методы определения третичной структуры}

Экспериментальные: обычный -- рентгеноструктурный анализ.
Современный -- криоэлектронная микроскопия.

Напомним, Никос читал нам об этом шесть лекций.

\subsection{ Молекулярная динамика }

Как же белок укладывается?

Молекулярная динамика -- компьютерный метод численных расчётов физических движений атомов и молекул.

История: Лос-Аламовская лаборатория (ядерный центр, это как унас Арзамас/Саров).

1955 год -- Лос-Аламовский суперкомпьютер.

Начальное состояние: задаётся число частиц и их позиций в системе, есть ли у них заряд.

Вводится такое понятие, как поле сил молекулярных взаимодействий.
Поле сил ксть сумма потенциальных взаимодействий (ковалентных и нековалентных: потенциальная энергия связей и вращений).

Ковалентные взаимодействия между атомами описываются как пружины с равновесным состоянием.
Нековалентные взаимодействия: их расчёт -- гораздо более затратный процесс.
Во многих приложениях достаточно описания Ван-дерВаальсовых взаимодействий, которые описываются потенциалом Леннарда-Джонса (формула весьма приближённая).

\[ U(r) = 4 \varepsilon \left[ \left(\frac{\sigma}{r}\right)^{12} - \left(\frac{\sigma}{r}\right)^6 \right] \]

Для расчёта электростатических взаимодействий используют кулоновские силы, что тоже не очень физически оправдано.

\subsubsection{Расчёт МД. Ограничения метода}

Ньютоновская механика.

Оптимизация.
Verlet integration -- что-то вроде симплекс-метода.

Используем законы механики, а не термодинамики, что ограничивает нашу точность вычислений.
Правильнее было бы ... % TODO список

Drug Discovery

Время сворачивания белков -- миллисекунды.
Но современные компьютеры могут рассчитывать только нано- или микросекунды.

Университет штата Вашингтон: \verb| folding@home | -- использование распределённых вычислительных мощностей для расчёта молекулярной динамики на разных компьютерах.

Кластеризация; метастабильные состояния. % TODO немного про скрытые марковские модели

Пакет, который наиболее часто используется для MD -- Rosetta.
В ней по сравнению с рассматриваемыми нами формулами используются улучшенные формулы.
Энергия дисульфидных связей, ... учитываются.

Подгонку Розетта делает с помощью оптимизации (алгоритм Метрополиса).

Чаще всего решается задача: имеется некий белок, он немного изменяется (замены каких-то аминокислот) и смотрим, так же он сворачивается или иначе.

Сейчас мы приходим в эру дизайна белков \textit{de novo}.
Какую-то структуру белка моделируем в компьютере, потом её ДНК синтезируем, внедряем в  \textit{Escherichia Coli}, выделяем и смотрим с помощью рентгеноструктурного анализа, что получилось.

Вакцины на основе синтетических белковых нанаконструкций: Walls et al, 2021.

\subsection{Решение задачи о сворачивании белков}

Моделирование по гомологии -- классический метод (сейчас уступил место новым).

Трёхмерные структуры белка более консервативные, чем первичная структура (замена отдельных аминокислот) -- принцип, положенный в основу моделирования по гомологии.

\begin{enumerate}[noitemsep]
	\item Находим белок, гомологичный белку интереса (он называется \emph{матрицей}).
	\item Делаем выравнивание
	\item Считаем, что выровненные участки укладываются так же
\end{enumerate}

Но вставки в белке моделируются таким методом плохо, и чем больше вставка, тем хуже.

До недавнего времени лучшей программой была программа I-TASSER, которая ищет гомологию сразу с несколькими матрицами.
Оптимизация -- с помощью немного модифицированного алгоритма Метрополиса.

\end{document}