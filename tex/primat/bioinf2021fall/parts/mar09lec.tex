\documentclass[main.tex]{subfiles}
\begin{document}

\section{Метаболомика. Продолжение}

Люди любят всё классифицировать.

Метаболиты делятся на сахара, липиды, стероиды и так далее.
Анаболизм и катаболизм.

Напомним, что важные показатели -- время удержания и <...>

\subsection{Цели и моды метаболомного анализа. Продолжение, Нетаргетированная метаболомика}

Проблема метаболомных данных: они очень сильно скоррелированы.

\subsection{Контроль ложноположительных предсказаний}

Обычная вещь -- поправка Бонферрони, но она не очень хороша, когда переменные коррелированы.
Поэтому предложен специальный метод.

% TODO a bit

Если бы данные были независимы, число образцов для достижения нужного уровня значимости можно было бы взять меньше.
Это можно интерпретировать так: доля образцов, которые мы должны добавить, отвечает доли зависимых процессов.

Используется регрессия.
Штрафные функции: гребневая регрессия, LASSO,

PCA --  один из методов сокращения 

PLS -- Partial Least Squares: находим оси, наилучшим образом объясняющие вариацию отклика

PLSR (partial least squares regression): среди предикторов находится такое линейное подпространство, которое наилучшим образом объясняет отклик.
Эти оси называют \emph{латентными переменными}.

PLS-DA: один из методов классификации данных.
Также в метаболомике применяется SVM.

\subsection{ Аэробное и анаэробное окисление глюкозы }

При аэробном разложении глюкозы получается 32 молекулы АТФ, а при анаэробном только 2 (и метаболомная сеть совсем другая).
Поэтому пути нужно учитывать.

Метаболиты связаны, и даже не только внутри одного пути.
Есть cross-talk между путями (одно и то же вещество может участвовать в нескольких путях).
Пример -- метаболизм глюкозы и <...>

\subsection{Гауссовская сеть}

Встаёт задача восстановления метаболических путей.
Существуют статистические методы.
Различают две ситуации: базовая сеть неизвестна; базовая сеть известна (второе мы изучали у Виталия Валерьевича).

Задача в первом случае ставится так: имеются многомерные нормально распределённые данные $ X = (X_1, ) $ % TODO a bit

Матрица точности -- обратная к матрице корреляции

\subsubsection{Частные корреляции}

% TODO a bit

Условная корреляция: переменные X и Y могут быть независимы при условии Z

Если корреляция Пирсона равняется нулю, то это показатель независимости.
Но высокая корреляция не является показателем зависимости.

Высокая условная корреляция не является показателем зависимости, но низкая условная корреляция является показателем (условной) независимости.

% TODO a bit

Метаболомные данные могут быть не только непрерывные, но и дискретные.
Тогда нельзя применять гауссовскую модель, и используются смешанные графические модели (MGM).
Дискретное марковское случайное поле (они же модели изинга)

Считаем корреляции отдельно в графах между непрерывной переменной и непрерывной переменной, дискретной переменной и дискретной переменной, непрерывной и дискретной переменной.

Пример: были выбраны дети с разным ИМТ, с разной степенью обучения; измерялись фенотипические признаки и метаболизм.
Составлен граф метаболизма для двух групп, найдена разница между двумя графами.

\subsection{Базы данных метаболических путей}

KEGG (Kyoto Encyclopedia of Genes and Genomes)

Можно восстановить метаболические пути по геному, если использовать аннотации.

\subsection{Overrepresentation analysis}

Гипергеометрическое распределение.
Односторонний тест Фишера.

Сверхпредставленность считается всегда.

\subsection{Анализ обогащённости набора генов (gene set enrichment analysis)}

Хотим понять, обогащён ли набор метаболитов веществами из какого-то пути.

Есть количественные данные (о концентрациях метаболитов или об уровнях экспрессии генов);
берём и ранжируем гены по представленности.
Если гены из какого-то пути находятся в начале списка или в конце списка (то есть слишком хорошо или слишком плохо представлены), говорим об обогащённости списка генами этого пути.

Количественная оценка обогащённости -- максимум бегущей суммы

\[ ES_i = \begin{cases}
	0, \quad i=0 \\
\end{cases} \]

Эта статистика называется GSEA (Gene Set Enrichment Analysis),  используется для анализа обогащённости.

Для метаболитов используется похожая статистика  MSEA (Metabolite Set Enrichment Analysis):
статистика $ Q $ описывает ковариацию между 

... Сверхпредставленность в контексте регрессионного анализа.
Хотим понять, есть ли какой-то



\end{document}