\documentclass[main.tex]{subfiles}
\begin{document}

\section{Обсуждение лекции 1}

На самом деле есть масса програм для SNP calling, и Gawk -- даже не лучшая.

Даже PacBio мы всегда верифицируем

ДНК-полимераза: берём какую-то фаговую!
(некоторые вирусы носят с собой аппарат репликации; есть ДНКовые вирусы, РНКовые вирусы...)
Конечно, полимераза кодируется каки есть отличия в последоо

Oxford Nanopore: они сейчас пытаются сделать синтетические поры, это Material Science.

Софт для PacBio: CANU, ...

\subsection{CYP79}
Этот ген принадлежит к семейству генов, отличающих за устойчивость к стрессу, защите от внешних воздействий, паразитов.
Глюкозиналат: в активном сайте в предковых популяциях метианин, в наследниках изолицин либо валин.

Вся фишка в том, что они смогли проанализировать район, в котором много повторов.
\end{document}