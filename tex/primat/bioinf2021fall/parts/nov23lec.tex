\documentclass[main.tex]{subfiles}
\begin{document}

\section{ Разбор лекции проф. Ли 24 ноября 2021 года }

Если в клетке положительно заряженная аминокислота, а мы заменим её на отрицательные,

Всё в природе на самом деле не является статичным, структура немного меняется из-за слабых колебаний атомов.
Прежде всего это Ван-Дер-Ваальсовы силы (не ковалентные связи).

Если рассматриваем два организма в филогенетическом древе и гены похожие, наверное, это ортологи.
Но может произойти перестройка генома -- дупликация, вместо одной копии гена становится две.
Эти копии гена начинают независимо эволюционировать; если они расходятся в своей структуре, то называются \emph{паралогами}.
Профессор говорит о том, что в результате мутаций один ген может оказаться вообще нефункциональным (\emph{псевдоген}).
А может у белка появиться новая функция (\emph{неофункциональность}).
Иногда для этого достаточно одной мутации.
\begin{leftbar}
Чем ближе к центру молекулы находятся аминокислоты, тем они важнее; на поверхности -- не такие важные.
Одна замена может привести к замене аминокислоты, и в плохом случае белок будет синтезироваться, но не сворачиваться правильно.
Изменение консервативных позиций особенно важно.
Такие белки маркируются убиквитином и разбираются.

Может быть ошибка в сайте посадки РНК-полимеразы, тогда ген вообще не будет транкрибироваться.
Может где-то возникнуть стоп-кодон.
\end{leftbar}

\subsection{Within a gene family...}

Главный фактор, на основе которого выделяют семейства генов -- функция.
Пример -- транскриптазы.
Новый белок может быть отнесён к какому-то семейству с помощью профилей (HMM, скрытые марковские модели).
База данных PFAM.

\subsection{Whole-genome duplication ()olyploidy)}

Твёрдая пшеница, из которой делают макароны -- тетраплоид.
Из которой делают хлеб -- гексаплоид.

Человек -- тоже древний полиплоид, но полиплоидизация была настолько давно, что уже незаметно.

Полиплоидизация -- большой шок для организма.
Если просто свой геном удвоился, это аутополиплоидизация.
Бывает ещё удвоение в результате скрещивания двух похожих видов (так, например, получились некоторые сорта Arabidopsis).
Может также в мейозе разойтись три хромосомы в одну клетку и одна в другую; после слияния получится непонятно что.

Не очень понятно до сих пор, почему полиплоидия иногда подхватывается эволюцей.
И непонятно, как такие хромосомы расходятся в мейозе.

Пшеница -- не аутополиплоид: у неё один геном -- близкого сорняка, два другие -- разных сортов пшеницы.

После полиплоидизации часто такое большое количество генетического материала оказывается ненужным, и появляются "молчащие" гены, некоторые гены из генома вырезаются.
Это интересная (и довольно хорошо изученная) тема.

Время расхождения оценивается в количестве мутационных событий.
Рассматриваются сайты, в которых произошли синонимичные замены (те, которые не изменяют функцию аминокислоты).
Напомним, мы рассматривали простую модель Джукса-Кантора, а также модель Кимуры (которая считает, что транзиции и трансверсии происходят с разной скоростью).
Есть и другие модели, которые учитывают, например, что нуклеотид в третьей позиции кодона мутирует быстрее других.

\subsection{The globin gene family}

Интересно, что у эмбриона работают совсем другие цепи.

... смотрели по слайдам...

% TODO a bit

Белки функционируют путём перестановки доменов.
Все домены есть в базе PFAM.
Домены типа Zink Finger отвечают за связывание с ДНК (их часто имеют транскрипционные факторы).
Белки с разными доменами типа KRAB A / B / ... -- совсем разные и имеют разную функцию.

Изменение доменов в белке может произойти в процессе рекомбинации в мейозе (иногда могут конъюгировать негомологичные хромосомы), а чаще могут перескакивать транспозоны с места на место.
\begin{leftbar}
Транспозоны существуют разные.
Есть вариант, когда с помощью транспозаз считывается копия участка и перемещается в другое место.
А есть ретротранспозоны...
\end{leftbar}

\subsection{Genes: RNA genes}

Есть гены, которые кодируют РНК: транспортную РНК, рибосомальную РНК, микроРНК...
Длинные некодирующие РНК: часто их функция -- в регуляции генов; они часто образуют дуплексы с регуляторными районами, что приводит к сайленсингу генов.
Ген цветения растений как раз регулируется таким образом; чтобы растение зацвело, оно должно испытать холод (во многих видах), иначе не получится.

Наш геном напичкан транспозонами.
Если бы они всё время перескакивали, часто возникали бы мутации, геном бы пошёл вразнос.
В клетках есть механизмы, которые борятся с активностью транспозонов с помощью интерференции.
piRNA борятся с активностью транспозонов; siRNA участвуют в сплайсинге.
Микро-РНК присоединяются к poly-A-концу % TODO a bit 
\\

Обычно промотор выключен.
Есть белки, которые отвечают за синтез РНК в момент холода.

\subsection{Epigenetic regulation}

Два компонента эпигенетической регуляции: метилирование и модификация гистонов (как правило -- ацетилирование).

Но генетические метки у млекопитающих не наследуются: в эмбрионе происходит репрограмминг.
Эпигеном формируется в первые годы развития, поэтому важно нормальное питание и прочие условия.
Какие-то родительские метки восстанавливаются; какие-то получаются заново.
А у растений эпигенетические метки сохраняются!

Можно подейстовать на семена неким агентом, убирающим эпигенетические метки, можно вырастить растения с другим временем цветения, и это время цветения будет передаваться потомству.

\subsection{Histone modification}

Гистон H1 -- линкерный гистон.
Саму нуклеосомную группу формируют четыре гистона  H2A, H2B, H3, H4.
Есть разные эпигенетические метки в разных позициях гистонов.

Метилирование гистонов чаще приводит к выключению, ацетилирование -- к активации.

Напомним, гетерохроматин -- внутри него гены неактивны, эухроматин -- активны.
Деление условное, поскольку хроматин -- динамическая структура и часто одни и те же участки в разные периоды жизни клетки могут быть активны либо неактивны.
Вирус AIDS часто встраивается в районе гетерохроматина.
Есть даже теория о том, что коронавирус может инегрироваться в геном (но не все с этим согласны).

% TODO a bit

\subsection{Types of mutations}

В основном изучаются точковые мутации, потому что их легко находить.
Сейчас благодаря методам секвенирования третьего поколения мы вплотную подошли к исследованиям структурных вариаций генома.

В результате инверсии могут быть произведены хромосомы с двумя центромерами (тогда хромосома разорвётся при делении) или без центромеры (тогда не будет расхождения в митозе -- к центромере прикрепляются тяжи веретена деления)

\subsection{Non-allelic recombiation}

Циклическая хромосома 

\end{document}