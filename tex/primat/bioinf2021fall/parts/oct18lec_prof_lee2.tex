\documentclass[main.tex]{subfiles}
\begin{document}
\section{Omics}

How do we "annotate" the genome?

\begin{enumerate}[noitemsep]
	\item Computational prediction: We know that a "coding sequence" of a protein-coding gene starts with ATG and ends with TAA, TAG, or TGA.
	\item % TODO a bit
\end{enumerate}

\subsubsection{Codon and translation}

Different types of proteins are just different combinations of amino acids.
Three base pair per unit.
TRNA grabs a codon...
Every protein starts from Met (ATG)

These are protein coding genes.
But the structure is much more complicated.

Not the whole primary RNA (pre-mRNA) transcript is used to code a protein.
Moreover, our genome is not packed with a lot of genes.
Only 3\% of our genome are coding proteins, other do not; they might carry some function (for example, control when a gene is turned on or off).

\subsubsection{How to sequence mRNA?}

We build a library of cDNA (complementary DNA).
Reverse transcription is used, for example, in RNA viruses that reverse-transcribe their genes into DNA.

We need primers to sequence the cDNA.
A rule: when a polymerase makes mRNA, it adds many A in the end (poly-A end).

Then they need a lot of cDNA, a lot of 

If a gene is expressed 
This is a problem of cDNA library.

Next generation sequencing changed everything.
% TODO a bit

\subsubsection{Then what?}

Traditional cDNA library + Sanger sequencing:

Next-gen / third-gen -- two strategies:

Thick parts on the picture are exons, thin are introns.
Grey: depth.
Blue: previously predicted genes.

\subsubsection{Alternative splicing}

When we make RNA out of this DNA, in some cases (e.g. in different cells in different tissues) we choose different exons (so we have various proteins).

This is why we use third-generation sequencing more and more:

\subsection{Amino acids}

Different amino acids differ in R-groups.
They can be divided in some groups, e.g. some acids are hydrophobic or not.
Based on their properties, we may guess which amino acid is likely to occur in which part of the protein.

Based on the DNA code, we may predict the protein structure.

A few months ago, there was a software called AlphaFold released.
It is very difficult.

\subsection{Gene duplication and gene family}

A chunk of DNA may become duplicated.

Suppose there is an ancestral gene which has four functions.
After the 

Subfunctionalization, neofunctionalization, loss of function / gene loss, degeneration.

\emph{Gene family}: ...

Orthologous genes (orthologs) -- originated from a single gene and are in different species.
Paralogous genes (paralogs) -- 

Whole genome duplication:
Bread wheat -- 6x

\subsubsection{How do we infer and date whole-genome duplication?}

If we compare orthologs and paralogs, we can find that their times of divergence is different.

Basically every paper on plant genomics have this kind of graph (a phylogenetic tree).

If two population have no common genome duplication peaks in some period, we may conclude that  by this time they are 

\subsubsection{ The globin gene family }

Myoglobin: obsorbs oxygen in muscules.
In different human age we have different genes respobsible for hemoglobin.
Why do we do this?
A fetus (=baby still in the womb) cannot directly exchange oxygen with the mother.
Its globin has more affinity to the oxygen.

In amphibian there is only one copy of the globin gene.
Marsupials (kangaroo, ...): only three copies of the hemoglobin gene; % why cannot they get away with three copies?
one of the reasons that allowed mammals is a fourth copy of the hemoglobin gene.

\subsubsection{Another good example: primate opsins}

In new world monkeys, there are three different alleles which result in different types of color blindness (in X chromosome).
Females in these species are trichromats if they are heterozygous; males are always dichromats!

\subsubsection{Duplication can result in a gene with multiple domains}

A special case is when the duplicated gene results not in many copies

\subsubsection{Protein swapping}

Proteins can evolve by swapping domains.
For example, zink fingers: they may duplicate many types.

This may happen due to gene exon shuffling during recombination.

\subsection{title}

10 last years: we learn more about other genes.

rRNA = ribosomal NRNA; tRNA = transfer RNA; miRNA = microRNA; ...

microRNA is a small piece of RNA (about 50 base pairs).
The first and the second part are reverse-compelmentary.

microRNA

long non-coding RNA and flowering time: a plant cannot flower when an FLC gene is expressed.
Cold temperature in winter suppresses FLC expression, and after winter

COLDAIR, COOLAIR RNA...

These things are still studied...

\subsubsection{Epigenetic change}

DNA methylation: the rule of thumb is that it happens only in Cytosine (C) base.

In a normal C there is a hydrogen; in methylated C there is an $H_3 C$ group.
It is \textbf{the most important information} in our genome despite the AGTC code itself: metylated parts are not expressed.

Next generation bisulfite sequencing: if we treat our DNA with sodium bisulfite, it changes un-methylated C into U(T).

\subsubsection{Histone modification}

We see a chromosome only when a cell wants to divide so it is easy to carry.
Usually the chromatin is open, not packed.

We have four different subunits in each histone.
Modification may happen on amino acids of histones.

The "chromatin landscape": a cell wants to decide which region is 

ATAC-seq: enzymes that can cut DNA here and there.
We can actually deduct more and more out of our DNA structure

HIV virus is an RNA virus.
Transponsable elements: are basically parasites.
We should not allow them to jump much.

Our genome is very-very dynamic!

There are ancient remainders of genes of viruses in our chromosomes!

We should suppress the expression, otherwise it will be bad.

Natural selection does not always give us the best thing.
It only gives us what lets us survive.

\end{document}
