\documentclass[main.tex]{subfiles}
\begin{document}
\section{Population genetics II: Selection and the consequences}

The most important thing about Hardy-Weinberg Equilibrium is that it tells us the condition under which evolution does not happen.

Evolution is the violation of HWE.

HWE assumes no drift (a large population), random mating, no mutations, no gene flow and no linkage disequilibrium.

What is \emph{fitness}?
This measure describes how well the individual survive, succeeds in finding a mate, and the number of his progeny.
If an individual does not survive or does not find a mate, his fitness is consider zero.

Absolute fitness: $ W_{A_1 A_1} = 20 $ seeds, $ W_{A_2 A_2} = 16 $ seeds.
Relative fitness: absolute divided by the largest in the population.

Consider one locus.

% TODO some 10 minutes

\subsubsection{Mutation-selection balance}

Suppose $ A_1 $ is the dominant allele, $ A_2 $ -- recessive.

$ A_1 A_1 $, $ A_1 A_2 $ -- fitness 1.

% TODO a bit

In the first location, the elevation rate is increased by

If the fruit flight comes from higher elevation, it takes higher time to recover...

Gene flow is too frequent...

Migration might wash away the mutation!

In the real world, we may see see local adaptations even in the presence of migration.
For example, mice in Arizona have different frequencies of genes responsible for skin color.

\subsubsection{Trade-off: selective advantage depends on environments}

A plant may grow quicker OR yield larger crop.
\emph{Boechera stricta} has more fruits in the West US than in the East.

We can use some phenotype-to-environment association.

\subsection{Detecting natural selection}

How to observe traces of previous selection (and on-going)?
Two methods: focus on divergence between population or polymorphisms inside a single population.

We cannot use the frequency of a single SNP to decide whether it is under selection because of the drift.
But loci under selection have extreme allele frequency divergence, so comparing to a neutral expectation we can detect these loci.
Recall mice: gene defining skin color differs from some random gene.

Look: this is not GWAS!
Here we see P-values of difference in allele frequencies of genes between Han Chinese and Tibetans.
Tibetans have an altitude adaptation in their genome.

Where the adaptation may come from?
A new mutation, or inherited from ancient people.
They have good genes from denisovans.

Adaptive introgression: during the Ice Age, there were 4 populations of people located in different parts of Southern Europe.

Japonica and Indica rice:
there is a gene which is very different among them and which has nothing to do with local adaptation. % TODO ask: What this gene is responsible to?

$ F_{ST} - Q_{ST} $ comparison

\subsubsection{Patterns of DNA polymorphism. Tajima's D}

$ \pi $ is a pairwise distance

Tajima's D statistics: tries to compare whether the level of polymorphism meets the neutral expectation.

Recall that branch lengths are bigger when the sequences differ more.
Singleton (an SNP found in a single individual), doubleton, tripleton, ...

Under the neutrality, we may calculate the number of singletons.

Site frequency spectrum: every bar is proportional to the branch length

The sum of all bars is the total number of SNP sites.

The population genetic parameters $ \Theta = 4 N_e u $, $ N_e $ is the effective population size.
It may be proven that under neutrality $ \pi = \hat k = 4 N_e u = \Theta $.
Also, under neutrality, $ \Theta = \frac{S}{a_n} $.
So we may calculate two different estimates of $ \Theta $.

\[ D \overset{def}= \frac{d}{\sqrt{\hat V (d)}} = \frac{\hat k - \frac{S}{a_1}}{\sqrt{[ e_1 S + e_2 S (S - 1) ]}} \]
If every is neutral and population size is constant, Tajima's D = 0.

$ S $ is very sensitive to SNPs with very low frequency alleles (particularly singletons).
If there are many terminal brances (recent rapid population expansion), there will be a lot of singletons, $ S $ will increase and Tajima's D will be negative.
If the population size decreases and we loose a lot of singletons, we have positive D values.

\subsection{Selection and genetic variation. Positive selection and selective sweep}

Very strong positive selection would decrease the variation in a particular place of a chromosome (\emph{selective sweep}).
This is the reason why we may, having observed a dip ...
% TODO a bit

Maize vs teosinte:
The corn in a totally artificial monster which did not exist in the wild before native Americans domesticated it.

The number of polymorphisms have been calculated in the sliding window along the genome.

Pi ratio:

If you read a lot of modern papers, you will see that they calculate Tajima's D statistics

% TODO a bit

A polygenic trait: example is the human height.
It is very difficult to find traces of evolution for the polygeinc trait because this evolution may be achieved with only a slight change in many sites at the same time.

\subsubsection{Other types of selection?}

Directional selection reduces genetic variation.
Balancing selection maintains variation.

Negative frequency-dependent selection:
the frequency of left-

\end{document}