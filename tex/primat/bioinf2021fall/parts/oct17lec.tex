\documentclass[main.tex]{subfiles}
\begin{document}
\section{}

% TODO a bit

Тромбоциты -- то, благодаря чему кровь не вытекает из нас (отвечают за свёртываемость); гепатоциты -- клетки печени.

Тип клетки определяется тем, какие гены в ней работают.
Биологи уже давным-давно это поняли.

Более новое знание: если мы возьмём популяцию клеток печени, она тоже гетерогенна (клетки различаются по уровню экспрессии генов).
Если усреднить результаты по популяции клеток, получим одно; если возьмём одну клетку, иное.

Пример: если измерить изменение экспрессии гена во времени в популяции клеток, получается, что она затухает во времени.
Но на самом деле в части клеток экспрессия, наоборот, возрастает.

Почему существует гетерогенность?
Экспрессия стохастична (это результат действия небольшой концентрации молекул), а реакция популяции есть усреднение по большому числу случайных реакций.
Что интересно, это может влиять на development decision (какой тип клетки получится из данной плюрипотентной).

% TODO a bit
Delta есть лиганд для Notch (Notch -- рецептор, который находится на поверхности клетки).
% TODO a bit

Небольшие различия в экспресии становятся всё более сильными.
В результате получаются клетки, экспрессирующие Notch, и другой тип клеток, которые экспрессируют Delta.

Такая гипотеза, на самом деле, была сформирована не на основе Single-Cell-секвенирования, но её можно этим методом подтверждать.

\subsection{Метод Single Cell}

Можно взять суспензию клеток и очень сильно развести (добиться такого разведения, чтобы при отборе пипеткой получалась одна молекула).
Раствор раскапывается по планшетам.
Это не очень хороший метод, поскольку оказывае

Второй метод -- взять молекулярную пипетку, втянуть ей клетку и перенести, но при этом клетка может быть механически повреждена.

Третий способ -- вырезать лазером клетку из ткани.

% TODO a bit

Можно метить частицы флюорофором, а можно магнитными частицами ()

Реально используются \emph{микрофлюидные устройства} (много каналов, ширина которых имеет порядок ширины клеток).
Каналы снабжены ловушками (например, гидродинамические ловушки): если давление жидкости в разных частях канала разное, клетка может задерживаться.

Из микрофлюидных систем наиболее распространены те, которые используют  \emph{microdroplets}:
к <...> присоединены бусины (beads), а бусины содержат олиги.
Каждый олиг состоит из нескольких частей: одна -- чтобы начать ПЦР; вторая -- <...>, ещё одна -- уникальный идентификатор конкретной клетки (UMI -- universal molecular index).
Наконец, есть олиг TTTT..., чтобы подцепить poly-A-содержащую ДНК. % TODO a bit

Получается капелька масла, внутри которой бусина с торчащими праймерами и клетка.

Наиболее распространённые системы -- InDrop и DropSeq.

Что происходит дальше?

% TODO a bit

Чтобы распознать 100 разных типов матричной РНК, нужно UMI длиной 4 нуклеотида.

Когда РНК присоединилась к иглам из олигов, надо синтезировать кДНК и амплифицировать их.
Так производится библиотека.

Первый этап после секвенирования -- проверка качества (quality control); считаем качество нуклеотидов (обычно с помощью FastQC).
Нужно убрать адаптеры \textit{in silico} (это повышает качество).
После этого картируем прочтения на геном и смотрим, что куда картируется (примитивный алгоритм картирования работает так: в суффиксных массивах ищем префиксы, потом их расширяем).

% TODO some

Read count for a gene: RPKM, FKPM, TPM

TPM -- Transcripts per megabase -- получается так: RPK делим на сумму всех RPK по всем транскриптам.
RPKM: сначала нормализуем по всем прочтениям, затем -- последнее деление -- нормализуем на длину гена.
TPM: сначала нормализуем на длину гена, т. о. % TODO

Это для нормализации данных внутри одного образца.
Для нормализации данных между образцами, напомним, используются TMM (Trimmed mean of M-values) и Relative Log Expression (последний метод реализован в Dseq2).

\subsubsection{Trimmed mean of M-values}

Мы ранее это разбирали.

Хотим сравнить различия между генами, используя неэкстремальные значения экспрессий генов.
Строим график в осях <<>> % TODO

\subsubsection{Relative log-expression}

Моделируем референс как геометрическое среднее между образцами и вычисляем скалирующий фактор.

\subsubsection{Картировщики. Kallisto}

Картировщики в RNA-Seq работают довольно долго (часы), но сейчас предложена программа, работающая быстро (программа Kallisto).
Суть алгоритма: тоже используем индекс, но этот индекс представляет собой цветной граф Де Брюйна (в узлах расположены k-меры, соединённые контигами или транскриптами, проходящими через k-меры).
Каждый контиг/k-мер имеет свой цвет (цвета -- классы эквивалентности контигов /к-меров).
Хэш-таблица устанавливает соответствие к-меров контигам. % TODO clarify a bit

За счёт чего достигается быстрота?
Не обязательно смотреть на соответствие <...> % TODO
Можно перескакивать через какие-то узлы.

\subsubsection{Confounding факторы}

Есть \emph{batch effect}: если берём в одной выборке данные из разных условий, получим неконтролируемое различие в экспрессии.
Также может быть \emph{drop-up effect}: некоторые гены могут экспрессироваться, но наступ
Есть и другие confounding факторы (например, в силу ошибок при обратной транскрипции, при амплификации).

В публикации: смотрели экспрессию генов в разных клетках мыши.
Оказалось, что большой вклад в гетерогенность вносит тот факт, что клетки находятся на разных стадиях клеточного цикла. % TODO a bit
Авторы попробовали решить проблему чисто математическими методами % TODO a bit

\end{document}