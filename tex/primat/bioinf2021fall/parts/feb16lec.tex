\documentclass[main.tex]{subfiles}
\begin{document}

\section{Моделирование структуры белка}

Напомним, в моделировании по гомологии нужно сперва сделать выравнивание белка на белок-матрицу, с которым наш схож.

При восстановлении структуры учитываются пространственные ограничения: каково локальное окружение, какой аминокислотный остаток стоит в белке-матрице.
Пространственные ограничения задаются в виде функции плотности вероятности.
Задача -- восстановить наименее противоречивую структуру -- решается с помощью оптимизации.

Мы знаем, что третичная структура более консервативна, чем первичная, и её можно хорошо восстановить даже при малой гомологии.

I-TASSER (2010): несколько матриц, алгоритм Метрополиса.
Предсказания кластеризуются, находятся центроиды кластеров.

\subsection{AlphaFold 2}

2 нейронные сети.
Первая (EvoTransformer) находит features.
Features можно находить и вручную, это коэволюционирующие пары аминокислотных остатков (замены коррелированы).
Коррелированные замены означают, что аминокислоты расположены пространствено близко.

На вход EvoTransformer приходит выравнивание.
Также можно на вход подавать уже известную матрицу-третичную структуру.
Найденные скоррелированные пары подаются снова на вход первой сети, и так 48 раз.

Вторая нейросеть предсказывает структуру и указывает степень уверенности в найденных участках последовательности.

AlphaFold Protein Structure Database -- все белки человека.

\subsection{Молекулярный докинг}

Лиганд -- часто лекарство -- например, ингибитор.

Задача:

\begin{enumerate}[noitemsep]
	\item ...
	\item Какова ориентация в сайте связывания?
	\item Какова энергия связывания?
\end{enumerate}

Докинг основан на молекулярной динамике.
Поэтому учитываем всё то же, что и при динамике.

Оценочная функция паетка Autodock -- сумма оцениваемых энергий.
Первая описывает Ван-Дер-Ваальсовы взаимодействия, вторая -- потенциал Леннарда-Джонса.
Четвёртая -- растворителя.

Белок содержит много атомов.
Тестируются миллиарды лигандов зараз.

Обычно сайт белка считается неподвижным относительно неменяющегося белка.
Лиганд обычно гибкий.

То, как проводится докинг -- \emph{виртуальный скрининг}.

Молекул очень много.
В белке делаем решётку, и в каждой клетке решётки делаем расчёт энергии.
По полученным энергиям суммируем.

Хорошо, когда есть предварительно известные положительные и отрицательные варианты ()
Это оценивается с помощью AUC/ROC-кривых.
Потом 

ZINC20 -- основная библиотека.
Для многих молекул оттуда рассчитаны все параметры, необходимые для докинга.

Пример: протеаза SARS-CoV-2.
Часто разработчики лекарств пробуют старые препараты.
В частности, одно из лекарств, которым пытаются лечить COViD, ранее использовалось для лечения СПИДа.

\section{Метаболомика}

Немного метаболомику обсуждали, когда разбирали  Flux Balance Analysis.

Метаболиты -- небольшие вещества.
Задача метаболомики -- узнать концентрацию метаболитов во времени...

\emph{Анаболизм} -- все реакции синтеза веществ.
\emph{Катаболизм} -- реакции расщепления.

Основные метаболиты: сахара (глюкозы, фруктозы), липиды, стероиды (гормоны, связанные с липидным обменом), изопреноиды, порифины.

Метаболиты могут иметь разную конформацию.

\emph{Таргетная метаболомика}: мы заранее сказали: <<нас интересуют сахара>>, или <<нас интересуют липиды>>.
\emph{Нетаргетная метаболомика}: 

Какие технологии используются?
\begin{enumerate}[noitemsep]
	\item NMR (ЯМР) -- полярные метаболиты, невысокое разрешение
	\item GC-MS -- газовая масс-спектрометрия
	\item LC-MS -- жидкостная масс-спектрометрия
\end{enumerate}

Прибор для высокопроизводительной (то есть с высокой степенью разделения) жидкостной спектрометрии (HPLC) состоит из двух частей: колонка и масс-спектрометр.

При прохождении колонки всё время происходят процессы сорбции и десорбции.
Время удержания на колонке разное у разных молекул (они выходят в разное время).

Прибор масс-спектрометрии состоит из трёх частей: ионизатор, магнит и 

% TODO image

Для каждого иона мы можем узнать из спектрометрии отношение массы к заряду.

\subsection{GC-MS vs LC-MS}

Газовая масс-спектрометрия -- дешёвый метод.
Но не всегда получается перевести метаболит в газ (он может быть нелетучий).

Жидкостная масс-спектрометрия используется, как правило, в нетаргетированной моде.

\subsection{Анализ сырых данных}

Два типа данных. % TODO a bit

Молекулы фрагментируются, и одной молекуле может соответствовать несколько пиков.

Метаболические данные сильно скоррелированы, и их очень много.
Поэтому это очень сложно анализировать.

\subsection{Этапы обработки и анализа}

\begin{itemize}[noitemsep]
	\item Препроцессинг (чистим данные, делаем так, что их можно сравнивать между разными экспериментами)
	\item Идентификация пиков, вычисление результирующих статистик
	\item ... % TODO
\end{itemize}

\subsubsection{Препроцессинг}

Часто положение пиков немного отличается у одного и того же метаболита.
Поэтому пики сглаживаются.

Но часто, когда два пика рядом, что-то оказывается посередине, и непонятно, к какому пику относится.

\subsubsection{Выравнивание пиков}

Надо выровнять данные (убрать сдвиги).
Вообще, задача выравнивания часто возникает при анализе изображений.

Методов несколько.
Один из них -- начинаем интервал расширять или сжимать; метрика качества -- коэффициент корреляции.

Или же можно разбивать на интервалы и считать кросс-корреляцию.

\begin{leftbar}
	ppm на слайде -- характеристика из ядерного магнитного резонанса, из масс-спектрометрии просто не было найдено примеров
\end{leftbar}

Другой способ выравнивания: спектрограмму сделаем бинарной, эти последовательности как-то выравниваем (Lokhov et al, 2011).

\subsubsection{Сглаживание пиков}

... Иногда используются вейвлеты.

Деконволюция пиков помогает решить проблему слияния пиков при сглаживании.
Данные -- обычно двумерная матрица, где строки-спектры и ... % TODO a bit
Исходную матрицу раскладывают на две.
Конечно, задача плохо определена, но как-то решают.

\subsubsection{Идентификация пиков}

Мы должны понять, какова масса вещества, 

Сравнение с библиотекой спектров эталонных соединений.
Библиотека NIST (для газовой хроматографии): по соотношению массы к заряду можно примерно оценить, какие метаболиты.

С жидкостной хроматографией всё хуже.

Хроматограф будет давать разные результаты в зависимости от того, новый он или уже колонка забилась.
Базы данных обычно ориентированы на один тип инструмента, или на один тип вещества, или на один организм.

\subsubsection{Контроль качества данных}

В поток молекул добавляются некие контрольные, калибровочные молекулы (для котрых мы знаем, когда они выходят из колонки).

\subsubsection{Нормализация данных}

Допустим, анализируем метаболиты мочи.
Концентрация зависит от того, сколько человек пил до анализа.

Часто, напомним, в метаболомных данных применяется центрирование по среднему (находим среднее каждой строки)...

Нормализация на постоянную сумму: каждое значение в таблице делим на некое эталонное значение (в прочем, в качесте эталона обычно берётся какое-то медианное значение)

% TODO a bit

\subsection{Цели и моды метаболомного анализа}

... обычно используются методы обучения без учителя.

Поскольку метаболитов много и они скоррелированы, сталкиваемся с проблемой Large P, Small N.
Тут нужна кросс-валидация.

Таргетированная метаболомика: стандартные методы анализа многомерных данных.

Нетаргетированная метаболомика: можно для каждого метаболита отдельно смотреть...

Недостатки: игнорируются

Часто в статьях люди не задаются вопросом, можно ли использовать тот или иной тест.

\end{document}