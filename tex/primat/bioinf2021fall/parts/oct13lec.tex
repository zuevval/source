\documentclass[main.tex]{subfiles}
\begin{document}
	
\section{Структура ДНК}
13 октября \\

Конфигурация  ДНК в интерфазе и других фазах различная.
Укладка ДНК -- динамический процесс.
Чтобы клетки правильно разошлись, хромосомы должны быть \emph{суперспирализованы}.
В процессе жизни клетки постоянно происходит то суперспирализация, то раскручивание разных участков.

Раньше, чтобы увидеть разные хромосомы (или участки хромосом) красили разными флюоресцентными красителями (те же флюорофоры, что используются в microarrays) -- метод FISH.
Прокрашиваем либо всю хромосому, либо участки.

На слайде -- хромосомы курицы (у неё их 76).
Гомологичные хромосомы в интерфазном ядре занимают разное пространство.

Сейчас же есть семейство high-throughput методов, направленных на идентификацию контактов между двумя участками геномов: 3C, 4C, 5C и HI-C.
Все методы имеют несколько общих этапов.
ДНК в клетке находится в связи с белком, и белки между собой связаны слабыми водородными связями (возможно, этот контакт функционально важные).
Хотим детектировать этот контакт, но он слабый, поэтому добавляем формальдегид, который сшивает белки вместе.
Затем режем ДНК ферментами рестрикции; производим легирование.
Т.о. получаем \emph{фрагменты-химеры}.
В зависимости от того, как разрежем, получаем лигационную связку.

One-to-many (4C): детектируем контакт какого-либо участка генома со всеми остальными возможными.

 3C: есть два района интереса (знаем их последовательности) и хотим посмотреть, есть ли между ними контакт.
 Делаем к этим районам праймеры, проводим рестрикцию и легирование; с помощью гель-фореза смотрим, присутствуют ли фрагменты. \\
 
 HI-C: после легирования режем ДНК, чтобы кольца разрезать. % TODO ? a bit
 Получаем химерный фрагмент, в середине лигатура и на концах, возможно, праймеры.
 Всего у нас получаются три типа фрагментов: контакт, отсутствие легирования или self-ligation (фрагмент замкнулся сам на себя).
 Фрагменты могут быть разной длины; но, в любом случае, они длиннее, чем 150 БП, которые умеет делать Illumina.
 Далее полученные фрагменты надо картировать на референсный геном.
 Проблема: сегменты могут быть оба из прямой цепи (Left), оба из антипараллельной (Right) или из разных (Inwards / Outwards).
 Если self-ligation, ориентация всегда "Outwards".
 Нам интересны только истинные химерные риды.
 
 \subsection{HI-C}
 
 Для HI-C мы используем праймеры, содержащие биотин.
 Перед секвенированием используем стрептавидиновые бусины, на которые вытягивяются легированные фрагменты.
 После обогащения последовательности подрезаются (берётся только 5'-участок до лигатуры).
 % TODO a bit
 Чтобы различить  self-ligation, no ligation и химеры, в HI-C вводятся пороги.
 Параметр $l_u$ (расстояние между рестрикционными фрагментами) используется, чтобы отсеять нелегированные фрагменты.
 Другой порог -- $l_s$.
 
 Какая модель используется для описания контактов?
 Есть ДНК-полимер, состоящий из соединённых мономеров.
 Можем задать локальную концентрацию -- посмотреть, как меняется концентрация сегмента с расстоянием вдоль хромосомы.
 % TODO not very clear
 
 От мономеров можно перейти к нуклеотидам.
 Для этого вводим зависимость... % TODO a bit
 
 Конечно, модель не очень хорошая, но используется для того, чтобы понять, сколько контактов можно иметь на определённом расстоянии.
 
 Когда фрагменты картировали на геном (например, программой BOWTIE), считаем количество контактов между областями как число химерных фрагментов.
 
 Берутся определённые окна.
 Может быть мегабаза, сотни килобаз, десяти килобаз...
 Внутрихромосомных контактов примерно в 40 раз больше, чем межхромосомных.
 Другой подход -- брать окно, по длине равное длине рестрикционного фрагмента.
 Чем больше размер окна, тем меньше деталей.
 
 Получаем матрицу сходства районов.
 Из-за 
 Также, когда рестриктаза режет, не все фрагменты получаются одинаковой длины; на распределение сайтов рестрикции иногда влияет GC-состав, покрытость белками участков белка.
 Эти ошибки можно исправлять с помощью bottom-up или top-down подходов. % TODO a bit
 
 \subsubsection{Iterative correction and eigenvector decomposition (ICE)}
 
 Идея: сдвиг (bias) может быть представлен как произведение индивидуальных ошибок.
 
Ожижаемая частота контактов вычисляется по формуле $ E_{ij} = B_i B_j T_{ij} $.
Если бы убрали ошибки, получили бы матрицу $ T_{ij} $ (всё нормализовано по рядам и столбцам).
Ошибки находим методом максимального правдоподобия.

Результат нормализации показан на тепловых картах.
Белые участки на исходной тепловой карте соответствуют разным хромосомам.
На новой тепловой карте полосы уходят, что считается хорошо.
 
 \begin{leftbar}
HI-C имеет много приложений.
Поиск структурных вариантов (примерно так же, как это происходит с оптическими картами); ...

Правда, это сравнительно дорого.
 \end{leftbar}

\subsection{Sequential component normalization}

Вначале убирают слабо представленные фрагменты.
Производят нормализацию (на норму или произведение сумм).
Так повторяют до тех пор, пока матрица не станет симметричной.
Утверждается, что достаточно 3-4 итераций.
В результате шум уходит, контакты становятся более заметными.

\end{document}