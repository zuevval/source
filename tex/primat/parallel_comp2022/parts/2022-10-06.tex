\documentclass[main.tex]{subfiles}
\begin{document}
	\section{MPI}
	
Создание типа данных в MPI -- двухступенчатый процесс.
\begin{enumerate}[noitemsep]
	\item Конструирование типа
	\item Регистрация типа
\end{enumerate}

Типы данных MPI создаются в runtime (а не во время трансляции)

Есть несколько конструкторов производных типов.
Самый общий -- \texttt{MPI\_Type\_struct} (наиболее общий конструктор типа), можно использовать полное описание каждого элемента типа.

Векторный тип:

\begin{verbatim}
int MPI_Type_vector(int count, int blocklen, int stride, MPI_Datatype oldtype, MPI_Datatype *newtype)
\end{verbatim}

Новый тип newtype есть вектор из oldtype. \\

Структурный тип:

\begin{verbatim}
int MPI_Type_struct(int count, int blocklengths[], MPI_Aint indices[], MPI_Datatype oldtypes[], MPI_Datatype *newtype)
\end{verbatim}

Индексированный тип:

\begin{verbatim}
int MPI_Type_indexed(int count, int blocklens[], int indices[], MPI_Datatype * <... TODO>
)
\end{verbatim}

Созданные типы нужно зарегистрировать (MPI\_Type\_commit),  а после использования удалить (MPI\_Type\_free).

\subsubsection{упаковка}

Альтернативный метод группировки -- \emph{упаковка}.
Кладём что-то в буфер; нам при этом возвращается адрес следующей ячейки памяти, куда следует в буфер класть следующую порцию данных.
Тип при этом не создаётся.

Упаковка удобна, когда нужно один раз что-то послать и принять.
Создание новых типов MPI -- когда какая-то сложная работа с данными (может быть, создание новых типов).

\subsection{Группы процессов и коммуникаторы}

До сих пор у нас был только один общий, глобальный коммуникатор.
Это не всегда удобно; иногда удобнее сгруппировать процессы.

Есть также понятие \emph{контекста}.
Сообщения в рамках одного контекста не конфликтуют с сообщениями в другом.
У каждого коммуникатора свой контекст.

Есть \emph{группы процессов}.
Операции с группами могут выполняться отдельно от операций с коммуникаторами.
Группы -- объединяем, разъединяем процессы; коммуникаторы -- обмен сообщениями.

% TODO a bit

Интра- и интеркоммуникаторы.

% TODO a bit

Если в коммуникаторе несколько групп, это интеркоммуникатор.

Включение в группу:

\begin{verbatim}
int MPI_Group_incl(MPI_Group oldgroup, int n, int *ranks, MPI_Group *newgroup)
\end{verbatim}

У процессов в новой группе другие ранги.

пустая группа -- \texttt{MPI\_GROUP\_EMPTY}

Кроме включения можно исключать процессы из групп.
Реализованы также операции на группах (пересечение, объединение, разность).  \\

Создав группу, можно создать коммуникатор.

\subsubsection{Создание нескольких коммуникаторов}

\begin{verbatim}
int MPI_Comm_split(...)
\end{verbatim}

Коммуникатору кроме указателя можно присвоить также имя.

Имея интеркоммуникатор, можно из своей группы можно получить доступ к удалённой группе.

\end{document}