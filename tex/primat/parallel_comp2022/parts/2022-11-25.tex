\documentclass[main.tex]{subfiles}
\begin{document}

GWDT это...

\section{Hybrid MPI}

Программы, которые сочетают MPI и что-то ещё (OpenMP, или просто потоки, или старая библиотека TBB, CUDA...)

OpenCL -- та же CUDA, но для видеокарт AMD.

MPI позволяет распараллеленную уже программу раскидать по разным узлам.
Конечно, это можно сделать с помощью HTTP, TCP и так далее вручную, но есть готовая библиотека MPI.

Поскольку MPI родился давно, его реализации не являются потокобезопасными.
То есть его копии, запущенные в разных потоках, могут порождать ошибки типа  Race Condition и так далее.

Реализована дополнительная функция для потокобезопасности, которая 

\section{Отладка MPI}

\begin{leftbar}
	OpenMP-то не так страшно отлаживать
\end{leftbar}

\end{document}