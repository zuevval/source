\documentclass[main.tex]{subfiles}
\begin{document}

GitLab:
\begin{enumerate}[noitemsep]
	\item Написать свой user ID GitLab
	\item repo, reporter
	
	Issue: open - doing - submit - closed
\end{enumerate}

Slurm -- несколько режимов:

locate -- просто выделить ресурсы на некоторое количество времени;

задание -- \\

MPI команды: mpirun (работает поверх Slurm)

\section{Параллельные программы}

Выделяют
\begin{enumerate}[noitemsep]
	\item Архитектура с общей памятью
	\item Архитектуры с распределённой памятью (кластеры): машины соединены быстрыми шинами и вместе что-то считают
	\item Гибридные архитектуры -- всё, что угодно
\end{enumerate}

Мы занимаемся вторым, притом 

FLOPS -- Floating Point Operations per Second

\subsection{Uniform Memory Access}

UMA (Uniform Memory Access):
Symmetrical Multiprocessing -- одинаковые процессоры подключены к одной памяти.
Архитектура MIPS сейчас используется в роутерах.

Non Uniform Memory Access (NUMA) -- архитектура легче расширяема 

\subsection{Устройства с реконфигурируемой архитектурой}

FPGA -- Field Programmable Gate Array (хороши для одновременного вычисления одной операции; дорогие)

\subsection{Grid}
Grid -- объединяет много кластеров.

\subsection{Закон Амдала}

Amdalh's Las, 1967: с увеличением числа вычислительных узлов рост производительности ограничен

Есть и другой, более оптимистичный закон (Густавсона-Барсиса)

\subsection{Топология и коммуникация}

МВС -- многопроцессорная вычислительная система.
Когда мы пересылаем данные между узлами, наиболее распространены топологии

\begin{enumerate}[noitemsep]
	\item Полный граф
	\item Линейка
	\item Решётка
	\item Кольцо -- пересылаем двум соседям (а также бывает тор).
	Кольцо часто используется для методов оптимизации.
	\item Звезда (есть master-процесс) и worker'ы
\end{enumerate}

Система управления памяти

Латентность -- задержка при передаче сообщения нулевой длины

Пропускная способность

\subsection{ Модели параллельного программирования }

\begin{enumerate}
	\item threads
	\item Процессы: fork (создать копию процесса с определённого места)  / execv (заместить процесс)
	\item Передача сообщений (message passing)
	\item Параллелизм по данным
\end{enumerate}

POSIX -- стандарт работы операционной системы.
Но его сложно целиком реализовать, и нет ни одной ОС, реализующих его целиком.

\begin{verbatim}
	// unistd.h

pid_t my_id = <...>

\end{verbatim}

\end{document}