\documentclass[main.tex]{subfiles}
\begin{document}

OpenMP используется широко (даже шире, чем MPI), ибо работает в привычных компьютерах с общей памятью.

Никакой пересылки в OpenMP не требуется! Есть общая память, к которой все обращаются.

Синхронизация -- есть явная; иногда требуется неявная.

Начинается исполнение с одной нити (которая называется master).
При входе в параллельную область возникают дополнительные нити.

Вложенные параллельные области -- общий враг для всех!
Если используем OpenMP, и ещё библиотеку, которая использует OpenMP -- можем создать больше потоков, чем хотим.

\begin{verbatim}
#pragma omp parallel for shared(n, a, b, c)
for (int i=0; i<n; i++)
    c[i] = a[i] + b[i]
\end{verbatim}

Два основных класса переменных:

\begin{enumerate}[noitemsep]
	\item shared (разделяемые) -- по умолчанию (все нити видят одну и ту же переменную)
	\item private -- своя копия для каждой нити
\end{enumerate}

В примере цикла выше $ n, a, b, c $ -- разделённые переменные (это можно было и не указывать явно), $ i $ -- локальная. 

Можно в \texttt{\#pragma omp parallel} использовать \texttt{if (condition)} (будет параллелиться, если условие верно).

\subsubsection{Редукция в OpenMP}

\begin{verbatim}
reduction (operator: list of common variables)

operator in {-, +, *, -, &, |, ^, &&, ||}
\end{verbatim}

Пример:

\begin{verbatim}
int a[100];
int sum = 0, k=4;

#pragma omp reduction(+: sum)
for (int i=0; i<100; i+k)
    sum += a[i];
\end{verbatim}

\begin{verbatim}
#pragma omp reduction(+: count)
{
	count++;
	printf("Count value: %i", count);
}
\end{verbatim}

\subsubsection{Другие функции OMP}

\begin{verbatim}
double omp_get_wtime(void); // wallclock time in seconds from some moment in the past
double omp_get_wtick(void); // timer resolution
\end{verbatim}

Можем получать и задавать максимальное число потоков.
Можно запрещать или разрешать вложенные области.

\subsubsection{Директива master}

\begin{verbatim}

#pragma omp parallel private(n)
{
	#pragma omp master n=2
	...TODO...
}
\end{verbatim}

\subsubsection{ Сложные циклы }

\texttt{collapse(n)} -- этот модификатор позволяет распараллелить несколько последовательных циклов.

\texttt{ordered} -- исполнение итераций в том порядке, в котором они идут в цикле.

Ограничения распараллеливания циклов/блоков: нельзя использовать параллельный выход (break)

\texttt{scheduled} -- динамическое распрееление итераций

\texttt{guided} -- ещё умнее

Быстрее всего работает 

\end{document}