\documentclass[main.tex]{subfiles}
\begin{document}

Oct 20, 2022 \\

Директива \texttt{ordered}

Директива \texttt{critical}: 

\begin{verbatim}
#pragma omp critical [name]
\end{verbatim}

Эта область кода выполняется всегда одним потоком.
Несколько одинаковых [name] схлопываются в одну область.

\begin{verbatim}
#pragma omp atomic
\end{verbatim}

\begin{verbatim}
#pragma omp parallel
{
	#pragma omp atomic
	count++;
}
printf(counter);
\end{verbatim}

Это атомарная операция, то есть в каждый момент управление только у одной нити.

\subsection{Замки}

\begin{enumerate}[noitemsep]
	\item Простой
	
	\begin{verbatim}
		void omp_init_lock(omp_lock_t * l);
		void omp_destroy_lock(omp_lock_t * l);
		int omp_test_lock // 1 if we can take the lock, 0 if we cannot
	\end{verbatim}
	
	\item Nested -- вложенный: у него не один оборот, а несколько оборотов
	
	\begin{verbatim}
		void omp_init_nest_lock (l);
		void omp_set_nest_lock
		void omp_set_unnest_lock
		void omp_destroy_nest_lock (l);
	\end{verbatim}

	Пример использования: ?? честно говоря, не до конца понял
	
\end{enumerate}

Замок защищает некую область памяти или какой-то другой участок кода.
Замок нужно открыть (если он наш) или захватить(закрыть), если мы начинаем с ним работать.

\subsection{Ещё о синхронизации}

\begin{verbatim}
#pragma omp flush [list of variables]
\end{verbatim}
Если есть какие-то буферы, в которых что-то ожидает, все операции завершаются и буферы сбрасываются.
Без списка - это относится ко всем переменным.

Сброс буферов происходит неявно по завершении всех параллельных секций. % Аня поняла, как пользоваться flush!

\subsection{Параметры}

\begin{verbatim}
	OMP_NUM_THREADS
	OMP_STACKSIZE // size of memory for processes declared as workers
	OMP_WAIT_POLICY // 
	OMP_THREAD_LIMIT // max number of threads for this program
	OMP_NESTED // permits nested OMP sections
\end{verbatim}

Как эти параметры реально влияют на процессы -- зависит от реализации, от ОС.


\end{document}