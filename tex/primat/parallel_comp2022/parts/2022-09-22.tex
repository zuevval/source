\documentclass[main.tex]{subfiles}
\begin{document}

% TODO some

Уравнение в частных производных для обработки изображений

\[ u(x, y, 0) = f(x, y) \overset{\text{добавляем время}}\to \boxed{\tau_t } \to \tau_t (f) (x, y) = u(x, y, t) \]

Бинарное изображение $ f(x, y) $

Фон $ S = { (x, y): f(x, y) = 1 } $

Передний план $ S' $

\begin{itemize}[noitemsep]
	\item Есть изображение
	\item Есть маска источников
	\item Есть маска $ k \times k $
\end{itemize}

\subsection{Лабораторная 3}

Фильтр, который (при некотрой доработке) может быть использован для отрисовки границ (если они плохо видны на изображении):

Пусть у нас изображение с нерезкими, нечёткими границами объектов + между объектами тоже серый шум.
Возьмём отсечение по порогу - получим либо шумные границы, либо слишком маленькие по размеру объекты.

\textbf{Нелинейная диффузия}: Canny

\[ I_t = div(c(x,y,t) \nabla I) = c(x,y,t) \Delta I + \nabla c \cdot \nabla I \]

Оператор Лапласа

\[ I_{TODO} \]

\subsection{MPI}

Point-to-Point (p2p-взаимодействия): всегда взаимодействуют два и только два процесса (один отправляет, другой получает)

Передача -- всегда синхронный процесс.
Приём может быть асинхронный. \\

Типичные ошибки: блокировка из-за отправки 

\subsection{Обмен с буферизацией}

На разных машинах операция локальная.

Нужно использовать, когда действительно требуется контроль над распределением памяти.

Буферизация -- копирование в памяти, то есть дополнительные накладные расходы.

\subsubsection{Буферизация в MPI}

Можно "псевдоодновременно" принимать несколько сообщений.

Системный буфер в MPI скрыт от программиста, вся работа осуществляется средствами библиотеки.
Он может быть на одной машине или на нескольких.

Сообщения поступают в буфер.
Система знает, кому надо отправить из буфера. \\

\subsubsection{А что, если без буферизации?}

Можно использовать отправку по готовности, это более быстро, но небезопасно. \\

Ожидание:

\begin{enumerate}[noitemsep]
	\item \texttt{MPI\_Wait}
	\item \texttt{  }
\end{enumerate}

\end{document}