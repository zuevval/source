\documentclass[main.tex]{subfiles}
\begin{document}

\subsection{Виртуальная топология}

Виртуальная топология -- структура соединений процессов в коммуникаторе (каналы обмена сообщениями).

Иногда удобно представить процессы расположенными каким-то определённым образом.
Например, решая ДУ, будет удобно представить процессы расположенными в виде решётки.

Либо мы сами помним, где какой номер процесса, либо можем вначале рассказать системе MPI, и она сама будет знать.

\subsubsection{Два вида топологии}

\textbf{Декартова топология} -- решётка произвольной размерности

\textbf{Топология графа} -- всё остальное (явно задаём, какие рёбра соединяют какие узлы).

Одна из специальных операций в декартовой топологии -- \emph{расщепление коммуникатора на гиперплоскости}.
Другая -- \emph{сдвиг} (информация на определённое число узлов перемещается вдоль оси).

\subsubsection{Пример создаия декартовой топологии}

\begin{verbatim}
	int MPI_Cart_create (MPI_Comm comm_old, int ndims, int *dims, int *periods, int reorder, MPI_Comm *comm_cart)
\end{verbatim}

Есть старый коммуникатор, на который мы накладываем декартову топологию и создаём его модификацию.

Зарегистрировавшись в решётке, каждый процесс может узнать, где он находится.

\begin{verbatim}
	int MPI_Cart_coords(MPI_Comm comm, int rank, int maxdims, int *coords)
\end{verbatim}
(декартовы координаты процессс по его рангу в группе) \\

Расщепление:
\begin{verbatim}
	int MPI_Cart_sub(MPI_Comm comm, int *remain_dims, MPI_Comm *comm_new)
\end{verbatim}

Узнать информацию о процессе:

\begin{verbatim}
	int MPI_Cart_get(...TODO...)
\end{verbatim}

Переупорядочить процесс:

\begin{verbatim}
	int MPI_Cart_map(...TODO...)
\end{verbatim}

Сдвиги: циклический (в позицию $ (J + K) mod N $) и просто линейный (в позицию  $ J + K $, если $ J + K \le N $)

\begin{verbatim}
	int MPI_Cart_shift(...TODO...)
\end{verbatim}

\subsection{ Топология типа графа }

\begin{verbatim}
	int MPI_Graph_create(MPI_Comm comm, int nnodes, int *index, int *edges, int reorder, MPI_Comm *comm_graph)
\end{verbatim}

\texttt{int *index}: сколько рёбер исходит из каждой вершины.
Например, 2 3 0 0 0 -- из первой вершины 2 ребра, из второй 3, далее по нулям.

Тогда в нашем примере массив \texttt{int *edges} таков: первые два элемента -- индексы вершин, соединённых рёбрами с первой, следующие 3 -- со второй и так далее.

\texttt{int reorder}: значение "истина" разрешает перемену местами вершин.



\end{document}