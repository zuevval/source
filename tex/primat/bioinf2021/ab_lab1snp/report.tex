% compile with XeLaTeX or LuaLaTeX

\documentclass[a4paper,12pt]{article}
\usepackage[utf8]{inputenc} % russian, do not change
\usepackage[T2A, T1]{fontenc} % russian, do not change
\usepackage[english, russian]{babel} % russian, do not change

% fonts
\usepackage{fontspec} % different fonts
\setmainfont{Times New Roman}
\usepackage{setspace,amsmath}
\usepackage{amssymb} %common math symbols
\usepackage{dsfont}

% utilities
\usepackage{systeme} % systems of equations
\usepackage{mathtools} % xRightarrow, xrightleftharpoons, etc
\usepackage{array} % utils for tables
\usepackage{makecell} % multirow for tables
\usepackage{subfiles}
\usepackage{hyperref}
\hypersetup{pdfstartview=FitH,  linkcolor=blue, urlcolor=blue, colorlinks=true}
\usepackage{framed} % advanced frames, boxes
\usepackage{graphicx}
\usepackage{caption}
\usepackage{subcaption} % captions for subfigures
\usepackage{color}
\usepackage{listings}
\lstset{
    language=bash,
    basicstyle=\ttfamily,
    showstringspaces=false,
    commentstyle=\color{red},
    keywordstyle=\color{blue},
    breaklines=true,
    postbreak=\mbox{\textcolor{red}{$\hookrightarrow$}\space},
    columns=fullflexible
}

% styling
\usepackage{float} % force pictures position
\floatstyle{plaintop} % force caption on top
\usepackage{enumitem} % itemize and enumerate with [noitemsep]
\setlength{\parindent}{0pt} % no indents!
\usepackage[left=25mm, top=20mm, right=25mm, bottom=20mm, nohead, footskip=10mm]{geometry}

% misc
\graphicspath{{./img/}}
\newcommand{\myPictWidth}{.9\textwidth}
\newcommand{\phm}{\phantom{-}}

\begin{document}

\subfile{titlepage}

\section{Постановка задачи}

Необходимо провести манипуляции с короткими последовательностями, полученными в результате секвенирования генома \textit{Escherichia coli} (\href{https://www.ncbi.nlm.nih.gov/sra/?term=SRR1770413}{NCBI: SRR1770413}), а именно
\begin{enumerate}[noitemsep]
    \item Выровнять прочтения на референсный геном (\href{https://www.ncbi.nlm.nih.gov/nuccore/NC_000913.3}{NCBI: NC\_000913.3})
    \item Выполнить поиск вариабельных позиций
    \item Посмотреть, как выявленные варианты распределены по генам (аннотация SNP).
\end{enumerate}

\section{Ход работы}

\subsection{Подготовка данных}
Парные прочтения в формате FASTQ были загружены при помощи утилиты \href{https://github.com/ncbi/sra-tools/wiki/01.-Downloading-SRA-Toolkit}{SRA-Toolkit}:
\begin{lstlisting}
    ../sratoolkit.2.11.0-win64/bin/fasterq-dump --split-files SRR1770413
\end{lstlisting}
Референсный геном в формате FASTA можно скачать через веб-интерфейс сайта NCBI.

\subsection{Выравнивание}

Выравнивание осуществлено с помощью программы \href{https://sourceforge.net/projects/bowtie-bio/files/bowtie2/}{Bowtie2}.

Сперва на основе FASTA-файла строится индекс референсного генома:

\begin{lstlisting}
./bowtie2-build ../lab1_snp/nc_000913_3_ref.fasta ../lab1_snp/ref_index
\end{lstlisting}

Затем производится выравнивание, на выходе которого генерируется файл с расширением sam:

\begin{lstlisting}
../bowtie2-2.3.3.1-mingw-x86_64/bowtie2 -x ref_index -1 SRR1770413_1.fastq -2 SRR1770413_2.fastq -S alignments.sam
\end{lstlisting}

Примерно $71\%$ прочтений удалось выровнять (ниже показан вывод программы):

\begin{verbatim}
643253 reads; of these:
643253 (100.00%) were paired; of these:
561440 (87.28%) aligned concordantly 0 times
79980 (12.43%) aligned concordantly exactly 1 time
1833 (0.28%) aligned concordantly >1 times
----
561440 pairs aligned concordantly 0 times; of these:
359323 (64.00%) aligned discordantly 1 time
----
202117 pairs aligned 0 times concordantly or discordantly; of these:
404234 mates make up the pairs; of these:
374955 (92.76%) aligned 0 times
11532 (2.85%) aligned exactly 1 time
17747 (4.39%) aligned >1 times
70.85% overall alignment rate
\end{verbatim}

\subsection{Поиск полиморфизмов}

Для поиска структурных вариантов были применены программы SAMtools и BCFtools.

Вначале преобразуем sam-файл, оставив только выровненные последовательности, и отсортируем его:

\begin{lstlisting}
samtools view -b -F 4 alignments.sam > mapped.bam
samtools sort mapped.bam --reference nc_000913_3_ref.fasta > sorted.bam
\end{lstlisting}

Затем определим степень правдоподобия наличия полиморфизма в каждой позиции генома и классифицируем позиции (SNP / не SNP):

\begin{lstlisting}
bcftools mpileup -Ou -f nc_000913_3_ref.fasta sorted.bam | bcftools call -mv -Ov -o variant_calls.vcf
\end{lstlisting}

\subsection{Аннотация полиморфизмов}

После нахождения структурных вариантов осуществлён поиск генов, обогащённых вариабельными позициями (код программы на языке R приведён в приложении).
Для этого использован список генов \textit{E. coli} (\href{https://biocyc.org/group?id=biocyc13-24029-3731963750}{K-12 MG1655}).

Выявлены 12 генов, в которых встречается не менее двух SNP (табл. \ref{tab:snp}, рис. \ref{fig:hist})

% latex table generated in R 4.0.5 by xtable 1.8-4 package
% Wed May 05 07:05:00 2021
\begin{table}[H]
    \centering
    \caption{Гены, в которых найдены 2 полиморфизма и более}
    \begin{tabular}{rllr}
        \hline
        & Gene.Name & Product & count \\
        \hline
        1 & ybfQ & inactive transposase YbfQ &  11 \\
        2 & ychS & putative uncharacterized protein YchS &   9 \\
        3 & ybcV & DLP12 prophage; DUF1398 domain-containing protein YbcV &   6 \\
        4 & nohD & DLP12 prophage; putative DNA-packaging protein NohD &   6 \\
        5 & borD & DLP12 prophage; prophage lipoprotein BorD &   5 \\
        6 & yhgF & putative RNA-binding protein YhgF &   3 \\
        7 & oppA & oligopeptide ABC transporter periplasmic binding protein &   3 \\
        8 & rpoD & RNA polymerase, sigma 70 (sigma D) factor &   2 \\
        9 & fliC & flagellar filament structural protein &   2 \\
        10 & nuoL & NADH:quinone oxidoreductase subunit L &   2 \\
        11 & ylcI & DLP12 prophage; DUF3950 domain-containing protein YlcI &   2 \\
        12 & rttR & small RNA RttR &   2 \\
        \hline
    \end{tabular}
    \label{tab:snp}
\end{table}

\begin{figure}[H]
    \centering
    \includegraphics[width=\myPictWidth]{snp_hist}
    \caption{Распределение числа полиморфизма в генах (показаны гены, в которых найдено не менее 2 полиморфизмов)}
    \label{fig:hist}
\end{figure}


\begin{thebibliography}{99}
    \bibitem{bowtie} Langmead, B., Salzberg, S. Fast gapped-read alignment with Bowtie 2. Nat Methods 9, 357–359 (2012). doi: \href{https://doi.org/10.1038/nmeth.1923}{10.1038/nmeth.1923}
\end{thebibliography}

\section{Приложение. Код программы, выполняющей аннотацию полиморфизмов}

\lstinputlisting[language=R, deletekeywords={count}]{E:/Users/vzuev/github-zuevval/source/r/bioinf2021/ab_lab1_snp/vcf_annotator.R}

\end{document}