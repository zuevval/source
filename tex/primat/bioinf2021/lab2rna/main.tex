% compile with XeLaTeX or LuaLaTeX

\documentclass[a4paper,12pt]{article}
\usepackage[utf8]{inputenc} % russian, do not change
\usepackage[T2A, T1]{fontenc} % russian, do not change
\usepackage[english, russian]{babel} % russian, do not change

% fonts
\usepackage{fontspec} % different fonts
\setmainfont{Times New Roman}
\usepackage{setspace,amsmath}
\usepackage{amssymb} %common math symbols
\usepackage{dsfont}

% utilities
\usepackage{systeme} % systems of equations
\usepackage{mathtools} % xRightarrow, xrightleftharpoons, etc
\usepackage{array} % utils for tables
\usepackage{makecell} % multirow for tables
\usepackage{subfiles}
\usepackage{hyperref}
\hypersetup{pdfstartview=FitH,  linkcolor=blue, urlcolor=blue, colorlinks=true}
\usepackage{framed} % advanced frames, boxes
\usepackage{graphicx}
\usepackage{caption}
\usepackage{subcaption} % captions for subfigures
\usepackage{color}
\usepackage{listings}
\lstset{
    language=bash,
    basicstyle=\ttfamily,
    showstringspaces=false,
    commentstyle=\color{red},
    keywordstyle=\color{blue},
    breaklines=true,
    postbreak=\mbox{\textcolor{red}{$\hookrightarrow$}\space},
    columns=fullflexible
}

% styling
\usepackage{float} % force pictures position
\floatstyle{plaintop} % force caption on top
\usepackage{enumitem} % itemize and enumerate with [noitemsep]
\setlength{\parindent}{0pt} % no indents!

% misc
\graphicspath{{./img/}}
\newcommand{\myPictWidth}{.99\textwidth}
\newcommand{\phm}{\phantom{-}}

\begin{document}

\subfile{titlepage}

\section{Исходные данные}

В данной работе проводится анализ информации об экспрессии генов в клетках людей, часть из которых больна психозом, который связан с употреблением метамфетамина \cite{breen}.
Соответствующий набор данных расположен в открытом доступе. Краткая сводка со ссылками на ресурсы опубликована в базе данных ArrayExpress (\href{https://www.ebi.ac.uk/arrayexpress/experiments/E-GEOD-74737/}{E-GEOD-74737}); результаты секвенирования доступны для скачивания в формате FASTQ с сайта European Nucleotide Archive (\href{https://www.ebi.ac.uk/ena/browser/view/PRJNA301364}{PRJNA301364}), матрицу с информацией об уровне экспрессии каждого гена можно найти в базе GEO (\href{https://www.ncbi.nlm.nih.gov/geo/query/acc.cgi?acc=GSE74737}{GSE74737})

\section{Оценка качества с помощью FastQC}

\subsection{Подготовка материалов}

С ftp-сайта European Nucleotide Archive были загружены два набора данных (прочтения с 5'-конца и 3'-конца), соответствующих эксперименту \href{https://www.ebi.ac.uk/ena/browser/view/SRX1421028}{SRX1421028} (клетки крови человека, образец из контрольной выборки):

\begin{lstlisting}
wget ftp://ftp.sra.ebi.ac.uk/vol1/fastq/SRR292/004/SRR2927744/SRR2927744_1.fastq.gz
wget ftp://ftp.sra.ebi.ac.uk/vol1/fastq/SRR292/004/SRR2927744/SRR2927744_2.fastq.gz
\end{lstlisting}

\subsection{Оценивание необработанных данных}

Проведён комплексный анализ качества образцов с помощью FastQC.
Результаты приведены в приложении.

Видно, что качество прочтения падает с увеличением номера позиции нуклеотида (рис. \ref{fig:raw:per_base_q}).
Среднее качество прочтения хорошее (рис. \ref{fig:raw:across_reads_q}).
В обоих наборах прочтений распределение GC-контента бимодальное (рис. \ref{fig:raw:gc}).
Заметно, что первые 9-10 нуклеотидов в прочтениях $ 5' \to 3'$ распределены неравномерно (рис. \ref{fig:raw:sequence_content}); скорее всего, это влияние адаптерной последовательности.

\subsection{Обработка последовательностей}

Прочтения обрезаны с помощью утилиты TrimmoMatic \cite{bogler}:

\begin{lstlisting}
java -jar "C:/Program Files/Trimmomatic-0.39/trimmomatic-0.39.jar" PE -threads 50 -phred33 SRR2927744_1.fastq.gz SRR2927744_2.fastq.gz SRR2927744_1_paired.fastq.gz SRR2927744_1_unpaired.fastq.gz SRR2927744_2_paired.fastq.gz SRR2927744_2_unpaired.fastq.gz  ILLUMINACLIP:truseq3pe2mod.fa:2:30:10:2:keepBothReads LEADING:3 TRAILING:3 SLIDINGWINDOW:4:15 MINLEN:36
\end{lstlisting}

Переданные аргументы имеют следующий смысл:
\begin{enumerate}[noitemsep]
    \item \texttt{-threads 50} -- программа запускается одновременно в 50 потоках максимум (по умолчанию в одном потоке).
    \item \texttt{-phred33} -- оценка качества записана в формате Phred 33 (впрочем, Trim\-mo\-Matic умеет сам это пределять, поэтому данный параметр можно не указывать).
    \item \texttt{ILLUMINACLIP:truseq3pe2mod.fa:2:30:10:2:keepBothReads} -- в ридах ищутся и вырезаются адаптерные последовательности из файла, который получен добавлением часто встречающейся в исходных данных последовательности CTTC\-GATG\-TCGG\-CTCT\-TCCT\-ATCA\-TTGT\-GAAG\-CAGAA\-TTCAC\-CAAG\-CGTT в файл \texttt{Tru\-Seq3-\-PE-2.fa} (поставляется вместе с программой в папке \texttt{adapters}), причём допускается максимально 2 несовпадения нуклеотидов.
    Порог для различия двух палиндромов - 30, порог для различия между адаптерами - 10.
    \item \texttt{LEADING:3 TRAILING:3} -- минимально допустимое качество отдельного нуклеотида.
    \item \texttt{SLIDINGWINDOW:4:15} -- в каждой четвёрке подряд идущих оснований среднее качество должно быть не меньше 15.
    \item \texttt{MINLEN:36} -- если последовательность короче 36 оснований, она отбрасывается.
\end{enumerate}

Фильтры применяются последовательно (сперва обрезка адаптеров, затем фильтрация по минимально допустимому качеству отдельного нуклеотида и так далее).

\subsection{Оценивание обработанных данных}

TrimmoMatic в режиме работы с парными прочтениями распределяет последовательности из одного набора в два: те, для которых была найдена пара, и те, для которых пары не нашлось.
В данной работе приведена оценка качества только для тех последовательностей, которые нашли свой палиндром.
Всего в исходном наборе $ 11, 852, 482 $ фрагментов, после фильтрации остались $ 9, 257, 785 $, т. е. чуть больше $ 78\% $.

После обработки в данных остаются только последовательности с хорошим качеством прочтения в последних позициях (рис. \ref{fig:trimmed:per_base_q}), но число последовательностей с хорошим средним качеством уменьшается (рис. \ref{fig:trimmed:across_reads_q}).
Второй пик в бимодальном распределении удаётся сгладить, оно становится почти унимодальным (рис. \ref{fig:trimmed:gc}), но не удаётся извлечь адаптерные последовательности (рис. \ref{fig:trimmed:sequence_content}) -- возможно, используются неверные предположения о том, какие последовательности используются в качестве праймеров.

% TODO maybe move to a new report?
\section{Анализ дифференциальной экспрессии}

Произведены манипуляции с данными об экспрессии генов, которые представлены двумя таблицами: число прочтений каждого гена в исходных и в нормализованных данных.
Обе таблицы собраны в файл .xls, который доступен для загрузки на сайте GEO.

В настоящей работе использована первая таблица (ненормализованные данные), поскольку только такие позволяют построить корректную модель \cite{love2014}.

Метаданные о столбцах таблицы теоретически можно взять из таблицы с описанием образцов (\texttt{E-GEOD-74737.sdrf.txt}), но там не содержится существенной информации о различиях между образцами, кроме состояния каждого экземпляра: контрольная группа / зависимые от метамфетамина, без психоза / зависимые, с диагностированным психозом.
Поэтому таблица с метаданными содержит только один столбец -- состояние. \\

\subsection{Обнаружение дифференциально экспрессирующихся генов}

С помощью пакета \texttt{DESeq2} проведён поиск генов, проявляющих диффееренциальную экспрессию, на основе модели негативного биномиального распределения.

Как выяснилось, экспрессии генов метамфетамин-зависимых пациентов не отличаются от контрольной группы при уровне значимости $ 0.05 $.
В приложении приведён график MA (рис. \ref{fig:ma}).

Если взять пороговое значени $ p_{thresh} = 0.2 $, остаются 6 генов с меньшим значением скорректированного p-value:
ARL6, \hspace{0pt}
CDYL2, \hspace{0pt}
ELK3, \hspace{0pt}
FAM169A, \hspace{0pt}
MAP1LC3B2.

\subsection{Анализ обогащённости Gene Ontology и KEGG}

\subsubsection{fgsea}

По скорректированным p-value дифференциальной экспрессии генов были составлены ранги.
С сайта \href{http://www.gsea-msigdb.org/gsea/downloads.jsp}{gsea-msigdb.org} загружены записи баз данных GO и KEGG для Homo Sapiens (\texttt{c5.go.v7.4.entrez.gmt} и \texttt{c2.cp.kegg.\-v7.\-4.\-entrez.gmt}).
После этого с помощью пакета FGSEA по рангам генов и спискам записей проведён анализ обогащённости.
Выявлены 588 терминов из базы GeneOntology и 51 путь из базы данных KEGG, которые сверхпредставлены в полученном наборе генов с уровнем значимости $0.05$.

\subsubsection{DAVID}

Осуществлена функциональная аннотация найденных десяти генов с помощью утилиты DAVID (\href{https://david.ncifcrf.gov}{david.ncifcrf.gov}).
В качестве параметра "background" \hspace{0pt} (фоновое распределение) использован список из 20267 генов, которые указаны в таблице с данными об экспрессии.

Анализ не выявил сверхпредставленности в генах элементов из базы данных Gene Ontology или KEGG.
Видимо, слишком малое число генов отмечены как дифференциально экспрессирующиеся.

\subsection{Сравнение с результатами, приведёнными в статье}

В статье \cite{breen} указаны несколько дифференциально экспрессирующихся генов: CLN3, FBP1, TBC1D2, ZNF821, ADAM15, ARL6, FBN1, MTHFSD (о них написано, что они относятся к множеству генов с аннотацией <<деградация РНК>>); ELK3, SINA3 (аннотированы общим термином <<нарушение циркадных ритмов>>) и PIGF, UHMK1, C7orf11 (<<убиквитин-зависимый протеолиз>>).

В данной работе выявлены пять генов, общие с описанными в статье: ARL6, CDYL2, ELK3, FAM169A и MTHFSD.
Один найденный ген (MAP1LC3B2) в статье не указан.
Не найдены ещё 19 генов, приведённые в статье в списке 25 генов с наиболее выраженной дифференциальной экспрессией.
Расхождение можно объяснить различиями в используемых программных инструментах (\texttt{limma} в статье, \texttt{DESeq2} в данной работе).

Когда авторы статьи провели поиск аннотаций и метаболических путей, которыми обогащены дифференциально экспрессирующиеся гены, они выявили две группы д. э. генов: ответственные за нарушение циркадных ритмов и связанные с убиквитин-зависимым протеолизом.
В данной работе по результатам анализа с помощью fgsea выявлены другие метаболические пути, но многие из них логично связать с зависимостью от метамфетамина и психозом: KEGG\_\-NEUROACTIVE\_\-LIGAND\_\-RECEPTOR\_\-INTERACTION (с нарушениями этого механизма связывают психоз \cite{adkins2011}), KEGG\_\-OLFACTORY\_\-\-TRANSDUCTION, KEGG\_\-DRUG\_\-METABOLISM\_\-CYTOCHROME\_\-P450.
Выявленные термины GO также относятся в основном к работе нервной системы и приёму препаратов: GOBP\_\-DETECTION\_\-OF\_\-CHEMICAL\_\-STIMULUS, GOBP\_\-DETECTION\_\-OF\_\-STIMULUS\_\-INVOLVED\_\-IN\_\-SENSORY\_\-PERCEPTION, GOBP\_\-NERVOUS\_\-SYSTEM\_\-PROCESS и так далее.

К сожалению, DAVID не смог найти пути, обогащённые шестью найденными в данной работе дифференциально экспрессирующимися генами. \\

Код программы на языке R, которая ищет дифференциально экспрессирующиеся гены и выполняет анализ обогащённости, \href{https://github.com/zuevval/source/tree/master/r/bioinf2021/lab2rna_seq}{доступен на сайте GitHub}.
html-страницу, сгенерированную этим скриптом, можно посмотреть на сайте \href{http://azuev.ddns.net/~valera/stu/2021bio/rnaseq/differ_expression.nb.html}{azuev.ddns.net/\string~valera/stu/2021bio/rnaseq/differ\_expression.nb.html}.

\newpage
\begin{thebibliography}{99}
    \bibitem{breen} Breen, M., Uhlmann, A., Nday, C. et al. Candidate gene networks and blood biomarkers of methamphetamine-associated psychosis: an integrative RNA-sequencing report. Transl Psychiatry 6, e802 (2016). doi: \href{https://doi.org/10.1038/tp.2016.67}{10.1038/tp.2016.67}
    \bibitem{bogler} Bolger, A.M., Lohse, M., Usadel B. Trimmomatic: a flexible trimmer for Illumina sequence data. Bioinformatics. 2014 Aug 1;30(15):2114-20. doi: \href{https://doi.org/10.1093/bioinformatics/btu170}{10.1093/bioinformatics/btu170}
    \bibitem{love2014} Love MI, Huber W, Anders S. Moderated estimation of fold change and dispersion for RNA-seq data with DESeq2. Genome Biol. 2014;15(12):550. doi:
    \href{https://doi.org/10.1186/s13059-014-0550-8}{10.1186/s13059-014-0550-8}. PMID: 25516281; PMCID: PMC4302049.

    \bibitem{young2010} Young, M.D., Wakefield, M.J., Smyth, G.K. et al. Gene ontology analysis for RNA-seq: accounting for selection bias. Genome Biol 11, R14 (2010).
    doi: \href{https://doi.org/10.1186/gb-2010-11-2-r14}{10.1186/gb-2010-11-2-r14}

    \bibitem{adkins2011} Adkins DE, Khachane AN, McClay JL, et al. SNP-based analysis of neuroactive ligand-receptor interaction pathways implicates PGE2 as a novel mediator of antipsychotic treatment response: data from the CATIE study. Schizophr Res. 2012;135(1-3):200-201. doi:\href{https://doi.org/10.1016/j.schres.2011.11.002}{10.1016/j.schres.2011.11.002}

    \bibitem{ropek2019} Ropek N, Al-Serori H, Mišík M, Nersesyan A, Sitte HH, Collins AR, Shaposhnikov S, Knasmüller S, Kundi M, Ferk F. Methamphetamine ("crystal meth") causes induction of DNA damage and chromosomal aberrations in human derived cells. Food Chem Toxicol. 2019 Jun;128:1-7. doi: \href{https://doi.org/10.1016/j.fct.2019.03.035}{10.1016/j.fct.2019.03.035}. Epub 2019 Mar 22. PMID: 30910685.

\end{thebibliography}

\newpage
\subfile{appendix}

\end{document}