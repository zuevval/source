\documentclass{article}

\usepackage[utf8]{inputenc} % russian, do not change
\usepackage[T2A, T1]{fontenc} % russian, do not change
\usepackage[english, russian]{babel} % russian, do not change

% fonts
\usepackage{fontspec} % different fonts
\setmainfont{Times New Roman}

\usepackage[]{graphicx}\usepackage[]{color}
% maxwidth is the original width if it is less than linewidth
% otherwise use linewidth (to make sure the graphics do not exceed the margin)
\makeatletter
\def\maxwidth{ %
  \ifdim\Gin@nat@width>\linewidth
    \linewidth
  \else
    \Gin@nat@width
  \fi
}
\makeatother

\definecolor{fgcolor}{rgb}{0.345, 0.345, 0.345}
\newcommand{\hlnum}[1]{\textcolor[rgb]{0.686,0.059,0.569}{#1}}%
\newcommand{\hlstr}[1]{\textcolor[rgb]{0.192,0.494,0.8}{#1}}%
\newcommand{\hlcom}[1]{\textcolor[rgb]{0.678,0.584,0.686}{\textit{#1}}}%
\newcommand{\hlopt}[1]{\textcolor[rgb]{0,0,0}{#1}}%
\newcommand{\hlstd}[1]{\textcolor[rgb]{0.345,0.345,0.345}{#1}}%
\newcommand{\hlkwa}[1]{\textcolor[rgb]{0.161,0.373,0.58}{\textbf{#1}}}%
\newcommand{\hlkwb}[1]{\textcolor[rgb]{0.69,0.353,0.396}{#1}}%
\newcommand{\hlkwc}[1]{\textcolor[rgb]{0.333,0.667,0.333}{#1}}%
\newcommand{\hlkwd}[1]{\textcolor[rgb]{0.737,0.353,0.396}{\textbf{#1}}}%
\let\hlipl\hlkwb

\usepackage{framed}
\makeatletter
\newenvironment{kframe}{%
 \def\at@end@of@kframe{}%
 \ifinner\ifhmode%
  \def\at@end@of@kframe{\end{minipage}}%
  \begin{minipage}{\columnwidth}%
 \fi\fi%
 \def\FrameCommand##1{\hskip\@totalleftmargin \hskip-\fboxsep
 \colorbox{shadecolor}{##1}\hskip-\fboxsep
     % There is no \\@totalrightmargin, so:
     \hskip-\linewidth \hskip-\@totalleftmargin \hskip\columnwidth}%
 \MakeFramed {\advance\hsize-\width
   \@totalleftmargin\z@ \linewidth\hsize
   \@setminipage}}%
 {\par\unskip\endMakeFramed%
 \at@end@of@kframe}
\makeatother

\definecolor{shadecolor}{rgb}{.97, .97, .97}
\definecolor{messagecolor}{rgb}{0, 0, 0}
\definecolor{warningcolor}{rgb}{1, 0, 1}
\definecolor{errorcolor}{rgb}{1, 0, 0}
\newenvironment{knitrout}{}{} % an empty environment to be redefined in TeX

\usepackage{alltt}
\oddsidemargin=0cm
\textwidth=17cm

\usepackage{hyperref}
\hypersetup{pdfstartview=FitH,  linkcolor=blue, urlcolor=blue, colorlinks=true}

\usepackage{caption}
\usepackage{subcaption} % captions for subfigures
\usepackage{graphicx}
\graphicspath{{./img/}}
\usepackage{float} % force pictures position with \begin{figure}[H]
\newcommand{\myPictWidth}{.99\textwidth}
\IfFileExists{upquote.sty}{\usepackage{upquote}}{}
\begin{document}

\title{Анализ данных DNA-Seq}
\author{Валерий Зуев, 3630102/70401}
\maketitle

\section{Задание}

Необходимо выполнить манипуляции с данными об исследовании экспрессии генов по образцам РНК, выделенным из моноцитов
\begin{enumerate}
    \item взятых из крови людей, больных системной красной волчанкой (SLE) и ревматоидным артритом (RA)
    \item взятых от здоровых пациентов, но обработанных в лабораторных условиях цитокинами
    \begin{enumerate}
        \item TNF-$\alpha$, который, как считается, отвечает за клиническое проявление ревматоидного артрита
        \item интерфероны IFN-$\alpha/\gamma$, которые связывают с SLE.
    \end{enumerate}
\end{enumerate}

В статье, где описываются результыты исследования \cite{smiljanovich2012}, утверждается, что пациенты с SLE и RA демонстрируют специфичные для болезни профили экспрессии генов, ответственных за синтез цитокинов: у пациентов, больных SLE, повышен уровень экспрессии многих генов, регулирующих синтез IFN-$\alpha$, у пациентов с ревматоидным артритом -- TNF-$\alpha$.

В данной работе необходимо воспроизвести исследования, выполненные с лабораторными данными:

\begin{enumerate}
    \item Загрузить результаты экспериментов (артефакты работы сканеров Affymetrix) с сайтов \href{https://www.ebi.ac.uk/arrayexpress/experiments/E-GEOD-38351/}{ArrayExpress} и \href{https://www.ncbi.nlm.nih.gov/geo/download/?acc=GSE38351}{GEO}
    \item Произвести оценку качества данных, при необходимости отсеять выбросы; произвести нормализацию.
    \item Найти дифференциально экспрессирующиеся гены.
    \item Провести кластеризацию профилей дифференциально экспрессирующихся генов в клетках пациентов, а также обработанных \textit{in vitro} клеток; построить тепловые карты.
    \item Выполнить анализ обогащённости генов метаболическими путями из базы данных KEGG и Reactome
    \item Сравнить результаты с результатами, полученными в статье \cite{smiljanovich2012}.
\end{enumerate}

\section{Ход работы}

\subsection{Загрузка данных}

Подготавливаем рабочее пространство (нужно изменить \texttt{data\_dir} на директорию, содержащую код на Вашем компьютере):
\begin{knitrout}
\definecolor{shadecolor}{rgb}{0.969, 0.969, 0.969}\color{fgcolor}\begin{kframe}
\begin{alltt}
\hlkwd{library}\hlstd{(dplyr)}
\hlstd{data_dir} \hlkwb{<-} \hlstr{"E:/Users/vzuev/github-zuevval/source/r/bioinf2021/lab1microarrays/out"}
\hlkwd{setwd}\hlstd{(data_dir)}
\end{alltt}
\end{kframe}
\end{knitrout}

Загружаем данные (они разделены на два массива, первый -- результат сканирования в формате AFFY-44, второй -- AFFY-33).

Loading data (it consists of two assays, first is the result of AFFY-44 and second of AFFY-33 scan).

Если Вы запускаетсе этот код первый раз, необходимо раскомментировать строки, в которых данные загружаются из базы.

\begin{knitrout}
\definecolor{shadecolor}{rgb}{0.969, 0.969, 0.969}\color{fgcolor}\begin{kframe}
\begin{alltt}
\hlstd{geod_list_filename} \hlkwb{<-} \hlstr{"geod38351.RData"}
\hlcom{# geod38351 <- ArrayExpress::getAE("E-GEOD-38351", type = "full")}
\hlcom{# save(geod38351, file=geod_list_filename)}
\hlkwd{load}\hlstd{(geod_list_filename)}
\hlstd{aesets} \hlkwb{<-} \hlstd{ArrayExpress}\hlopt{::}\hlkwd{ae2bioc}\hlstd{(}\hlkwc{mageFiles} \hlstd{= geod38351)}
\end{alltt}

\begin{alltt}
\hlstd{aeset1} \hlkwb{<-} \hlstd{aesets[[}\hlnum{1}\hlstd{]]} \hlcom{# AFFY-44}
\hlstd{aeset2} \hlkwb{<-} \hlstd{aesets[[}\hlnum{2}\hlstd{]]} \hlcom{# AFFY-33}
\end{alltt}
\end{kframe}
\end{knitrout}

Посмотрим, какие факторы поставлены в соответствие образцам:
\begin{knitrout}
\definecolor{shadecolor}{rgb}{0.969, 0.969, 0.969}\color{fgcolor}\begin{kframe}
\begin{alltt}
\hlstd{colnames1} \hlkwb{<-} \hlkwd{colnames}\hlstd{(}\hlkwd{pData}\hlstd{(aeset1))}
\hlstd{fac} \hlkwb{<-} \hlstd{colnames1[}\hlkwd{grep}\hlstd{(}\hlstr{"Factor"}\hlstd{, colnames1)]}
\hlstd{fac}
\end{alltt}
\begin{verbatim}
## [1] "Factor.Value.stimulus." "Factor.Value.disease."
\end{verbatim}
\end{kframe}
\end{knitrout}

\subsection{Оценка качества}
\begin{knitrout}
\definecolor{shadecolor}{rgb}{0.969, 0.969, 0.969}\color{fgcolor}\begin{kframe}
\begin{alltt}
\hlcom{# devnull <- mapply(function(outdir = ? str, eset = ? ExpressionSet) \{}
\hlcom{#  arrayQualityMetrics::arrayQualityMetrics(}
\hlcom{#    expressionset = eset,}
\hlcom{#    outdir = outdir,}
\hlcom{#    force = FALSE,}
\hlcom{#    do.logtransform = TRUE,}
\hlcom{#    intgroup = fac)}
\hlcom{#\}, list("qaraw_geod_38351_1", "qaraw_geod_38351_2"), c(aeset1, aeset2))}
\end{alltt}
\end{kframe}
\end{knitrout}

\begin{figure}[H]
    \centering
    \begin{subfigure}{.5\textwidth}
        \centering
        \includegraphics[width=\myPictWidth]{box1}
        \caption{boxplot (assay 1)}
    \end{subfigure}%
    \begin{subfigure}{.5\textwidth}
        \centering
        \includegraphics[width=\myPictWidth]{box2}
        \caption{boxplot (assay 2)}
    \end{subfigure}
    \begin{subfigure}{.5\textwidth}
        \centering
        \includegraphics[width=\myPictWidth]{ma1}
        \caption{MA plot (assay 1)}
    \end{subfigure}%
    \begin{subfigure}{.5\textwidth}
        \centering
        \includegraphics[width=\myPictWidth]{ma2}
        \caption{MA plot (assay 2)}
    \end{subfigure}
\caption{Метрики качества}
\end{figure}

Массив \texttt{aeset1} содержит выброс (образец  \textnumero 1).
Уберём из данных это наблюдение и проведём нормализацию:

% computation,results=hide,cache=T
\begin{knitrout}
\definecolor{shadecolor}{rgb}{0.969, 0.969, 0.969}\color{fgcolor}\begin{kframe}
\begin{alltt}
\hlstd{c_aeset1} \hlkwb{<-} \hlstd{oligo}\hlopt{::}\hlkwd{rma}\hlstd{(aeset1[,} \hlopt{-}\hlnum{1}\hlstd{])}
\hlstd{c_aeset2} \hlkwb{<-} \hlstd{oligo}\hlopt{::}\hlkwd{rma}\hlstd{(aeset2)}
\end{alltt}
\end{kframe}
\end{knitrout}

\subsection{Поиск дифференциально экспрессирующихся генов}

Обучим линейные модели, описывающие зависимость факторов от профилей экспрессии.
В первом массиве в основном профили клеток, стимулированных \textit{in vitro}, во втором -- пациентов, поэтому будем строить одну модель по первому массиву и одну по второму.

Для удобства изменим имена факторов на более короткие:

\begin{knitrout}
\definecolor{shadecolor}{rgb}{0.969, 0.969, 0.969}\color{fgcolor}\begin{kframe}
\begin{alltt}
\hlstd{c_aeset_colnames} \hlkwb{<-} \hlkwd{c}\hlstd{(}\hlstr{"stimulus"}\hlstd{,} \hlstr{"disease"}\hlstd{)}
\hlstd{groups1} \hlkwb{<-} \hlkwd{pData}\hlstd{(c_aeset1)[, fac]}
\hlstd{groups1[groups1} \hlopt{==} \hlstr{"incubated for 1.5 hour without any stimulation"}\hlstd{]} \hlkwb{<-} \hlstr{"incub"}
\hlstd{groups1[groups1} \hlopt{==} \hlstr{"no incubation"}\hlstd{]} \hlkwb{<-} \hlstr{"noincub"}
\hlstd{groups1[groups1} \hlopt{==} \hlstr{"stimulated with IFN-gamma for 1.5 hour"}\hlstd{]} \hlkwb{<-} \hlstr{"ig"}
\hlstd{groups1[groups1} \hlopt{==} \hlstr{"stimulated with IFN-alpha-2a for 1.5 hour"}\hlstd{]} \hlkwb{<-} \hlstr{"ia"}
\hlstd{groups1[groups1} \hlopt{==} \hlstr{"stimulated with TNF-alpha for 1.5 hour"}\hlstd{]} \hlkwb{<-} \hlstr{"ta"}
\hlstd{groups1}
\end{alltt}
\begin{verbatim}
##                                   Factor.Value.stimulus. Factor.Value.disease.
## GSM940522_Un_0h_7_133P.CEL                       noincub                normal
## GSM940521_Un_0h_6_133P.CEL                       noincub                normal
## GSM940520_Un_0h_5_133P.CEL                       noincub                normal
## GSM940519_Un_0h_4_133P.CEL                       noincub                normal
## GSM940518_Un_0h_3_133P.CEL                       noincub                normal
## GSM940517_Un_0h_2_133P.CEL                       noincub                normal
## GSM940516_Un_0h_1_133P.CEL                       noincub                normal
## GSM940515_Un_90min_11_133P.CEL                     incub                normal
## GSM940514_Un_90min_10_133P.CEL                     incub                normal
## GSM940513_Un_90min_9_133P.CEL                      incub                normal
## GSM940512_Un_90min_8_133P.CEL                      incub                normal
## GSM940511_Un_90min_7_133P.CEL                      incub                normal
## GSM940510_Un_90min_6_133P.CEL                      incub                normal
## GSM940509_Un_90min_5_133P.CEL                      incub                normal
## GSM940508_Un_90min_4_133P.CEL                      incub                normal
## GSM940507_Un_90min_3_133P.CEL                      incub                normal
## GSM940506_Un_90min_2_133P.CEL                      incub                normal
## GSM940505_Un_90min_1_133P.CEL                      incub                normal
## GSM940504_IFNg_90min_7_133P.CEL                       ig                normal
## GSM940503_IFNg_90min_6_133P.CEL                       ig                normal
## GSM940502_IFNg_90min_5_133P.CEL                       ig                normal
## GSM940501_IFNg_90min_4_133P.CEL                       ig                normal
## GSM940500_IFNg_90min_3_133P.CEL                       ig                normal
## GSM940499_IFNg_90min_2_133P.CEL                       ig                normal
## GSM940498_IFNg_90min_1_133P.CEL                       ig                normal
## GSM940497_IFNa2a_90min_7_133P.CEL                     ia                normal
## GSM940496_IFNa2a_90min_6_133P.CEL                     ia                normal
## GSM940495_IFNa2a_90min_5_133P.CEL                     ia                normal
## GSM940494_IFNa2a_90min_4_133P.CEL                     ia                normal
## GSM940493_IFNa2a_90min_3_133P.CEL                     ia                normal
## GSM940492_IFNa2a_90min_2_133P.CEL                     ia                normal
## GSM940491_IFNa2a_90min_1_133P.CEL                     ia                normal
## GSM940490_TNF_90min_3_133P.CEL                        ta                normal
## GSM940489_TNF_90min_2_133P.CEL                        ta                normal
## GSM940488_TNF_90min_1_133P.CEL                        ta                normal
## GSM940487_RA_98_Mo_test_133P.cel                    none  rheumatoid arthritis
## GSM940486_RA_88_Mo_test_133P.cel                    none  rheumatoid arthritis
## GSM940485_RA_18_Mo_test_133P.cel                    none  rheumatoid arthritis
## GSM940484_RA_1_Mo_test_133P.cel                     none  rheumatoid arthritis
\end{verbatim}
\begin{alltt}
\hlkwd{colnames}\hlstd{(groups1)} \hlkwb{<-} \hlstd{c_aeset_colnames}
\hlstd{c_aeset_stimulus} \hlkwb{<-} \hlstd{c_aeset1[, groups1}\hlopt{$}\hlstd{disease} \hlopt{==} \hlstr{"normal"}\hlstd{]} \hlcom{# drop ill patients}
\hlstd{groups_stimulus} \hlkwb{<-} \hlstd{groups1} \hlopt
  \hlkwd{filter}\hlstd{(disease} \hlopt{==} \hlstr{"normal"}\hlstd{)} \hlopt \hlcom{# drop ill patients}
  \hlkwd{select}\hlstd{(}\hlopt{-}\hlstd{disease)}

\hlstd{groups2} \hlkwb{<-} \hlkwd{pData}\hlstd{(c_aeset2)[, fac]}
\hlstd{groups2[groups2} \hlopt{==} \hlstr{"rheumatoid arthritis"}\hlstd{]} \hlkwb{<-} \hlstr{"ra"}
\hlstd{groups2[groups2} \hlopt{==} \hlstr{"systemic lupus erythematosus"}\hlstd{]} \hlkwb{<-} \hlstr{"sl"}
\hlkwd{colnames}\hlstd{(groups2)} \hlkwb{<-} \hlstd{c_aeset_colnames}
\hlstd{groups2}
\end{alltt}
\begin{verbatim}
##                                  stimulus disease
## GSM940483_ND_114_Mo_133A.CEL         none  normal
## GSM940482_ND_113_Mo_133A.CEL         none  normal
## GSM940481_ND_24_Mo_133A.CEL          none  normal
## GSM940480_ND_20_Mo_133A.CEL          none  normal
## GSM940479_ND_18_Mo_133A.CEL          none  normal
## GSM940478_ND_14_Mo_133A.CEL          none  normal
## GSM940477_ND_13_Mo_133A.CEL          none  normal
## GSM940476_ND_12_Mo_133A.CEL          none  normal
## GSM940475_ND_11_Mo_133A.CEL          none  normal
## GSM940474_ND_9_Mo_133A.CEL           none  normal
## GSM940473_ND_7_Mo_133A.CEL           none  normal
## GSM940472_ND_3_Mo_133A.CEL           none  normal
## GSM940471_RA_11_Mo_133A.CEL          none      ra
## GSM940470_RA_9_Mo_133A.CEL           none      ra
## GSM940469_RA_7_Mo_133A.CEL           none      ra
## GSM940468_RA_5_Mo_133A.CEL           none      ra
## GSM940467_RA_4_Mo_133A.CEL           none      ra
## GSM940466_RA_3_Mo_133A.CEL           none      ra
## GSM940465_RA_2_Mo_133A.CEL           none      ra
## GSM940464_RA_1_Mo_133A.CEL           none      ra
## GSM940463_SL_21_Mo_test_133A.CEL     none      sl
## GSM940462_SL_16_Mo_test_133A.CEL     none      sl
## GSM940461_SL_14_Mo_test_133A.CEL     none      sl
## GSM940460_SL_13_Mo_test_133A.CEL     none      sl
## GSM940459_SL_11_Mo_test_133A.CEL     none      sl
## GSM940458_SL_10_Mo_133A.CEL          none      sl
## GSM940457_SL_9_Mo_133A.CEL           none      sl
## GSM940456_SL_8_Mo_133A.CEL           none      sl
## GSM940455_SL_7_Mo_133A.CEL           none      sl
## GSM940454_SL_5_Mo_133A.CEL           none      sl
## GSM940453_SL_4_Mo_133A.CEL           none      sl
## GSM940452_SL_3_Mo_133A.CEL           none      sl
## GSM940451_SL_2_Mo_133A.CEL           none      sl
## GSM940450_SL_1_Mo_133A.CEL           none      sl
\end{verbatim}
\begin{alltt}
\hlstd{c_aeset_disease} \hlkwb{<-} \hlstd{c_aeset2} \hlcom{# just define a new name for convenience}
\hlstd{groups_disease} \hlkwb{<-} \hlstd{groups2} \hlopt \hlkwd{select}\hlstd{(}\hlopt{-}\hlstd{stimulus)}
\end{alltt}
\end{kframe}
\end{knitrout}

Составив дизайн моделей, можно обучить их и вычислить набор статистик с помощью \texttt{eBayes}:

\begin{knitrout}
\definecolor{shadecolor}{rgb}{0.969, 0.969, 0.969}\color{fgcolor}\begin{kframe}
\begin{alltt}
\hlstd{f_stimulus} \hlkwb{<-} \hlkwd{factor}\hlstd{(}\hlkwd{unlist}\hlstd{(groups_stimulus))}
\hlstd{design_stimulus} \hlkwb{<-} \hlkwd{model.matrix}\hlstd{(}\hlopt{~}\hlstd{f_stimulus)}
\hlkwd{colnames}\hlstd{(design_stimulus)} \hlkwb{<-} \hlkwd{colnames}\hlstd{(design_stimulus)} \hlopt \hlkwd{lapply}\hlstd{(}\hlkwa{function}\hlstd{(}\hlkwc{x}\hlstd{)} \hlkwd{gsub}\hlstd{(}\hlstr{"f_stimulus"}\hlstd{,} \hlstr{""}\hlstd{, x))}
\hlkwd{print}\hlstd{(design_stimulus)}
\end{alltt}
\begin{verbatim}
##    (Intercept) ig incub noincub ta
## 1            1  0     0       1  0
## 2            1  0     0       1  0
## 3            1  0     0       1  0
## 4            1  0     0       1  0
## 5            1  0     0       1  0
## 6            1  0     0       1  0
## 7            1  0     0       1  0
## 8            1  0     1       0  0
## 9            1  0     1       0  0
## 10           1  0     1       0  0
## 11           1  0     1       0  0
## 12           1  0     1       0  0
## 13           1  0     1       0  0
## 14           1  0     1       0  0
## 15           1  0     1       0  0
## 16           1  0     1       0  0
## 17           1  0     1       0  0
## 18           1  0     1       0  0
## 19           1  1     0       0  0
## 20           1  1     0       0  0
## 21           1  1     0       0  0
## 22           1  1     0       0  0
## 23           1  1     0       0  0
## 24           1  1     0       0  0
## 25           1  1     0       0  0
## 26           1  0     0       0  0
## 27           1  0     0       0  0
## 28           1  0     0       0  0
## 29           1  0     0       0  0
## 30           1  0     0       0  0
## 31           1  0     0       0  0
## 32           1  0     0       0  0
## 33           1  0     0       0  1
## 34           1  0     0       0  1
## 35           1  0     0       0  1
## attr(,"assign")
## [1] 0 1 1 1 1
## attr(,"contrasts")
## attr(,"contrasts")$f_stimulus
## [1] "contr.treatment"
\end{verbatim}
\begin{alltt}
\hlkwd{print}\hlstd{(limma}\hlopt{::}\hlkwd{is.fullrank}\hlstd{(design_stimulus))} \hlcom{# should be TRUE}
\end{alltt}
\begin{verbatim}
## [1] TRUE
\end{verbatim}
\begin{alltt}
\hlstd{fit_stimulus} \hlkwb{<-} \hlstd{limma}\hlopt{::}\hlkwd{lmFit}\hlstd{(c_aeset_stimulus, design_stimulus)}
\hlstd{fit_stimulus2} \hlkwb{<-} \hlstd{limma}\hlopt{::}\hlkwd{eBayes}\hlstd{(fit_stimulus)}
\hlkwd{head}\hlstd{(fit_stimulus2}\hlopt{$}\hlstd{coefficients)}
\end{alltt}
\begin{verbatim}
##           (Intercept)          ig       incub    noincub           ta
## 1007_s_at    5.486107  0.13449020  0.03166571 -0.0767085  0.185413075
## 1053_at      7.437284 -0.19053967 -0.24036103 -0.2383999 -1.260466493
## 117_at       7.540072 -0.22570990  0.73219823  0.9689542 -0.365928565
## 121_at       7.622349 -0.06832876 -0.18950035 -0.2561430 -0.445371906
## 1255_g_at    2.581148  0.04617147 -0.13434369 -0.1802197 -0.001689818
## 1294_at      8.192515 -0.19773569  0.03052780  0.1171849  0.247536641
\end{verbatim}
\begin{alltt}
\hlkwd{png}\hlstd{(}\hlkwc{filename} \hlstd{=} \hlstr{"./img/volcanoplot_stimulus.png"}\hlstd{)}
\hlkwd{par}\hlstd{(}\hlkwc{mfrow} \hlstd{=} \hlkwd{c}\hlstd{(}\hlnum{1}\hlstd{,} \hlnum{2}\hlstd{))}
\hlstd{limma}\hlopt{::}\hlkwd{volcanoplot}\hlstd{(fit_stimulus2,} \hlkwc{coef} \hlstd{=} \hlstr{"ta"}\hlstd{,} \hlkwc{highlight} \hlstd{=} \hlnum{15}\hlstd{,} \hlkwc{main} \hlstd{=} \hlstr{"stimulus: TNF-alpha"}\hlstd{)}
\hlstd{limma}\hlopt{::}\hlkwd{volcanoplot}\hlstd{(fit_stimulus2,} \hlkwc{coef} \hlstd{=} \hlstr{"ig"}\hlstd{,} \hlkwc{highlight} \hlstd{=} \hlnum{15}\hlstd{,} \hlkwc{main} \hlstd{=} \hlstr{"stimulus: INF-gamma"}\hlstd{)}
\hlkwd{dev.off}\hlstd{()}
\end{alltt}
\end{kframe}
\end{knitrout}

\begin{figure}[H]
    \centering
    \includegraphics[width=0.8\textwidth]{volcanoplot_stimulus}
    \caption{график-вулкан (моноциты, стимулированные в лабораторных условиях)}
\end{figure}

\begin{knitrout}
\definecolor{shadecolor}{rgb}{0.969, 0.969, 0.969}\color{fgcolor}\begin{kframe}
\begin{alltt}
\hlstd{f_disease} \hlkwb{<-} \hlkwd{factor}\hlstd{(}\hlkwd{unlist}\hlstd{(groups_disease))}
\hlstd{design_disease} \hlkwb{<-} \hlkwd{model.matrix}\hlstd{(}\hlopt{~}\hlstd{f_disease)}
\hlkwd{colnames}\hlstd{(design_disease)} \hlkwb{<-} \hlkwd{colnames}\hlstd{(design_disease)} \hlopt \hlkwd{lapply}\hlstd{(}\hlkwa{function}\hlstd{(}\hlkwc{x}\hlstd{)} \hlkwd{gsub}\hlstd{(}\hlstr{"f_disease"}\hlstd{,} \hlstr{""}\hlstd{, x))}
\hlkwd{print}\hlstd{(design_disease)}
\end{alltt}
\begin{verbatim}
##    (Intercept) ra sl
## 1            1  0  0
## 2            1  0  0
## 3            1  0  0
## 4            1  0  0
## 5            1  0  0
## 6            1  0  0
## 7            1  0  0
## 8            1  0  0
## 9            1  0  0
## 10           1  0  0
## 11           1  0  0
## 12           1  0  0
## 13           1  1  0
## 14           1  1  0
## 15           1  1  0
## 16           1  1  0
## 17           1  1  0
## 18           1  1  0
## 19           1  1  0
## 20           1  1  0
## 21           1  0  1
## 22           1  0  1
## 23           1  0  1
## 24           1  0  1
## 25           1  0  1
## 26           1  0  1
## 27           1  0  1
## 28           1  0  1
## 29           1  0  1
## 30           1  0  1
## 31           1  0  1
## 32           1  0  1
## 33           1  0  1
## 34           1  0  1
## attr(,"assign")
## [1] 0 1 1
## attr(,"contrasts")
## attr(,"contrasts")$f_disease
## [1] "contr.treatment"
\end{verbatim}
\begin{alltt}
\hlkwd{print}\hlstd{(limma}\hlopt{::}\hlkwd{is.fullrank}\hlstd{(design_disease))} \hlcom{# should be TRUE}
\end{alltt}
\begin{verbatim}
## [1] TRUE
\end{verbatim}
\begin{alltt}
\hlstd{fit_disease} \hlkwb{<-} \hlstd{limma}\hlopt{::}\hlkwd{lmFit}\hlstd{(c_aeset_disease, design_disease)}
\hlstd{fit_disease2} \hlkwb{<-} \hlstd{limma}\hlopt{::}\hlkwd{eBayes}\hlstd{(fit_disease)}
\hlkwd{head}\hlstd{(fit_disease2}\hlopt{$}\hlstd{coefficients)}
\end{alltt}
\begin{verbatim}
##           (Intercept)          ra           sl
## 1007_s_at    8.798228 -0.08031188 -0.148079381
## 1053_at      7.547945  0.03210987 -0.024725947
## 117_at       9.509895  0.06833807  0.114255460
## 121_at       9.982004 -0.11442005 -0.283091830
## 1255_g_at    5.185456  0.01852113 -0.009584626
## 1294_at      9.620198  0.20138900  0.174358416
\end{verbatim}
\begin{alltt}
\hlkwd{png}\hlstd{(}\hlkwc{filename} \hlstd{=} \hlstr{"./img/volcanoplot_disease.png"}\hlstd{)}
\hlkwd{par}\hlstd{(}\hlkwc{mfrow} \hlstd{=} \hlkwd{c}\hlstd{(}\hlnum{1}\hlstd{,} \hlnum{2}\hlstd{))}
\hlstd{limma}\hlopt{::}\hlkwd{volcanoplot}\hlstd{(fit_disease2,} \hlkwc{coef} \hlstd{=} \hlstr{"ra"}\hlstd{,} \hlkwc{highlight} \hlstd{=} \hlnum{15}\hlstd{,} \hlkwc{main} \hlstd{=} \hlstr{"rheumatoid arthritis"}\hlstd{)}
\hlstd{limma}\hlopt{::}\hlkwd{volcanoplot}\hlstd{(fit_disease2,} \hlkwc{coef} \hlstd{=} \hlstr{"sl"}\hlstd{,} \hlkwc{highlight} \hlstd{=} \hlnum{15}\hlstd{,} \hlkwc{main} \hlstd{=} \hlstr{"systemic lupus erythematosus"}\hlstd{)}
\hlkwd{dev.off}\hlstd{()}
\end{alltt}
\begin{verbatim}
## pdf
##   2
\end{verbatim}
\end{kframe}
\end{knitrout}

\begin{figure}[H]
    \centering
    \includegraphics[width=0.8\textwidth]{volcanoplot_disease}
    \caption{график-вулкан (пациенты,больные RA / SLE)}
\end{figure}

Отфильтруем гены по значению $ p-value $ сверхпредставленности (для наборов \texttt{c\_aeset\_stimulus}, \texttt{c\_aeset\_ra}, \texttt{c\_aeset\_sl} -- в образцах, обработанных INF-$\gamma$, взятых у пациентов с RA, взятых у пациентов с SLE соответственно) и изобразим тепловые карты профилей генов, которые демонстрируют дифференциальную экспрессию:
\begin{knitrout}
\definecolor{shadecolor}{rgb}{0.969, 0.969, 0.969}\color{fgcolor}\begin{kframe}
\begin{alltt}
\hlstd{draw_clust_heatmap} \hlkwb{<-} \hlkwa{function}\hlstd{(}\hlkwc{c_aeset}\hlstd{,} \hlkwc{groups}\hlstd{,} \hlkwc{fit}\hlstd{,} \hlkwc{pval_thresh}\hlstd{,} \hlkwc{coef}\hlstd{,} \hlkwc{title}\hlstd{)\{}
  \hlstd{results} \hlkwb{<-} \hlstd{limma}\hlopt{::}\hlkwd{topTable}\hlstd{(fit,} \hlkwc{coef} \hlstd{= coef,} \hlkwc{adjust} \hlstd{=} \hlstr{"BH"}\hlstd{,} \hlkwc{number} \hlstd{=} \hlkwd{nrow}\hlstd{(c_aeset))}
  \hlkwd{print}\hlstd{(}\hlkwd{as.integer}\hlstd{(results[,} \hlstr{"P.Value"}\hlstd{]} \hlopt{<} \hlstd{pval_thresh)} \hlopt
          \hlstd{na.exclude} \hlopt
          \hlstd{sum)} \hlcom{# how many differentially expressed genese were found?}
  \hlstd{topgenes} \hlkwb{<-} \hlstd{results[results[,} \hlstr{"P.Value"}\hlstd{]} \hlopt{<} \hlstd{pval_thresh,]}
  \hlstd{diffexpr_ids} \hlkwb{<-} \hlkwd{rownames}\hlstd{(topgenes)}

  \hlstd{diffexpr_ids} \hlopt
    \hlstd{unlist} \hlopt
    \hlkwd{write.table}\hlstd{(}\hlkwd{paste0}\hlstd{(}\hlstr{"./genes_diffexpr_"}\hlstd{, title,} \hlstr{".csv"}\hlstd{),}
                \hlkwc{sep} \hlstd{=} \hlstr{","}\hlstd{,} \hlkwc{row.names} \hlstd{= F,} \hlkwc{col.names} \hlstd{= F,} \hlkwc{quote} \hlstd{= F)}
  \hlstd{results} \hlopt
    \hlstd{rownames} \hlopt
    \hlstd{unlist} \hlopt
    \hlkwd{write.table}\hlstd{(}\hlkwd{paste0}\hlstd{(}\hlstr{"./genes_background_"}\hlstd{, title,} \hlstr{".csv"}\hlstd{),}
                \hlkwc{sep} \hlstd{=} \hlstr{","}\hlstd{,} \hlkwc{row.names} \hlstd{= F,} \hlkwc{col.names} \hlstd{= F,} \hlkwc{quote} \hlstd{= F)}

  \hlstd{m} \hlkwb{<-} \hlkwd{exprs}\hlstd{(c_aeset[diffexpr_ids,])}
  \hlkwd{ncol}\hlstd{(m)}
  \hlkwd{colnames}\hlstd{(m)} \hlkwb{<-} \hlkwd{unlist}\hlstd{(groups[}\hlnum{1}\hlstd{])}
  \hlkwd{png}\hlstd{(}\hlkwc{filename} \hlstd{=} \hlkwd{paste0}\hlstd{(}\hlstr{"./img/heatmap_"}\hlstd{, title,} \hlstr{".png"}\hlstd{))}
  \hlkwd{heatmap}\hlstd{(m)}
  \hlkwd{dev.off}\hlstd{()}
\hlstd{\}}

\hlstd{c_aeset_ra} \hlkwb{<-} \hlstd{c_aeset_disease[, groups_disease}\hlopt{$}\hlstd{disease} \hlopt{!=} \hlstr{"sl"}\hlstd{]}
\hlstd{c_aeset_sl} \hlkwb{<-} \hlstd{c_aeset_disease[, groups_disease}\hlopt{$}\hlstd{disease} \hlopt{!=} \hlstr{"ra"}\hlstd{]}
\hlstd{groups_ra} \hlkwb{<-} \hlstd{groups_disease} \hlopt \hlkwd{filter}\hlstd{(disease} \hlopt{!=} \hlstr{"sl"}\hlstd{)}
\hlstd{groups_sl} \hlkwb{<-} \hlstd{groups_disease} \hlopt \hlkwd{filter}\hlstd{(disease} \hlopt{!=} \hlstr{"ra"}\hlstd{)}

\hlstd{devnull} \hlkwb{<-} \hlkwd{mapply}\hlstd{(draw_clust_heatmap,}
                  \hlkwd{list}\hlstd{(c_aeset_stimulus, c_aeset_ra, c_aeset_sl),}
                  \hlkwd{list}\hlstd{(groups_stimulus, groups_ra, groups_sl),}
                  \hlkwd{list}\hlstd{(fit_stimulus2, fit_disease2, fit_disease2),}
                  \hlkwd{list}\hlstd{(}\hlnum{1e-7}\hlstd{,} \hlnum{5} \hlopt{*} \hlnum{1e-3}\hlstd{,} \hlnum{2} \hlopt{*} \hlnum{1e-3}\hlstd{),}
                  \hlkwd{list}\hlstd{(}\hlstr{"ig"}\hlstd{,} \hlstr{"ra"}\hlstd{,} \hlstr{"sl"}\hlstd{),}
                  \hlkwd{list}\hlstd{(}\hlstr{"stimulus"}\hlstd{,} \hlstr{"ra"}\hlstd{,} \hlstr{"sl"}\hlstd{))}
\end{alltt}
\begin{verbatim}
## [1] 1732
## [1] 723
## [1] 1056
\end{verbatim}
\end{kframe}
\end{knitrout}

\begin{figure}[H]
    \centering
    \includegraphics[width=0.8\textwidth]{heatmap_stimulus}
    \caption{тепловая карта (различные виды стимуляции в лаборатории)}
\end{figure}

\begin{figure}[H]
    \centering
    \includegraphics[width=0.8\textwidth]{heatmap_sl}
    \caption{тепловая карта (пациенты с SLE и здоровые)}
\end{figure}

\begin{figure}[H]
    \centering
    \includegraphics[width=0.8\textwidth]{heatmap_ra}
    \caption{тепловая карта (пациенты с RA и здоровые)}
\end{figure}

\subsection{ Анализ обогащённости }

Произведена функциональная аннотация найденных дифференциально экспрессирующихся генов с помощью инструмента \href{https://david.ncifcrf.gov/}{DAVID}.

Для этого полученные списки дифференциально экспрессирующихся генов были выгружены в формате \texttt{AFFYMETRIX\_\-3PRIME\_\-IVT\_\-ID}
и подверглись анализу функциональной обогащённости путями KEGG и Reactome (фоновый набор генов -- Homo Sapiens).

Были найдены два метаболических пути из базы Reactome, значимо обогащённые генами, дифференциально экспрессирующимися при стимулировании белком INF-$\gamma$:
\begin{enumerate}
    \item \href{http://www.reactome.org/content/detail/R-HSA-877300}{R-HSA-877300} (interferon gamma signaling), $p-adjusted \approx 1.5\cdot 10^{-5}$
    \item \href{http://www.reactome.org/content/detail/R-HSA-909733}{R-HSA-909733} (interferon alpha/beta signaling), $p-adjusted \approx 1.5\cdot 10^{-5}$
\end{enumerate}

В базе данных KEGG также были найдены несколько метаболических путей, но они, по-видимому, не имеют прямого отношения к экспериментальному воздействию.
Среди прочих найдена сверхпредставленность генов из пути "TNF signaling pathway"
($p-adjusted \approx 0.05$).

Для пациентов с волчанкой в базе данных Reactome найдены те же два пути R-HSA-877300 и R-HSA-909733   $p-adjusted \approx 4.2\cdot 10^{-5}$ и $1\cdot 10^{-12}$ соответственно).

В клетках, взятых у пациентов с артритом, не нашлось значимо обогащённых путей из базы Reactome, но в базе данных KEGG показана свехпредставленность генов из пути "TNF signaling pathway".
Впрочем, значимость этого наблюдения невысока ($p-adjusted \approx 0.1$).

\section{Выводы}

После фильтрации данных были построены тепловые карты, на которых отчётливо заметно  различие в экспрессии некоторых генов между пациентами с артритом и здоровыми людьми; между пациентами с волчанкой и здоровыми людьми; между группами клеток, обработанных разными веществами.
Такое же чёткое разделение было получено и авторами статьи.

Функциональная аннотация дифференциально экспрессирующихся генов выявила сверхпредставленность генов из путей, регулирующих синтез соответствующих цитокинов (как и в статье); в большей степени это заметно в случае волчанки, чем в случае ревматоидного артрита.

\begin{thebibliography}{99}
    \bibitem{smiljanovich2012} Smiljanovic, B., Grun, J.R., Biesen, R. et al. The multifaceted balance of TNF-$\alpha$ and type I/II interferon responses in SLE and RA: how monocytes manage the impact of cytokines. J Mol Med 90, 1295-1309 (2012). doi: \href{https://doi.org/10.1007/s00109-012-0907-y}{10.1007/s00109-012-0907-y}
\end{thebibliography}

\end{document}
