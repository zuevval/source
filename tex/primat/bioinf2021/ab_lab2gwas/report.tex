% compile with XeLaTeX or LuaLaTeX

\documentclass[a4paper,12pt]{article}
\usepackage[utf8]{inputenc} % russian, do not change
\usepackage[T2A, T1]{fontenc} % russian, do not change
\usepackage[english, russian]{babel} % russian, do not change

% fonts
\usepackage{fontspec} % different fonts
\setmainfont{Times New Roman}
\usepackage{setspace,amsmath}
\usepackage{amssymb} %common math symbols
\usepackage{dsfont}

% utilities
\usepackage{systeme} % systems of equations
\usepackage{mathtools} % xRightarrow, xrightleftharpoons, etc
\usepackage{array} % utils for tables
\usepackage{makecell} % multirow for tables
\usepackage{subfiles}
\usepackage{hyperref}
\hypersetup{pdfstartview=FitH,  linkcolor=blue, urlcolor=blue, colorlinks=true}
\usepackage{framed} % advanced frames, boxes
\usepackage{graphicx}
\usepackage{caption}
\usepackage{subcaption} % captions for subfigures
\usepackage{color}
\usepackage{listings}
\lstset{
    language=bash,
    basicstyle=\ttfamily,
    showstringspaces=false,
    commentstyle=\color{red},
    keywordstyle=\color{blue},
    breaklines=true,
    postbreak=\mbox{\textcolor{red}{$\hookrightarrow$}\space},
    columns=fullflexible
}

% styling
\usepackage{float} % force pictures position
\floatstyle{plaintop} % force caption on top
\usepackage{enumitem} % itemize and enumerate with [noitemsep]
\setlength{\parindent}{0pt} % no indents!
\usepackage[left=25mm, top=20mm, right=25mm, bottom=20mm, nohead, footskip=10mm]{geometry}

% misc
\graphicspath{{./img/}}
\newcommand{\myPictWidth}{.7\textwidth}
\newcommand{\phm}{\phantom{-}}

\begin{document}

\subfile{titlepage}

\section{Постановка задачи}

Даны две таблицы: файл \texttt{Transferrin\_males\_chr1.vcf} содержит информацию о найденных однонуклеотидных полиморфизмах в лабораторных пробах ДНК различных организмов; файл \texttt{Tr\_males.pheno} -- уровень экспрессии трансферрина в тех же организмов в формате <<ID семьи, ID таксона, уровень экспрессии>>.

Необходимо провести анализ ассоциации, выявив полиморфизмы, наличие которых коррелирует с уровнем экспрессии трансферрина.

\section{Ход работы}

\subsection{Фильтрация полиморфизмов}

С помощью утилиты \href{https://zzz.bwh.harvard.edu/plink/download.shtml}{PLINK} выбраны только полиморфизмы с частотой минорного аллеля $ >3\% $ и более $90\%$ всех значений в строке таблицы присутствуют:

\begin{lstlisting}
./plink --vcf ../Transferrin_males_chr1.vcf --maf 0.03 --geno 0.1 --make-bed --out tr_males_chr1_filtered
\end{lstlisting}

Вывод программы:

\begin{verbatim}
<...>
21078 variants loaded from .bim file.
1589 people (0 males, 0 females, 1589 ambiguous) loaded from .fam.
<...>
Total genotyping rate is 0.99941.
0 variants removed due to missing genotype data (--geno).
66 variants removed due to minor allele threshold(s)
(--maf/--max-maf/--mac/--max-mac).
21012 variants and 1589 people pass filters and QC.
\end{verbatim}

Отсеяны 66 SNP (менее $1\%$), т. е. данные качественные.
После фильтрации на выходе получается набор из .bed, .bim и .bam-файлов, которые мы снова преобразуем в .vcf:

\begin{lstlisting}
./plink --bfile tr_males_chr1_filtered --recode vcf --out tr_males_chr1_filtered
\end{lstlisting}

\subsection{Поиск генов, значимо ассоциированных с фенотипом}

Проведём полногеномный анализ ассоциаций с помощью TASSEL \cite{tassel}.

\subsubsection{Предобработка файла фенотипов}

Составим программу на языке R, которая переводит файл фенотипа в формат, соответствующий vcf-файлу (объединим ID семьи и ID вида, добавим необходимые для TASSEL метаданные):

\begin{lstlisting}[language=R]
library(dplyr)
pheno <- read.csv("Tr_males.pheno", header = F, sep = "\t")
colnames(pheno) <- c("familyid", "taxid", "ph")
out_filename <- "tr_males_tassel.pheno"
cat( "<phenotype>\ntaxa\tdata\n", file = out_filename)
pheno %>%
   transmute(taxid = paste0(familyid, "_", taxid), ph = ph) %>%
   write.table("tr_males_tassel.pheno", sep = "\t", quote = F, row.names = F, append = T)
\end{lstlisting}

\subsubsection{Работа в TASSEL}

Построим смешанную линейную модель, используя матрицу сходства и данные в пространстве пяти первых главных компонент.

\begin{enumerate}[noitemsep]
    \item Загрузим данные в рабочее пространство TASSEL GUI: File $\rightarrow$ Open as $\rightarrow$ ok $\rightarrow$ выбираем файлы .pheno и .vcf
    \item По vcf-таблице построим матрицу сходства (Analysis $\rightarrow$ Relatedness $\rightarrow$ Kinship)
    \item Вычислим главные компоненты той же таблицы (Analysis $\rightarrow$ Relatedness $\rightarrow$ PCA)
    \item Построим пересечение результата PCA с данными о фенотипе и генотипе (Data $\rightarrow$ IntersectJoin)
    \item Построим линейную модель, используя матрицу сходства и таблицу, полученную на предыдущем шаге (Analysis $\rightarrow$ Association $\rightarrow$ MLM).
\end{enumerate}

\subsubsection{Анализ зависимости между полиморфизмами и фенотипом}

В результате подбора линейной модели, описывающей связь генотипа с фенотипом, были получены $p-values$ гипотезы о сцепленности каждого полиморфизма с проявлением признака (экспрессия трансферрина).

Проведём поправку на множественное тестирование методом Бенджамини-Хохберга:

\begin{lstlisting}[language=R]
library(dplyr)
mlm_res <- read.csv("mlm_results.txt", sep = "\t") %>%
   na.omit %>%
   select(p, Marker)
mlm_res$padj <- p.adjust(mlm_res$p, method = "BH")
xtable::xtable(head(mlm_res %>% arrange(padj), 3),
               display = list("d", "e", "s", "f"))
\end{lstlisting}

Результат -- список SNP с самым низким значением скорректированных $p-value$.
\begin{table}[H]
    \centering
    \caption{Полиморфизмы с наиболее низкими значениями $p_{adj}$}
    \begin{tabular}{rrlr}
        \hline
        & p & Marker & padj \\
        \hline
        1 & 1.17e-05 & rs12079041 & 0.25 \\
        2 & 3.72e-02 & rs3737728 & 0.97 \\
        3 & 2.59e-02 & rs3736330 & 0.97 \\
        \hline
    \end{tabular}
    \label{tab:snp}
\end{table}

Необходимо отобрать те SNP, у которых скорректированное значение $p-value$ меньше порогового: $ p_{adj} < 0.05 $.
Таких полиморфизмов не оказалось, как видно из табл. \ref{tab:snp}.

\begin{thebibliography}{99}
    \bibitem{tassel} Bradbury PJ, Zhang Z, Kroon DE, Casstevens TM, Ramdoss Y, Buckler ES. (2007) TASSEL: Software for association mapping of complex traits in diverse samples. Bioinformatics 23:2633-2635. doi: \href{https://doi.org/10.1093/bioinformatics/btm308}{10.1093/bioinformatics/btm308}
\end{thebibliography}

\end{document}