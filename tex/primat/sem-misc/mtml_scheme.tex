% !TeX spellcheck = en_US
% !TeX program = xelatex

\documentclass[a4paper,12pt]{article}
\usepackage[utf8]{inputenc} % russian, do not change
\usepackage[T2A, T1]{fontenc} % russian, do not change
\usepackage[english, russian]{babel} % russian, do not change

% fonts
\usepackage{fontspec} % different fonts
\setmainfont{Times New Roman}
\usepackage{setspace,amsmath}
\usepackage{amssymb} %common math symbols
\usepackage{dsfont}

% utilities
\usepackage{systeme} % systems of equations
\usepackage{mathtools} % xRightarrow, xrightleftharpoons, etc
\usepackage{array} % utils for tables
\usepackage{makecell} % multirow for tables
\usepackage{subfiles}
\usepackage{hyperref}
\hypersetup{pdfstartview=FitH,  linkcolor=blue, urlcolor=blue, colorlinks=true}
\usepackage{framed} % advanced frames, boxes
\usepackage{graphicx}
\usepackage{caption}
\usepackage{subcaption} % captions for subfigures
\usepackage{color}
\usepackage{chngcntr} % change counters
%\counterwithout*{section}{chapter} % continue sections enumeration with chngcntr
\usepackage{theorem}


% styling
\usepackage{float} % force pictures position
\floatstyle{plaintop} % force caption on top
\usepackage{enumitem} % itemize and enumerate with [noitemsep]
\setlength{\parindent}{0pt} % no indents!

% misc
\graphicspath{{./img/}}
\newcommand{\myPictWidth}{.95\textwidth}
\newcommand{\phm}{\phantom{-}}

\begin{document}
	
\begin{gather*}
	\eta = B \eta + \Pi g + \varepsilon \\
	p = \Lambda \eta + K y + \delta
\end{gather*}

\begin{gather*}
\begin{pmatrix}
	\eta_1 \\
	\eta_2 \\
	\eta_3
\end{pmatrix} =
 \begin{pmatrix}
 	0 & 0 & 0 \\
 	b_{21} & 0 & 0 \\
 	b_{31} & b_{32} & 0
 \end{pmatrix}
\begin{pmatrix}
	\eta_1 \\
	\eta_2 \\
	\eta_3
\end{pmatrix} + 
\begin{pmatrix}
	\pi_{11} & \pi_{12} & \pi_{32} & 0 & 0 & 0 \\
	0 & 0 & 0 & \pi_{24} & 0 & 0 \\
	0 & 0 & 0 & 0 & \pi_{35} & \pi_{36}
\end{pmatrix}
\begin{pmatrix}
	g_1 \\ g_2 \\ g_3 \\ g_4 \\ g_5 \\ g_6
\end{pmatrix} +
\begin{pmatrix}
	e_{11} & e_{12} & e_{13} \\
	e_{21} & e_{22} & e_{23} \\
	e_{31} & e_{32} & e_{33}
\end{pmatrix} \\
\begin{pmatrix}
	p_1 \\ p_2 \\ p_3 \\ p_4 \\ p_5 \\ p_6 \\ p_7
\end{pmatrix} =
\begin{pmatrix}
	\lambda_{11} & 0 & 0 \\
	\lambda_{21} & 0 & 0 \\
	\lambda_{31} & \lambda_{32} & 0 \\
	0 & \lambda_{42} & 0 \\
	0 & \lambda_{52} & 0 \\
	0 & 0 & \lambda_{63} \\
	0 & 0 & \lambda_{73} \\
\end{pmatrix} 
\begin{pmatrix}
	\eta_1 \\
	\eta_2 \\
	\eta_3
\end{pmatrix} + 
\begin{pmatrix}
	k_{11} & k_{12} & 0 & 0 & 0 & 0 \\
	0 & 0 & k_{23} & 0 & 0 & 0 \\
	0 & 0 & 0 & 0 & 0 & 0 \\
	0 & 0 & 0 & 0 & 0 & 0 \\
	0 & 0 & 0 & k_{54} & 0 & 0 \\
	0 & 0 & 0 & 0 & k_{65} & 0 \\
	0 & 0 & 0 & 0 & 0 & k_{76} \\
\end{pmatrix}
\begin{pmatrix}
	y_1 \\ y_2 \\ y_3 \\ y_4 \\ y_5 \\ y_6
\end{pmatrix} + \delta
\end{gather*}

Не может ли какой-то SNP влиять на $p_i$ как посредством латентных переменных, так и напрямую? \\

LISREL:

\begin{verbatim}
	# structural part
	eta1 ~~ g1 + g2 + g3
	eta2 ~~ eta1 + g4
	eta3 ~~ eta1 + eta2 + g5 + g6
	
	#  measurement part (not sure about it)
	eta1 =~ p1 + p2 + p3
	eta2 =~ p3 + p4 + p5
	eta3 =~ p6 + p7
	y1 =~ p1
	y2 =~ p2
	y3 =~ p2
	y4 =~ p5
	y5 =~ p6
	y6 =~ p7
\end{verbatim}

	
\end{document}
