\documentclass[main.tex]{subfiles}
\begin{document}

\section{Лекция 9. Часть 1. Гены и болезни}

ДНК каждого человека или другого огранизма у каждого своя (вариабельность у человека -- примерно 1 SNP на 100 баз).
SNP во многих исследованиях используются как маркёры районов генома.
Но надо учитывать, что большинство SNP не проявляются в фенотипах

Впрочем, могут быть SNP, вызывающие функциональные изменения (изменять структуру белка -- если расположены в кодирующей области, или влиять на стабильность мРНК).

Частота SNP приводится в минорной аллели (то есть она не больше 50\%).

\subsubsection{SNP в ДНК человека}

Какие существуют вариации?
SNP -- самые распространённые структурные вариации, но есть и более крупные (инсерции, делеции, ...)

SNP, как правило, не приводят к функциональным изменениям, но некоторые из них -- \emph{rare genetic variants} (редкие генетические варианты) -- приводят к болезням (например, один из SNP приводит к кистозному фиброзу).

Болезней, вызванных только одной мутацией (т. н. \emph{менделевские} болезни), известно не более 1000.
Их проявление -- редкость (вымывается отбором).
Диабет и т. д. -- \emph{полигенные} болезни.

\subsubsection{Стратегии поисков генов предрасположенности}

\emph{Гены предрасположенности} -- то, что повышает вероятность болезни.

Они ищутся двумя путями: % TODO a bit

В исследованиях интересны однояйцевый близнецы (Linkage studies):
допустим, оба ребёнка больны.
Генотипируем полиморфные маркеры (считываем ДНК и находим все SNP) и смотрим, как ассоциирован маркер с болезнью (есть ли корреляция в разных семьях между наличием болезни и наличием SNP). % TODO ask: зачем именно двойняшки?

\subsubsection{Статистические тесты на ассоциацию}

Анализ проводится с использованием тр

Нулевая гипотеза: ассоциации нет.
Используем критерий хи-квадрат, чтобы проверить истинность гипотезы.

Пример: королева Виктория и её потомки -- гемофилия.
Королева Виктория была гетерозиготна по гемофилии.

Другой пример исследования -- целиакия (непереносимость глютена).

\subsubsection{Гипотеза common disease -- common variant}

Тезис: на общие широко распространённые (неменделевские) заболевания, вероятно, влияют генетические полиморфизмы, которые также часто встречаются в популяции.

Возможность выявить гены предрасположенности зависит от частоты встречаемости гена и от

\subsubsection{ Неравновесное сцепление (Linkage disequilibrium, LD) }
Чем больше в каждом следующем поколении произойдёт рекомбинаций, тем более хромосома будет мозаичной (перемешаны куски от разных предков).
Но есть некий предел разбиения, который не может быть исправлен рекомбинацией (неравновесное сцепление): некоторые SNP наследуются обязательно вместе.
LD зависит от многих факторов: места на хромосоме, размера популяции и так далее.

Как оценить LD?
Берём два соседних SNP и считаем между ними корреляцию.

Геном человека - 3 гигабазы.
Секвенировать с хорошим покрытием дорого (порядка 120 000 на один геном).
Придумали методы \emph{SNP-типирования}: есть некая карта генетических полиморфизмов (знаем, где они находятся).
У нас есть кусочки, содержащие полиморфизмы; гибридизуем ДНК с известными полиморфизмами и смотрим, нашлись ли соответствующие кусочки.

Есть разные базы SNP.

\subsection{Изучение вариабельности генома в биомедицинском контексте (GWAS)}

GWAS (Genome-Wide Association Study): с помощью этого метода мы находим лишь ассоциированные с болезнью участки, но не сами казуальные (вызывающие болезнь) мутации, потому что найденные SNP могут находиться в неравновесном сцеплении.

GWAS требует больших объёмов выборок и

С помощью GWAS в нулевые годы искали гены аутоиммунных заболеваний (например, диабет, ревматоидный артрит).

\subsubsection{Обобщённая линейная модель (GLM)}

$$ y = g(\sum \alpha_i x_i) + e $$

Чтобы понять, связан ли SNP с болезнью, используем тест отношения правдоподобия: сравниваются вероятности двух моделей (нулевая -- нигде нет ассоциации, альтернативная -- в качестве эффекта последовательно проверяется первый SNP, второй...) % TODO a bit

\subsubsection{Анализ структуры популяции}

Для предотвращения стратификации (помните, в какой-то лекции говорили про неё в контексте разделения образцов на дорожки), % TODO a little
STRUCTURE интегрирован во многие программы, которые делают GWAS.

\subsubsection{Влияние ОНП на экспрессию генов}

% TODO

\subsubsection{Какие результаты удалось получить?}

Думали, что с помощью  GWAS удастся найти все гены предрасположенности, но в итоге GWAS выявил гены, которые в среднем увеличивают риск заболевания лишь на $ 10-30 \% $.

\begin{leftbar}
    Полигенные болезни -- те, которые обусловлены изменениями во множестве генов.
    Многофакторные болезни -- те, у которых много факторов (может быть и не генетические).
\end{leftbar}

\subsubsection{ Почему GWAS объясняет только незначительный процент риска, ассоциированного с болезнью? }

\section{Лекция 9. Часть 2. Методы обнаружения структурных вариантов}

SNP -- не единственный источник вариабельности.
В ДНК много инсерций, делеций и дупликаций.

Говорят о больших структурных вариантах (Large SV), когда размер от 1 КБ до 5 МБ.
Примерно $ 12 \% $

Есть методы поиска, основанные на paired-end mapping, на глубине прочтения, сборке генома...
(глубина прочтения -- сколько раз прочли каждый участок).
Эффективнее комбинация этих подходов.

\subsubsection{ Paired-End mapping approach }

Если нет инсерции/делеции, при парном прочтении все соответствующие кусочки будут картироваться на  участки, расположенные на равном расстоянии от центра.
Если же есть структурная вариация,
Притом SV должна быть не длиннее, чем кусочек, который мы картируем на геном.

% TODO про splitRead. Не очень понял. И почему SV должна быть не длиннее ... ?

Расстояние между splitRead'ами указывает на то, какой величины произошла инсерция.

Может быть в одном районе не одна инсерция, а несколько.
Тогда split read разносится не на два участка, а на три.

\subsubsection{Read-depth-based approach}

Допустим, произошла дупликация.
Тогда покрытие генома (глубина прочтения) в этом месте будет вдвое больше.

Можно пройти по геному и попробовать выделить районы, где аномально высокая глубина прочтения.

Четырёхступенчатая процедура:

\begin{enumerate}[noitemsep]
    \item Картирование прочтений на геном
    \item Нормализация
    \item Определение покрытия (проходим каким-то окном)
    \item Сегментация генома на участки
\end{enumerate}

Нормализация нужна, т. к. ПЦР вносит погрешность (разные участки генома могут по-разному амплифицироваться).
Нормализацию обычно проводят по GC-контенту.

\subsubsection{Assembly-based approach}

Можно собирать геном индивидуума de novo и потом картировать на референс.
Так найдём инсерции, делеции и другие вариации.

\section{Лекция 9. Часть 3. Геномная селекция (файл lecture9.pdf)}

В GWAS каждый SNP маркируется отдельно.
Метод геномной селекции (GS) предназначен для растений.
Предположения:
% TODO

Предсказание количественных признаков должно базироваться на учёте большого числа маркеров / генов.
Как и у человека, присутствуют

В основе геномной селекции метод машинного обучения.
(хорошо работает на коровах, даёт большой выигрыш в стоимости и времени).

\begin{enumerate}[noitemsep]
    \item Формируем обучающую выборку
    \item Все растения обучающей выборки должны быть фенотипированы и генотипированы
    \item Восстанавливаем неявную зависимость между маркерами и фенотипами путём построения модели $ \Rightarrow $ можем предсказать новые фенотипы.
\end{enumerate}

\subsubsection{ Частные случаи модели GS }

Какие модели МО используются для геномной селекции?

\begin{enumerate}[noitemsep]
    \item $ y_i = \mu + \sum_{j=1}^{p} x_ij \beta_j $
    \item неявные модели (нейронные сети, SVM...)
\end{enumerate}

\subsubsection{Методы GS, использующие штрафы}

Можно использовать регуляризацию.
Можно использовать байесовские модели.

В байесовских моделях важно априорное распределение эффектов маркеров.

\subsubsection{ Процесс обучения }

Используем кросс-валидацию: обучаем модель несколько раз, причём для тестирования каждый раз берём другую часть выборки.

\subsubsection{}

\begin{enumerate}[noitemsep]
    \item
    \item
\end{enumerate}

\end{document}
