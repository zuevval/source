\documentclass[main.tex]{subfiles}
\begin{document}

\section{ Выравнивание прочтений на референсный геном и поиск вариабельных позиций}

Алёна Борисовна

\begin{enumerate}[noitemsep]
	\item
	\item Результат секвенирования: формат fastq, референсный геном -- в формате fasta.
	$$ \begin{cases}
	fastq \\
	fasta
	\end{cases} \to \text{sam, bam -- выравнивание} $$  (bam = бинарный вид sam)
	\item Поиск вариабельных позиций (SNP, делеции, дупликации, инсерции и т. д). $ \Rightarrow $ .vcf
	
	Можно его отфильтровать и далее работать со снипами (нас интересуют именно они).
	
	Формат vcf: столбцы pos (позиция), id (идентификатор), <Name1> <Name2> ... (названия образцов, которые мы исследуем) и в каждом из них числа: либо 0/0 (замены нет, референсная гомозигота), либо 0/1 (гетерозигота, замена в одном аллеле) или 1/1 (альтернативная гомозигота).
	Могут быть другие обозначения: 0|0 вместо 0|1 (это значит, что файл \emph{фазированный}: известно, в каком именно аллеле произошла мутация).
	
	В нашем задании только один образец.
	Тогда vcf-файл принимает вид: вместо <Name1> <Name2> ... -- только <Name> 
	
	\item Анализ: можно подсчитать число снипов на каждой хромосоме (построить гистограмму распрделения количества снипов по хромосомам).
	
	Мы будем анализировать так: берём аннотацию \textbf{генома} (.gff3), смотрим, попадают снипы в какой-то ген или нет.
	
	Если найдём гены, обогащённые снипами, можно посмотреть в базах, что эти гены делают.
	
	
	\item vcf можно отфильтровать по пороговому значению качества...
	
\end{enumerate}

Задание: написать небольшой код, который строит пересечение vcf с аннотацией в gff3.
\begin{leftbar}
	Ещё есть программа SNPEFF (в задании это не нужно использовать), которая смотрит не только на то, попадает ли снип в ген, но и попадает ли снип в область рядом с геном (upstream / downstream регион).
\end{leftbar}

Данные: \textit{E. Coli}, даны ссылки в задании.

\begin{enumerate}[noitemsep]
	\item SRA toolkit: программа, чтобы скачать fastq
	\item Команды для SRA toolkit: \texttt{fastq-dump --split-files}
	\item скачать референсный геном
	\item Выравнивание (Bowtie2). Можно BLAST.
	\begin{itemize}[noitemsep]
		\item Построить индекс референсного генома (см. самим, как это делается)
		\item \subitem Собственно выравнивание: нужно помнить, одиночные прочтения или парные.
		На выходе -- формат sam.
		Там будут последовательности, которые выровнялись, и невыровненые.
	\end{itemize}
	\item С помощью программы SamTools: убрать из .sam невыровненные прочтения.
	На выходе тоже .sam или .bam (хорошо бы перевести в .bam, чтобы файл весил меньше).
	
	Делается в одну команду.
	Нужно только указать параметры фильтрации, и нужно указать индекс...
	
	\item Сортируем с помощью SamTools
	\item Поиск вариабельных позиций (тоже можно с помощью SamTools одной командой).
	Или 
	На выходе .vcf
	\item Интерпретация: распределение SNP по хромосомам; аннотация снипов относительно генов.
\end{enumerate}

Отчёт: написать, как это делали, все промежуточные команды указать.
Чтобы посторонний человек взял отчёт и мог повторить.
В качестве выводов: написать про аннотацию и распределение по хромосомам;
может быть, гистограмму распределения снипов относительно генов или таблицу.
Также можно посмотреть на функции генов.
В общем, пофантазировать.

Фиксированного дедлайна нет.

\end{document}
