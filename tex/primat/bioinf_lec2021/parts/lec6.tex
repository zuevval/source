\documentclass[main.tex]{subfiles}
\begin{document}

\section{Секвенирование }
RNASeq.pdf

Подготовка библиотеки: РНК разрезают, пришивают адаптеры, потом секвенируют.
Особенность: библиотеки готовятся по-разному в зависимости от того, какую РНК секвенируем.

Матричную ДНК $\Rightarrow$ у неё poly-A концы, и их

Есть более древняя технология -- ДНК-чипы.
EST -- тоже древняя технология, её применяли, когда не умели делать microarrays. Выделяли РНК из клеток, переводили в ДНК и секвенировали ДНК по методу Сэнгера.

Динамический диапазон -- какое количество материала можем секвенировать за один раз.
У RNA-Seq он на порядок выше.

Напомним, сплайсинг генов может происходить по-разному.
Разные сплайсы называют \emph{изоформами}.
RNA-Seq позволяет их видеть. % TODO ask: how is it?

Стоимость: ДНК-чипы гораздо более дешёвые по сравнению с microarrays.

\subsubsection{RNASeq vs MA (microarrays)}

% TODO

\subsubsection{Причины перехода от MA к RNASeq}

Для MA нужен референсный геном.
С помощью RNASeq можно просто собрать транскриптом (так же, как мы собираем геном).

Методами РНК-секвенирования можно обнаружить новые транскриб, ..., РНК-редактирование (когда ) % TODO

\subsubsection{Трудности использования RNA-Seq:}

Главное -- высокая стоимость

\subsubsection{Картирование прочтений на геном}

Можно собирать транскриптом DeNovo % TODO ask: транскриптом -- по мРНК длинный, как геном?  или по всем РНК -- отдельные молекулы РНК?

\subsubsection{Картирование сплайсированных прочтений}

Обычно референсный геном дополняют библиотекой экзонных границ

\subsubsection{Результат картирования: SAM / BAM}

Показано, сколько совпадений, вставок и несовпадений (match, insert, mismatch).

\subsubsection{Суммирование данных}
% TODO

\subsubsection{Нормализация данных}

В разных библиотеках отдельные гены могут по-разному секвенироваться.

Можно нормализовать по-разному:
\begin{enumerate}[noitemsep]
 \item Нормализация на размер библиотеки:
 \begin{enumerate}[noitemsep]
     \item скалирующий фактор -- число прочтений в образце, делённое на среднее число прочтений в образце: $ \hat s_j = \frac{N_j}{\frac{1}{n} \sum_l N_l} $, $y_{ij} \to \frac{y_{ij}}{\hat s_j}$
     \item % TODO
 \end{enumerate}

 \item Нормализация по типу распределения
 \begin{enumerate}[noitemsep]
     \item Квантильная нормализация (рассказывали в теме <<ДНК-чипы>>)
     \item Медианная нормализация
 \end{enumerate}
    \item Нормализация по эффективному размеру библиотеки

    Один ген в библиотеке может амплифицироваться.


\end{enumerate}

Два основных пакета, которые используются для нормализации: edgeR / DESeq2

\subsubsection{Trimmed Mean} % TODO title

Реализовано в edgeR.

edgeR: идея -- оценить различие между образцами с использованием неэкстремальных значений генов.

На рисунке -- график MA (показывает, как среднее зависит от дисперсии: абсцисса -- сумма логарифмов, ордината -- разность).

\subsubsection{...} % TODO

Этот метод реализован в библиотеке DESeq2.

\subsubsection{Сравнение методов}

Видно, что наиболее прилично ведут себя методы, реализованные в edgeR и DESeq2, также неплохой результат у медианной нормализации и нормализации по третьему квартилю.
К тому же, у результатов DESeq2 и edgeR примерно одинаковое число дифференциально экспрессирующихся генов.


\subsubsection{Оценка дифференциальной экспрессии}
% TODO many

Чтобы оценить дифференциальную экспрессию, делаем репликаты.
Процедура дорогая; как правило, принято делать три репликата.
По трём репликатам сложно понять, какова дисперсия.

edgeR: дисперсия между репликатами (геноспецифичная дисперсия) обычно оценивается методом максимального правдоподобия по сложным формулам.

DESeq2: объединяют гены с похожей экспрессией и считают суммар

Дифференциальная экспрессия: оцениваем по t-критерию Стьюдента или с помощью теста Фишера.

Строим четырёхпольные таблицы...

Пуассоновская модель нам не подходит (есть сверхдисперсия).
Нужно делать коррекцию, чтобы не получить много False Positive.
Если есть репликаты и пуассоновская модель

\subsection{Дизайн эксперимента для RNASeq}
\subsubsection{Репликация, рандомизация и стратификация данных}

Планирование эксперимента направлено на уменьшение дисперсии оцениваемых параметров.

Фишер показал, что нужно для получения качественных результатов соблюсти три условия:

\begin{enumerate}[noitemsep]
    \item репликация (пациенты взяты из разных стационаров, в одинаковой пропорции с разных отделений и так далее )
    \item рандомизация
    \item стратификация: например, в Illumina важно решить, как распределяют библиотеки ДНК по репликатам.

    Плохой дизайн эксперимента: каждый из репликатов в отдельной дорожке.
    Тогда репликаты сравнивать нельзя (будут разные погрешности).

\end{enumerate}

\subsubsection{Batch effect / lane effect}

Эффект партии (batch effect): ошибки, вызванные приготовлением библиотеки (неодинаковая в разных экспериментах).

Эффект дорожки (lane effect):
ошибки, происходящие от % TODO

Борьба с lane effect: можно пришить каждой библиотеке (каждому репликату) соответствующий баркод (дополнительную последовательность).
Это и есть стратификация.
Смешиваем в пробирке и помещаем одинаковое количество ДНК каждого образца в каждую дорожку.

\subsubsection{Сбалансированная неполная стратификация}

Если число уникальных последовательностей (баркодов) меньше числа пришедших образцов, полная стратификация невозможна.

% TODO ask: почему лимитирующий фактор -- число дорожек, а не число штрих-кодов?

$ T = \frac{sL}{JI} $, $I$ -- число условий

\subsubsection{}

\begin{enumerate}[noitemsep]
    \item
    \item
\end{enumerate}

\end{document}