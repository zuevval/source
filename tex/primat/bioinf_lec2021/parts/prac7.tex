\documentclass[main.tex]{subfiles}
\begin{document}

\section{Лабораторная 2 от АБ. Полногеномный поиск ассоциаций}
11 марта 2021 г.

Genome-wide association study.

Анализ однонуклеотидных полиморфизмов.
В прошлой работе финальный результат был -- vcf-файл.
В нём находятся снипы: 0/0 (референсная гомозигота), 0/1, 1/1

В прошлой ЛР искали связи этих значений с тем, какие там гены.
В полногеномном анализе ассоциаций -- соответствие фенотипическим признакам.

Будет несколько вариантов задания: несколько фенотипических признаков (например, рост) и несколько датасетов, где в каждом надо взять только 1 хромосомы.

Нужно сделать предобработку (разную в зависимости от программы).
Либо vcf, либо $vcf \to bed$.
Формат bed: тот же vcf, но записанный наоборот; бинарный вариант формата ped.
По сравнению с vcf появляются дополнительные столбцы: family\_id (например, если есть семья из 4 человек, у них общий family\_id), individ\_id.
Далее идут столбцы SNP1,SNP2, ..., где стоят значения нуклеотидов: если SNP1: A$\to$T, то 0/0 соответствует AA, 0/1 -- AT, 1/1 -- TT

ped-файл должен сопровождаться map-файлом (там для каждого SNP\_i указано, какая у этого снипа позиция в геноме).
Бинарный вариант map-файла -- bim.

Перевод .vcf в .bed: можно программой \textbf{PLINK}: \texttt{plink --vcf <...>.vcf --make-bed}

Вопрос: все ли снипы нам подходят?
Если у нас 400 образцов и только для двух известно, есть SNP или нет, мы такой не берём.
Стандартно отсеивают те SNP, о которых известно наличие/отсутствие в 90-95\% организмов (call\_rate > 90-95\%).
Фильтрация снипов по отсутствующим значениям: либо в .bed с помощью того же PLINK, либо ещё до этого в .vcf (программа \textbf{vcftools}).

Далее -- фильтрация по минорному аллелю (минорный -- там, где 1, где альтернативный вариант).
Стандартно: частота минорного аллеля должна быть больше 3-5\% (maf -- minor allele frequency).

Команды для фильтрwации: PLINK -- \texttt{plink --bfile <...>.bed --maf 0.03 --gene 0.1}, vcftools -- \texttt{vcftools --vcf <...>.vcf --maf 0.03 --max-missing 0.9} \\

\subsubsection{GWAS-анализ}

Программа \textbf{TASSEL} (ан. кисточка) -- рекомендуется (простой вариант).
Есть CLI и GUI.
Нужно открыть файл со снипами (vcf) и файл с фенотипами.

Просьба использовать линейную модель -- MLM (K + PCA), K -- Kinship Matrix.

\emph{Матрица родства} (Kinship matrix) рассчитывается на основе матрицы снипов.
(некоторые снипы находятся близко друг от друга, между ними есть корреляция).

Главные компоненты добавляются, чтобы учитывать популяционную структуру.

\begin{leftbar}
	Выделить мышкой vcf.
	В меню сверху опция -- провести PCA.

	Далее -- kinship matrix

	После этого будет три файла, которые надо объединить (вкладка IntersectJoin) $ \Rightarrow $ ещё один файл.

	Далее надо выделить получившийся
\end{leftbar}
Всё это можно сделать в командной строке.

Появится дополнительный файл с результатами, который надо сохранить.
Там, среди прочего, будет столбец со снипами и столбец с фенотипами, а также результат -- ассоциация (значение p-value)
После этого надо сделать поправку на множественное тестирование и получить из  $ p\_values$ $ p\_adjusted $ (например, в R можно).
Методы -- Борферрони, Бенджамини, ...

К $ p-adjusted $ можно применять уже пороговую фильтрацию.
Увы, может ни одного снипа не остаться; тогда говорим, что ни один SNP значимо не ассоциирован с фенотипами.
Если же найдём какие-то SNP, посмотреть их в базе по идентификаторам (какие базы? Ищите в Google; может быть HDDGP)

\end{document}
