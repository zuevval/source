\documentclass[main.tex]{subfiles}
\begin{document}

\subsubsection{Эксперименты высокого выхода}

Главный принцип таких экспериментов -- \emph{параллелизация}, возможность

Мы расмотрим применение и, кроме того, подробнее поговорим о % TODO

Анализируем все метаболиты в клетке -- метаболомный анализ, гены -- геномный анализ, все белки в клетке -- протеомный анализ.

Процесс реализации записанной в ДНК информации -- экспрессия (= активность) гена. В узком смысле под экспрессией понимают преобразование информации из ДНК в белок. Основные механизмы, реализирующие экспрессию -- транскрипция и трансляция.

\subsubsection{Анализ транскриптома: ДНК-чипы}
ДНК-чипы -- более старая технология. Разработана в 80-х годах, пионеры -- Московский институт биоорганической химии (в Перестройку они уехали в США).

Существует подложка (некая мембрана), на которые наносятся множество одноцепочечных проб, соответствующих некоторому участку ДНК (больше нигде не встречаются).
Потом через мембрану с пробами пропускается РНК (тоже одноцепочечная), выбранная из ткани.
Надо детектировать образование связей (дуплексов).
Как правило, РНК метится флюорофорами и сканер считывает флюоресценцию.
Полученная при этом численная оценка, как считают, отражает содержание в образце РНК, комплементарной пробе.

\subsubsection{Платформы: чипы кДНК (cDNA arrays)}
Одна из первых технологий. Поверхность -- нейлоновая мембрана, пробы -- ПЦР-продукты клонов ДНК; мРНК помечена радиоактивным фосфором (это опасно). Такие чипы были большие.

\subsubsection{Платформы: двухцветные микрочипы (two color arrays)}

Пропускаемую через чип мРНК помечают флюорофорами; пробы -- ДНК, помещённая на поверхности стекла.
Преимущество: можно использовать один чип для двух проб (например, опытная и контрольная), но нужно метить разными флюорофорами (оттого и называется двухцветным).

Технология довольно шумная, поэтому хорошо, что можно анализировать две пробы на одном чипе (на разных чипах могли бы получить результаты, которые сравнивать некорректно).

\subsubsection{Платформа: олигонуклеотидные чипы (oligonucleotide chips)}

Наиболее известная технология -- Affymetrix Gene Chip

Олигонуклеотид -- пробы до  20 пар оснований. Готовим ДНК \textrightarrow готовим РНК

Одного короткого нуклеотида недостаточно (будет неспефичное связывание), поэтому для детекции одного гена

В каждой клетке чипа $ N $ <<олигов>>. Для каждого, как правило, присутствует пара -- тот же олиг, но центральный нуклеотид изменён (пробы полностью и частично идентичные: Perfect Match / Mismatch). Нужно для отсева неспецифичных гибридизаций.

Амплификация = усиление сигнала. % TODO
...

Изготовление чипов Affymetrix высоко стандартизовано, поэтому они широко используются в фарминдустрии. Например, есть какая-то болезнь; можно сделать для неё специфический чип.

Главное отличие от двухцветных чипах -- образцы не смешиваются, для эксперимента нужно два чипа.

\textbf{Выводы}:

\subsubsection{Сравнение платформ}

Чипы бывают полногеномные и специализированные (custom).
Последние используются для диагностики заболеваний.

Результаты измерений слабо коррелированы.
Это связано с тем, что даже в одном эксперименте измерения высоко чувствительны к условиям, которые привносят систематическую ошибку, могут варьировать на разных стадиях эксперимента. Этот феномен называется \emph{эффектом серии} (batch effect).

\subsubsection{Условия эксперимента, привносящие ошибку в технологию}

\begin{enumerate}[noitemsep]
    \item нанесение пробы (наносим специальной иголкой, не всегда одинаково)
    \item амплификация при ПЦР
    \item протокол подготовки образца \label{item:protocol}
    \item Покрытие на чипе (не всегда идеальное)
    \item Ошибки сканирования и обработки изображений \label{item:scan_err}
\end{enumerate}

Влияние факторов \ref{item:protocol} -- \ref{item:scan_err} % TODO

\subsubsection{Анализ изображений и контроль качества данных}

Используется зелёный и красный флюорофор.
При наложении получается жёлтый.
Процессинг изображений не представляет труда, делается автоматически.
Есть шаблоны; накладываем их так, чтобы центр шаблона совпадал с центром массы сигнала в ячейке. % TODO

После обнаружения делаем сегментацию изображения. Усредняем цвет пикселя по сегменту.

\subsubsection{Относительная экспрессия. Базовые понятия}

В ДНК-чипах мы считаем всегда лишь относительную экспрессию.

$$ T_k = \frac{R_k}{G_k} $$

% TODO

Вывод: результат гибридизации РНК/кДНК ... % TODO
Сканер генерирует изображение, которое обрабатывается программами с целью % TODO
Т. о. уровень экспрессии

\subsubsection{Условия эксперимента, привносящие ошибку}
\begin{enumerate}[noitemsep]
    \item нанесение пробы (наносим специальной иголкой, не всегда одинаково)
    \item амплификация при ПЦР
    \item протокол подготовки образца
    \item Покрытие на чипе (не всегда идеальное)
    \item Ошибки сканирования и обработки изображений
    \item неспецифическая гибридизация (кросс-гибридизация олигонуклеотидных проб)
    \item различная эффективность мечения образца разными флюоресцирующимися красителями
\end{enumerate}

Первые пять мы уже рассматривали ранее.

Что делать?

\subsubsection{Нормализация}
Есть <<гены домашнего хозяйства>> (housekeeping genes), которые есть в каждой клетке.

% TODO

\subsubsection{Линейные модели шумов}
Различают аддитивные и мультипликативные шумы.

$$ y_{ki} = $$ % TODO

Одна часть приносится условиями эксперимента, другая -- чипом

\subsubsection{Учёт мультипликативного шума}

В общем случае

$$ log Y_{kjm} = g_k + s_j + d_i + v_m + [gs]_{kj} + [gv]_{km} + \varepsilon_{kjm} $$

$ j $ -- проба, $ s $ -- слайд, ... % TODO

Можно переписать для олигонуклеотидных чипов: % TODO

Можно учитывать и аддитивный шум.

\subsubsection{Дисперсия}
Важно, что существует зависимость дисперсии от абсолютного значения интенсивности свечения (гетероскедастичность, heteroskedastisity). % TODO

\subsubsection{Asinh- и log2-трансформации}
Преобразовав данные с помощью гиперболического арксинуса, получим, что значения близки к диагонали. % TODO

Какие ещё есть алгоритмы нормализации?
\subsubsection{Алгоритмы нормализации} % TODO

\subsubsection{Алгоритм Loess}

Локальная гегрессия (кусочная): вы не слышали? <<Ну, вы посмотрите>>.

\subsubsection{Квантильная нормализация} % TODO
Если два распределения одинаковы, то все данные на графике <<квантиль-квантиль>> будут лежат на диагонали.

\begin{enumerate}[noitemsep]
    \item Для $n$ чипов с массивами данных длиной $ p $ формируем матрицу $ X $ размерности $ p \times n $ % TODO
    \item
\end{enumerate}

\end{document}
