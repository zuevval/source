\documentclass[main.tex]{subfiles}
\begin{document}
    \section{Картирование коротких последовательностей на референсный геном}

    Современные секвенаторы дают множество прочтений, и все они короткие.
    Нужно научиться смотреть, откуда пришёл фрагмент (например, искать инсерции, делеции, инверсии...)
    Поэтому картирование -- необходимый этап.
    Мы на уроках не изучаем, т. к. это требует больших объёмов данных.

    Напомним, есть программа BLAST.
    Картирование прочтений на геном делают похожим образом.
    Правда, допускается лишь малое число несовпадений (1-2 нуклеотида), поэтому сделали отдельный класс программ -- картировщики прочтений.

    Причины несовпадений:
    \begin{enumerate}[noitemsep]
        \item Структурная вариабельность (SNP или небольшие инделы).

        Инсерции научились предсказывать программно, делеции -- пока не очень.
        \item Ошибки секвенирования
    \end{enumerate}

    Чаще всего секвенируют по технологии Illumina/Solexa (примерно 100 несовпадений максимум).
    Точность секвенирования убывает к концу рида.

    % TODO про картировщики, трейдофф скорость VS чувствительность

    Один и тот же участок генома читается несколько раз.
    Образуются <<стопки>> фрагментов, среди которых надо выбрать наилучшим образом совпадающие.

    Для повышения качества картирования помимо самих последовательностей полезно знать дополнительную информацию.
    В секвенаторах для каждого прочтения известна оценка качества.
    Хорошие секвенаторы позволяют заменять нуклеотиды, прочтённые с плохим качеством.

    \subsubsection{Использование парных прочтений}

    Обычно мы знаем, сколько секвенируем с каждого конца.
    Известна общая длина прочтения, поэтому, хотя мы не знаем,  % TODO

    Геномы содержат огромное количество повторов (мест, обогащённых, например,  A-T).
    \subsubsection{TODO}
    % TODO индексация

    Индексировать недостаточно.
    % TODO

    Как допустить небольшое число несовпадений?
    Два способа:

    \begin{enumerate}[noitempsep]
        \item Делают репликацию хэш-таблиц
        \item Одновременный поиск нескольких k-меров (Ukkonen, 1991)
    \end{enumerate}

    \subsubsection{Преимущества и недостатки алгоритмов, использующих хэш-таблицы}

    \begin{enumerate}[noitemsep]
        \item
    \end{enumerate}

    Недостатки:

    \begin{enumerate}[noitemsep]
        \item
    \end{enumerate}

    \subsubsection{Суффиксные массивы}

    Вы это не знаете?
    Чему же вас учат на кафедре прикладной математики?

    Альтернатива поиску с помощью хэш-таблиц.
    \emph{Суффиксный массив} $A$ подпоследовательности $S$ -- массив целочисленных значений, соответствующий всем возможным окончаниям $S$, идущим в лексикографическом порядке.

    Пример: рассмотрим последовательность banana\$ (\$, как будем считать, лексикографически меньше всех остальных )

    Недостаток: для работы с суффиксными массивами нужно $O(n \log \sigma)$

    % TODO подумать: почему при алфавите длины 4 суффиксные массивы будут занимать в 4 раза больше места, чем сам геном?

    \subsubsection{FM-индекс}

    Развитие идеи суффиксных массивов.

    В основе BWT (Burrow-Wheeler Transform): получаемая последовательность символов = суффиксный массив.

    Пусть есть последовательность $ T $ и ещё одна  $ T' $. Определим...

    Важное свойство матрицы BWT -- <<картирование последнего на первое>>: перестановки циклические, следовательно, встречаемость символа в последней строке будет равна встречаемости в первой строке.

    Пусть $T$ -- исходный текст, $ B = b_1, ..., b_{n+1} $ -- BWT-матрица перестановок.
    Ряды обозначают префиксами.
    Пусть $ L(W) $ -- наименьший индекс $ k $:

    В суффиксных массивах все суффиксы, имеющие одинаковый префикс, группируются вместе.
    То же верно для матрицы  BWT.

    % TODO

    Пример: ищем <<aac>> в тексте <<acaacg>>

    Преобразование: gc\$aaac

    Таблица $ C(x) $

    \subsubsection{Как выбрать алгоритм для картирования}

    Хэш-таблицы удобно использовать, если проводилось картирование с использованием <<затравок>>.

    Требование для любого картировщика -- нахождение \textbf{всех} существующих совпадений c заданным числом несовпадений нуклеотидов.
    Самый первый картировщик для Illumina обладал этим свойством, но потом длина Illumina

    Самое сложное в этом материале: алгоритмы нигде нормально не описаны.

\end{document}

