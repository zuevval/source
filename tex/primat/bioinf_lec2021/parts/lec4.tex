\documentclass[main.tex]{subfiles}
\begin{document}

\section{Технология секвенирования геномов}

Секвенирование есть считывание молекул ДНК (даже при секвенировании РНК)

В геноме длинные молекулы, которые сложно упакованы (десятки тысяч килобаз).
Нет ни одного метода секвенирования, который работает с молекулой целиком.
Перед анализом ДНК всегда разрезается.
Затем происходит склеивание обратно (с помощью алгоритмов, программно).

\subsubsection{5' и 3'-концы}

Напомним, направление ДНК считается по тому, какой остаток сахарофосфатного остова свободен.
Цепи ДНК антипараллельны.
В базах хранится цепь от 5' к 3'.

Прочитывание ДНК происходит с помощью синтеза комплементарной цепи.
Синтез ДНК осуществляется молекулой ДНК-полимеразы, которая сама не может

Состав праймеров: основания РНК и ДНК

\subsubsection{История развития технологии}

Метод Сэнгера (1977):

Проводим реакцию с одной и той же матрицей в четырёх разных пробирках (по одной на нуклеотид).

В одной пробирке: небольшое количество меченого радиацией нуклеотида + нормальный нуклеотид + три оставшихся - все нормальные.

Тот, что меченый, не обычный, а дидезоксинуклеотид: он останавливает синтез.

Таким образом, в каждой из четырёх пробирок оказываются последовательности, оканчивающиеся только на определённый нуклеотид.
После этого фрагменты помещаются в агарозный гель и включается

Максимальная длина -- примерно 30 килобаз (мало)!

Усовершенствования:
вместо радиации использовали флюорофор; разные цвета меток позволили заменить четыре реакции одной.
Именно такой метод использовался в проекте <<Геном Человека>>.

Параллельно с международным консорциумом геном секвенировала компания Celera (руководитель -- Крейг Вентор), и международное сообщество не хотело дать последней секвенировать геном раньше консорциума (тогда бы она смогла запатентовать последовательность).

Недостатки метода Сэнгера:

\begin{enumerate}[noitemsep]
    \item % TODO
    \item
\end{enumerate}

Технологии секвенирования нового поколения:
\begin{enumerate}[noitemsep]
    \item 2005 -- технология 454 секвенирования
    \item 2006 -- Sollexa (компания Illumina)
    \item 2007 -- Solid (компания Applied Biosystems)
\end{enumerate}

Технология секвенирования нового поколения (ТСНП) включает три этапа:
\begin{enumerate}[noitemsep]
    \item Приготовление образца ДНК (т. н. составление библиотеки)
    \item Иммобилизация ДНК с помощью белков-адаптеров (они же служат праймерами).
    Все существующие технологии нового поколения, кроме PacBio RS, требуют амплификации фрагментов (когда фрагмент прикрепляется к подложке, с помощью ПЦР-рекции мы в одном месте получаем множество его копий).
    \item Секвенирование: используют меченые флюоресцентными метками нуклеотиды + реакцию синтеза ДНК с помощью ДНК-полимеразой (или в методе  SOLiD реакцию легирования).
\end{enumerate}

ТСНП используют множество технологий: струйные/жидкостные, микрофлюидные устройства, оптические системы.

Компании не очень заинтересованы в удешевлении своих технологий.
Китайцы сумели сделать сравнительно дешёвый вариант секвенатора.

\subsubsection{Секвенирование ДНК происходит циклически}
% TODO просто текст на слайде

\subsubsection{GS FLX 454}

Адаптер (<<бусина>>) прикрепляется к последовательности.
С адаптером молекула помещается в пору (лунку) в планшете.

Через планшет пропускается раствор, содержащий нуклеотиды.
Сперва пропускаем р-р, содержащий аденин, затем гуанин, ...

Каждый нуклеотид несколько модифицирован так, чтобы при соединении в цепочку испускать

Почему от технологии отказались?
Не используется система с терминирующими нуклеотидами, которые бы препятствовали присоединению в одном цикле секвенирова

\subsubsection{Genome Analyzer (Solexa, ныне Illumina)}

\begin{enumerate}[noitemsep]
    \item Молекулы прикрепляются адаптерами к платформе
    \item Bridge Amplification: адаптер -- на самом деле праймер.
    % TODO a bit
    Проводится примерно 35 циклов амплификации.
    \item Секвенирование в итерациях: на процесс смотрит камера с высокопроизводительной системой (разновидность конфокального микоскопа).
    В растущей цепи можно контролировать, чтобы на каждой итерации присоединялся только один нуклеотид: к подсоединяющемуся нуклеотиду добавляется молекула, блокирующая синтез.
\end{enumerate}

Технология годится только для анализа коротких фрагментов % TODO

\subsubsection{Парные прочтения}

Синтез всегда идёт в одном направлении: от 5' к 3'-концу.
Можно секвенировать не одну цепь, а обе (параллельную и антипараллельную).

\subsubsection{SoLiD}

Эта технология использует ПЦР в эмульсии для приготовления библиотеки:
\begin{enumerate}[noitemsep]
    \item Сперва фрагменты дробятся на участки длиной
    \item К обеим концам прикрепляют по адаптеру-праймеру
    \item Используется не синтез, а легирование. Раствор содержит октонуклеотиды (не отдельные нуклеотиды), но не все, а 16 штук: сочетания из 4 нуклеотидов по 2.
    % TODO последовательное секвенирование со сдвигом...
\end{enumerate}

Так можно считывать только фрагменты длиной 35

Сейчас в основном применяют технологии третьего поколения: PacBio (наиболее точная), или нанопоры.

Исключая этап амплификации, мы уменьшаем вероятность ошибки (ошибки могут быть и при амплификации).

Библиотека тоже составляется, но не совсем обычным образом.

% TODO

\subsubsection{Другие технологии}
Ion Torrent:
Довольно дешёвая.
Основана на регистрации изменения pH.
Даёт много ошибок и практически ушла в прошлое.

Технология на основе нанопор

Менее точная, чем PacBio.
Получается некоторый выигрыш по стоимости в сравнении с PacBio (аппарат недорогой, реактивы к нему дорогие)

PacBio и технология на основе нанопор позволяют секвенировать длинные молекулы.

\subsubsection{Качество секвенирования ДНК (Base-calling)}

Секвенатор может одномоментно выполнять миллиарды реакций секвенирования.
На реакции смотрит камера, которая, во-первых, детектирует по флюоресценции, какой нуклеотид встроился (A, G, T или C), и, во-вторых, оценивает качество секвенирования.

Обычно алгоритмы распознавания нуклеотидов поставляются вместе секвенаторами разработчиками технологии.
Есть и открытая программа Phread

Результат распознавания -- файл в формате FastQ, который содержит последовательность и QualityScore.

FastQ:
\begin{enumerate}[noitemsep]
    \item уникальный код секвенатора. В конце $ /1 $ (непарное секвенирование) или $ /2 $ (парное секвенирование).
    \item
    \item
    \item
\end{enumerate}

Каждый участок молекулы ДНК желательно прочесть не один раз, а больше.
Хорошее покрытие одного нуклеотида, как считается, начинается с 20.


\begin{enumerate}[noitemsep]
    \item
    \item
\end{enumerate}

\end{document}
