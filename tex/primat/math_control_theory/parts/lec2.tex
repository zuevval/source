\documentclass[main.tex]{subfiles}
\begin{document}

\subsection{Законы управления (задачи)}

\begin{enumerate}
	\item Задачи позиционирования (пример: посадка ракеты)
	\item Задачи стабилизации (например, летит самолёт, надо поддерживать курс)
	\item Отслеживание параметров (реализуется в т. н. следящих системах).
	Пример: в катере между штурвалом и рулём посредник -- следящая система, которая позволяет многократно увеличивать усилие (примерно как гидроусилители руля в автомобиле).
	\item Самый высокоразвитый уровень управления: управление наилучшим образом (?)
\end{enumerate}

\subsubsection{Качество управления}

Для приведённых выше задач вводим критерий качества $ J $.

\begin{enumerate}[noitemsep]
	\item $J= ||\vec e(T)|| $; $ e = y_\alpha - y $, $ t \in [0;T] $
	\item $ J =||e_{\text{уст}}|| $ - установившаяся ошибка
	\item 
	\item $ J = \int_{0}^{T}(q ||e(t)||^2 +  \sigma||u||^2)$ (коэффициенты нужны для приведения к одной размерности)
\end{enumerate}

Уменьшение ошибки ведёт, как правило, к увеличению стоимости системы.

\subsection{Принципы управления}

\begin{enumerate}[noitemsep]
	\item \emph{Программное управление} -- простейшее. Управляя, не имея модель объекта, нельзя, поэтому считаем, что известна зависимость $ y = f(u, w) $. Предположим, что мы можем обратить зависимость: $ u = \phi(y, w) $ (возмущение $ w $ знаем и оно постоянно, $ y $ задано), т. о. можем найти программно необходимое $ u_{\text{прогр}}(t) $
	
	Пример.
	Во время Второй мировой войны самолёты сбивали с помощью программного управления.
	Но эффективность чрезвычайно низка (управление без обратной связи).
	Иногда в очень простых системах применять можно.
	\item \emph{Управление по возмущению} -- примерно то же, что программное.
	Предполагаем, что мы можем измерять и учитывать $ W $ (к примеру, знаем распределение ветров в течение года).
	\item Управление с обратной связью (ОС): измеряем выходную величину. Измеренное значение $ y $ поступает в регулятор, и туда же мы подаём желаемое значение выхода $ y_d $.
\end{enumerate}

Какие виды управления мы рассматриваем?

\begin{enumerate}[noitemsep]
	\item Пропорциональный закон управления (p-регулятор): $ u = k_1 e $
	\item Интегральный закон управления: $ u = k_2 \int_{0}^{t} e(\tau)d\tau \longrightarrow (e  \to 0) $ 
	
	Этот закон уменьшает накопленную ошибку.
	\item Дифференциальный закон управления: $ u = k_3 \dot e $
	
	Уменьшаются колебания ошибки.
	
	\item Естественно, может быть комбинация: $ u = k_1e + k_2 \int... + k_3 \dot e $ -- \emph{ПИД-регулятор} (может быть и \emph{ПИ-регулятор}).
\end{enumerate}

посмотрим на примере эти законы

\subsubsection{Задача о нагреве тела в ёмкости}

Пусть имеется некое тело (например, чайник), который надо нагреть ( $T_0 \to T^*$; есть горелка с некой интенсивностью подводимого тепла $ q_0 $).
К примеру, $ T_0=20, T^*=100 $. Пусть тело имеет некую полную теплоёмкость.

\begin{enumerate}
	\item Подведённая энергия: $ C \Delta T \ q_0 \Delta t $ -- баланс энергии.
	
	Переходим к ДУ: $ C \dot T = q_0 \longrightarrow T(t) \to \infty $ при $ q_0 = const $
	
	В этой модели мы не учитываем теплоотдачу.
	
	\item Пусть температура окружающей среды постоянна ($ T_{\text{ср}} = const $)
	$$ C \dot T = q_0 + q_T $$
	Закон Ньютона для теплопередачи: Стефана-Больцмана -- $ E \sim T^4 $ (в кельвинах); Ньютона -- его линеаризация в области $ T \approx 300-380 K $: $ q_T = - \alpha(T - T_c) $.
	\begin{align*}
		& C\dot T = q_0 - \alpha (T - T_c) \\
		& T(t) = Ae^{-\frac{\alpha}{C}t} + T_c + \frac{q_0}{\alpha} \xrightarrow{t \to \infty} T_c + \frac{q_0}{\alpha} = T \\
	\end{align*}
	
	Можно тот же закон получить на школьном уровне: сколько тепла приходит, столько и уходит: $q_0 = \alpha (T^* - T_c) \longrightarrow T \ne T^* $
	
	К сожалению, нам неизвестные точные значения параметров $ \overline \alpha, \thickspace \overline T_c $
	
	Хотим посмотреть, насколько влияет разница предполагаемого значения и реального.
	
	$$ C\dot T = \alpha (T^* - T_c) - \overline{\alpha}(T - T_c) $$
	$$ T(t) = Be^{- \frac{\overline{\alpha}}{C}t} + \frac{\alpha(T^* - T_c) + \overline{\alpha} \overline{T}_c}{\overline{\alpha}} \ne T^* $$
	
	Отсюда видим, что программное управление неэффективно (даёт ошибку), причём в случае неустойчивости ошибка может нарастать до бесконечности.
	
	\item Сделаем обратную связь. Предположим, что мы можем измерять $ T $; теперь $ u = k_1 e $, $ q_0 = q_0^{\text{прогр}} +  $
	
	% TODO всё до второго перерыва пропущено
		
\end{enumerate}

\subsection{Структура ТАУ}

ТАУ:

\begin{enumerate}[noitemsep]
	\item Анализ (система работает плохо; выяснить, почему)
	\item Синтез (конструируем модель системы управления, затем и саму систему) -- обратная задача.
	Зачастую решается через многократное решение задач анализа с последующим изменением системы.
\end{enumerate}

\subsubsection{Анализ систем автоматического управления}

\begin{enumerate}[noitemsep]
	\item Построение математической модели (ММ). От адекватности модели зависит эффективность управления.
	\item Собственно анализ (установившегося процесса):
	\begin{enumerate}[noitemsep]
		\item Устойчивость: все системы должны быть устойчивы.
		Нужно узнавать приближение неустойчивости и уметь с ней бороться.
		\item Точность: с какой точностью функционирует наша система, какова величина ошибки управления?
	\end{enumerate}
	\item Переходные процессы (пример: у самолёта это взлёт / посадка, где происходит подавляющее количество аварий).
	Нас при этом интересуют:
	\begin{enumerate}[noitemsep]
		\item Время переходного процесса $ t_{\text{ПП}} $. При перелёте из СПб в Москву, к примеру, после взлёта сразу следует посадка.
		\item Режим этого процесса
	\end{enumerate}
\end{enumerate}

Если процесс безинерционный (подаём сигнал на вход и мгновенно получаем изменение выхода), к примеру, $ y = au + bw $ -- \emph{статические} (безинерционные) системы.

Если же в процессах есть инерция, системы называются \emph{динамическими}.
Они описываются не алгебраическими, а дифференциальными уравнениями.

Иногда в разные моменты времени одну и ту же систему можно рассматривать как статическую (например, в установившемся режиме) и как динамическую.

\subsubsection{Основные принципы построения ММ}

Модель, конечно же, лучше всего пытаться разделить на блоки (\textbf{принцип разбиения на более мелкие звенья}).
Но надо помнить: каждое звено должно иметь \emph{однонаправленное действие}, то есть выход звена не влияет на его вход.

Последнее требование позволяет отдельно работать с математической работой каждого звена.

Пример: в автомобиле (в первом приближении) можно выделить спидометр как подсистему, у которой вход не зависит от выхода.

\textbf{Пример: задача}. Рассмотрим катер или автомобиль с электроусилителем руля.

$ \phi_{\text{вход}} \text{(угол)} \xrightarrow{\text{датчик 1}}
 \phi_{\text{вход}}^V \text{(напряжение)} \xrightarrow{\text{УН}} \xrightarrow{\text{УМ}} \xrightarrow{\text{ДВ (двигатель)}} \xrightarrow{\text{Р (регулятор)}}  $ - далее стрелочка замыкается, приходя обратно в датчик. % TODO fix representation in PDF (missing words)
 
\subsection{Линейные непрерывные стационарные системы (ЛНСС)}

Линейные, т. к. проще: можем получить общие результаты аналитически (выполняется \emph{принцип суперпозиции}: $ \alpha u_1 + \beta u_2 \to \alpha y_1 + \beta y_2 $).
В нелинейном анализе, как правило, можно получить только частные результаты.

Известен <<принцип фонаря>>: если ночью потеряли чёрный кошелёк, искать его надо под фонарём, т. к. там есть хоть какая-то вероятность его найти.

Нелинейность на входе можно компенсировать нелинейностью в цепи обратной связи.

Когда нельзя компенсировать нелинейность, прибегаем к \emph{линеаризации}.

Линеаризовать сигнал можно только на маленьком участке.

Итак,
\begin{enumerate}[noitemsep]
	\item Линейные системы: это проще
	\item Непрерывные: в современном мире, конечно, всё чаще дискретные; но, отдавая дань традиции, вначале изучаем непрерывные.
	\item Стационарные системы -- системы с постоянными параметрами (постоянные коэффициенты в уравнениях, например, жёсткость пружины в колебательной системе со временем считаем переменной).
	Ракету нельзя считать стационарной системой: масса меняется.
	
	\begin{itemize}[noitemsep]
		\item Статическая система: $ y = f(u^*, w^*) + \frac{\partial f}{\partial u} |_* \Delta u + \frac{\partial f}{\partial w} |_* \Delta w + ... $
		
		Слагаемые более высокого порядка отбрасываем
		
		\item Динамическая система: $ f(y, y', y'', u, u', w) = 0 $
		
		После линеаризации: $ \alpha_2 \ddot{\Delta y} + \alpha_1 \dot{\Delta y} + \alpha_0 y = \beta_1 \dot{\Delta u} + \beta_0 \Delta u + \gamma_0 w $
		
		Это уравнение в отклонениях. Далее будем рассматривать все уравнения в отклонениях.
	\end{itemize}

\end{enumerate}

\subsection{Равномерно ограниченные ЛНСДС (линейные непрерые стационарные динамические системы)}

Рассмотрим ДУ порядка $ n $

$ D^n $ -- оператор дифференцирования

\begin{equation}
	(\alpha_n D^n + ... + \alpha_0) y = (\beta_n D^m + ... + \beta_0) u
\end{equation}

В векторном виде:
\begin{equation}
	\alpha (D) y = \beta(D) u
\end{equation}

Определение. $ n > m \to $ \emph{строго реализуемая} система; $ n = m \to $ \emph{нестрого реализуемая} (физическая) система; $ n < m $ -- \emph{нереализуемая} система (такое мы не можем себе позволить).

\textbf{Пример.} Грузик тащим по поверхности, действует сила трения $ to m \ddot x = F $. Здесь $ \alpha(D) = mD^2, \beta(D) = 1, n = 2 > m = 0 $

\textbf{Пример 2.} Грузик, подвешенный на пружине с демпфером: $ m \ddot y = - c y - b \dot y + F  \to \alpha(D) = mD^2 + bD + c, \beta(D) = 1 $  \textbf{Если на экзамене спрошу: <<а где $ m\vec{g} $?>>, то ответ <<сократилось, мы считаем отклонение от положения равновесия>>}

\textbf{Пример 3.} Закон Ома для участка цепи: $ I = \frac{U}{R} \to \alpha = 1, \beta = \frac{1}{R}; n = m = 0 $

\textbf{Пример 3а.} Рассматриваем RC-контур. Реактивное сопротивление конденсатора (в отличие от активного сопротивления, на конденсаторе мощность не выделяется): $ \frac{d q}{d t} = C \frac{d U}{d t} \to I = C DU \to X_c = \frac{1}{CD} $ -- реактивное сопротивление.

Определение. $ \alpha(D) $ называют \emph{характеристическим полиномом} системы (в нём заложены основные свойства системы: устойчивость и т. д)., а его порядок -- \emph{степенью} системы.

\textbf{Пример 3б.} Заметим, что в модели 3а в нуле разрыв производной. Природа не терпит не только бесконечные величины, но и бесконечные производные. % TODO
Учитываем индуктивность проводов: RC-контур становится RLC-контуром.

Устройство электрошокера: батарея + катушка индуктивности; ключ в нормальном режиме замкнут, когда нужно кого-то ударить током, ключ размыкают. $ U_L = \frac{d I}{d t} $


\begin{enumerate}[noitemsep]
	\item
	\item
\end{enumerate}

\subsubsection{}

\begin{enumerate}[noitemsep]
	\item
	\item
\end{enumerate}

\end{document}