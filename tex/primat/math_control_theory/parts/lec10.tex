\documentclass[main.tex]{subfiles}
\begin{document}

\section{ Лекция 10.  }
\subsection{ Описание дискретных систем в виде разностного (рекуррентного) уравнения высокого порядка }
2 апреля 2021 г.

\subsubsection{ Вспоминаем прошлую лекцию }

Непрерывные системы:

$$ \begin{cases}
\dot x = Ax + Bu \\
y = Cx
\end{cases} \Leftrightarrow \alpha(D) y = \beta(D) u $$

Дискретные системы: разностные уравнения

$$ \begin{cases}
x_{k+1} = A x_k + B u_k \\
y_k = C x_k
\end{cases} $$
\[ \alpha_n y_{k+n} + ... + \alpha_0 y_k = \beta_m u_{k+m} + ... + \beta_0 u_k \]
\[ k = 0, 1, 2, ...; \text{ н. у. } y_0, y_1, ..., y_{n-1} \]
$ n $ -- порядок разностного уравнения, $ k $ -- дискретное время.

Условие строгой реализуемости: $ \boxed{ n > m } $ (не может сигнал зависеть от управления в будущие моменты времени!)

\[ y_{k+n} = \frac{1}{\alpha_n} \left[ - \alpha_0 y_k - ... - \alpha_{n-1} y_{k+n-1} + \beta_m u_{k+m} + ... \right] \]

$ ] k = 0 \Rightarrow $ в нуле знаем значение из начальных условиях:
\[  \] % TODO
Раскручиваем рекурренцию:
% TODO

То же самое, если система описана в пространстве состояний.

НС (непрерывная система): оператор дифференцирования
 
ДС: вводим \emph{ оператор сдвига } $ \zeta f_k := f_{k+1} $ $ \Rightarrow $ $ \zeta^2 f_k = $ % TODO

Запись разностного уравнения $n$-го порядка в операторном виде:

\[ \alpha_n \zeta^n y_k + ... + \alpha_0 y_k = \beta_m \zeta^m \] % TODO

Далее вводим \emph{ дискретную операторную передаточную функцию }, с помощью которой ещё проще записать систему:
\[ y_k = \frac{\beta(\zeta)}{\alpha(\zeta)} \]

\subsubsection{ Общее решение разностного уравнения }

Непрерывная система:
\[ y(t) = y^*(t) + y^{**}(t), \thickspace y^*(t) = e^{\lambda t} \]
Дискретная система:
\begin{align*}
	& y_k = y_k^* + y_k^{**} \\
	& y_k^* \text{ -- решение однородного уравнения } \\
	& y_k^{**} \text{ -- частное решение неоднородного уравнения }
\end{align*}
Подстановка -- $ y_k^* = \lambda^k $:
\begin{align*}
	& \alpha(\zeta) y_k = 0 \\
	& y_k^* = \lambda^k \\
	& \alpha(\zeta) \cdot \lambda^k = 0 \Rightarrow \boxed{ \alpha(\lambda) = 0 } \text{ -- характ. ур-е }
\end{align*}

$ ] \lambda_i $ -- простые различные корни.
Тогда
\[ \boxed{ y_k^* = \sum_{i=1}^{n} C_i \lambda_i^k } \]

Если же $ \lambda_1 $ -- корень кратности $ q $, а остальные простые, то
\[ y_k^* = [C_1 + ] \] % TODO

Пример.
\[ (\zeta - a)(\zeta - b) y_k = 1[k] \]

где $1[k]$ -- дискретная функция Хевисайда
% TODO img

Пусть $ a \ne b; \thickspace a,b \ne 1 $

Будем искать решение в виде
\[ y_k^{**} = C \cdot 1[k] \]
Подставляем:
\[ \zeta y_k^{**} = C \cdot 1[k + 1] = X \cdot 1[k] = 1 \cdot y_k^{**} \]
\[ (1-a)(1-b) \cdot C = 1 \Rightarrow C = \frac{1}{(1-a)(1-b)} \]
Тогда
\[ y_k = C_1 a^k + C_2 b^k + \frac{ 1 }{(1-a)(1-b)} \]
$ C_1, C_2 $ получаем из н. у.

\textbf{ Задача. Числа Фибоначчи }
\[ y_{k+2} = y_{k+1} + y_k \]
(это уже однородное уравнение).
Н. У.: $ y_0 = 0, y_1 = 1 $
\[ 0, 1, 1, 2, 3, 5, 8, 13, 21, ... \]

\begin{align*}
	& y_{k+2} - y_{k+1} - y_k = 0 \\
	& \\ % TODO
	& \lambda_{1,2} = \frac{1}{2} \pm \frac{\sqrt{5}}{2} \\
	& \lambda_1 \approx - 0.618, \lambda_2 = 1.618 \\
	& \\ % TODO
\end{align*}

\[ \lim\limits_{k \to \infty} \frac{y_{k+1}}{y_k} = \lim\limits_{k \to \infty} \frac{ \lambda_1^{k+1} - \lambda_2^{k+1} }{\lambda_1 - \lambda_2} \frac{ \lambda_1 - \lambda_2 }{\lambda_1^k - \lambda_2^k} \] % TODO

\subsubsection{ Дискретное преобразование Лапласа }

НС:
\[ \mathfrak{L} \{f(t)\} = \bar f(p) = \int_{0}^{\infty} e^{-pt} f(t) dt \]

ДС:

% TODO img

Предположим, что точки -- результат дискретизации некой непрерывной функции.
Чем меньше шаг дискретизации $ h $, тем больше мы знаем о функции.

Введём \emph{ модулирующую функцию }, которая определяется через дельта-функцию Дирака.
Напомним, дельта-функция определяется как бесконечно высокий и бесконечно узкий импульс, площадь под которым равна единице; это производная функции Хевисайда.
Фильтрующее свойство дельта-функции: $ \int_{- \infty}^{\infty} f(t) \delta(t-t_0) dt = f(t_0) $

Введём $ f^*(t) = \sum_{k=0}^{\infty} f(t) \delta(t-kh) $
Это псевдонепрерывная функция с пиками, площадь под каждым из которых есть $ f(kh) $
\begin{align*}
	\mathfrak{L} \{ f^*(t) \} & = \int_{0}^{\infty} \sum f(t) \delta(t-kh) \\
	& = \sum_{k=0}^{\infty} \int_{0}^{\infty} % TODO
\end{align*}

(\emph{z-преобразование})
Ряд сходится при $ |z| > r > 1 $

\subsubsection{ Обратное z-преобразование }

% TODO

Этой формулой не пользуются, как не пользуются точной формулой для обратного преобразования Лапласа от непрерывной функции. \\

Раскладываем в ряд Лорана по отрицательным степеням:
\[ \bar f(z) = a_0 + \frac{a_1}{z} + \frac{a_2}{z^2} + ... \]
Как раскладывать в ряд Лорана?
Представляем $ z = \frac{1}{x} $ и раскладываем в ряд Тейлора по $ x $.
Тогда получается:
\[ \bar f(z) = f_0 + \frac{ f_1 }{ z } + \frac{f_2}{z^k} \] % TODO a bit

\textbf{ Примеры }

\begin{enumerate}[noitemsep]
	\item $ f_k $ % TODO
	\item $ f_k = a^k \Rightarrow \bar f(z) = \sum a^k z^k  = \sum \left( \frac{ a }{ z } \right)^k =  $% TODO a bit
\end{enumerate}

\subsubsection{ Свойства z-преобразования }

\begin{enumerate}[noitemsep]
	\item Линейность: $ z \{ \alpha_1 f_k + \alpha_2 \phi_k \} = \alpha_1 \bar f(z) + \alpha_2 \bar \phi(z) $
	\item Теорема смещения: в НС было
	\[ \mathfrak{L} \{ \dot f \} = \mathfrak{L} \{ D f \} = p \bar f(p) - f(0) \]
	Дискретная система:
	% TODO
	\[ z\{ f_{k+2} \} \overset{\phi_k := f_{k+1}}= z\{ \phi_{k+1} \} = \] % TODO
	
	Общая формула:
	\[ z\{ f_{k+n} \} = z^n \bar f(z) - z^n f_0 - ... - z f_{n-1} \]
	% TODO a bit
	
	\item Теорема о свёртке: в НС было
	\[ \int_{0}^t f_1(\tau) f_2(t - \tau) d \tau \]
	
	ДС:
	\[ \sum_{l=0}^k f_1[l] f_2[k-l] = \sum f_1 [k-l] f_2 [l] \]
	
	Покажем, что свёртка коммутативна.
	
	Чтобы понять, какие пределы суммирования, посмотрим на область суммирования:
	% TODO img
	
	\begin{align*}
		z\{ \sum_{l=0}^k f_1[l] f_2[k-l] \} & = \sum_{k=0}^\infty \sum_{l=0}^k \\ % TODO
		& = \sum_{l=0}^{\infty} \sum_{k=l}^{\infty} f_1[l] f_2[k-l] \\ % TODO
		& \overset{k_1 := k-l}= \sum_{l=0}^{\infty}
	\end{align*}
	
	Т. о. z-преобразование от свёртки есть произведение преобразований.
	
	\item Предельная теорема
	
	\begin{align*}
		& \bar f(z) = f_0 + \frac{f_1}{z} + \frac{f_2}{z^2} + ... \\
		& \\ % TODO
	\end{align*}
	
	\begin{enumerate}[noitemsep]
		\item % TODO
		\item \label{item:lim_th2}
	\end{enumerate}

	Докажем \ref{item:lim_th2} % TODO
	
	\begin{align*}
		\lim\limits_{  }
	\end{align*}
	
\end{enumerate}

\subsection{ Вторая часть лекции }
\subsubsection{ Таблицы z-преобразований }

\begin{align*}
	& z\{ 1[k] \} = \frac{ z }{ z - 1 } \\
	& \\ % TODO
	& z\{ \sin(bk) \} = \frac{ z \sin b }{z^2 - 2z \cos b + 1} \\
	& z\{ \cos(bk) \} = \frac{z (z - \cos b)}{z^2 - 2z \cos b + 1}
\end{align*}
Как получить $ z\{ k \} $?
\[ z\{ 1[k] \} = \frac{ z }{ z - 1 } = \sum_{k=0}^{\infty} 1 \cdot z^k  \] % TODO

Выводим $ z\{sin(bk)\} $:
\[ sin(bk) = \frac{ e^{jbk} - e^{-jbk} }{2j} \]
\[ z\{sin(bk)\} = \frac{1}{2 j} \left\{  \right\} \] % TODO

Преобразование косинуса:
\[ z \{ cos(bk) \} = \frac{1}{2} z\{ e^{jbk} + e^{-jbk} \} = \frac{1}{2} \frac{z^2 - z e^{-jb} + z^2 - }{den} \]
% TODO

\subsubsection{ Решение ДУ через z-преобразование }

Для начала рассмотрим случай нулевых начальных условий.

 $ z\{\zeta y_k \} \overset{0 \text{ н. у.} }=  $
\[ \alpha(z) \bar y(z) = \beta(z) \bar u(z) \]
\[ \bar y(z) \]%  TODO

Это третий вид описания: при помощи передаточной функции

\subsubsection{ Дискретная весовая функция }
Как и в НС, весовая функция есть обратное преобразование от передаточной функцией.
\begin{equation}\label{eq:weight_discr}
	h[k] \overset{def}= z^{-1} \{ H^*(z) \}
\end{equation}
\[  \]
\[ y[k] = \sum_{l=0}^{k} h[k-l] \cdot u[l] \]
\[ z\{ y[k] \} = \bar h(z) \cdot \bar u \overset{\ref{eq:weight_discr}}= H^*(z) \cdot \bar u(z) \]

Физический смысл $ h_k $: в НС это реакция на единичный дельта-импульс при нулевых начальных условиях.
В дискретных системах:

\[ y[k] = \sum_{l=0}^{k} h[k-l] h[l] \]
\[ ] u[l] = \delta_{l0} \]
(здесь $ \delta_{ij} $ -- символы Кронекера).

% TODO img 

Это дискретный аналог дельта-функции Дирака.
Если подадим её на вход, то выход есть весовая функция: $ \boxed{ y_k = h_k } $

% TODO img

\subsubsection{ Эквивалентное описание системы }

$$ \begin{cases}
x_{k+1} = A x_k + B u_k \\
y_k = C x_k
\end{cases} $$

При нулевых начальных условиях:
$$ \begin{cases}
z \bar x = A \bar x + B \bar u \\
\bar y = C \bar x \\
\end{cases} $$
\[ \bar x = (zE - A)^{-1} B \bar u \]
\[ \bar y = C(zE - A)^{-1} B \bar u \]
% TODO

Метод последовательного понижения порядка:
\[ \alpha(\zeta) y_k = \beta(\zeta) u_k \to A, B, C, X_0 \]
Пример на метод ППП:
\[ \]
% TODO

\subsubsection{ Критерии устойчивости }

Алгебраические критерии довольно хитрые.
Обычно их напрямую не применяют, а делают \emph{ конформное преобразование }.

Напомним, н. и д. условие устойчивости: $ |\lambda_A| < 1 \& |\lambda_i| < 1 $

% TODO img

Конформное преобразование ставит в соответствие точкам круга точки полуплоскости.
Внутренняя точка переходит во внутреннюю, внешняя во внешнюю, граничная в граничную.

\[ ] z = \frac{p+1}{p-1} \Leftrightarrow p = \frac{z+1}{z-1} \]
\[ ] z = r e^{j \phi}, \phi \in [0; 2 \pi] \]
\[ p = \frac{r e^{j \phi} + 1}{r e^{j \phi} - 1} =  \] % TODO

Минимальное значение знаменателя -- когда $ \cos \phi = 1 $.
При этом знаменатель $ = r^2 - 2r + 1 = (r-1)^2 $.
Значит, знаменатель всегда положительный.

$ \Re p = r^2 - 1 $

$$ \Re p \begin{cases}
> 0, r > 1 \text{ -- неустойчивая система } \\
= 0, r = 0 \text{ -- граница устойчивости } \\
< 0, r < 1 \text{ -- устойчивая система }
\end{cases} $$

% TODO

\textbf{ Пример. }

\[ \alpha(z) = \alpha_1 z + \alpha_0 \]
\[ \alpha \left( \frac{p+1}{p-1} \right) = \alpha_1 \frac{ p + 1 }{ p - 1}  + \alpha_0 = \frac{ (\alpha_1 + \alpha_0) p + (\alpha_1 - \alpha_0) p  }{den} \] % TODO

По критерию Стодола: % TODO

\subsubsection{ Построение точной дискретной модели по непрерывному описанию }

Пусть есть непрерывная система, но управление носит дискретный характер (пример: шаговый двигатель).

% TODO img

Управление на каждом интервале характеризуется одной амплитудой (масштабным множителем).
\[ u(t) = u_k p(t-t_k), k = 0, 1, ... \]

К примеру, фиксатор:
% TODO img

\[ x_{k+1} = A^* x_k + B^* u_k \]

Вопрос: как найти $ A^*, B^* $?

Напомним, когда приближённо описывали непрерывную систему дискретной, получали
\[ x_{k+1} \approx (E + Ah) x_k + Bh u_k \]
$ E + Ah \approx A^* $ и $ Bh \approx B^* $

Получаем точный вид $ A^*, B^* $ -- первый способ:

\[ x(t) = e^{At} x_0 + \int_{0}^{t} e^{A(t-\tau)} B u(\tau) d \tau \]
\[ x(h) = x_1 = e^{Ah} x_o + \int_{0}^{h} e^{A(h-\tau)} B p(\tau) u_0 d \tau \]
% TODO a bit
Но матричную экспоненту искать тоже нелегко.

Второй способ: решим непосредственно нашу систему на интервале $ [0;h] $ с н. у. $ x_0, u_0 $
\[ x(h) = x_1 = x_1(x_0, u_0) = A^* x_0 + B^* u_0 \]
меняем: $ x_1 \to x_{k+1}, x_0 \to x_k, u_0 \to u_k $

\textbf{ Третий способ: }
\[ \bar y = H(p) u \]
Пусть передаточная функция имеет такой вид: $ H(p) = \sum_{i=1}^{n} \frac{c_i}{p-p_i} $

... мы выводить не будем ...

\[ H^*(z) = \sum_{i=1}^{n} a_i  \] % TODO

\subsubsection{ Примеры }

\textbf{ Пример 1. }

\[ \ddot x + x = u \]

Надо проинтегрировать один раз систему на интервале $ [0;h] $ при произвольных начальных условиях.

% TODO

$ A^* = \begin{pmatrix}
 \\
\end{pmatrix} $

\textbf{ Пример 2. Числа Фибоначчи через z-преобразование }

% TODO
\[ \bar y = \frac{z}{\lambda_1 - \lambda_2} \left[ \frac{1}{z - \lambda_1} - \frac{ 1 }{z - \lambda_2} \right]  \]
\[ \boxed{ y_k = \frac{1}{\lambda_1 - \lambda_2} [ \lambda_1^k - \lambda_2^k ] } \]

\end{document}
