\documentclass[main.tex]{subfiles}
\begin{document}
	
\section{Н. и д. условия асимптотической устойчивости линейных систем. Продолжение}

Напомним, если система устйчивая, $\alpha(D)y =$ % TODO a bit

Разобрали случай различных корней.
Если же существуют кратные корни, в решении появится, в частности, слагаемое $ C_kt^k e^{\lambda_k t} $

Если вещественная часть меньше нуля, то экспонента <<задавит>> на бесконечности любой полином: $ y(t) \to 0 $ при $ t \to \infty $.
То есть н. и д. условие асимптотической устойчивости -- $ \boxed{Re \lambda < 0} $ (корни характеристического полинома лежат в левой полуплоскости).

В МТУ нас интересует, как правило, асимптотическая устойчивость (а не устойчивость по Ляпунову).
Хотим, чтобы все колебания, неизбежно появляющиеся в  переходных процессов (например, флаттер в самолёте), затухали на бесконечности.

Напомним, $ \dot y - y = u + w \Rightarrow \lambda_1 = 1 > 0 \Rightarrow $ неустойчивая система.

\subsubsection{Важное следствие}

Напомним, мы нашли решение для возмущённого решения, для невозмущённого, вычли, получили однородное уравнение относительно $ \Delta y = \sum c_i e^{\lambda_i t} $...

Попробуем сразу найти общее решение: $ y(t) = y^+(t) + y^{++}(t) $, где $y^+$ -- общее решение однородного уравнения, $y^{++}$ -- частное решение неоднородного.

Левая часть уравнения $ \alpha(D)y = \beta() $ называют свободными % TODO a bit

Вывод: если система устойчивая, решение при $ t \to \infty $ стремится к частному решению.
Иногда это даже даётся как определение

Т. о. устойчивая система <<забывает>> начальные условия.
Пример: маятник с трением вне зависимости от начального угла и скорости в конце концов придёт к положению равновесия (покой).
Следовательно, можно начальные условия положить равными нулю, если нас интересует установившееся решение.

\subsection{Описание системы в пространстве состояний}

Напомним, в пространстве состояний три параметра: матрица $A$, вектор-столбец $B$ и строка $C$.
Что отвечает за устойчивость?

$$ \begin{cases}
\dot x = Ax + Bu \\
y = Cx
\end{cases} $$

Перейдём к преобразованию Лапласа:

% TODO formulas a bit

$$ (pE-A) \bar x = B \bar u \Rightarrow \bar x = (pE-A)^{-1} B \bar u $$

% TODO
$$  $$

Т. о. за устойчивость отвечает матрица $ A $ (её спектр): условие устойчивости -- $ Re \lambda_A < 0 $

\subsubsection{Устойчивость составных систем}

\begin{enumerate}[noitemsep]
	\item Последовательное соединение: $\alpha = \alpha_1 \alpha_2 \Rightarrow H = H_1 H_2$, н. и д. условие устойчивости -- устойчивость обоих звеньев 
	\item Параллельное соединение: знаменатель будет таким же...
	\item Система с обратной связью: $ H = \frac{\frac{\alpha_1}{TODO}}{1 + TODO} $ % TODO
\end{enumerate}

\subsection{ Критерии устойчивости линейных систем }
Рассмотрим устойчивую систему:
$$ \alpha(p) = 0, Re p_i < 0 $$

Даже для квадратного уравнения $ x^2 + bx + c = 0 $ можно найти такие такие $b,c$, что численные методы нахождения корней дадут сколь угодно далёкий от решения результат.

Научное сообщество разработало \emph{критерии} устойчивости линейных систем.
Они работают с коэффициентами полинома и значительно упрощают определение устойчивости.

У нас будут три алгебраических критерия и два частотных.  

\subsubsection{Необходимое условие устойчивости Стодола}
Необходимое условие: позволяет браковать заведомо неустойчивые системы.

Стодола -- словацкий инженер (1890).
Следуя автору, запишем $\alpha(p) = a_0 p^n + ... + a_n $

Условие:
$$\boxed{a_i > 0} \text{ (точнее, коэффициенты одного знака)}$$

Доказательство.
$$ \alpha(p) = a_0 (p-p_1) \cdot ... \cdot (p - p_n); \alpha(p_i) = 0 $$

\begin{enumerate}[noitemsep]
	\item $ ] p_i \in \mathds{R} \to p_i = - \gamma_i^2 < 0 $ (необходимое условие устойчивости)
	$$ \alpha(p) = a_0(p+\gamma_1^2) \cdot .. \cdot (p + \gamma_n^2) $$
	
	Раскроем мысленно скобки: будет полный полином (все коэффициенты присутствуют), и их знак такой же, как знак $a_0$. 
	\item $ ] p_i \text{ комплексно сопряжённые: } p_1 = - \gamma_1^2 + j \delta_1, p_2 = - \gamma_1^2 - j \delta_1  $
	% TODO формулы
	
	Полный полином второго порядка; если раскроем скобки, все коэффициенты будут иметь знак $a_0$.
\end{enumerate}

Пример:

$ ] \alpha(p) = p^4 + 2 p^3 + 3p + 4 \Rightarrow $ неустойчивая система, т. к. нулевой коэффициент при $ p^2 $ и условие Стодола не выполняется.

Если же $ \alpha(p) = p^4 + 2 p^3 + 5 p^2 + 3p + 4 $, условие не работает и сказать о системе по критерию не получает.

\subsubsection{Н. и д. условия устойчивости Гурвица}

Гурвиц (1895 г): ему написал Стодола (второму не хватало математического образования).

Строим матрицу Гурвица:

$$ \Gamma_{n \times n} = \begin{pmatrix}
a_1 &  a_3 & a_5 & ... & 0 \\
a_0 &  a_2 & a_4 & ... & 0 \\
0   &  a_1 & a_3 & ... & 0 \\
... &&&& \\
0  &&& a_{n-2} & a_n
\end{pmatrix} $$

Критерий:

\begin{enumerate}[noitemsep]
	\item $ a_0 > 0 $
	\item $ \Delta_i > 0 $ (все главные миноры положительные)
\end{enumerate}

Пример:

$ ] n = 1 $ % TODO

$ ] n = 2, \alpha = a_0 p^2 + a_1 p + a_2, p_{1,2} = \frac{-a_1 \pm \sqrt{a_1^2 - 4a_0 a_2}}{2a_0} $
% TODO
\begin{leftbar}
	Для систем I и II порядка необходимое условие совпадает с достаточным.
	Начиная с третьего порядка это не работает (необходимо дополнительное условие), и чем больше порядок, тем больше дополнительных условий.
\end{leftbar}

$ ] n = 3, \alpha = a_0 p^3 + a_1 p^2 + a_2 p + a_3 $

$$ \Gamma_{3\times3} = \begin{pmatrix}
a_1 & a_3 & 0 \\
a_0 & a_2 & 0 \\
0   & a_1 & a_3 \\
\end{pmatrix} $$

\begin{enumerate}[noitemsep]
	\item $ a_0 > 0 $
	\item $ Delta_1 = a_1 > 0 $
	\item $ Delta_2 = a_1 a_2 - a_0 a_3 \overset{a_0 > 0, a_3 > 0} > 0 $ -- этого условия нет в критерии Стодола!
	\item $ Delta_3 = a_3 \Delta_2 > 0 \Rightarrow a_3 > 0 $ 
\end{enumerate}

$$ \boxed{a_1 a_2 > a_0 a_3} \text{ -- дополнительное условие к критерию Стодола для систем III порядка} $$

Пример:
$$ p^3 + 2 p^2 + 3p + 5 \Rightarrow 2 \cdot 3 > 1 \cdot 5, \text{ система устойчива} $$

\subsubsection{Н. и д. условия Льенара-Шипара, 1914}

Пусть выполнены необходимые условия Стодола ($ a_i > 0 $).
Тогда можно упростить достаточное условие.

Составляем матрицу Гурвица $ \Gamma_{n \times n}  $.
Условия:

\begin{enumerate}[noitemsep]
	\item $ a_i > 0 $ (это условие Стодола)
	\item $ \Delta_{2k} > 0 $ ИЛИ $ \Delta_{2k-1} > 0 $ (можно проверять только чётные или нечётные миноры).
\end{enumerate}

То есть достаточно проверять вдвое меньше миноров.

Пример: если $ a_i > 0 $, мы посчитали миноры и оказалось, что все положительные, кроме одного, значит, мы ошиблись! 

\subsubsection{Следящая система}

% TODO img & description

$ \bar e(p) = H_{\phi_{\text{вх}} e}(p) \cdot (\bar \phi_{\text{вх}}(p)) \xrightarrow[k \to \infty]{} 0 $, т. е. $ e(t) \to 0 $
(т. е. выход есть функция от входа на передаточную функцию; нам интересен не выход $\phi_{\text{вых}}$, а $e$ -- ошибка).

$ H_{} $ % TODO write some


\Large \textbf{Часть 2 лекции} \normalsize
Пример.

Пусть $\phi_{\text{вх}} = at \Rightarrow \bar \phi_{\text{вх}} = \frac{a}{p^2}$

Хотим найти по предельной теореме ошибку в установившемся режиме:

$$ e_\infty = \lim\limits_{p \to 0} p \bar e(p) = \frac{a}{k} $$

% TODO

Ошибка не может быть меньше величины

\subsection{Частотные характеристики}

Частотные характеристики отражают поведение линейной динамической устойчивой системы при гармоническом воздействии.

Планеты движутся вокруг Солнца, электроны по орбитам, человек дышит...
Всё это колебания.

Пример:
\begin{enumerate}[noitemsep]
	\item Соотношение популяций волков и зайцев находится в динамическом равновесии (они находятся в противофазе и колеблются почти гармонически).
	\item Плавание: плотность тела такая, что мы не тонем, но уровень воды выше рта и носа, поэтому человек, не умеющий плавать, принимает вертикальное положение и пытается, работая руками и ногами, всё время держаться на поверхности.
	
	Но правильная стратегия -- привести себя в колебательное состояние: когда лицо над водой, делать вдох.
	Затраты на поддержание колебаний незначительные!
\end{enumerate}

Задача.

$$ \xrightarrow{u = u_0 \cos (\omega t)} \boxed{\frac{\beta (D)}{\alpha (D)}} \xrightarrow{y_\infty(t) = ?} $$

$y_\infty(t)$ -- установившаяся реакция

Система линейная, поэтому можно добавить мнимую часть (синус), чтобы легче было дифференцировать:

$ U = u_0 e^{j \omega t} = u_0 \cos (\omega t) + j \sin (\omega t) \to V(t) $

Линейная система $ \Rightarrow $ выход от вещественной части -- вещественная часть выхода, т. е. $ u_0 \cos (\omega t) \to Re(V(t)) $

$$ \alpha(D) Y = \beta(D) U $$

Устойчивая система, поэтому $ Y_\infty(t) = Y_{\text{частн}}(t) $

$$ \alpha(D)Y = (\alpha_n D^n + ... + \alpha_1 D + \alpha_0) Y = \alpha_n (j \omega)^n Ce^{j \omega t} + ... + \alpha_1 (j\omega) C e^{j \omega t} + \alpha_0 Ce^{j \omega t} $$

% TODO

$$ C = \frac{\beta(j\omega)}{\alpha(j\omega)} u_0 $$

Это -- \emph{частотная передаточная функция}: то, на что нужно умножить входной гармонический сигнал, чтобы получить выход.

Частотная передаточная функция может быть представлена так: $ H(j\omega) = |H(j\omega)| e^{j \phi} $, $\phi$ -- аргумент:
$$ \phi = arg H(j \omega) = arctg \frac{Im(H)}{Re H}, \frac{-\pi}{2} \le \phi \le \frac{\pi}{2} $$
Т. о. $ \boxed{ Y_\infty(t) = |H(j\omega)| e^{j \phi} u_0 e^{j \omega t} = |H(j\omega)|u_0 e^{j(\omega t + \phi)} }  $

$$  $$ % TODO a bit

Выводы: если на входе гармонический сигнал, то на выходе гармонический сигнал той же частоты $ \omega $, амплитудой, увеличенной в $ |H(j \omega)| $ раз; сдвиг фазы $\phi$ (обычно $ \phi < 0 $).

\subsubsection{Частотные характеристики}

\begin{enumerate}[noitemsep]
    \item Амплитудно-частотная характеристика (резонансная кривая): зависимость амплитуды от частоты % TODO img
    \item Фазо-частотная (фазово-частотная): зависимость сдвига фазы от частоты % TODO img
    \item (Всеобъёмлющая характеристика) Амплитудно-фазовая характеристика (она же годограф ЧПФ -- частотной передаточной функции): траектория в осях $ Re (H(j\omega)) $, $ Im(H(j\omega)) $ % TODO img
    
    Часто это спирали, заканчивающиеся в нуле.
    Длина вектора из нуля в точку на АФК равна частоте $ \omega $, а угол поворота -- фазе $ \phi $. 
\end{enumerate}

Примеры.

\begin{enumerate}[noitemsep]
	\item Интегрирующее звено:
	% TODO 
	
	\begin{leftbar}
		Просьба в работах на осях писать $ Im(H(j\omega)), Re(H(j\omega))  $, а не просто $ Im, Re $
	\end{leftbar}
	
	\item Инерционное звено:
	% TODO img
	
	$$ U_{\text{вых}} = U_C $$
	\begin{align*}
		& \frac{dq}{dt} = C\frac{du_C}{dt} \\
		& I = C \dot U_C \\
		& \dot U = \dot U_R + \dot U_C, \thickspace U_R = IR \\
		& DU = DIR + \frac{I}{C} \\
		% TODO
	\end{align*}
	
	$$ \boxed{H(p) = \frac{1}{Tp + 1}} $$
	
	Построим все частотные характеристики.
	
	$$ H(j\omega) = \frac{1}{1 + j \omega T} = \frac{1 - j \omega T}{1 + \omega^2 T^2} $$
	$$ |H(j \omega)| = \frac{1}{\sqrt{1 + \omega^2 T^2}} $$
	
	АЧХ:
	
	% TODO img
	
	Считается, что, если $ \omega < \frac{1}{T} $, то сигнал (например, звук) проходит почти без искажений.
	Если выше, то это фильтр низких частот.
	
	ФЧХ:
	$$ \phi = arctg \frac{- \omega T}{1} = -arctg(\omega T) $$
	
	% TODO img
	
	АФХ:
	
	% TODO img
	
	$$ (Re H - \frac{1}{2})^2 + (Im H)^2 = \frac{1}{4} $$

\item (к примеру, стиральная машина):
$ m \ddot y = -b \dot y - c y + F(t) $, $ F(t) = F_0 cos(\omega t) $
$$ H(p) = \frac{1}{p^2 + 2np + k^2}, \thickspace k^2 = \frac{c}{m}, 2n = \frac{b}{TODO} $$
$$ $$

АЧХ:

Если трение мало и частота равна собственной, будет пик.
Чем меньше $ n $, тем выше пик (вплоть до бесконечности).
% TODO img
Кому интересно, можно вывести дома: собственная частота $ = \sqrt{k^2 - 2n^2} $

АФХ:

% TODO img

Когда $ \omega = k $, вечественная часть передаточной нулевая; $ \omega > k \Rightarrow $ отрицательная, $ < $ -- положительная.

При $ \omega \to \infty $ приходим, касаясь абсциссы, в ноль.

$$ \phi = \begin{cases}
arctg \frac{- 2n \omega}{k^2 - \omega^2}, \omega < k \\
arctg \frac{- 2n \omega}{k^2 - \omega^2} - \pi, \omega \ge k
\end{cases} $$

ФЧХ:

% TODO img

Если $n$ мало, перегиб более крутой, в нуле -- функция Хевисайда.

Замечание: по АЧХ видно, что лучше всего работать на частоте больше, чем резонансная частота.
Стиральная машина быстро разгоняется, но при торможении (так называемый режим выбега) медленно проходит вниз по графику частот и может сильно трястись.

\item <<Раскидай>>: мячик на резинке. Диаметр примерно 4-5 мм.

Замечание: если подвесить раскидай как маятник и колебать с малой частотой, колебаться будет в такт.
Если частота велика $ \omega \approx 5 $ Гц, колебаться будет в противофазе.

В самом деле, раскидай, фактически -- грузик на пружинке.
Посмотрим на ФЧХ.

\item Светофор и машины: те, что стоят дальше от светофора в очереди, трогаются на красный.
	
\end{enumerate}

Оказывается, для определения устойчивости достаточно строить характеристики весьма приближённо.

\end{document}
