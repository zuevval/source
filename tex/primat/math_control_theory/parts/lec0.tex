\documentclass[main.tex]{subfiles}
\begin{document}

\subsection{Введение}

Суханов Александр Алексеевич, МПУ \href{mailto:A.A.Sukhanov@gmail.com}{A.A.Sukhanov@gmail.com}

Два расчётных задания.
Уже выложены в Teams.
Все они привязаны к порядковому номеру.

Две страницы на каждое задание.
На второй странице -- подробная инструкция по офрмлению задания.
Пожалуйста, читайте её внимательно, иначе фильтры могут отправить письмо в спам.

На экзамене будут вопросы в полном соответствии с пунктами материала.

Литература ():
\begin{enumerate}[noitemsep]
	\item Первозванский А. А. Курс теории автоматического управления.
	\item Попов Е. П. Теория систем автоматического регулирования и управления. \emph{Регулирование -- то же, что управление}.
	\item Теория автоматического управления. Ч.1. /Под ред. А.А.Воронова. - М.: Высш. шк., 1986. – 368 с.
	\item Бесекерский В.А., Попов Е.П. Теория систем автоматического регулирования. - СПб: Профессия, 2004. – 752 с.

	\item Пугачев В.С. Основы автоматического управления. – М.: Наука, 1968.

	\item Юревич Е.И. Теория автоматического управления. – СПб: БХВ-Петербург, 2007. – 560 с.

	\item Мирошник И.В. и др. Теория автоматического управления. Линейные системы. – СПб; Питер, - 2005. – 336 с.
	\item Теория автоматического управления /Под ред. В.Б.Яковлева. – М: Высшая школа, 2005. – 568 с.
	\item Сборник задач по теории автоматического регулирования и управления. /Под ред. В.А.Бесекерского. – М.: Наука, 1978. – 512 с. \emph{Люблю брать примеры и отсюда. Чем хороша? Структура: теория, затем примеры, задачи. В конце ответы}.
\end{enumerate}

Лекция и практика плавно перетекают друг в друга.

\end{document}