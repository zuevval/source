\documentclass[main.tex]{subfiles}
\begin{document}
\section{ Лекция 11. Частотные характеристики ДС }

\subsection{ Организационные вопросы. О дифф. зачёте }
Дифф. зачёт онлайн: несложные, скорее качественные.
Отвечать нужно сходу.

Можно провести подобие консультации прямо перед зачётом.

\subsection{ Частотные характеристики ДС }
Частотные характеристики возникают тогда, когда

Напомним, в НС
% TODO img
\[ u = u_0 \cos \omega t \Rightarrow y = y_\infty(t) = |H(j \omega)|(u_0 \cos(\omega t + \phi)) \]
где $ \phi = \arg H(j \omega) $ -- аргумент частотной передаточной функции. \\

Пусть теперь вход дискретный: $ u = u_0 \cos \omega t $.

Пусть $ t_k = kh, k = 0, 1, ... $; $ u_k = u_0 \cos \omega t_k $; $ y_k^{\infty} = ? $

% TODO img

Описание системы: $ \alpha(\tau) y_k = \beta(\tau) u_k, u_k = u_0 \cos \omega k h $.
Обозначим $ \Omega := \omega h $ -- безразмерная \emph{ дикретная частота } ($ \omega: \frac{1}{sec} $, $  $)

Будем искать решение в виде % TODO a bit

В силу принципа суперпозиции, очевидно, $ y_k = \Re Y_k $.
Это очевидно.

Далее говорим: у нас установившийся режим $ \Rightarrow Y_k \xrightarrow[k \to \infty]{} Y_{k \text{частн}} = Y_k^\infty $

\[ ] Y_{k \text{частн}}  \]
% TODO

Как и в непрерывном случае, экспонента в любой степени не ноль.
Сокращаем (здесь умножение <<честное>>) $ \Rightarrow $

\[ C = \frac{\beta(e^{j \Omega})}{\alpha(e^{j \Omega})} u_0 = H^*(e^{j \Omega}) u_0 \]
а это -- передаточная функция от аргумента $ e^{j \Omega} $.
$ H^*(e^{j \Omega}) $ -- \emph{ дискретная частотная передаточная функция } (ЧДПФ).
В непрерывных системах было от $ j \omega $; здесь немного посложнее.
% TODO a bit

\[ \boxed{ Y_k = | H^*(e^{j \Omega}) | u_0 e^{j (\Omega k + \phi)} } \]
\[ y_k^\infty = | H^*(e^{j \Omega}) | u_0 \cos(\Omega k + \phi) \]

Т. о. мы мгновенно вычисляем выход, если вход -- гармоническая функция.

\textbf{ Особенности: }

$ e^{j \Omega} = e^{j (\Omega + 2 \pi)} \Rightarrow \Omega in [- \pi; \pi] $ % TODO maybe some comments?

Построим АФХ (зависимость мнимой части передаточной функции в координатах $ \Im H^* $) % TODO a bit + img

Покажем на примере системы второго порядка, что график симметричен относительно вещественной оси.
\[ ] n = 2; H^*(\zeta) = \frac{\beta(\zeta)}{den} = \frac{ a \zeta^2 + b \zeta + c }{ d \zeta^2 + e \zeta + f } \]
\[ \zeta = e^{j \Omega} = \cos \Omega + j \sin \Omega \]
\[ \zeta^2 = \cos 2 \Omega + j \sin 2 \Omega \]
\[ \beta(e^{j \Omega}) = a(\cos 2 \Omega + j \sin 2 \Omega) + b (\cos \Omega + j ) \] % TODO
$ X_1 $ -- чётная функция, $ Y_1 $ -- нечётная.
Т. о.
\[ H^*(e^{j \Omega}) = \frac{ X_1 + j Y_1 }{ X_2 + j Y_2 } = \frac{ X_1 X_2 + Y_1 Y_2 + j () }{den} \] % TODO
$ \Re H^* (e^{j \Omega}) $ чётная; $ Y_2 X_1 $ -- нечётная $ \Rightarrow $  $ \Im H^*(e^{j \Omega}) $ -- нечётная.

% TODO img

Т. о. функция симметрична относительно вещественной оси.
В силу симметрии можно исследовать годограф только на промежутке $ \Omega \in [ 0; \pi ] $ и при необходимости восстановить вторую половину. \\

\textbf{ Пример: }
Расмотрим дискретное инерционное звено.
Построим для него частотную характеристику.

Напомним, непрерывное передаточное звено имеет передаточную функцию $ H(D) = \frac{1}{T D + 1} $, т. о. ДУ
\[ T \dot y + y = u \]
% TODO img

У нас 

Период $ T = \frac{2 \pi}{\omega} \Rightarrow \Omega = \frac{2 \pi}{T} = h $.

$ ] \frac{h}{T} = \frac{1}{10} $.
Тогда на промежутке $ [0;T] $ будет $ 10 $ интервалов ($ 11 $ точек).

% TODO img

На последнем занятии мы говорили, что можно построить точный аналог непрерывной системы, если на входе дискретная функция.

Пусть на входе ступенчатая функция: $ u = u_k p( t - t_k ) $, $ p(\tau) = 1 $ -- фиксатор.

Посмотрим, как выглядит эквивалентная система, если имеем непрерывное инерционное звено, где на вход подаётся дискретный ступенчатый сигнал.

% TODO img

Используем третий подход: сразу восстанавливаем $ H^*(\tau) $ по $ H(D) $.

Вспоминаем формулу с прошлого занятия:
\[ H^*(\zeta) = \sum_{i=1}^n a_i \frac{c_i}{\zeta -e^{p_i h}} \] 
\[ H^*(\zeta) = \frac{a_1 c_1}{\zeta - e^{- \frac{h}{T}}} \]
\[ a_1 = \frac{1}{-\frac{1}{T}} [ e^{-\frac{h}{T}} - 1 ] \Rightarrow H^*(\zeta) = \frac{1 - e^{- \frac{h}{T}}}{\zeta - e^{- \frac{h}{T}}} = \frac{1 - b}{\zeta - b} \]
\[ H^*(\zeta) =  \] % TODO

\textbf{ Пример: }
$ ] b = \frac{1}{2} $; при $ \frac{h}{T} = \frac{1}{10}$ $ b \approx 0.9 $.
\[ H^*(\zeta) = \frac{1 - b}{\zeta - b} = \frac{1}{2} \frac{1}{\zeta - \frac{1}{2}} \]
\[ H^*(e^{j \Omega}) = \frac{1}{2} \frac{1}{e^{j \Omega} - \frac{1}{2}} = \frac{1}{2} ... = ... =  \] % TODO

Построим АФХ по двум точкам:

$ \Omega = 0 \Rightarrow H^*(e^{j \Omega}) = 1 + 0j  $; $ \Omega = \pi \Rightarrow H^*(e^{j \Omega}) = - \frac{1}{3} + 0j $

Для интереса посмотрим значение в промежуточной точке: $ \Omega = \frac{\pi}{2} \Rightarrow H^*(e^{j \Omega}) = -\frac{1}{5} - \frac{2}{5}j $

% TODO img

Можно показать, что АФХ -- полуокружность.

Если хотим построить АЧХ:
в силу симметрии и периодичности можно продолжить график на всю прямую.

% TODO img

\subsection{ Вторая часть лекции. Частотные критерии устойчивости }

Напомним, для определения устойчивости можно использовать конформное преобразование:
\[ z = \frac{ p+1 }{ p-1 } \to \bar \alpha(p) \]

Но есть также алгебраические и частотные критерии непосредственно для дискретных систем.

\subsubsection{ Частотный критерий Михайлова для дискретных систем }

Напомним, для непрерывных систем мы строим годограф:  $\alpha_{\text{з}}(j \omega)$ в осях $ \Re \alpha_{\text{з}}(j \omega) $, $ \Im \alpha_{\text{з}}(j \omega) $.
\[ \Delta \arg \alpha_{\text{з}}(j \omega) |_0^\infty = \frac{\pi}{2} (n - m) \]
где $ m $ -- число неустойчивых корней; т. о. для устойчивости н. и д. условие $\Delta \arg \alpha_{\text{з}}(j \omega) |_0^\infty = \frac{\pi}{2} n $. \\

Для дискретных систем:
\[ \alpha(z) = \alpha(j \Omega) \overset{\text{т. Безу}}= a_0  \] % TODO a bit: formula

% TODO img

В случае непрерывных осей мы шли вдоль мнимой оси (граница области).
Здесь же мы хотим идти вдоль единичной окружности.
\[ p = e^{j \Omega}, \Omega \in [- \pi; \pi] \]
Пусть вне окружности $ m $ корней, а всего $ n $.
При обходе окружности приращение аргумента от каждого внутреннего корня
\[ \Delta \arg (z - z_1) = 2 \pi \]
а от каждого внешнего --
\[ \Delta \arg (z - z_2) = 0 \]
Т. о.
\[ \Delta \arg \alpha(e^{j \Omega}) |_0^\infty =  \] % TODO

Заметим, что годограф $ \alpha_{\text{з}} $ будет симметричен относительно вещественной оси.

% TODO img 

Т. о. можно исключить отрицательную часть.

% TODO maybe add smth, maybe not

Формула для половины годографа принимает вид:
\[ \Delta \arg \alpha_{\text{з}}(e^{j \Omega})|_0^\pi = \pi n \]

\textbf{ Пример: }
\[ ] \alpha_{\text{з}}(z) = \left( z -\frac{1}{2} \right) \left( z + \frac{3}{2} \right) = z^2 + z - \frac{3}{4} \]
Система неустойчива (есть корень, лежащий вне единичной окружности).
Сделаем вид, что мы это не заметили, и воспользуемся критерием Михайлова для ДС.

\[ \alpha_{\text{з}}(e^{j\Omega}) = \cos 2 \Omega + j \sin 2 \Omega + \cos \Omega + j \sin \Omega - \frac{3}{4} = \]
\[ = \cos 2 \Omega + \cos \Omega - \frac{3}{4} = (\cos 2 \Omega + \cos \Omega - \frac{3}{4}) + j (\sin 2 \Omega + \sin Omega) \]
При $ \Omega = 0 $ % TODO

$ \Delta \arg \alpha_{\text{з}}(e^{j \Omega}) = \pi \overset{n=2}= \pi(2-1) $

\subsubsection{ Границы устойчивости }

Пусть есть корни на границе, т. е. на окружности.
Тогда в какой-то момент будет $ z - z_i = 0 $

% TODO img

Тогда годограф проходит через начало координат.
В таком случае нужно немного пошевелить годограф.
Если при отклонении в одну сторону будет неустойчивая система, в другую -- устойчивая, то мы на границе устойчивости.

% TODO img

\subsubsection{ Критерий Михайлова }

\[\Psi = 1 + H_{\text{ раз }}(z) = 1 + \frac{\beta}{\alpha} = \frac{\alpha + \beta}{\alpha} = \frac{\alpha_{\text{з}}}{\alpha} \]
\[ \Delta \arg \Psi(e^{j \Omega})|_0^\pi = \Delta \arg \alpha_{\text{з}}(e^{j \Omega})|_0^\pi - \Delta \arg \alpha(e^{j \Omega})|_0^\pi \]
% TODO a bit

Полностью совпадает с критерием Найквиста для НС.
Следовательно, сохраняется и правило Цыпкина.

% TODO img

$ \sum X = \frac{l}{2} \Leftrightarrow $ система устойчива.

\textbf{ Задача: }
рассмотрим систему первого порядка:

% TODO img

Разомкнутая система устойчива (единственный корень лежит внутри единичного круга).
Замкнутая: $ \alpha_{\text{з}} = z - \frac{1}{2} + 1 = z + \frac{1}{2} \Rightarrow $ устойчива.

% TODO a bit

\[ ... = \frac{\cos \Omega - \frac{1}{2} - j \sin \Omega}{\frac{5}{4} - \cos \Omega } \]
$ \Omega = 0 \Rightarrow H_{\text{ раз }}(e^{j \Omega}) = 2 $; $ \Omega = \pi \Rightarrow H_{\text{ раз }}(e^{j \Omega}) = - \frac{2}{3} $; $ \Omega = \frac{\pi}{2} \Rightarrow H_{\text{ раз }}(e^{j \Omega}) = - \frac{2}{5} - \frac{4}{5} j $

% TODO img

Сумма пересечений левее $ -1 $ $ \sum X = 0 $.

\textbf{ Пример: критерий Михайлова. Граница устойчивости }

Пусть $ z_1 = - \frac{1}{2} $, $ z_2 = 1 $.

% TODO img

\[ \alpha(z) =  \]
\[ \alpha(e^{j \Omega}) = e^{2j \Omega} - \frac{1}{2} e^{j \Omega} - \frac{1}{2} = (\cos 2 \Omega - \frac{1}{2} \cos \Omega - \frac{1}{2}) + j (\sin 2 \Omega - \frac{1}{den}) \] % TODO

\[  \]
\subsubsection{  }

\begin{itemize}[noitemsep]
	\item 
	\item 
\end{itemize}
\end{document}