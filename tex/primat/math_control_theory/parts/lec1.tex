\documentclass[main.tex]{subfiles}
\begin{document}

\subsubsection{Введение в управление}

ТАУ \\
МТУ \\
САУ \\
ТСАУ \\
САР \\
...

Управление -- одна из самых широких и вместе с тем размытых категорий. Воздействие чего-то на что-то и есть управление.

Пример: в современном автомобиле много автоматического управления.

Будем понимать под \emph{управлением} всякое разрешённое воздействие $ u $ (обозначаем входящей стрелочкой) на некий объект (ОУ -- объект управления), которое приводит к желаемому результату.
Этот желаемый выход будем обозначать как $ y_d $; хотим: $ y \to y_d $.

На объект может действовать нежелательное воздействие (внешнее возмущение) $ W $.

\emph{Автоматические} системы: автомобиль с шофёром -- ручное управление, а мы рассматриваем автоматику.
Пример: система автоматической парковки. Т. о. мы рассматриваем системы, обеспечивающие заданный выход без вмешательства человека.

АСУ -- автоматизированные системы управления.
В чём отличие от автоматических?
Главным модулем всё-таки является человек, оператор (принимает окончательное решение).

\subsubsection{Бытовые примеры автоматического управления}

\begin{enumerate}[noitemsep]
    \item Часы (в том числе механические)
    \item Ток в сети: у нас $220$ В, $50$ ГЦ; за океаном $110$ В, $60$ ГЦ. 220 В опаснее, но меньше затраты на производство проводов. Если мы увеличим напряжение, ту же мощность можно получать при меньшем токе; значит, меньше будут потери при передаче электроэнергии. \\
    Напряжение поддерживать сложнее, чем частоту. \\
    Почти все лампочки перегорают в момент включения. Почему? Сопротивление холодной лампочки больше, чем горячей, и ток  $ I = \frac{U}{R} $ больше. \\
    Что такое 220 Вольт? Это среднее (эффективное) напряжение.
    $$ P = \frac{1}{T} \int_0^T P_{\text{мгновенн}} dt =  $$ % TODO write equation till end
    Какой постоянный ток $ U_0 $ даст ту же мощность?
    $$ U = \frac{U_0}{\sqrt 2} $$
    \item Холодильник (или нагревательный прибор) имеет \emph{термостат}, то есть ручка, с помощью которой мы поддерживаем постоянную температуру.
    \item Smart-круиз-контроль: держим дистанцию до впереди идущей машины.
\end{enumerate}

% TODO leftbar
% \begin{leftbar}
	Никогда не говорите <<я думаю>>, <<мне кажется>>	
% \end{leftbar}

\subsubsection{Четыре кита автоматического управления}

\begin{enumerate}
	\item Первый кит -- \textbf{обратная связь}.
	\begin{enumerate}[noitemsep]
		\item Нужно уметь измерять выход системы (например, скорость автомобиля).
		\item Нужно уметь сравнивать получаемый выход с желаемым
		\item Измеренная ошибка поступает в \emph{исполнительное устройство}, которое выдаёт сигнал.
	\end{enumerate}
	% TODO схему вставить
	Если обратной связи нет, говорят, что система \emph{разомкнутая}.
	\item Второй кит -- % TODO
	\begin{enumerate}[noitemsep]
		\item Объект управления
		\item Измерительные устройства
		\item Устройство управления (или регулирования) -- преобразует значение ошибки в управляющий сигнал \label{item:regulating}
		\item  Исполнительное устройство
	\end{enumerate}
	
	Наша наука занимается частью \ref{item:regulating}.
	\item Третий кит -- \textbf{независимость принципов управления от физической природы объектов управления}.
	
	Примеры:
	
	\begin{tabular}{m{0.2\linewidth} | m{0.2\linewidth} | m{0.2\linewidth} | m{0.2\linewidth}}
		\hline \hline
		ОУ & $ u $ & $ y $ & $ W $ \\
		\hline
		МС (механические) & $ F, H $ & $ q, q', q'' $ & $ F_{\text{тр}} $ \\
		ЭТС (электротехнические) & $ U, R, C $ & $ I, P, ... $ & $ R_H $ \\
		ТТС (теплотехнические) & $ T_{H_2 0} $ & $ T_{\text{ср}} $ & вирус и т. д. \\
		и прочие... & & & \\
		\hline
	\end{tabular}
	
	Невзирая на широту спектра ОУ, есть некая общность в управлении этими системами. Эту связь будем обозначать так: $ y = f(u, W) $ -- \emph{математическая модель}.
	
	Тезис: если системы могут быть описаны похожими ММ, ими можно управлять одинаково.
	
	Пример: $ m\ddot{x} + Cx = 0 $ -- уравнение колебаний грузика на пружинке, $ \ddot I + \frac{1}{LC}I = 0 $ -- LC-контур.
	
	Решения: $ x = x_0 \cos kt $; $ I = I_0 \cos kt $ \\
	
	
	Пример САУ на примере электрического утюга. Если в утюге есть только теплоэлемент, обратной связи нет (не можем регулировать температуру).
	Если ввести в конструкцию термостат (биметаллическая пластина, которая изгибается от температуры и размыкает цепь). Коэффициент теплового расширения меди примерно вдвое больше, чем железа. 
	
	\item Четвёртый кит -- \textbf{оптимальное управление}
	
	Рассмотрим задачу: как добраться из дома до Политехнического института?
	Можно доехать на такси, на общественном транспорте или пешком.
	Каждый вариант может быть оптимальный в определённых условиях.
	
	Мы будем формулировать некий скалярный функционал качества, который зависит от входных и выходных данных: $ J = J(u, y) $.
	
\end{enumerate}

\subsubsection{Бытовые задачи теории управления}

Задача 1. Пусть мы занимаемся сменой квартиры и у нас много мебели. Как правило, цена пропорциональна площади.

Пусть есть квадратная комната площади $ S $ и прямоугольная площади $ S $.
В какую поместится больше мебели? - В прямоугольную! Ведь мебель в основном пристенная.

Задача 2. Пусть есть книжная полка. Под действием груза (книг) она может сломаться, и сломается, наверное, по центру.

Оптимальное расположение опор -- не по краям, а ближе к центру. Какое оптимальное смещение к центру $ \xi $ (величина в интервале $ [0; l/2] $)? Если в центре полки разрыв, оптимум $ \xi^* = \frac{1}{4} $, но, поскольку его нет, то на самом деле меньше.

Во сколько в задаче 2 увеличивается прочность? В шесть раз!

Задача 3: на базе всё той же полочки решим геодезическую задачу.
Пусть на полочке стоят приборы, например, телескопы.
Под нагрузкой максимальный прогиб при крайнем расположении опор $ h_1 $ будет, конечно, в центре.
Хотим уменьшить прогиб.
Казалось бы, хотим сместить опоры, чтобы минимизировать максимальный прогиб.

Это \emph{минимаксная} задача; уже сложнее минимизации нагрузки.
Как думаете, оптимальное расположение опор в этой задаче такое же или другое?

Оказывается, решение примерно такое же ($ \approx \frac{1}{5} $), хотя задача формально другая.

Главный вопрос: как соотносится минимальный прогиб $h_2$ и прогиб при крайнем расположении опор $ h_1 $?

Оказывается, $ h_2 \approx \frac{1}{50} h_2 $!



\begin{enumerate}[noitemsep]
	\item 
	\item
\end{enumerate}

\subsubsection{}

\begin{enumerate}[noitemsep]
	\item
	\item
\end{enumerate}

\end{document}
