\documentclass[main.tex]{subfiles}
\begin{document}

$$ \frac{\alpha(D)y = \beta(D)U}{ } $$ % TODO

\subsection{Описание с помощью операторной передаточной функции}

В первом выражении разделим на $ \alpha(D) $; обозначим $ H(D) := \frac{\alpha(D)}{\beta(D)} $ -- \emph{операторная передаточная функция} (ОПФ).

У нас будет несколько передаточных функций; это -- операторная, т. к. зависит от оператора дифференцирования.

Пусть $ \beta(D) = (D - a), \alpha(D) = (D-a)(D-b) $. 
При этом нельзя сократить на $ (D - a) $, т. к. можем сократить неустойчивость. % TODO

\subsubsection{Преобразование Лапласа}

$$ \mathds{L}\{f(t)\} = \bar{f}(p) = \int_0^{+\infty} e^{-pt} f(t) dt $$
$$ \mathds{L}\{\dot f (t)\} = p \bar{f}(p)- f(0) $$

Устойчивые системы со временем <<забывают>> начальные условия, поэтому частая ситуация -- н. у. $= 0$

% TODO

$ \boxed{\bar{y}(p) = H(p) \cdot \bar{u}(p) }  (2b) $ % TODO

Итак, у нас появились два варианта записи:

$$ \xrightarrow{u(t)} \boxed{H(D)}\xrightarrow{y(t)} $$
$$ \xrightarrow{\bar u(t)} \boxed{H(D)} \xrightarrow{ \bar y(t)} $$

Преобразования некоторых функций: % TODO

\begin{enumerate}[noitemsep]
    \item
    \item
\end{enumerate}

\subsubsection{Реализуемость}

Напомним, $ \alpha(D)y = \beta(D) u $, порядок $ \alpha n $, порядок $ \beta m $; $ n < m \Rightarrow $ нереализуемая система.

Покажем это, используя преобразование Лапласа.

$$ \bar y = H(p) \bar u = \frac{\alpha(p)}{\beta} $$

Были некие начальные условия.
Хотим найти реакцию системы в момент $t^0$

$$ y(t^0) = \lim_{p \to \infty} $$ % TODO

\subsubsection{ Дельта-функция Дирака }

$$ \delta(t) = \begin{pmatrix}
	0, t \ne 0 \\
	\infty, t = 0
 \end{pmatrix} $$
+ условие нормировки: $ \int_{- \infty}^{+ \infty} \delta(t) dt $

Свойства:


\begin{enumerate}[noitemsep]
	\item Дельта-функция есть производная от функции Хевисайда.
	\item $ \int $ % TODO
\end{enumerate}

\subsection{Третий способ описания: интегральное описание (описание при помощи весовой функции)}

$$ f(t) = \int_0^t h(t - \tau) u(\tau) d \tau $$

% TODO

$$ y(t) = \int_{0}^{+ \infty}h(t - \tau)u(\tau) d\tau =  $$

Естественно, мы предполагаем, что н. у. $ =0 $.

Вторая форма удобна, если $ u = 1 $; третья, интегральная -- когда $ h $ есть простая функция.

\subsubsection{Физический смысл h(t)}

Найдём реакцию произвольной динамической системы, если на входе дельта-функция Дирака (и нулевые начальные условия).
Воспользуемся третим способом описания.

$$ y(t) = \int_0^t  $$

В соответствии с фильтрующим свойством $ y(t) = h(t-\tau)|_{\tau=\tau_0=0}=h(t)^0 $ -- весовая функция.

Какой физический смысл?
Если у нас быстрое узконаправленное воздействие (например, удар), то выход равняется весовой функции.
Узнав весовую функцию $ h(t) $, можем узнать преобразование Лапласа $ H(p) = \mathds{L}\{h(t)\} = \frac{\beta}{\alpha} \Rightarrow $ уже можем решать д. у.

Есть и иной подход, чтобы не потребовался удар:
$$ u = \mathds{L}[t]  $$ % TODO

$\Pi(t) = \int_0^t h(\tau)d\tau$ -- \emph{переходная} функция.

$ y(t) = \Pi[t],  \Rightarrow h(t) = \dot \Pi $

\subsubsection{}

\begin{enumerate}[noitemsep]
	\item
	\item
\end{enumerate}

\subsection{Четвёртый вид описания: описание в пространстве состояний (в фазовом пространстве)}

Формально \emph{фазовое пространство} -- это когда задана система дифференциальных уравнений первого порядка (в общем случае нелинейных) в нормальной форме.
У нас система линейная и уравнения линейные.

$$ \dot x = Ax + Bu $$
$$ y = cx $$

$ A, B, C $ -- числовые матрицы, $ c $ -- вектор,

Нет программных пакетов, которые умеют решать дифференциальные уравнения произвольных порядков, зато системы линейных -- можно.
Иметь дело с числовыми матрицами гораздо проще,

К тому же, свойство управляемости, задачи оптимизации и так далее -- всё это основано на матрицах.

\subsection{Эквивалентность и преобразование описаний}

\begin{enumerate}[noitemsep]
	\item $ \alpha(D)y = \beta(D)u $ \label{represent:1}
	\item  \label{represent:2}
	\begin{enumerate}[noitemsep]
		\item $ y(t) = \frac{\beta(D)}{\alpha(D)}u(t) = H(D)u(t) $
		\item $ \bar y(p) = \frac{\beta(p)}{\alpha(p)}u(t) = H(p) \bar u(t) $
	\end{enumerate}
	\item \label{represent:3} % TODO
	\item $ \dot y = Ax + Bu, y = cx $ \label{represent:4} % TODO system of equation
\end{enumerate}

\ref{represent:1}, \ref{represent:2}, \ref{represent:3} практически очевидно сводятся друг к другу.
Особняком стоит \ref{represent:4}

% TODO

Как получить $ (\ref{represent:1}) \to (\ref{represent:4}), (\ref{represent:2}) \to (\ref{represent:4}) $?

Преобразование $ (\ref{represent:1}) \to (\ref{represent:4}) $ \emph{множественное}: есть бесконечно много способов сделать такую трансформацию.
Поэтому оно сложнее (и оно важнее).
В обратную сторону -- преобразование Лапласа.

Покажем неоднозначность преобразования.
$$ \dot X = Ax + Bu, y = Cx $$
Сделаем следующее неособое преобразование $ x = Dz $ (то есть $ \exists D^{-1} \Rightarrow z = D^{-1}x $)
$$ D \dot z = ADz + Bu, y = CDz $$
$$ z = D^{-1}ADz + D^{-1}Bu, y = CDx $$
Матрицы $ A, A':=DAD $ называются \emph{подобными}. Обозначим также $B':=D^{-1}B, C':=CD$

Примеры:
\begin{enumerate}[noitemsep]
	\item $ \let x_1 = y \Rightarrow C = (1, 0, ...) $
	\item $ A = \begin{pmatrix}
	 0 & 1 & 0 \\ % TODO
	\end{pmatrix} $
	$ A $ есть матрица в форме Фробениуса.
\end{enumerate}

Осталось согласовать начальные условия.

Зачастую второй этап (перевод начальных условий) сложнее первого.

\subsubsection{ Метод последовательного понижения порядка }
Автор -- сам Александр Алексеевич.

Он легко запоминается, естественный; позволяет быстро получить преобразование, без матриц

\begin{enumerate}[noitemsep]
	\item
	\item 
	\item 
	\item 
	\item 
	\item 
\end{enumerate}

Пример 1: $ \ref{represent:4} \to \ref{represent:1} $ % TODO

Пример 2:



\section{Элементарные звенья}

\emph{Звено} -- часть системы, или соединение частей системы, или вся система в целом.
То есть всё, что угодно.

Напомним, мы разбивали на однонаправленные звенья...

\emph{Элементарное звено} -- звено, которое описывается дробью (передаточной функцией) не выше второго порядка.

Элементарные звенья: $ K, p, \frac{1}{p}, \frac{1}{Tp+1}, \frac{1}{T^2p^2 + 1}, ... $

\begin{enumerate}[noitemsep]
	\item Идеальный усилитель: $ \xrightarrow{u} \boxed{K} \xrightarrow{y} $, где $ K: n=m=0 $.
	Почему идеальный? В реальном усилителе на графике $ y(u) $ начиная с какого-то момента начинается нелинейная область, <<завал>>
	
	$ H(p) = K \Rightarrow h(t) = K \delta(t) $
	
	\item Дифференцирующее звено: $ \xrightarrow{u} \boxed{p} \xrightarrow{y} $, $ p: m=1, n=0 $. Это нереализуемое звено! (формально). 
	Зачем же оно нужно, если реализовать его нельзя?
	Полезно в связи с другими.
	
	$  $ % TODO
	
	\item Интегрирующее звено: 
\end{enumerate}

\begin{enumerate}[noitemsep]
	\item
	\item
\end{enumerate}

\end{document}
