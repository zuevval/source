\documentclass[main.tex]{subfiles}
\begin{document}
	
\section{ Частотные критерии устойчивости Михалкова и Найквиста }
5 марта 2021 г. \\

Темы важные, непростые.
Возможно, даже центральные! \\

Устойчивость -- основное свойство системы.
Как правило, все системы должны быть устойчивы.
Истребитель, в отличие от гражданского самолёта, в свободном полёте неустойчив, НО с помощью обратной связи мы вновь делаем его устойчивым!

Напоминание:

Для устойчивости линейной динамической системы достаточно условия $ \boxed{Re \lambda < 0} $, где $ \lambda : \alpha(\lambda) = 0 $

Проверять корни характеристического полинома -- занятие неблагодарное.
Есть необходимое условие Стодола, а также н. и д. условие Гурвица.
Их дополняет критерий Льенара-Шипара.

\subsubsection{Частотные критерии. Введение}

На прошлом занятии мы разобрали, что такое частотные характеристики.
Зачем они нужны, если есть алгебраические?
Когда порядок системы большой (10, 100, ...), задача вычисления большого числа определителей также неблагодарная.
А на базе некоторых характеристик, которые мы будем называть \emph{годографов}, можно определить устойчивость проще.
Причём будет достаточно найти годографы лишь приближённо.
Знак определителя не найти с помощью приближённых вычислений.

\subsubsection{Критерий Михайлова}
1938 год.

Частотные характериристики основаны на \textbf{принципе аргумента} из теории функции комплексной переменной: приращение аргумента функции при обходе области
$$ \Delta \arg f(z) = 2 \pi (N - P) $$
здесь $N$ -- число нулей и $ P $ -- число полюсов внутри области.

Докажем для полинома (нам этого достаточно):

Воспользуемся теоремой Безу.
$$ \alpha(p) = a_0 p^n + ... + a_n = a_0 \prod_i (p - p_i), \thickspace \alpha(p_i) = 0 $$
$$ H(p) = \frac{\beta(p)}{\alpha(p)} $$
% TODO img
Аргумент произведения есть сумма аргументов.
Аргумент $ a_0 $ равен нулю, если это положительное вещественное число.
$$ \arg \alpha(j \omega) = \sum_{i=1}^{n} \arg (j \omega - p_i) $$
$$ \delta \arg \alpha(j \omega) |_{-\infty}^\infty = \sum_i \Delta \arg (j \omega - p_i) $$

Изобразим корни уравнения на ком:
% TODO img2

Пусть в правой полуплоскости $ m $ корней $ \Rightarrow $ в левой -- $ n - m $ (общий порядок $n$).

От каждого вектора из левой полуплоскости приращение угла поворота при проходе $ \omega $ от $ - \infty $ до $ + \infty $ есть $ \pi $, от каждого правого -- $ \pi $
% TODO 1 formula
$$ $$
Доказано. \\

Нарисуем один из возможных годографов.
% TODO img
Геометрическое место точек, образованное значениями полинома $ \alpha(j \omega) $ (точнее, её половину), называют \emph{кривой Михайлова}.

Пример.
$$ \alpha(p) = p^2 + p + 1 $$
$$ \alpha(j \omega) = 1 - \omega^2 + j \omega $$
обозн. $ X(\omega) = 1 - \omega^2, \thickspace Y(\omega) = \omega $ % TODO
$$  $$

\begin{equation}\label{eq:mikhailov}
	\Delta \arg \alpha (j \omega) |_0^\infty = \frac{num}{den}
\end{equation}

Для устойчивости необходимо и достаточно, чтобы $ m = 0 $ (нет правых, неустойчивых корней).
\begin{equation}\label{eq:mikh1}
\boxed{}	% TODO
\end{equation}
Для асимптотической устойчивости также нужно, чтобы не было корней на мнимой оси:
\begin{equation}\label{eq:mikh2}
	\boxed{ \alpha(j \omega) \ne 0 }
\end{equation}
(т. о. годограф не должен проходить через начало координат).

Эти две формулы вместе и есть критерий Михайлова.
% TODO img3 годограф, img4 корни на мнимой оси

Объяснение:
если корень на мнимой оси, то приращение аргумента при её обходе -- $\pi$, но мы не знаем, с каким знаком.
То есть неопределённость: $ \Delta \arg j \omega = ? $

% TODO img5 годограф, проходящий через ноль

Это как деление на ноль -- если определим как предел, будет непонятно, какой знак: $ \frac{1}{0} = \pm \infty $

Замечания:

\begin{enumerate}[noitemsep]
	\item Для устойчивой системы кривая Михайлова имеет вид раскручивающейся спирали, которая всё время раскручивается в одном направлении и заканчивается в квадранте $ n $ (т. е. $ \Delta \phi = т\frac{\pi}{2} $).
	
	Пример.
	% TODO img7 годограф при n=5
	система устойчива.
	
	Почему спираль монотонно раскручивается?
	Изменение аргумента от всякого корня с $ Re < 0 $ есть монотонная функция, и сумма их тоже.
	Если  
	\item % TODO
	\item Если система неустойчива, то критерий Михайлова -- единственный, который позволяет найти $ m $.
	$m$ можно найти из \eqref{eq:mikhailov}
\end{enumerate}

Пример.
% TODO img8

Иной пример.
% TODO img9 неверный годограф
Вывод: приращение аргумента зависит от чётности порядка системы.
Если порядок системы нечётный, годограф должен заканчиваться в нечётном квадранте (вне зависимости от того, устойчива ли система).

Более конструктивный пример: инерционное звено
% TODO img10 годограф инерционного звена
\begin{align*}
	&  H(p) = \frac{1}{Tp + 1} \\
	& \alpha(p) = Tp + 1 \\
	& \alpha(j \omega) = 1 + j \omega T \\
	% TODO a bit
\end{align*}
% TODO img11 годограф звена
% TODO пример

Как с помощью критерия Михайлова идентифицировать систему, находящуюся на границе устойчивости?
Необходимое, но не достаточное условие: годограф проходит через ноль.
% TODO img12 годограф для границы усточивости при n=4
Пример: $ n=4 $.
Пошевелим немного годограф в одну сторону (красная кривая) $ \Rightarrow $ система становится устойчивой ($ \Delta \phi = 2 \pi $).
Пошевелим в другую $ \Rightarrow  \Delta \phi = 0$, система неустойчива.

Смотрите-ка, мы на границе устойчивости!
Если бы порядок был $ n = 6 $, то при смещении в одном случае (синяя кривая) было бы 3 неустойчивых корня с $Re \lambda > 0$; при смещении в другую сторону (красная кривая) был бы 1 неустойчивый корень.
\subsubsection{Примеры. Критерий Михайлова для передаточной функции замкнутой цепи}

Пример 1.
% TODO
\begin{enumerate}[noitemsep]
    \item $ \alpha_{\text{з}}() = Tp^2 + p + K $
    \item $ \alpha_{} $ 
\end{enumerate}

Пример 2. Требуется найти $ K_{\text{крит}} $
\begin{align*}
	& H_{\text{разомкн}} = \frac{K}{p}\\ % TODO
	& K_{\text{крит}} \\
\end{align*}

Система устойчива, если $ K < K_{\text{крит}} $.

\subsubsection{Критерий Найквиста}
Найквист (1932) -- американец.
Чуть посложнее.

Сравнение: Михайлов работает со знаменателем замкнутой цепи $ \alpha_{\text{з}}(j \omega) $, Найквиста -- с дробью разомкнутой $ H_{\text{разомкн}} = \frac{\beta(j \omega)}{\alpha(j \omega)} = \prod H_i $.
По сложности в среднем примерно одно и то же.

Сейчас получим :
$$ \Psi(p) = 1 + H_{\text{разомкн}} = 1 + \frac{\beta}{\alpha} = \frac{\alpha + \beta}{\alpha} = \frac{\alpha_{\text{з}}}{\alpha} $$
Порядок знаменателя $n$, числителя -- $ m \le n \Rightarrow $ порядок $  $ % TODO
$$ \Delta \arg \Psi() $$ % TODO

здесь $ l $ -- количество строго неустойчивых корней знаменателя $ H_{\text{разомкн}} $.

Cтроим годограф $ H_{\text{разомкн}} $ в осях $ Re( \Psi(j \omega) ), Im( \Psi(j \omega)) $

Т. о. для устойчивой системы необходимо и достаточно, чтобы годограф $ H_{\text{разомкн}} $ сделал $ \frac{l}{2} $ оборотов вокруг точки $ (-1;0) $ (это оси $ \Psi $)

Примеры.
% TODO img, img
Число неустойчивых корней знаменателя $ l = 0 $, годограф делает ноль оборотов вокруг точки $ (-1;0) $ (на обеих картинках), и 

\subsubsection{Геометрическое правило Цыпкина}
Будем считать число пересечений с лучом $ ( - \infty; -1 ]) $ на вещественной оси: если пересекаем из второго квадранта в третий, добавляем +1; из третьего во второй -- -1; если стартуем из точки на отрицательной вещественной полуоси, добавляем + или $ - \frac{1}{2} $ в зависимости от того, в какую сторону стартуем.
% TODO img

Пример.
$$ H_{\text{раз}} = \frac{2}{p-1}, \thickspace l = 1 $$
$$ H_{\text{раз}}(j \omega) = \frac{2}{j \omega - 1} = \frac{-2 - 2 j \omega}{\omega^2 + 1} $$
% TODO img
% TODO вывод из примера, проверка

\subsubsection{Критические случаи}
У критерия Михайлова нет особых случаев.
У критерия Найквиста есть.

\begin{itemize}[noitemsep]
	\item $ H_{\text{раз}} = \frac{\beta(p)}{p^s \alpha(p)} $ -- плохо, т. к.
	Тогда принцип аргумента применить нельзя, $ \Delta \arg (1 + H_{\text{раз}}(j \omega)) |_0^{\infty} $ -- величина не определённая.
	
	Как быть? % TODO запутался, не уверен в следующей строке.
   %	Если на мнимой оси $ l $ корней у $ \alpha_{\text{разомкн}} $, $ \Delta \arg \alpha(j \omega) |_{- \infty}^\infty = \pi (n - 2 l) $
   % TODO
   Обходим неприятную точку $ (0;0) $ справа (слева нам неудобно): % TODO
   Модификация принципа аргумента для знаменателя: $\Delta \arg \alpha(p) |_{\text{trajectory}}$
   
   $$ trajectory = \begin{cases}
   j \omega, \omega < - \varepsilon or \omega > \varepsilon \\
   \varepsilon e^{- j \frac{\pi \omega}{ 2 \varepsilon}}, \omega \in [- \varepsilon; \varepsilon]
   \end{cases} $$
   % TODO
   $$ \Delta \arg [1 + H_{\text{раз}}(p)] |_{traject} =  $$
   
   $$ H_{\text{раз}}(p) = \begin{cases}
   H_{\text{раз}}(j \omega), \omega > \varepsilon \\
   H_{\text{раз}}(p), p = % TODO a bit
   \end{cases} $$
   $$ H_{\text{раз}}(p) |_{traj, \varepsilon \to 0} = \begin{cases}
   content...
   \end{cases} $$
   
   Т. о. надо дополнить годограф дугой бесконечно большого радиуса с фазой $ - \frac{\pi}{2}s $ ($ s $ -- кратность нуля как корня знаменателя), которая должна начинаться на положительной полуоси.
   Найдя эту траекторию, считаем число пересечений:
   
   Итак, разрыв в нуле % TODO
   
   Для применения правила Цыпкина достаточно брать не бесконечно большую дугу, а любую с радиусом $ > 1 $.
   
   Пример: интегратор.
   $$ H_{\text{раз}} = \frac{1}{p} $$
   Годограф интегратора, напомним, таков:
   % TODO img
   
   $ s = 1, l = 0 $.
   Дополняем эту траекторию дугой.
   % TODO
   
   Ещё один пример:
   $$  $$
   
   Проверка: $ \alpha_{\text{з}} = p^{} $ % TODO
\item Второй критический случай: комплексно сопряжённые 
$$ H_{\text{раз}}(p) = \frac{\beta(p)}{\prod(p^2 + \omega_\mu^2) \alpha(p)} $$
% TODO img
<<Прыжок>> на $\pi$ или $-\pi$, разрыв

Выход: обходим корень по дуге бесконечно малого радиуса.
Тогда изменение аргумента вспомогательной функции от $p$ есть % TODO

Итак, если у нас второй критический случай (в знаменателе есть чисто мнимые корни), то дополняем годограф дугой бесконечно большого радиуса с радиусом $ - \pi \cdot \text{кратность корня} $, после чего уже можно пользоваться правилом Цыпкина.

Пример.
\begin{align*}
	& H_{\text{раз}}(p) = \frac{p}{p^2 + 1}, \thickspace l = 0  \\
	% TODO 
\end{align*}

То же самое, но со знаком минус:
\begin{align*}
& H_{\text{раз}}(p) = \frac{-p}{p^2 + 1}, \thickspace l = 0  \\
% TODO 
\end{align*}
\end{itemize}

\subsubsection{Границы устойчивости в критерии Найквиста}

Если в критерии Михайлова точка, подозрительная на границу устойчивости -- $ (0;0) $, то здесь -- $ (-1;0) $.
Как исследовать?
Точно так же: пошевелить.

Пример:
% TODO
\begin{align*}
	& \phi = 0 - \left[\frac{num}{den}\right] \\
	& \\ % TODO
	& K_{\text{крит}} = \frac{T_1 + T_2}{T_1 T_2} \text{ (совпадает с результатом, полученным по критерию )} \\
\end{align*}

\begin{itemize}[noitemsep]
	\item 
	\item 
\end{itemize}

\end{document}
