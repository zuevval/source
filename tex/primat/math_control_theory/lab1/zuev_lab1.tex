% compile with XeLaTeX or LuaLaTeX

\documentclass[a4paper,12pt]{article}
\usepackage[utf8]{inputenc} % russian, do not change
\usepackage[T2A, T1]{fontenc} % russian, do not change
\usepackage[english, russian]{babel} % russian, do not change

% fonts
\usepackage{fontspec} % different fonts
\setmainfont{Times New Roman}
\usepackage{setspace,amsmath}
\usepackage{amssymb} %common math symbols
\usepackage{dsfont}

% utilities
\usepackage{systeme} % systems of equations
\usepackage{mathtools} % xRightarrow, xrightleftharpoons, etc
\usepackage{array} % utils for tables
\usepackage{subfiles}
\usepackage{hyperref}
\hypersetup{pdfstartview=FitH,  linkcolor=blue, urlcolor=blue, colorlinks=true}
\usepackage{framed} % advanced frames, boxes
\usepackage{graphicx}
\usepackage{color}
\usepackage{float} % force pictures position
\usepackage{subcaption} % captions for subfigures


% styling
\usepackage{enumitem} % itemize and enumerate with [noitemsep]
\setlength{\parindent}{0pt} % no indents!

% misc
\graphicspath{{./img/}}
\newcommand{\myPictWidth}{.95\textwidth}
\newcommand{\phm}{\phantom{-}}

\begin{document}

\subfile{titlepage}

\section{Постановка задачи}
\subsection{Исходная постановка задачи}

Дана передаточная функция разомкнутой цепи:
\[ H_{\text{раз}}(p) = \frac{ k(T_1 p + 1) }{p (p^2 + \omega_0^2)(T_2 p + 1) (T_3 p + 1)} \]
где
\[ k = \frac{i}{4}, \omega_0 = i, T_1 = 0.01, T_2 = 0.02i, T_3 = 0.01i \]
($ i = 4 $ -- порядковый номер студента). \\

Определить устойчивость системы тремя способами:

\begin{enumerate}[noitemsep]
    \item По критерию Гурвица или Льенара-Шипара
    \item По критерию Михайлова
    \item По критерию Найквиста
\end{enumerate}

\subsection{Постановка задачи с конкретными числами}
\[ i = 4 \Rightarrow k = 1, \omega_0 = 4, T_1 = 0.01, T_2 = 0.08, T_3 = 0.04 \]
\[ H_{\text{раз}}(p) = \frac{ (0.01 p + 1) }{p (p^2 + 16)(0.08 p + 1) (0.04 p + 1)} \]

Определить устойчивость системы с подобной $ H_{\text{раз}} $ указанными выше способами.

\section{ Результаты расчётов }

\subsection{ Критерий Льенара-Шипара }

Преобразуем для удобства передаточную функцию:
\[ H_{\text{раз}}(p)  = \frac{25(p+100)}{8p(p^2+16)(p+12.5)(p+25)}  \]
Знаменатель передаточной функции замкнутой цепи есть сумма числителя и знаменателя разомкнутой:
\[ \alpha_{\text{замкн}} = \alpha_{\text{раз}} + \beta_{\text{раз}} = 8 p^5 + 3000 p^4 + 2628 p^3 + 4800 p^2 + 40025 p + 2500 \]
Обозначим коэффициенты $ \alpha_{\text{замкн}} $ как $ a_0, a_1, ..., a_5 $.
Тогда матрица Гурвица

$$
\Gamma_{5 \times 5}=
\begin{pmatrix}
    a_1 & a_3 & a_5 & 0   & 0 \\
    a_0 & a_2 & a_4 & 0   & 0 \\
    0   & a_1 & a_3 & a_5 & 0 \\
    0   & a_0 & a_2 & a_4 & 0 \\
    0   & 0   & a_1 & a_3 & a_5 \\
\end{pmatrix}=
\begin{pmatrix}
    3000 & 4800 & 2500  & 0     & 0 \\
    8    & 2628 & 40025 & 0     & 0 \\
    0    & 3000 & 4800  & 2500  & 0 \\
    0    & 8    & 2628  & 40025 & 0 \\
    0    & 0    & 3000 & 4800   & 2500 \\
\end{pmatrix}
$$

Необходимое и достаточное условие Льенара-Шипара:

\begin{enumerate}[noitemsep]
    \item Коэффициенты $ \alpha(p) $ $ a_i > 0 $ \label{item:stodola}
    \item Взятые через один главные угловые миноры матрицы Гурвица $ \Delta_{2k} > 0 $ (равносильное утверждение: $ \Delta_{2k-1} > 0 $) \label{item:lienard}
\end{enumerate}

Условие \ref{item:stodola} выполнено. Проверим условие \ref{item:lienard}.

Чётные миноры: $ \Delta_2 \approx 7.846 \cdot 10^6$, $ \Delta_4 \approx -1.296 \cdot 10^{16} \Rightarrow $ система неустойчива.

\subsection{ Критерий Михайлова }

Система устойчива $ \Leftrightarrow $ приращение аргумента годографа знаменателя передаточной функции замкнутой цепи
\[ \Delta \phi(\alpha_{\text{замкн}}(j\omega))|_{\omega = 0}^{\omega = \infty} = \frac{\pi}{2} n \]
$ n = 5 $ -- порядок полинома.

Построим годограф в осях $ \Re [ \alpha_{\text{замкн}}(j\omega) ] $, $ \Im [ \alpha_{\text{замкн}}(j\omega) ] $ (рис. \ref{fig:mikhailov}).


Видно, что приращение аргумента $ \frac{pi}{2} \ne 5 \frac{\pi}{2} \Rightarrow $ система неустойчива.

\begin{figure}[H]
    \centering
    \begin{subfigure}{.5\textwidth}
        \centering
        \includegraphics[width=\myPictWidth]{mikhailov}
        \caption{В крупном масштабе}
    \end{subfigure}%
    \begin{subfigure}{.5\textwidth}
        \centering
        \includegraphics[width=\myPictWidth]{mikhailov_near0}
        \caption{Вблизи нуля}
    \end{subfigure}

    \begin{subfigure}{.5\textwidth}
        \centering
        \includegraphics[width=\myPictWidth]{mikhailov_near0_1}
        \caption{Совсем вблизи нуля}
    \end{subfigure}%
    \begin{subfigure}{.5\textwidth}
        \centering
        \includegraphics[width=\myPictWidth]{mikhailov_near0_2}
        \caption{Совсем-совсем вблизи нуля}
    \end{subfigure}
    \caption{Годограф Михайлова}
    \label{fig:mikhailov}
\end{figure}

\subsection{ Критерий Найквиста }

Система устойчива $ \Leftrightarrow $ число пересечений годографа $ H_{\text{раз}}(p)|_{trajectory} $ с вещественной осью левее $ -1 $ равняется $ \frac{l}{2} l $, $ l $ -- число строго неустойчивых корней знаменателя, $ trajectory $ -- путь, на котором $ p $ пробегает значения от $ 0 $ до $ + \infty $ по мнимой оси (за исключением особых точек, где строится обходной путь). \\

В нашем случае $ l = 0 $.
Изобразим годограф в осях  $ \Re(H(p)), \Im(H(p)) $ (рис. \ref{fig:nyquist}), сделав дополнительные построения:
\begin{enumerate}[noitemsep]
    \item При $ p \to 0 $ дополним годограф дугой с фазой $ - \frac{\pi}{2} $ (поскольку ноль является простым корнем знаменателя).
    Можно взять любой радиус $ r > 1 $; положим, к примеру, $ r = 1.5 $.
    \item При $ p \to 4j $ дополним годограф дугой с фазой $ - \pi $.
\end{enumerate}

Число пересечений левее $ -1 $: $ \sum X = -1 \ne \frac{l}{2} \Rightarrow $ система неустойчива.

\begin{figure}[H]
    \centering \includegraphics[width=\myPictWidth]{nyquist}
    \caption{Годограф Найквиста}
    \label{fig:nyquist}
\end{figure}

\section{ Ответ }

Номер студента $ i = 4 $.

Ответ: система \textbf{неустойчивая}, как утверждают хором А. Льенар, М. Шипар, А. В. Михайлов, Г. Найквист и Я. З. Цыпкин.

\end{document}