% compile with XeLaTeX or LuaLaTeX

\documentclass[a4paper,12pt]{article}
\usepackage[utf8]{inputenc} % russian, do not change
\usepackage[T2A, T1]{fontenc} % russian, do not change
\usepackage[english, russian]{babel} % russian, do not change

% fonts
\usepackage{fontspec} % different fonts
\setmainfont{Times New Roman}
\usepackage{setspace,amsmath}
\usepackage{amssymb} %common math symbols
\usepackage{dsfont}

% utilities
\usepackage{systeme} % systems of equations
\usepackage{mathtools} % xRightarrow, xrightleftharpoons, etc
\usepackage{array} % utils for tables
\usepackage{subfiles}
\usepackage{hyperref}
\hypersetup{pdfstartview=FitH,  linkcolor=blue, urlcolor=blue, colorlinks=true}
\usepackage{framed} % advanced frames, boxes
\usepackage{graphicx}
\usepackage{color}
\usepackage{float} % force pictures position
\usepackage{subcaption} % captions for subfigures


% styling
\usepackage{enumitem} % itemize and enumerate with [noitemsep]
\setlength{\parindent}{0pt} % no indents!

% misc
\newcommand{\phm}{\phantom{-}}

\begin{document}

\subfile{titlepage}

\section{Постановка задачи}
\subsection{Исходная постановка задачи}

Дана линейная система второго порядка
\begin{equation}\label{eq:task}
    \begin{cases}
        \ddot x - \omega_0^2 x = u \\
        x(0) = x_0 \\
        \dot x(0) = \nu_0
    \end{cases}
\end{equation}

Требуется определить оптимальное управление, минимизирующее интегрально-квадратичный функционал
\begin{equation}\label{eq:func}
    J = \int_{0}^{+ \infty} \left( \dot x^2 + q \ddot x^2 + r \dot u^2 \right) dt \to \min_u
\end{equation}
а именно, найти $ u_{opt}(x, \dot x) $ и соответствующее значение $ J_{min} $.
В приведённой выше формуле
\[ \begin{cases}
    q = \frac{1}{i} \\
    r = \frac{1}{i^2} \\
    \omega_0 = \frac{i}{2} \\
    x_0 = 1 \\
    \nu_0 = i + 5
\end{cases} \]

$ i=4 $ -- номер студента.

\subsection{Постановка задачи с конкретными числами}

\[ \begin{cases}
    q = \frac{1}{4} \\
    r = \frac{1}{16} \\
    \omega_0 = 2 \\
    x_0 = 1 \\
    \nu_0 = 9
\end{cases} \]
\[ \begin{cases}
    \ddot x - 4 x = u \\
    x(0) = 1 \\
    \dot x(0) = 9
\end{cases} \]
\[ J = \int_{0}^{+ \infty} \left( \ddot x^2 + \frac{1}{4} \dot x^2 + \frac{1}{16} \dot u^2 \right) dt \to \min_u  \]
Найти $ J_{min} $ и выразить $ u_{opt} $.


\section{ Результаты расчётов }

Приведём систему к пространству состояний:
\[ ] x_1 := x, x_2 := \dot x \overset{\eqref{eq:task}}\Rightarrow \begin{cases}
    \dot x_1 = x_2 \\
    \dot x_2 = 4 x_1 + u
\end{cases} \]
Тогда
\[ A = \begin{pmatrix}
    0 & 1 \\
    4 & 0
\end{pmatrix}, B = \begin{pmatrix}
0 \\ 1
\end{pmatrix} \]
притом, как следует из \eqref{eq:func},
\[ Q = \begin{pmatrix}
    1 & 0 \\
    0 & q
\end{pmatrix} = \begin{pmatrix}
1 & 0 \\
0 & \frac{1}{4}
\end{pmatrix}, R = r = \frac{1}{16} \]

$ R > 0 $, $ Q > 0 $.
Проверим, является ли система \eqref{eq:task} полностью управляемой.
\[ A_u := (B, AB) = \begin{pmatrix}
    0 & 1 \\
    1 & 0
\end{pmatrix} \]
$ rank(A_u) = 2 \Rightarrow (A, B) $ -- невырожденная пара матриц $ \Rightarrow $ \eqref{eq:task} -- полностью управляемая система.

Значит, согласно теореме Лурье, существует единственное положительно определённое решение уравнения Риккати

\[ A^T P + PA - PB R^{-1} B^T P + Q = 0 \]

Притом
$ P = \begin{pmatrix}
    p_{11} & p_{12} \\
    p_{12} & p_{22}
\end{pmatrix} = P^T, R = r \Rightarrow $ уравнение можно переписать так:
\[ A^T P + (A^T P)^T - rPB(PB)^T + Q = 0 \]

Подставляем и решаем.

\[ A^T P = \begin{pmatrix}
    0 & 4 \\
    1 & 0
\end{pmatrix} \begin{pmatrix}
p_{11} & p_{12} \\
p_{12} & p_{22}
\end{pmatrix} = \begin{pmatrix}
4 p_{12} & 4 p_{22} \\
p_{11} & p_{12}
\end{pmatrix} \]
\[ (A^T P)^T = \begin{pmatrix}
4 p_{12} &  p_{11} \\
4 p_{22} & p_{12}
\end{pmatrix} \]
\[ PB = \begin{pmatrix}
    p_{11} & p_{12} \\
    p_{12} & p_{22}
\end{pmatrix} \begin{pmatrix}
0 \\ 1
\end{pmatrix} = \begin{pmatrix}
p_{12} \\ p_{22}
\end{pmatrix} \Rightarrow (PB)^T =  \begin{pmatrix}
p_{12} & p_{22}
\end{pmatrix} \]

\[ r^{-1} PB (PB)^T = 16 \begin{pmatrix}
    p_{12} \\ p_{22}
\end{pmatrix} \begin{pmatrix}
p_{12} & p_{22}
\end{pmatrix} = 16 \begin{pmatrix}
p_{12}^2 & p_{12} p_{22} \\
p_{12} p_{22} & p_{22}^2
\end{pmatrix} \]

Собираем всё воедино:
\[ \begin{pmatrix}
    4 p_{12} & 4 p_{22} \\
    p_{11} & p_{12}
\end{pmatrix} + \begin{pmatrix}
4 p_{12} &  p_{11} \\
4 p_{22} & p_{12}
\end{pmatrix} - 16 \begin{pmatrix}
p_{12}^2 & p_{12} p_{22} \\
p_{12} p_{22} & p_{22}^2
\end{pmatrix} + \begin{pmatrix}
1 & 0 \\
0 & \frac{1}{4}
\end{pmatrix} = \begin{pmatrix}
0 & 0 \\
0 & 0
\end{pmatrix} \]

Приходим к системе:
\[ \begin{cases}
    8 p_{12} - 16 p_{12}^2 + 1 = 0 \\
    4 p_{22} + p_{11} - 16 p_{12} p_{22} = 0 \\
    2 p_{12} - 16 p_{22}^2 + \frac{1}{4} = 0
\end{cases} \]

Система была решена с помощью пакета SymPy в среде Python3.9;
было найдено единственное положительно определённое решение (главные угловые миноры положительны):
\[ P^* = \begin{pmatrix}
        1.7071 &  0.6036\\
        0.6036 &  0.3018
    \end{pmatrix} \]

Тогда, если положить  $ \vec x := \begin{pmatrix}
    x_1 \\
    x_2
\end{pmatrix} $, то
\[ u_{opt} = - K \vec x, \thickspace K = R^{-1} B^T P^* \]
\[ J_{\min} = \vec x_0^T P^* \vec x_0, \thickspace \vec x_0 = \begin{pmatrix}
    x_1(0) \\
    x_2(0)
\end{pmatrix} = \begin{pmatrix}
x_0 \\
\nu_0
\end{pmatrix} = \begin{pmatrix}
1 \\
9
\end{pmatrix} \]
После подстановки и вычислений получаем:
\[ K = 16 \begin{pmatrix}
    0 & 1
\end{pmatrix} \begin{pmatrix}
1.7071 &  0.6036\\
0.6036 &  0.3018
\end{pmatrix} = \begin{pmatrix}
9.6569 &  4.8284
\end{pmatrix} \]
\[ J_{\min} = \begin{pmatrix}
    1 & 9
\end{pmatrix} \begin{pmatrix}
1.7071 &  0.6036\\
0.6036 &  0.3018
\end{pmatrix} \begin{pmatrix}
    1 \\
    9
\end{pmatrix} = 37.015 \]

Проверим систему с найденным оптимальным управлением на устойчивость.
Пусть $ K = \begin{pmatrix}
    k_1 & k_2
\end{pmatrix} = \begin{pmatrix}
9.6569 &  4.8284
\end{pmatrix} $.
Тогда уравнение \eqref{eq:task} имеет следующий вид:
\begin{align*}
    & \ddot x - \omega_0^2 x = u = - k_1 x - k_2 \dot x \Rightarrow \\
    & \ddot x - 4 x = -9.6569x - 4.8284 \dot x \Rightarrow \\
    & \boxed{ \ddot x + 4.8284 \dot x + 5.6569 x = 0 }
\end{align*}

Коэффициенты положительны $ \Rightarrow $ согласно условию Стодола, которое является достаточным для систем II порядка, система асимптотически устойчива.

\section{ Ответ }

Номер студента $ i = 4 $.

Ответ:
\begin{align*}
    & u_{opt} = - 9.6569 x - 4.8284 \dot x \\
    & J_{\min} = 37.015 \\
\end{align*}

\end{document}