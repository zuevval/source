% !TeX spellcheck = en_US
% !TeX program = xelatex

\documentclass[a4paper,12pt]{article}
\usepackage[utf8]{inputenc} % russian, do not change
\usepackage[T2A, T1]{fontenc} % russian, do not change
\usepackage[english, russian]{babel} % russian, do not change

% fonts
\usepackage{fontspec} % different fonts
\setmainfont{Times New Roman}
\usepackage{setspace,amsmath}
\usepackage{amssymb} %common math symbols
\usepackage{dsfont}

\usepackage{enumitem}

\begin{document}
	
\section{14.11.2024}

Наука:
\begin{itemize}[noitemsep]
	\item 
	\item 
\end{itemize}
- эпистема: эпистемология - знание о системе
- гносис: гносиология - познание
Современное понимание науки не очень однозначное.
Наука - в первую очередь система знаний; также система действий, ориентированных на познание; также практическая сила.

Научная рациональность - тоже достаточно общая конструкция. Подразумевает особый язык познания, объяснения.
Научное знание - также абстрактное понятие, так как речь не о конкретной науке, а об обобщённом варианте. Отвечает т.н. критериям истинности и обоснованности.
Ещё одно из исходных понятий науки - научный дискурс: язык описаний и явлений, который не совпадает с обыденным языком.

Начиная со 2 половины XXв науку стали рассматривать в том числе как социокультурный феномен. Раньше наука была обращена только к описанию природы, то к середине XXв появляется т.н. "большая" наука: работают не одиночки, а большие коллективы. Это влияет на отношение общества к науке и науки к обществу - их уже не оторвать друг для друга.

Одна из задач философии науки - согласование научного дискурса и научного метода познания.
---
Научная картина мира -- предмет философии науки.
Формирование картины мира важно для интеграции знания

Научная картина мира носит исторический характер, соответственно, имеет исторические ограничения. Как минимум 4 раза в истории происходила смена научной картины мира, но до XX века она базировалась на физической и космологической картине, а в XXв появляются более фундаментальные соображения.

\hrule

Философские принципы:
\begin{itemize}[noitemsep]
	\item онтологический принцип единства мира - исторически был заявлен в античную эпоху. Следствием стала одна из первых проблем науки - проблема стуктурного единства мира, поиск структурного основания, из которого проистекает всё многообразие наблюдаемых явлений.
	\item Принцип детерминизма: требует всегда иметь в виду причинную обусловленность явлений. Система Демокрита постулирует, что нет беспричинных явлений.
\end{itemize}

Философские основания науки:
\begin{itemize}[noitemsep]
	\item онтологические категории
	\item гносеологические категории
\end{itemize}

Идеальный уроверь реальности может распадаться на то, что существует независимо от вас (идеальный газ, модель маятника и т.д.) и второй уровень - то, что существует в воображении (феноменология, феноменологическая реальность): "у меня есть мысль, и я её думаю". Социальная феноменология: можем описывать искусственный интеллект; можем учёного, который мыслит.

--
Наука может оперировать структурами, субстратами и свойствами. Например, крастота - это абстрактное свойство, а не структура и не субстрат, но может исследоваться.

Функциональные системы.
Функция может рассматриваться как самостоятельный объект исследования.

Одна из первых изученных функциональных систем -- \emph{биосфера}.
Её устойчивое положение обеспечивается круговоротом живого вещества.
<<С точки зрения космических влияний биосфера - очень тонкая и очень устойчивая плёнка>>.

\hrule

XVII в. -- "век часов".
В основе часов принцип маятника.

Первая половина XIX в. -- паровые машины; возникает идея отделить \emph{позитивную} науку, т.е. науку, которая имеет утилитарное, практическое применение (то, что сейчас называют прикладными науками).
Позитивная наука отделяется от философии.

\subsection{Проблема начала науки}

Историю можно начинать с периода неолита.
<<Наука>> передаётся с помощью знаков, созданных орудий.
Но тогда научное знание неотличимо от любого другого.

Поэтому считается, что наука начинается в XVII-XVIII вв., когда появляются принципы и инструменты: наблюдение, расчёт, эксперимент (специально спланированная ситуация).
Например, гелиоцентрическая система была введена с помощью расчёта, то есть ввели не явно наблюдаемые, но принципиально наблюдаемые сущности.

\begin{itemize}[noitemsep]
	\item Пранаука -- традиционные культуры (Древний Восток, Египет)
	\item Протонаука (античная наука, базируется на умозрительном доказательстве)
	\item Преднаука (XV-XVI) -- Кеплер, Коперник, Галилей, Декарт, Ньютон. Декарт, Ньютон: как мы понимаем пространство, время и материю?
\end{itemize}

\subsection{Эволюция современной науки}
\begin{itemize}[noitemsep]
	\item Классический период
	\item Неклассический период (начало XX в): принцип дополнительности, принцип неопределённости (в микроявлениях нельзя пренебречь связью исследователя и исследуемого). Принцип детерминизма пришлось подвергнуть пересмотру
	\item Постнеклассический период (вторая половина XX в): формируются междисциплинарные познавательные стратегии.
	Формулируются принципы системности, эволюции, самоорганизации.
\end{itemize}

Исторический тип научной рациональности.
Философия науки называет \emph{научной рациональностью} стиль познавательной деятельности, в котором обязательно присутствует математическое доказательство (расчёт) и проверка.

\subsection{Глобальные научные революции}

\begin{enumerate}[noitemsep]
	\item XVII век -- становление экспериментальных и математических методов классической науки. Постулируется идеал построения истинной картины природы, основанной на наблюдении, а не на 
	\item конец XVIII - первая половина XIX века: можно исследовать, измерять разнородные процессы (например, тепловые и механические) одними законами. Основы современной химии, биологии; клеточная теория, теплородная теория горения. Микроскоп. Теория эволюции, Ламарк, Дарвин.
	\item конец XIX - начало XX века: изобретения и открытия в области физики, в частности, открытие атома, рентгеновское излучение, радиоактивность. Атом определяется как мельчайшая структура химического элемента, сохраняющая его свойства. Новое представление о пространстве-времени как четырёхмерном континууме; теория относительности, гипотеза нестационарной (расширяющейся) вселенной,
	\item Четвёртая научная революция началась во второй половине XX века и связана с пост-неклассическим типом научной рациональности. Т.н. матричный принцип организации.
\end{enumerate}

Это всё было "галопом по европам", потом будем возвращаться на конкретных примерах.

\end{document}