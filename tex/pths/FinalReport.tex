\NeedsTeXFormat{LaTeX2e}
\documentclass[a4paper,12pt]{article}
%\textwidth=16cm
%\oddsidemargin=0cm
%\topmargin=-2cm
%\textheight=23cm
\renewcommand{\bfdefault}{b}
\newcommand{\ovr}[1]{\overrightarrow{#1}}
\newcommand{\lk}{\guillemotleft}
\newcommand{\rk}{\guillemotright}

\usepackage[utf8]{inputenc}
\usepackage[english, russian]{babel}
\usepackage{amssymb,amsmath}
\usepackage{gensymb}
\usepackage{color}
\usepackage{graphicx}
\usepackage{subfiles}

\addto\captionsrussian{\def\refname{}}

\begin{document}

\subfile{Parts/Titlepage}
\setcounter{page}{2}
\subfile{Parts/Intro}

\section {Методика исследований}
\subfile{Parts/PhotolumTest}
\subfile{Parts/Hall}


\section {\textbf{Результаты и выводы}}
\begin{tabular}{| p{.25\textwidth} | p{.25\textwidth} | p{.25\textwidth} | p{.25\textwidth} |}
\hline
Номер структуры&Параметр&Расчётное значение&Реальное значение \color{red} С погрешностями?\color{black}\\
\hline
&&&\\
\hline
\end{tabular}
\newpage

\section{Использованная литература}
\vspace{-25 pt}
\begin{thebibliography}{99}
\bibitem{barrier}{Левинштейн, М. Е. Барьеры (От кристалла до интегральной схемы) / М. Е. Левинштейн, Г. С. Симин. - М.: Наука, 1987. - 320 с. - (Б-чка \lkКвант\rk, Вып. 65.)}
\bibitem{hall}{Гладышев А. Г. Исследование электрофизических свойств полупроводниковых слоев (n- и p- тип) и структур с двумерным электронным газом методом Холла и вихретоковым методом: методические указания по исследовательской работе / А. Г. Гладышев. - СПб.: СПбАУ НОЦ нанотехнологий РАН, 2010. - 40 с.}
\bibitem{0}{Денисов Д. В. Эпитаксиальный синтез базовых полупроводниковых материалов: методические указания по исследовательской работе / Д. В. Денисов. - СПб.: СПбАУ НОЦ нанотехнологий РАН, 2010. - 21 с.}
\bibitem{soros2}{Агекян В. Ф. Фотолюминесценция полупроводниковых кристаллов / В. Ф. Агекян // Соросовский образовательный журнал. - 2000. №10. - С. 101-107.}
\bibitem{soros1}{Беляевский В. И. Физические основы полупроводниковой нанотехнологии / В. И. Беляевский // Соросовский образовательный журнал. - 1998. №10. - С. 92-98.}
\end{thebibliography}
\newpage
\section{Благодарности}
Авторы работы благодарят:\\
\begin{itemize}
\item{} Научного руководителя \bfseriesКсению Юрьевну Шубину\normalfont\
за своевременную поддержу, заботу и подробные объяснения;
\item{} Преподавателя по практике \bfseriesНиколая Михайловича Химина\normalfont\ 
за организационную работу и помощь в подготовке отчёта;
\item{} Сотрудника лаборатории наноэлектроники НОЦ СПбАУ РАН \bfseriesМаксима Сергеевича ...\normalfont\ за популярно изложенные основы эпитаксиальной технологии и навигацию в запутанном мире сложных научных материяй.
\end{itemize}
\end{document}
