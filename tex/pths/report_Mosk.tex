\NeedsTeXFormat{LaTeX2e}
\documentclass[a4paper,12pt]{article}
\pagestyle{empty}
\textwidth=16cm
\oddsidemargin=0cm
\topmargin=-1cm
\textheight=23cm
\renewcommand{\bfdefault}{b}
\newcommand{\lk}{\guillemotleft}
\newcommand{\rk}{\guillemotright}
\renewcommand{\thesection}[1]{
\vspace{5 pt}
\Large
\noindent \textbf{{#1}}
\normalsize
\par
}


\usepackage[utf8]{inputenc}
\usepackage[english, russian]{babel}
\usepackage{amssymb,amsmath}
%\usepackage{ulem}\normalem
\usepackage{graphicx}
%\userpackage{color}
\usepackage{longtable}
\addto\captionsrussian{\def\refname{Использованные источники:}}

\begin{document}
\thesection{Купеческие места в Москве}
\begin{enumerate}

\item{Покровский бульвар:} На этой улице и прилегающих переулках уже в XIX столетии строили себе дома богатейшие купцы: Савва Морозов, Хлебниковы, Оловянишниковы, Расторгуевы, Бахрушины... Близ Покровского бульвара, однако, располагалось "самое туманное место Москвы" - сплошь криминальный Хитров рынок, большая площадь с ночлежками, пристанище оборванцев, торговок объедками и преступников всех мастей. Большинство беглых каторжников в Москве задерживали именно на Хитровке. Домовладельцы возмущались страшным соседством, но сделать ничего не могли - главари Хитровки имели рычаги влияния в Думе.
\item{Сандуновские и Центральные бани:} Бани были неотъемлимой частью жизни любого русского поселения вообще и Москвы в частности.\\
Хлудов - купец, прославившийся своими кутежами в конце 19 века - построил Центральные бани, приносившие ему немалые барыши. Скряга Фирсанов, жалевший каждую копейку, владелец Сандуновских бань, был однажды обижен - кто-то сказал ему - А вот у Хлудовых Центральные бани выстроены! Хлудовых надо перешибить!
Купеческое самолюбие было задето; из Вены был выписан архитектор, и вместо типичных для бань одноэтажных деревянных домиков вырос дворец.
\item{Гостиный двор}: На Ильинке (в Китай-городе) ещё с 16 века велась торговля, были расположены склады. При Иване Грозном деревянный Гостиный Двор заменили на каменный; современное здание было построено по проекту Джакомо Кваренги в 1830 году.

\item{Дом Пороховщикова:} Деревянный дом с каменным полуподвалом был построен по заказу славянофила А. А. Пороховщикова в 1871-1872 годах. В 1880-х годах в особняке размещается «Общество воспитательниц и учительниц с бесплатной школой коллективных уроков по естествознанию и математике, иностранным языкам, пению», здесь читали лекции физиолог И. М. Сеченов, зоолог М. А. Мензбир, энтомолог К. Э. Линдеман. 5 марта 1880 здесь на средства меценатов открыта женская Воскресная школа, библиотека и педагогический музей. С конца 1890-х годов здание передано под жильё для состоятельных людей.

\item{Особняк Морозовых:} Морозовы - текстильные фабриканты, благотворители, коллекционеры. Особняк был построен в 90-х гг. 19 века в стиле эклектики с преобладанием английской неоготики. Интерьеры разрабатывал М. Врубель.\\
В особняке Морозов некоторое время укрывал революционера Баумана. В 1905 г. после самоубийства Саввы Морозова дом был продан купцам Рябушинским - банкирам, фабрикантам, общественным деятелям.

\item{Купеческий клуб}\\
В Москве на рубеже XIX-ХХ веков по ночам карточная игра шла в  Английском, Охотничьем, Немецком и Купеческом клубах, а также в Литературно-художественном кружке. В каждом клубе состояло от 300 до 800 членов. Существовали членские взносы, но, на самом деле, доходы заведений пополнялись главным образом от игры; накопленных капиталов хватало на роскошные помещения. Официально считалось, что в клубах играли "по маленькой", однако, по слухам, именно в Купеческом клубе Михаил Морозов проиграл за ночь миллион рублей.\\
Купеческий клуб был создан в 1786 году, и до 1909 г. находился в бывшем особняке графа П.С.Салтыкова на Большой Дмитровке, где после расположилось кабаре Максим.\\
Надо сказать, что как-то упорядочить и поставить под контроль азартные игры в России пытались постоянно. Например, по указу Александра II воспрещалось продавать карты иностранные и игранные. Всякий раз играли только новыми картами. На территории России допускались к использованию только карты, изготовленные на государственной (Императорской) фабрике, и все доходы от продажи передавались на содержание приютов для детей-сирот. Использованные карточные колоды запечатывались и отправлялись из клубов для уничтожения. С 1905 года, по новым правилам, была введена обязательная перемена карт в 12 часов ночи, а затем через 4 часа от начала игры.

В 1888 году генерал-губернатор распорядился о воспрещении в московских клубах карточной игры под названием "железная дорога" (в просторечии, "железка"). Но с наступлением глубокой ночи в "железку" играли на самые крупные ставки. В 1897 году в Купеческом клубе ежегодная выручка от карточных игр возросла и составила по 200 и более тысяч рублей в год. В 1908 году - повторный строжайший приказ о запрещении "железной дороги", который тоже не возымел никаких результатов. 29 января 1909 года в Купеческий клуб внезапно нагрянула полиция. Был составлен акт о том, что в "железку" играли 30 человек. Клуб был закрыт по приказу московского генерал-губернатора на 1 месяц.

В самом начале 1900-х у Купеческого клуба истекал срок аренды здания на Большой Дмитровке, и его владельцы (тоже члены клуба) Бахрушины подняли годовую арендную плату с 14 до 36 тысяч рублей. Было решено, что надо завести собственный дом, благо к тому времени у клуба накопился капитал более, чем в 500 тысяч рублей. В 1909 г. новое здание в стиле модерн приняло первых посетителей. 

Членами клуба были тогда представители новой генерации купечества с блестящим образованием, знанием иностранных языков: Прохоровы, Мамонтовы, Карзинкины, Третьяковы, Кузнецовы, Боткины, Алексеевы, Солдатенковы. Интересно, что этот клуб охотно посещали и представители московской аристократии: Долгоруковы, Апраксины, Волконские, Зубовы, Трубецкие. Гостями бывали такие известные люди как композитор Н.Г.Рубинштейн, адвокат Ф.Н.Плевако, артист М.С.Щепкин и даже К.П.По\-бе\-до\-нос\-цев. Членство в клубе было пожизненным.



\end{enumerate}
\end{document}