\NeedsTeXFormat{LaTeX2e}
\documentclass[a4paper,12pt]{article}
%в две колонки - \documentclass[a4paper, twocolumn,12pt]{article}
\textwidth=16cm
\oddsidemargin=0cm
\topmargin=-1cm
\textheight=25cm
\newcommand{\lk}{\guillemotleft}
\newcommand{\rk}{\guillemotright}

\usepackage[utf8]{inputenc}
\usepackage[english, russian]{babel}
\usepackage{amssymb,amsmath}
\usepackage{gensymb}
\usepackage{array}

\begin{document}
Здравствуйте. Мы хотим рассказать вам о том, как мы разными способами изучали полупроводниковые структуры, зачем мы это делали и что у нас в итоге получилось. \\
Наша практика проходила в лаборатории наноэлектроники Академического университета. В университете методом молекулярно-пучковой эпитаксии производятся полупроводниковые гетероструктуры, то есть многослойные конструкции, выращиваемые на цельной монокристаллической подложке. В установке МПЭ есть камера сверхвысокого вакуума (1), где расположены полупроводниковые пластины (2). Снизу установлены эффузионные источники (7) - тигли с элементами, которые испаряются и поступают на пластину; пластина в держателе вращается, и потоки элементов поступают на её поверхность равномерно. В каждой ячейке есть атомы одного элемента - например, алюминия, мышьяка или галлия. Чем выше температура ячейки, тем больше атомов поступает на подложку и тем выше будет молярная доля элемента в будущем твёрдом растворе. На практике подложку нагревают выше точки кипения мышьяка, но ниже точки плавления галлия, и дают мышьяку поступать в избытке. Если атом мышьяка встретит атом галлия, он встроится в решётку, если нет - десорбируется и осядет на стенке реактора. Управление составом и скоростью роста структуры осуществляется путём регулировки температуры ячеек с галлием и другими элементами второй, третьей и четвёртой групп.\\ 
Нам было дано пять структур, и мы должны были измерить такие их параметры, как плотность поверхностных дефектов (механические дефекты могут сильно влиять на проводимость),в одной из них - гетероструктуре - истинную толщину, скорость роста и мольные доли элементов в отдельных слоях. Для двух других - легированных донорной примесью  кремния или акцепторной примесью бериллия - имело смысл измерить подвижность и концентрацию свободных носителей заряда.\\
Данная работа не уникальна, но нужна. Подобные измерения и расчёты, как правило, проводятся при всяком производстве структур. Вначале примерно рассчитывают, как долго, при какой температуре и какие эффузионные ячейки надо держать открытыми, чтобы на поверхности подложки выросли слои с нужными электрофизическими характеристиками. Далее выращивают опытные образцы структур и проверяют на соответствие; по результатам проверки параметры роста коректируются. Процесс повторяется до тех пор, пока результаты не будут признаны удовлетворительными; тогда производится нужное количество структур, и они вновь проходят проверку.\\ 
(Поверхностные дефекты 2 слайда)\\
(Фотолюминесценция 2 слайда)\\
(Холл)\\
В процессе работы было установлено, что значения измеренных параметров приблизительно соответствуют ожидаемым. Это значит, что температуру эффузионных ячеек изменять не нужно. Единственное существенное отклонение - плотность дефектов на пластине R2469A более чем вдвое превысила допустимую норму. Поскольку все структуры выращивались в одинаковых условиях, скорее всего, дефекты обусловленны плохим качеством подложки.\\
Спасибо за внимание!\\
\end{document}