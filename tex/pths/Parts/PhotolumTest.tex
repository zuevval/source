\documentclass[../FinalReport.tex]{subfiles}
\begin{document}


\subsection{Измерение спектров фотолюминесценции и определение скоростей роста слоёв различного состава.}
\normalfont

\subsubsection{Описание эксперимента:}
Структура под номером R2469A была исследована при помощи установки автоматического измерения спектров фотолюминесценции (далее - ФЛ) RPMSigma. Прибор определяет интенсивность вторичного излучения от кристаллов полупроводника в некотором диапазоне, где ожидаются максимумы интенсивности; для этого сигнал проходит через систему фильтров, непрозрачных для первичного (возбуждающего) излучения, после чего попадает в монохроматор, содержащий вращающуюся дифракционную решётку (300 штрихов / мм), разлагающая спектр на узкие по длине волны промежутки.

\begin{figure}[h!]
\center{\includegraphics[width=\textwidth]{Pictures/SurfScan.jpg}}
\caption{Принципиальная схема установки измерения спектров ФЛ: 1 - возбуждающий лазер, 2 - оптоволокно, 3 - фокусирующая система, 4 - образец, 5 - система фильтров, 6 - фокусирующая система, 7 - монохроматор}
\end{figure}

В процессе работы структуру последовательно облучали инфракрасным ($\lambda_1=778 нм$) и ультрафиолетовым ($\lambda_2=266 нм$) лазерами.  В результате были получены графики зависимости напряжения на фотодетекторе (пропорционально интенсивности ФЛ) от длины волны от длины волны излучения (пропорционально энергии фотона).\\

\begin{figure}[h!]
\center{\includegraphics[width=\textwidth]{photolumData/peacs1.jpg}}
\caption{Зависимость интенсивности ФЛ от длины волны ФЛ на примере пластины R2469A и возбуждающих лазеров с длинами волн 266 нм (слева) и 778 нм (справа).}
\end{figure}
\subsubsection{Теоретическое обоснование:} Процесс фотолюминесценции заключается в поглощении фотонов с энегией, большей или равной ширине запрещённой зоны материала $E_g$ и испускании фотонов с энергией, равной $E_g$. \\
Исследуемый образец является комбинацией прямозонных полупроводниковых слоёв, т. е. в которых дно зоны проводимости расположено строго над потолком валентной зоны и потому разрешены прямые переходы \lkзона-зона\rk\- с испусканием фотона, энергия которого точно равна E$_g$.\\
%Вопрос к КЮ: Хорошо, а что тогда такое постоянная решётки?
%Вопрос к КЮ и Паше: я правильно понимаю физику явления - оси энергия/координата?
%Здесь надо знать, что есть непрямозонные полупроводники (напр., фосфид галлия). В таких материалах всё сложнее.
%Здесь надо объяснить форму кривой на графике.
Для определения молярного состава тройного твёрдого раствора был использован закон Вегарда - эмпирически выведенное правило, утверждающее, что при постоянной температуре в твёрдых растворах есть линейная зависимость между свойствеми кристаллической решётки твёрдого раствора и концентрацией отдельных его элементов:

\begin{equation}
\label{vegard}
a_{In_{y}Ga_{1-y}As}=x\cdot a_{InAs} + (1-x)\cdot a_{GaAs}
\end{equation}
%(\ref{vegard})
где x - молярная концентрация, a (в данном случае)  - постоянная решётки соединения.
%Вопрос к КЮ: что такое поатоянная решётки?
\begin{equation}
E_{ph}=\frac{hc}{\lambda_{GaAs}}
\end{equation}
\color{red}
%Возможный вопрос: 1 - А как могут излучать слои, находящиеся в глубине? Ведь верхние слои непрозрачны!
%Возможный ответ от КЮ: Верхние слои оказываются как раз-таки прозрачными. Есть такое понятие как "край фундаметального (собственного) поглощения". Это значит, что поглощает полупроводник только те фононы, энергия которых  больше, либо равна ширине запрещенной зоны (условие для перехода электронов из валентной зоны в зону проводимости), если говорить о длинах волн, то все наоборот, т.к. длина волны обратно пропорциональна величине энергии фотона. Таким образом, если у нас самым нижним из исследуемых слоев (не считаем подложку, буферный слой) является In{x}Ga{1-x}As, у которого на спектре ФЛ позиция пика соответствует 975нм, затем GaAs (845,1нм) и наш Al{x}Ga{1-x}As (730,3 нм), то соответственно GaAs и Al{x}Ga{1-x}As будут прозрачны для света, излучаемого In{x}Ga{1-x}As, т.к. длина волны света, излучаемого им, больше, чем длина волны, соответствующая краю фундаментального поглощения и благодаря МПЭ, присутствие нежелательных примесей, которые могли бы поглощать свет с этой длиной волны, исключено. 
\color{black}

\subsubsection{Полученные данные:}
\begin{flushleft}
\begin{tabular}{p{.25\textwidth} | p{.28\textwidth} | p{.28\textwidth}}
Слой&Ожидаемая длина волны в максимуме&Реальная длина волны в максимуме\\
\hline
\vspace{5 pt}$In_{y}Ga_{1-y}As$&\vspace{5 pt}970 нм&\vspace{5 pt} 975.3 нм\\
GaAs&850 нм&843.1 нм\\
$Al_{x}Ga_{1-x}As$&750 нм&730.3 нм\\
\end{tabular}
\end{flushleft}

\subsubsection{Обработка данных и расчёты:}

\begin{figure}[h!]
\center{\includegraphics[width=\textwidth]{Pictures/caliber.jpg}}
\caption{Калибровочные графики}
\end{figure}

Значения положений пиков переводятся из нм (длина волны) в Эв (энергия фотона). $$E_p\approx1000\cdot \frac{1240}{\lambda_{Ga}}$$\\
Молярная доля AlAs определяется по формуле: $$x_{Al}=\frac{E_0-1.424}{0.0127}=0.22$$\\
%(Бред какой-то) С помощью калибровочного графика (зависимость (при н. у.) энергии испускаемого слоем GaAs фотона от толщины слоя в нм и проведёнными изолиниями молярных долей GaAs) определяется молярная доля алюминия в твёрдом растворе $Al_{x}Ga_{1-x}As$:\\ $x_{Al}=0.22$.\\
По калибровочному графику для GaAs - квантовой ямы (зависимость энергии фотона в ФЛ от слоя с содержанием алюминия с проведёнными изолиниями долей AlAs в тройном растворе от толщины слоя GaAs)определяется толщина слоя GaAs:\\ $d_{GaAs}=98.9298\phantom{-}\mbox{\AA}$\\
%А как были получены калибровочные графики?\\
$t_{AlGaAs}=55.6\phantom{-}\mbox{сек}$ - время роста алюминийсодержащего слоя (время, в течение которого была открыта заслонка эффузионной ячейки с алюминием)\\
Зная последовательность расположения слоёв в структуре, качественный состав каждого слоя и время роста соединений, найдём скорость роста GaAs:\\
$$v_{GaAs}=d_{GaAs}/t\approx1.78\phantom{-}\mbox{\AA/с} \phantom{-}$$
$v_{GaAs}$ - не только скорость роста чистого слоя GaAs, но и эффективная скорость роста арсенида галлия на протяжении всего роста, поэтому верна следующая формула, выведенная из закона Вегарда:
$$v_{AlAs}=v_{GaAs}\cdot\frac{x_{Al}}{1-x_{Al}}=1.78\cdot\frac{0.22}{0.78}\approx0.50\phantom{-}\mbox{\AA/с}$$
Также\\
$t_{InGaAs}=55.3\phantom{-}\mbox{сек}$\\
И, следовательно,
$$d_{InGaAs}=v_{GaAs}\cdot t_{InGaAs}=1.78\cdot55.3=98.39\phantom{-}\mbox{\AA}$$
По другому калибровочному графику (зависимость толщины слоя InGaAs от эффективной толщины InAs в InGaAs) определяем:\\
$d_{InAs}=20.67\phantom{-}\mbox{\AA}$\\
А зная время, которое была открыта заслонка эффузионной ячейки с индием и толщину индийсодержащего слоя, найдём скорость роста и этого слоя:
$$v_{InAs}=d_{InAs}/t_{InGaAs}\approx0.37\phantom{-}\mbox{\AA/с}$$
$$v_{InGaAs}=v_{InAs}+v_{GaAs}=2,15\phantom{-}\mbox{\AA/с}$$
$$$$
%Затем на основании закона Вегарда (\ref{vegard}) был установлен элементный состав подложки и определена скорость роста.

\end{document}