\documentclass[../FinalReport.tex]{subfiles}
\begin{document}

\subsection{Вычисление подвижности и концентрации свободных носителей заряда модифицированным четырёхточечным методом Ван-Дер-Пау.}

Структуры №R2329A (GaAs:Si) и №V85 (GaAs:Be) были исследованы контактным разрушающим методом Ван-Дер-Пау, основанном на эффекте Холла.\\

\subsubsection{Описание эксперимента:}
От исследуемой пластины скальпелем откалывается прямоугольник размерами приблизительно $10\times10$ мм. В углах прямоугольника паяльником ставятся индиевые точки, после чего образец приблизительно 30 секунд отжигается в печи при температуре $220\degree\-C$ ($593\degree\-K$), для того чтобы индий продиффундировал в приповерхностный слой структуры. Затем образец припаивается индиевыми точками к клеммам специальной платы-переходника.\\
На установке Ecopia HMS-3000 производятся измерения: через противоположные контакты пропускается ток, и одновременно между другими двумя измеряется разность потенциалов. Далее процесс повторяется, но контакты, бывшие измерительными, служат для пропускания тока и наоборот. Измерения проводятся в отсутствие магнитного поля, после чего повторяются в магнитном поле сперва одной, затем противоположной ориентации с известной индукцией $B=0.55$ Тл (поле создаётся постоянным магнитом). Затем установка автоматически определяет концентрацию и подвижность носителелей заряда.

\subsubsection{Теоретическое обоснование:}
Рассмотрим полупроводник - кристалл, в которомПусть $\tau_0$ - среднее время между двумя соударениями свободного электрона или дырки с атомами кристаллической решётки, $m$ - масса электрона, $\overline{e}$ - заряд электрона. Тогда \emph{подвижностью} электронов и дырок будем называть такой параметр $\mu$, что
\begin{equation}
\mu=\frac{\overline{e}\tau_0}{m}
\end{equation}
Поясним это на примере электрона. В электрическом поле напряжённости $\ovr{E}$ свободный электрон  ускрояется под действием электрической составляющей силы Лоренца:
$$\ovr{a}=\frac{\ovr{F}}{m}=\frac{\overline{e}\ovr{E}}{m}$$
Тогда максимальная скорость, набранная электроном в свободном пробеге,
$$\ovr{v}_{max}=\ovr{a}\cdot\tau_0=\frac{\overline{e}\tau_0}{m}\cdot\ovr{E}=\mu\ovr{E}$$

\begin{figure}[h]
\begin{center}
\begin{minipage}[h]{0.49\linewidth}
\includegraphics[width=\linewidth]{Pictures/hall2.jpg}
\end{minipage}
\hfill
\begin{minipage}[h]{0.49\linewidth}
\includegraphics[width=\linewidth]{Pictures/hall1.jpg}
\end{minipage}
\caption{Слева: исследуемая пластина. Справа: Образцы, отколотые от пластин № R2329A и V85 c нанесёнными индиевыми контактами}
\end{center}
\end{figure}

\begin{figure}[h!]
\center{\includegraphics[width=.7\textwidth]{Pictures/hallHeat.jpg}}
\caption{Установка для отжига образцов}
\end{figure}

\subsubsection{Полученные данные:}
\begin{tabular}{| p{.20\textwidth} | p{.20\textwidth} | p{.20\textwidth} | p{.20\textwidth} | p{.20\textwidth} |}
\hline
\vspace{0 pt} Структура&\vspace{0 pt} Сила тока&\vspace{0 pt} Концентрация свободных носителей заряда $p$, см$^{-3}$& \vspace{0 pt} Подвижность $\mu$, $\frac{\mbox{см}^{2}}{\mbox{В}\cdot\mbox{с}}$& \vspace{0 pt} Удельное сопротивление, Ом$\cdot$см\\
\hline
R2329A&1.0 mA&$-3.309\cdot10^{18}$&$1.958\cdot10^3$&$9.633\cdot10^{-4}$\\
\cline{2-5}
&0.1 mA&$-3.307\cdot10^{18}$&$1.968\cdot10^3$&$9.593\cdot10^{-4}$\\
\hline
V85&1.0 mA&$1.354\cdot10^{18}$&$1.631\cdot10^2$&$2.827\cdot10^{-2}$\\
\cline{2-5}
&0.1 mA&$1.363\cdot10^{18}$&$1.627\cdot10^2$&$2.814\cdot10^{-2}$\\
\hline
\end{tabular}
\end{document}
