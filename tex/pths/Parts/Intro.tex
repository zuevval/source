\documentclass[../FinalReport.tex]{subfiles}
\begin{document}
\section*{Аннотация}
\emph{В ходе практической работы были измерены электрофизические параметры тестовых полупроводниковых структур, \color{red}проведено сравнение с ожидаемыми значениями и сделаны выводы (это пока желаемое выдаётся за действительное). \color{black}}
\newpage
\tableofcontents
\newpage
\section{Введение}
Молекулярно-пучковая эпитаксия (далее - МПЭ) - технологический процесс, при котором в условиях высокого вакуума на полуизолирующую кристаллическую подложку осаждаются испарённые в источниках (т. н. \emph{эффузионных ячейках}) вещества.  В настоящее время, благодаря относительной несложности контролирования ростового процесса и другим преимуществам МПЭ широко используется для изготовления полупроводниковых структур.\\
Результат ростового процесса зависит от того, в какое время и сколько вещества поступило на поверхность структуры из источников. Скорость поступления вещества, в свою очередь, определяется температурой источников - чем больше температура, тем активнее идёт испарение.\\
Эффузионная ячейка состоит из тигля с испаряемым веществом, нагревательного элемента и заслонки. Открывая/закрывая заслонку и нагревая вещество в тигле, можно управлять процессом роста. Изменение температуры ячеек отвечает за плавную регулировку состава выращиваемого слоя. Таким образом, задача выращивания структуры определённого состава сводится к подбору необходимой температуры источников; вначале они рассчитываются теоретически, затем выращиваются различные тестовые структуры, по которым проверяется соответствие параметров роста предполагаемым. Если выявляется несоответствие, в расчёты вводятся поправки и процесс повторяется.
\newpage

\section {Постановка задачи}
Задача данной работы - для данных тестовых (т. е. выращенных с целью калибровки температуры эффузионных ячеек установки молекулярно-пучковой эпитаксии - далее МПЭ) структур состава GaAs, $In_{y}Ga_{1-y}As$, $Al_{x}Ga_{1-x}As$ определить:
\begin{enumerate}
\item{} Плотность дефектов на поверхности структур (бесконтактным методом).
\vspace{-6 pt}
\item{} Вычисление скорости роста отдельных слоёв в структуре по спектрам фотолюминесценции.
\vspace{-6 pt}
\item{} Концентрацию и подвижность свободных носителей заряда (контактным методом).
\vspace{-6 pt}
\end{enumerate}
\newpage
\end{document}