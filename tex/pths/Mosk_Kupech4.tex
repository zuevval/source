\NeedsTeXFormat{LaTeX2e}
\documentclass[a4paper,12pt]{article}
\textwidth=16cm
\oddsidemargin=0cm
\topmargin=-1cm
\textheight=23cm
\renewcommand{\bfdefault}{b}

\usepackage[utf8]{inputenc}
\usepackage[english, russian]{babel}
\usepackage{amssymb,amsmath}
\usepackage{graphicx}
\usepackage{color}
\usepackage{textcomp}

\begin{document}
\newcommand{\lk}{\guillemotleft}
\newcommand{\rk}{\guillemotright}
\renewcommand{\thesection}[1]{
\vspace{5 pt}
\Large
\noindent \textbf{{#1}}
\normalsize
\par
}
\begin{titlepage}

\begin{centering}

\textbf{Лицей \lkФизико-техническая школа\rk}

\textbf{Санкт-Петербургского  Академического  университета }
\vspace{10 pt}
\hrule
\vfill

\Large{Практическая работа}

\Large{по литературе}

\vfill

\LARGE \textbf {Москва Купеческая}

\normalsize
\vspace{10 pt}
Зарождение, становление и жизнь купечества Москвы

от средних веков до 1917 года
\vfill
\vfill


\begin{flushright}
Работу выполнили:

ученики 11 Б класса

Зуев Валерий и Горох Антон

~\
Научный руководитель:

Вирина Галина Львовна
\end{flushright}

\vfill
\vfill

Санкт-Петербург
\par
2017

\end{centering}
\end{titlepage}
\emph{Основная цель настоящей работы - проследить общие закономерности развития московского купечества как слоя общества, не останавливаясь подробно на отдельных личностях. В статье рассмотрены основные места Москвы, где, как правило, жили или которые часто посещали купцы в период с XVIII по начало XX века.}\\

\thesection{Определение купечества в России\\}
Купечество - торговое сословие. Ещё в Древнерусском государстве были известны \lkкупцы\rk (горожане, занимающиеся торговлей)  и \lkгости\rk \\ (фактически - те же купцы, но занимавшиеся торговлей с другими городами и странами) \cite{dic}. До 90-х гг. XIX в. купцами справедливо можно считать предпринимателей, основной профессией которых была торговля; после издания манифеста 17 марта 1775 появляется формальный признак принадлежности к купечеству -  купец должен состоять в одной из трёх купеческих гильдий, различавшихся размерами капитала\cite{rustrana}.
Неоднозначность при классификации московских дельцов на купцов и не купцов возникает после принятия в 1892 г. Городового положения и в 1898 г. Положения о государственном промысловом налоге. По новым законам, в купечество можно было записаться \lkна основе соображений, посторонних торговой деятельности\rk \cite{burishk}. Поэтому в данной работе мы не будем рассматривать людей, живших на рубеже XIX-XX веков и занимавшихся торговлей или промышленностью, но не вышедших из купеческой семьи. 

\thesection{Купечество Москвы в допетровское время\\}

Географическое расположение Москвы на пересечении множества водных (Москва, Яуза и др.) и сухопутных (Рязань - В. Новгород, Смоленск - Владимир) дорог издревле делало благоприятной всяческую тогровлю, что и заметил Юрий Долгорукий, основав в XII веке свой детинец на Боровицком холме \cite{besed}. Как и в большинстве крупных городов, экономический успех во многом обеспечивали торговые люди - богатые горожане, вначале политически бесправные (и в правление Ивана Грозного нередко страдавшие от опричнины), а с конца XVI  разделённые на три \emph{корпорации} по размеру капитала: гостей (самые высокие), затем тогровых людей гостиной сотни и торговых людей суконной сотни \cite{burishk}.


\color{red} 
%\emph{Надписи на красном языке оставил Горох\\}
\color{black}

Вначале купцы вели торг на северном берегу у Всехвятского (большого) моста, частично захватывая территорию заречья \cite{zamosk}. Богатые \lkгости\rk, торговавшие в основном шёлковыми тканями, держали подвалы и амбары близ церкви Иоанна Златоуста \cite{burishk}. К концу XV в. основная торговля сосредоточилась на Большом посаде (близ будущей Красной площади), а замоскворечье оказалось плотно застроенным и включало слободы ткачей, овчинников, бочаров; внутренняя структура была запутанной, даже улицы заменялись \lkровушками\rk - дренажными канавами \cite{besed}. Впрочем, на южной окраине местности оседали казанские и ногайские торговцы, образовавшие Татарскую слободу. На рубеже XVI-XVII столетий все эти слободы обслуживали важнейшую торговую магистраль, проходившую по трассам современных улиц Большой и Малой Татарских \cite{zamosk}. 

\color{black}

В Смутное время район выгорел и был отстроен заново с некоторыми измениями. Общественно-торговым центром округи стал Пятницкий рынок. Вдоль прилегающей к нему Пятницкой улицы богатыми купцами были выстроены каменные особняки; улица стала главной артерией района \cite{zamosk}.\\
\color{red}

\color{black}

\thesection{XVIII - начало XIX века\\}

\color{black}
В результате петровских реформ купеческое сословие стало выделяться административно: по регламенту 1721 г. горожане разделились на две гильдии и чернорабочих; к более высокой из двух гильдий были причислены и купцы. Теперь записанный в гильдию купец получал многие сословные преимущества - например, мог покупать приписных крестьян \cite{burishk}. Купцы, однако, вплоть до манифста 1775 года (одним из итогов которого стала отмена для купцов подушной подати) считали себя ущемлёнными, несмотря на возросшую по сравнению с XVII веком роль (по некоторым данным, в то время в Москве проживало не менее 12 тысяч купеческих семей).

Административному выделению сопутствовало территориальное обособление. После переноса Петром царского двора в Петербург из Замоскворечья уехали ремесленники, работавшие на правительство; из коренных жителей остались только торговые люди. Купечество стало основным населением этой части города, послужив материалом для творчества Островского, таких художников, как П. А. Федотов и В. Г. Перов.  Район приобрёл облик провинциального захолустья - во многом именно ему Москва обязана имиджу \lkбольшой деревни\rk \cite{zamosk}.

\begin{figure}[h!]
\center{\includegraphics[width=.7\textwidth]{svatovstvo-maiora.jpg}}
\caption{П. А. Федотов. Сватовство майора}
\end{figure}

С середины XVIII в. купечество, в том числе и миллионеры, активно заселяет Замоскворечье как спальный район; торговали купцы в Зарядье. Так писал об этих местах Белинский: \lk...окна занавешены занавесками, ворота - на запор, при ударе в них раздается сердитый лай цепной собаки, все мертво или, лучше сказать, сонно: дом или домишко похож на крепость, приготовившуюся выдержать долговременную осаду\rk. Миллионер Козьма Матвеев воздвигнул в Замоскворечье церковь Климента на Пятницкой \cite{zamosk}.\\

\thesection{Война 1812 года и послевоенный период\\}

В войну 1812 года впервые зафиксированы акты благотворительности среди купечества: в результате воззвания к пожертвованию 1812 г. москвичами была собрана сумма в 1 млн. рублей - половина от дворянства, половина от купечества \cite{burishk}.

По возвращении в Москву богатые купцы начали отстраивать заново свои дома в Пятницкой, Якиманской, Сретенской и Таганской частях города. «Лесу для строения везут множество, - писал Ростопчин 13 февраля 1813 г. - Купечество намерено приводить сгоревшие свои дома в первобытное состояние. Дворяне же мало о сем промышляют — причина та, что многие разорены» \cite{1812}.

\thesection{Конец XIX - начало XX века\\}

По всей видимости, до освобождения крестьян (и, соответственно, постепенной потери влияния дворянства), интерес историков и литераторов к купечеству был едва ли выше, чем, например, к ремесленникам или крестьянам. Но когда бывшие материальные, да и культурные  ценности дворянства мало-помалу оказались сосредоточены в руках торгового сословия, \lkте же мужики, но в синих сюртуках\rk приобрели большой вес в обществе и политике. Наметилась противоположная тенденция; теперь уже, как писала \lkНовая газета\rk, \lkв Москве один знаменатель - купец, всё на свою линию загибающий\rk \cite{burishk}.

Забегая вперёд, следует отметить, что всё-таки полупрезрительное отношение к купцам как к \lkтретьему сословию\rk сохранилось вплоть до Октябрьской революции. Даже после Февральской \lkбуржуазной\rk революции купцы не вышли на первый план; положение оставалось примерно таким же, с той лишь разницей, что теперь господствующим классом стала социалистическая интеллигенция \cite{burishk}.\\



%Богатейшие купеческие династии XIX века В X веке в Москву из различных губерний в Москву приезжают родоначальники бдущих богатейших купеческих династий XIX века: Бахрушины (1835, из Рязанской губернии), Найдёновы (1765, из Суздальского), Третьяковы, Щукины.\\



%\thesection{Заключение\\}
\newpage
\begin{thebibliography}{99}
%С одним автором - в алфавитном порядке по автору, с двумя и более - по названию
\bibitem{burishk}{Бурышкин, П. А. Москва купеческая: Мемуары / Вступ. ст., коммент. Г. Н. Ульяновой, М. К. Шацилло. - М.: Высшая школа, 1991. - 352 с.}
\bibitem{Gilar}{Гиляровский, В. А. Москва и москвичи / В. А. Гиляровский. - М.: Художественная литература, 1981. - 382 с.}
\bibitem{besed}{Беседина, М.Б. Прогулки по допетровской Москве [Электронный ресурс] / М. Б. Беседина. -
М.: Астрель; Олимп, 2009. - 318 с. URL: http://www.twirpx.com/file/1500560 (Дата обращения: 23.01.2017)}
\bibitem{zamosk}{Замоскворечье [Электронный ресурс] // Московский день: Сайт о Москве. URL: http://mosday.ru/info/ region.php?moscow\_zamoskvorechie (Дата обращения: 21.01.2017).}
\bibitem{1812}{Застройка Москвы в 1812 - 1825 годах [Электронный ресурс] // Федеральный портал protown.ru. http://www.protown.ru/russia/city/articles/4629.html}
\bibitem{chul}{Чулков, Н. П. Московское купечество XVIII - XIX веков [Электронный ресурс] // Н.П. Чулков. - Русский архив, 1907. № 12, с. 553 – 567. URL: http://elar.urfu.ru/handle/10995/37207 (Дата обращения: 03.02.2017)}
\bibitem{dic}{Купечество [Электронный ресурс] // Энциклопедия \lkРусская цивилизация\rk. URL: http://dic.academic.ru/dic.nsf/russian\_history/4048 /\%D0\%9A\%D0\%A3\%D0\%9F\%D0\%95\%D0\%A7\%D0\%95\%D0\%A1 \%D0\%A2\%D0\%92\%D0\%9E (Дата обращения: 21.01.2017).}
\bibitem{rustrana}{О купеческом сословии вообще [Электроный ресурс] // Рустрана, 2007. URL: http://рустрана.рф/article.php?nid=9918 (Дата обращения: 21.01.2017).}
\end{thebibliography}

\end{document}