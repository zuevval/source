%C:\Users\Valery\AppData\Local\MiKTeX\2.9\miktex\log
\documentclass[a4paper,12pt]{article}
\usepackage[utf8]{inputenc}
\usepackage[english, russian]{babel}
\usepackage{gensymb}
\usepackage[dvips]{graphicx}
\graphicspath{{LogoPTHS/}}
\begin{document}
\begin{minipage}{0.45\textwidth}

\begin{center}
\noindent \Large \textbf{DESIGN CONSTRAINTS}
\end{center}
\noindent{Primary constraints imposed on the design of a refreshable Braille display include the dimensional constraints of Braille dots and characters, and the operational constraints of a refreshable display. These constraints ensure convenience to users of both existing Braille displays and paper-based Braille and allow a new display to penetrate better among them. There are a few secondary constraints as well which are essential to overcome the challenges imposed by SMA based actuation.}

\begin{center}
\noindent{DIMENSIONAL CONSTRAINTS}
\end{center}

\noindent{Figure 2 illustrates the different dimensions for size and spacing of Braille on paper and on a Braille cell. Dimensional requirements of Braille dot and character are derived from various standards practiced internationally. Prescribed sizes and spacings according to different standards are given in $[28]$. It is however, impossible to conform to all standards, with a single choice for each dimension. Therefore, Braille readers have been consulted and acceptable ranges of each dimension have been
determined.  }
\end{minipage}
\hfill
\begin{minipage}{0.45\textwidth}
A single value for each dimension conforming to the acceptable range has been chosen such that each fulfills most international standards. Table $1$ gives the acceptable ranges and chosen values for each dimension. Profile of a Braille dot has been constructed with a dome shaped top for smooth tactile perception, as recommended in $[29]$, and for its similarity to paper-based Braille.

\begin{center}
\noindent{Table $1$. RANGE AND VALUE OF VARIOUS DIMENSIONS}
\end{center}

\begin{tabular}{lcc}
\hline
Dmnsn & Range (mm) & Value\\ [5 pt]
\hline
a & $2.3-2.6$ & $2.5$\\
b& $2.3-2.6$ & $2.5$\\
c& $10-13$ & $12$\\
d& $6.0-7.0$ & $12$ \\
e& $1.3-1.6$ & $1.4$\\
f & $0.7-0.8$ & $0.75$\\
\hline
\end{tabular}

\begin{center}
\noindent{Table $2$. OPERATIONAL CONSTRAINTS}

\begin{tabular}{p{2 cm} p{3 cm}}
\hline
Parameter & Value\\ [5 pt]
\hline
Resisting force & min. $20$ gf per Braille pin, preferably $30$ gf\\ [3 pt]
Refresh frequency & min. 5 Hz\\ [3 pt]
Minimum actuator life & min. $10^6$ cycles\\ [3 pt]
Operating temperature & $0-40$ C\\
\end{tabular}
\end{center}
\end{minipage}
\begin{minipage}{0.45\textwidth}
\begin{center}
\noindent{OPERATIONAL CONSTRAINTS}
\end{center}
\noindent{Operational constraints define the minimum desired performance of a Braille display and are derived primarily through user interaction. Table $2$ enlists critical operational parameters.
Resisting force is defined as the resistance offered by each perceivable Braille dot to pressure exerted by a reader's finger.
%проблема - неправильно воспроизводит апостроф
It is quantified as the amount of force needed to push a Braille dot to a position just above the display surface. Resisting force
in commercial Braille displays is approximately $12$gf $[30]$, which is lower than the preferred value observed.
Refresh frequency is defined as the number of actuation cycles that can be completed by a Braille dot in one second. It
is an important parameter for user convenience as it directly affects reading speed. A lower limit on refresh frequency has
been determined by observing users. A $40$-character display is expected to refresh within a maximum of one second, while
individual cells even earlier. Minimum actuator life is defined as the number of cycles
each actuator must perform without failure. A minimum life of $106$ cycles comes out as equivalent to two to three years of
undisturbed above average use.}
\end{minipage}
\hfill
\begin{minipage}{0.45\textwidth}
\begin{center}
\noindent{SECONDARY CONSTRAINTS}
\end{center}
\noindent {A controlled cooling system is imperative to satisfy the operational constraint of actuation frequency. However, all
cooling systems increase the net electrical power consumed by the display as well as increase the overall device cost. It should
be noted that actuator packing and display design form an essential part of the cooling system.
SMA actuator elements are usually heated at actuation to beyond $50\degree$C to as high as $90\degree$C, depending on the alloy
chosen. The display needs to be such that it prevents accidental user contact with actuator elements. Further, excess heat needs
to be expelled immediately and its accumulation, especially near display surface and Braille dots, needs to be prevented.
SMA elements are usually available at low-cost. They are thus only a fraction of the net Bill-of-Material cost. Hence,
failure of SMA elements due to fatigue or shock need not lead to the discard of a whole Braille display. A modular design of
the display with individually replaceable cells is necessary. Further, Braille cells need to be designed such that individual
actuator elements are repairable and replaceable. The use of touch to read Braille leads to continuous
damage of Braille dots and the display surface because of wear.}
\end{minipage}
\hfill
\begin{minipage}{0.45\textwidth}
\noindent{ Maintenance, repair and replacement of these components
should be facilitated. Further, to prevent premature failures and reduce manufacture cost, moving parts should be kept at a minimum.}
\begin{center}
\noindent \Large \textbf{SYSTEM DESIGN}
\end{center}
\noindent{This section presents the detailed system design of a refreshable Braille display using SMA based actuation. Key
challenges in SMA based actuation have been targeted with methods discussed above.}
\begin{center}
\noindent{ACTUATOR CONFIGURATION}
\end{center}
\noindent{A precursor to actuator design is the selection and testing of a suitable shape memory alloy. A variety of SMAs are
available, varying in alloy composition, heat treatment and conditioning processes employed. Binary and ternary alloys of
NiTi are usually preferred for thermo-mechanical actuation applications over other SMAs $[31]$. Other popular SMAs are
CuZn alloys and CuAl alloys. Specifically, NiTiCu, a ternary alloy with $5-10 at\%$ Cu, is suitable for actuation applications
due to smaller hysteresis in thermo-mechanical cycling $[8]$ and higher fatigue stability $[32]$.}
\noindent{Flexinol\textregistered NiTi alloy procured from Dynalloy Inc. has been used for the purpose of developing prototypes due to its ready availability.}
\end{minipage}
\hfill
\begin{minipage}{0.45\textwidth}
\noindent{The characteristic stress - strain relation of Flexinol\textregistered HT wires was
%после знака регистрации не ставит пробел
obtained experimentally using isothermal tensile tests performed on Instron\textregistered MicroTester.
Flexinol\textregistered HT wires were found to be highly-conditioned to exhibit two way shape memory effect. The obtained relation was verified with experimental isotherms given in $[33]$.}

\noindent{SMA wire is chosen as the actuator element, for its advantages discussed earlier. Figure $3$ gives a schematic of the
actuator configuration. Different from a regular spring-biased SMA wire actuator, the proposed actuator uses the SMA wire
element in an inverse-U configuration. With such a configuration, all the terminals are placed on a single plane,
simplifying electrical connections. Further, it allows the wire to have removable connections to the structure, allowing easy
assembly and disassembly.}

\noindent{First order calculations were performed to determine the natural length $L_n$ of the SMA wire, stiffness $K_s$ of the bias compression spring and the stress $\sigma_A$ and strain $\varepsilon_A$ in the actuated SMA wire. Following design equations were used:}
$$K_Sx=2A_w\sigma_M$$
$$K_S(x+\Delta x)=2A_W \sigma_A$$
$$L_n(\varepsilon_M-\varepsilon_A)=\Delta x$$
$$K_s(x+\Delta x) \ge F_R$$
\end{minipage}
\hfill
\begin{minipage}{0.45\textwidth}
\begin{thebibliography}{199}
\bibitem{1}{Strobel, W. A., et al., 2004. "Industry Profile on Visual Impairment".}
\bibitem{2}{Cryer, H., and Home, S., 2011. "Use of Braille Displays". Research report \#15. RNIB Centre for Accessible Information, Birmingham, UK.}
\bibitem{3}{Ryles, R., 1996. "The Impact of Braille Reading Skills on Employment, Income, Education and Reading Habits". Journal of Visual Impairment and Blindness, 90(3), May-June, pp. 219-226.}
\bibitem{4}{Tetzlaff, J. F., 1981. "Electromechanical Braille Cell". United States Patent 4,283,178.}
\bibitem{5}{Metec AG. "Braille Module". On the WWW. URL http://web.metec-ag.de/braille\%20cells.html.}
\bibitem{6}{Blazie, D., 2000. "Refreshable Braille Now and in the Years Ahead". The Braille Monitor 43 (1), January.}
\end{thebibliography}
%иначе ругается!
\end{minipage}
\hfill
\begin{minipage}{0.45\textwidth}
\begin{thebibliography}{199}
\bibitem{7}{Vidal-Verd\'u, F., and Hafez M., 2007. "Graphical Tactile Displays for Visually-Impaired People". IEEE Transactions on Neural Systems and Rehabilitation Engineering, 15(1), March, pp. 119-130.}
\bibitem{8}{Kumar, P. K., and Lagoudas, D. C., 2008. "Introduction to Shape Memory Alloys". In Shape Memory Alloys: Modeling and Engineering Applications, D. C. Lagoudas, ed., Vol. 1. Springer US, Chapter 1, pp. 1-51.}
\end{thebibliography}

\end{minipage}
%\center{\includegraphics{gilbert.eps}}
\end{document}