% compile with XeLaTeX or LuaLaTeX

\documentclass[a4paper,12pt]{article} %14pt - extarticle
\usepackage[utf8]{inputenc} % russian, do not change
\usepackage[T2A, T1]{fontenc} % russian, do not change
\usepackage[english]{babel}

% fonts
\usepackage[document]{ragged2e}
\usepackage{fontspec} % different fonts
\setmainfont{Times New Roman}
\usepackage{setspace}
\usepackage{dsfont}

% utilities
\usepackage{hyperref}
\hypersetup{pdfstartview=FitH,  linkcolor=blue, urlcolor=blue, colorlinks=true}

\usepackage{graphicx}
\usepackage{caption}
\usepackage{subcaption} % captions for subfigures
\usepackage{array} % utils for tables
\usepackage{listings}
\usepackage{color}

% styling
\usepackage{float} % force pictures position
\floatstyle{plaintop} % force caption on top
\usepackage{enumitem} % itemize and enumerate with [noitemsep]
\usepackage[left=25mm, top=20mm, right=25mm, bottom=20mm, nohead, footskip=10mm]{geometry}


\begin{document}

\title{Motivation Letter}
\date{}
\maketitle

\justifying
\thispagestyle{empty}

\noindent Valerii Zuev\\
\href{mailto:valera.zuev.zva@gmail.com}{valera.zuev.zva@gmail.com} \\

\noindent \textbf{To:} \\
Yurii S. Aulchenko, Dr. Prof. \\

Dear sir Yurii S. Aulchenko,

I would like to be honored to attend the online course Introduction to Quantitative Genetics.

Currently I am in my first-year of Master's degree at St. Petersburg Polytechnic University.
Our master program is organized by the Mathematical Biology \& Bioinformatics Lab and we study life sciences (previously I also studied bioinformatics here as an undergraduate student).
We studied biology, molecular biology and various aspects of bioinformatics (including population genetics and data processing techniques, including GWAS).
The mathematical part of the curriculum included probability, statistics; this summer I've completed an online course on probability (\href{https://stepik.org/cert/1025955}{stepik.org/cert/1025955}). I suppose that the knowledge gained in the aforementioned set of courses will be sufficient to understand the lectures in your discipline.

This year I am supposed to start working on my Master thesis which I would like to devote to improving the library for structural equation modeling \texttt{semopy}, which is being developed in our laboratory for the purposes of SEM-GWAS (and which many researchers from all around the globe use nowadays).
Unfortunately, by this moment I am not very proficient in GWAS and would like to study this method in detail.
Moreover, I am interested in population and quantitative genetics because much work in these disciplines requires non-trivial mathematics.
As I expect, understanding genetic variation will help me to solve my current and future challenges more effectively (both in the short and long term).

At this time I am not employed, so I suppose that I will have enough time to complete the course.
I will be able to devote from 4 to 6 hours a week to these classes. \\

\noindent Sincerely, \\
\textbf{Valerii Zuev}



\end{document}