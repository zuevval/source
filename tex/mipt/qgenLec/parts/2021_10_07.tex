\documentclass[main.tex]{subfiles}
\begin{document}

\url{https://www.youtube.com/watch?v=a4o-XXRCaQoURL}

\section{ Introductory lection }

Will there be a recording? - No, probably not.

My name is Yurii Aulchenko and I am the chief scientist at a company <...>.

\section{Нанопоровое секвенирование: обзор технологии}

Демиденко Наталья, SkyGene.

Очень маленький секвенатор.
Нужен только ноутбук с USB-портом.

Недавно вышла статья, где сравнивалась стоимость прочтения за гигабазу.
Сейчас есть три основных прибора, доступных для покупки:

MinION
GridION

Ещё один,

Проточная ячейка для Nanopore не одноразовая!
Если в процессе, например, видим во время секвенирования, можно нажать "stop", промыть ячейку и загрузить новую.
В других технологиях (PacBio) такого нет.

Областей применения данной технологии очень-очень много.
Сейчас 2448 публикаций по Nanopore Sequencing.
Недавно Оксфорд вышел на биржу и буквально за первый день акции выросли на 40\%.

В зависимости от длины фрагмента можно решать разные задачи (Оксфорд может давать как короткие, так и длинные и даже ультрадлинные прочтения).
Понятно, что длинные прочтения облегчают сборку контигов.
Павлин, угорь и т.д.: секвенировали Oxford'ом, сверху закрывали Illumina.

Оксфорд позволяет определять связь вариантов.

Ещё одно применение -- улучшение существующей сборки.

Секвенирование генома секвойи: благодаря длинным чтениям покрытие выросло (N50).

Можно напрямую смотреть альтернативный сплайсинг, идентифицировать изоформы.
Точно так же смотрят модификации РНК.

Много приложений в онкодиагностике.
Один запуск на MinION даёт больше, чем анализ с помощью Сэнгера + микрочипы

% "демонстрация того, как надо рекламировать продукт своей компании"

\section{A crash course in Genome Wide Association Studies in Plants}

G. Laurent

Association genetics: ussually it means that you use natural genetic variability to identify loci of interest.

\subsection{ Linkage desiquilibrium }

Assumption (without linkage desiquilibrium): if we have enough evolution time, all the recombination will be produced in the population.
But sometimes a part of chromosome always follows a mutations (LD, Linkage Desiquilibrium).

LD occurs due to a gigantic MAGIC.

How do we measure LD?
Basically there is 2x2 table: four classes of genotypes (AB, Ab, aB, AB).
If there is equilibrium, we expect the proportion is $ f_{AB} = f_A f_B $, but when there is disequilibrium ($D$), there are deviations.

\[ D' = D / D_{max} \text{ -- normalized disequilibrium} \]

The \emph{haplotype} is a possible combination of SNPs.
We saw previously that we do not have all haplotypes in the populations.
When we do the SNP analysis, it is useful to look not only at possible SNPs, but also

Breeders\footnote{селекционеры} use haplotypes.

If there is no disequilibrium between two SNPs, information about one provides us no information about the other.

What if we have thousands of SNPs?
We compute the covariance 
\[ Cov(X*, Y*) = [X*, Y*]^T [X*, Y*] = \begin{pmatrix}
	D_1 & 1 - f_{X*,Y*} \\
	1 - f_{X*, Y*} & D_2
\end{pmatrix} \] % TODO  check

What forces LD shape? Genetic drift, ... % TODO a bit

\subsubsection{How LD is created and evolve?}

What causes LD is mutation.
If there is no mutation, there is no LD.
But recombination destroys LD.

More mutations $\Rightarrow$ more LD.
More recombination $ \Rightarrow $ less LD.
So the pattern of LD is shaped by the balance of two.

\begin{leftbar}
	loci  [lō′sī΄]
\end{leftbar}

Some SNPs go together in blocks.
What is more, some blocks go together!

\subsection{LD within a locus}

We discussed LD within the chromosome, but we may talk about LD within a specific locus.

In the picture, we see a very strong disequilibrium in the center of the gene.
The biological reason for this is an \textbf{intron}: variations are less likely within exons.

LD generally decreases with distance between two SNPs, but it varies from gene to gene.

\begin{leftbar}
 "Зелёная революция": inbreds $ \rightarrow $ landracers
\end{leftbar}

LD decreases more slowly in inbreds

\subsection{Example}

Assotiation study: we associate phenotypes with genotypes.

If the trait (e.g. height) is significantly different between two SNP variants, this trait is associated with this phenotype.

\subsection{Association mapping}

There is little sense in detecting a single marker.
We may lose it during recombination in next generations!
It is much more useful to detect groups of markers that are closely linked and target them.

Let's discuss the difference between \textbf{high LD} and \textbf{medium LD}.

Suppose there is an SNP (we do not know it) that explains $ 10 \% $ of total variance in the trait.

% TODO did not understand

In academia we may publish that, but we cannot sell it to any corporation.

\subsection{Admixtured populations: another problem}
Admixture is likely to destroy LD.

Sometimes we are not sure about our GWAS screening because in the beginning we had two mixed population.

% TODO two cross-tables and the third - a sum of two

\subsubsection{Population structure and quantitative trait}

Incorrect association % TODO

\end{document}
