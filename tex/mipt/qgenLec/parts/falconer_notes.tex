\documentclass[main.tex]{subfiles}
\begin{document}
\section{Falconer: Quantitative Genetics}

\subsection{Chapter 8. Variance}

\begin{leftbar}

Recall some facts from chapter 7. \\

Consider an a/d model ($ A_1 A_1 \to p^2$, genotypic value $ +a $; $ A_1 A_2 \to 2pq$, genotypic value $+d$; $ A_2 A_2 \to q^2$, genotypic value $ -a $).

A "breeding value"\hspace{0pt} is associated with genes carried by an individual and transmitted to the offspring.
It is known as the \emph{average effect} $ A $ (means "additive"). In a single gene, $ G = A + D $, $ D $ is \emph{dominance deviation}; in a sum of gene due to interactions $ G = A + D + I $.

\end{leftbar}

When genetic and environmental effects are uncorrelated, phenotype variance

\[ V_P = V_G + V_E = V_A + V_D + V_I + V_E  \]
$ V_G $ -- genetic variance, $ V_E $ -- environmental variance; $ G = A + D + I $, $ A $ -- sum of all breeding values, $ D $ -- sum of all dominance deviations.

In case of correlated $ G $ and $ E $ $ V_P = V_G + V_E + 2 cov(G,E) $.

$ V_G / V_P $ -- heritability in the broad sense (a degree of genetic determination).

$ V_A / V_P $ -- heritability in the narrow sense (or just heritability): a degree of resemblance between parents (is of the great importance).

\end{document}