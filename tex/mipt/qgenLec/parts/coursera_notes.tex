\documentclass[main.tex]{subfiles}
\begin{document}

\section{From Disease to Genes and Back}

Testing for multifactorial diseases is not yet used in the clinic.

\emph{Mendelian} disorders VS complex disorders.

\subsection{Anatomy of Human Genome}

In fact, we have two genomes, not one.
23 molecules of DNA (3 billion base pairs).

There are approximately 30 000 transcribed genes, of which 10 000 are \emph{non-coding genes} (the final product is RNA) and \emph{coding genes} (the final product is a protein).

Genes are so-called \emph{splits}: introns are eleminated during \emph{splicing}.
Exons are relatively small compared to introns: they form only $\approx 1.5 \%$ of the human genome.

\emph{Gene switches} -- parts of the genome which defines whether the gene is "on" or "off".
The main is a \emph{basal promotor}.

Whether the gene will be included in the transcrpt is defined by \emph{epigenetic marks} (annotations), of which the most studied are post-translation modifications of histones (histones constitute DNA).
Their terminal ends may be methylated, phosphorylated, acetylated, ubiquitinated.
Other epigenetic marks are DNA methylation and positive feedback loops.

$95 \%$ of the genome that is not transcribed is invaded by retroviruses or littered with pseudo-genes (many genes have multiple copies which are not used, e.g. those from the SINE family, especially Alu repeats in human).

Some elements -- \emph{satellites} -- are \emph{tandomly repeated} (in a head-to-tail fashion).
Some satellites (macrosatellites) are located at the ends of the DNA and are called \emph{telomeres}, wheres others appear in centromeres.

\subsection{Identifying functionally important elements in the human reference genome}

\begin{leftbar}
    ORF -- open reading frame -- a part of the genome that can be transcribed (start codon $ \rightarrow $ functional codons $ \rightarrow $ stop codons).
\end{leftbar}

To annotate genes, we may first mask the repeats (using, for example, the RepeatMasker software).
Then we sequence the \emph{transcriptome} (all RNA molecules) and try to map them back to genome. It becomes increasingly clear, however, that \textbf{not everything that is transcribed is a gene}, for example, promoters and enhancers generate small transcripts when they are active.

Identifying gene switches is much more complicated.
One technique for finding gene switches is comparison of highly conserved genome elements by comparing genomes across species (if the part of genome is conserved, it is likely very important, i.e. a switch).

Of course, there are some switches specific to human, so there are other methods to identify switches.
The first is \textbf{ChIP-seq} (Chromatin ImmunoPrecipitation) identifies all the gene switches to which a specific transcription factor is binded.
We cut the regions to which the TF is binded from the DNA (using antibodies to the TF), put them into NGS scanner, then map the fragments back to the human genome.
But this can work only for well-known TFs to which we can obtain antibodies.
There are alternative methods: (1) using antibodies not to TFs but to histones ends (this way we can recognize enhancers, silencers and promotors by means of the histone code). Applying HMMs to the histone code, we can identify active enhancers, poised enhancers, silenced enhancers, etc.

The chromatin around regulatory elements is usually more open. In \texttt{DNase-seq} or \emph{ATAC-seq}, in-vitro generated transposons are integrated into genome; they are incorporated usually into places where the chromatin is more open.

Absence of necessity of finding the TF helps, but sometimes we want to know which TFs bind to a specific site.
This is done by \emph{footprinting}: during mapping to the genome in the DNase-seq experiments, not all bases are equally covered.
Those that are covered less are protected from sequencing and therefore might contain TF binding sites.

One more technique is the \emph{chromatin confirmation capture with HI-C}.
When a gene switch is active, distant parts (promotor + enhancer) of the genome are bound by \emph{looping}, so we can try to identify these parts.

\subsection{Genetic Polymorphism}

A \emph{haplotype} is a set of  specific co-inherited alleles (typically they are present in neighboring loci).
Homozygous, heterozygous organisms.
Transitions: A-T, C-G; transversion: A-C / A-G / T-C / T-G; most common are transitions. \\

Genetic variants (genetic polymorphisms): the most common are SNPs.
In addition to SNPs, there are small insertions and deletions; variations in number of segmental duplications; reciprocal translocations and so on.

Where this variance comes from?
It's the result of balance between generation of new mutations VS loss of mutations in the population.
Approximately 40 new mutations from father and 20 from mother will come to a child (generation of mutations).
Some variances are not selected for the next generation (random genetic drift, loss of the mutation).

\subsubsection{Linkage disequilibrium and haplotype blocks}

A haplotype is a combination of markers (say, $A_1 B_1$) brought by sperm or oocyte.
If $A_1$ and $B_1$ are encountered more frequently together than it is expected by chance, we say that they are in the \emph{linkage disequilibrium}.

Genomes are split into \emph{haplotype blocks} within which SNPs are highly linked, whereas between the blocks they are less linked.

Genome variants may occur in evolutionary non-constrained parts or in constrained parts (so-called \emph{causative variants}); later are divided into \emph{coding} and \emph{non-coding} (occurring in genetic switches) variants.
Some mutations lead to a loss of function, whereas other might lead to a gain of function.
Interestingly, humans have much more mutations leading to loss of function than it might be expected in theory for normal living.

Aside from germ-line, there are \emph{somatic} mutations which occur during the mitosis.
Cancer is a result of accumulation of multiple mutations.

\subsection{Interrogating Genetic Variation}

A \emph{restriction enzyme} ("DNA is the king in a bacterium; if a foreign king invades the bacterium, this servant can cut him in small fragments, but he does not do any harm to his own king").
Southern blot is a method allowing to detect the presence of a specific sequence in a DNA sample (based on hybridization principle).
Electrophoresis. \\

1980, RFLP (Restriction Fragment Length Polymorphism) allowed to build first human linkage maps. % TODO what is meant by genetic linkage maps?
DNAs are extracted, cleaved with a restriction enzyme, separated by gel electrophoresis, and then by the process of Southern blotting we find fragments corresponding to specific locations (usually marked by radioactive markers).
This was a very hard and tedious procedure.

Mini-satellites were discovered; then micro-satellites + PCR.
This allowed to build sophisticated genome linkage maps and to identify genes associated with many Mendelian diseases.

Still, the capabilities of electrophoresis were limiting. Affymetrix and Illumina developed various chips (SNP arrays).

You can only find an SNP in an array if you know its location previously because you have to design an array!
A substantial number of disorders are caused by low-frequency SNPs, so to discover them, we need to sequence the whole genome of the patient (and perhaps his parents).
Fortunately, \textbf{sequencing is advancing extremely fast, outpacing the Moore's law} (that is, the cost of sequencing is decreasing rapidly).

\subsection{Week 2 | Introduction to population genetics }

P.G. deals with changes which happen in a population under the influence of evolutionary factors.

Evolutionary factors:
\begin{enumerate}[noitemsep]
    \item Mutation
    \item Selection (a process of \emph{differential} reproductive success)
    \item Gene flow and mating structure
    \item Genetic drift (random fluctuations in allele frequencies)
\end{enumerate}

Two individuals A and B are said to \emph{belong to the same population} if the probability that they would have a common offspring is greater than zero.
(of course, it is relative; we may speak about a genetic population of island A and a separate population of island B; but if there is yet another island C much farther, we may speak about a joint population of A+B).

\subsubsection{Hardy-Weinberg Equilibrium}

A \emph{panmixia} is a situation when any pair of different gender in the population has a chance to have an offspring.

We assume that
\begin{enumerate}[noitemsep]
    \item The population is infinitely large
    \item Generations are independent (one generation will give rise to a set of gametes which will give rise to the next generation and so on)
    \item Panmixia (random sampling / segregation and aggregation of gametes)
\end{enumerate}

Genotype A \& B $ \rightarrow $ genotypes AA, AB, BB.
Allele frequencies:
\begin{gather*}
	p = p(A) = p(AA) + \frac{1}{2} p(AB), \medskip q = p(B) = 1 - p(A) \medskip \Rightarrow \\ \medskip \boxed{ p(AA) = p^2, p(AB) = 2pq, p(BB) = q^2 } \quad \text{(Hardy-Weinberg equilibrium)}
\end{gather*}

Despite these formulas are based on very simple assumption, normally loci follow HWE.

\subsubsection{Linkage Disequilibrium}

Covariance and correlation: correlation is scaled by variances of variables and ranges from -1 to 1.

Loci are located on a chromosome and frequently the distribution of the genotypes within two close loci is statistically dependent.

Consider a system which consists of two loci presumably linked together, and in each locus there are two alleles (say, A/a and B/b).

Two loci on the same chromosome: alleles are in \emph{cis}; on two different chromosomes: \emph{trans}.

Consider a locus with two alleles, A and a. If some another loci used to have a single allele B, but at some point a mutation occurred which led to another (lucky) allele "b" (and that individual happened to have A genotype), then in a short period A and b genotypes will be correlated.
This is \emph{linkage disequilibrium}.

The measure of LD is $ D := P(AB)p(ab) - p(Ab)p(aB) $ (it is covariance).
Statisticians also often use $ r^2 := \frac{D}{p(A) p(B) p(a) p(b)} $ (this is squared correlation).

\textbf{Lewontin's D'}: $ D' \overset{def}= \frac{D}{D_{\max}} $

What reasons could cause LD?

\begin{enumerate}[noitemsep]
    \item Loci are close (physical proximity)
    \item Loci are far away, but there was a mapping error
    \item Genetic structure (e.g., our study is a mixture of two different population, or there was a recent mutation).
\end{enumerate}

Genetic maps.
HapMap project; its successor is the 1000 genomes project.
HapMap: researchers could genotype about 10 million SNPs in the human genome, but HapMap identifed about 250-500 000 "tag" SNPs that explain most of the variance (they provide always as much information as the whole set of 10 000 000 SNPs).

\subsection{Genetic Drift}

There are stochastic factors affecting allele frequencies -- a \emph{genetic drift}.
Recall that the model of genetic drift (generation are not overlapping, each generation is a random sampling of $ 2n $ gametes, ...) is a simple but often good model.

Consider a population with 10 diploid individuals and a single copy of a gene with mutation, i.e. $95\%$ usual gametes and $5\%$ mutants.
What are the chances that
\begin{enumerate}[noitemsep]
	\item A mutation will get lost;
	\item In the next generation, exactly one copy of a mutant allele will be present;
	\item In the next generation, the number of copies will increase?
\end{enumerate}

Answers: $ \left( \frac{19}{20} \right)^{20} = 0.95^{20 \approx 0.36}, \quad \left( \frac{19}{20} \right)^{19} \cdot \frac{1}{20} \cdot \binom{20}{1} = \left( \frac{19}{20} \right)^{19} \approx 0.38, \quad 1 - (0.36+0.38) = 0.26 $.

\subsubsection{Bottleneck effect and founder effect}

Bottleneck: a population size reduces dramatically.
Founder effect: a very small number of founders diverge and form a new population.
Mathematically these effects are the same.

\subsubsection{Long-term consequencies of drift}

Consider a population of size 1.
Let the initial individual be heterozygotic.
The chances that the heterozygocity persists in the next generation is $0.5$.
Sooner or later, each allele will be either fixed or lost.

Expected times of loss and fixation: let $ p_0 $ denote the original frequency, $ n $ -- the \textbf{effective} population size.
Then

\begin{gather*}
	\mathds E [t_{lost}] = -4 n \frac{p_0}{1-p_0} \cdot \ln(p_0) \\
	\mathds E [t_{fixed}] = -4n \frac{1-p_0}{p_0} \cdot \ln(1-p_0) \\
	\mathds E[t_{segregate}] = -4n [(1-p_0) \ln (1-p_0) + p_0 \ln (p_0)] \\
	P(lost | t, p_0) = \exp \left( -4n \frac{p_0}{t} \right) \\
	Var(p) = p_0 (1-p_0) \left( 1 - \left( 1 - \frac{1}{2n} \right)^t \right)
\end{gather*}


\end{document}
