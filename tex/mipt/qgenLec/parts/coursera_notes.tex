\documentclass[main.tex]{subfiles}
\begin{document}

\section{From Disease to Genes and Back}

Testing for multifactorial diseases is not yet used in the clinic.

\emph{Mendelian} disorders VS complex disorders.

\subsection{Anatomy of Human Genome}

In fact, we have two genomes, not one.
23 molecules of DNA (3 billion base pairs).

There are approximately 30 000 transcribed genes, of which 10 000 are \emph{non-coding genes} (the final product is RNA) and \emph{coding genes} (the final product is a protein).

Genes are so-called \emph{splits}: introns are eleminated during \emph{splicing}.
Exons are relatively small compared to introns: they form only $\approx 1.5 \%$ of the human genome.

\emph{Gene switches} -- parts of the genome which defines whether the gene is "on" or "off".
The main is a \emph{basal promotor}.

Whether the gene will be included in the transcrpt is defined by \emph{epigenetic marks} (annotations), of which the most studied are post-translation modifications of histones (histones constitute DNA).
Their terminal ends may be methylated, phosphorylated, acetylated, ubiquitinated.
Other epigenetic marks are DNA methylation and positive feedback loops.

$95 \%$ of the genome that is not transcribed is invaded by retroviruses or littered with pseudo-genes (many genes have multiple copies which are not used, e.g. those from the SINE family, especially Alu repeats in human).

Some elements -- \emph{satellites} -- are \emph{tandomly repeated} (in a head-to-tail fashion).
Some satellites (macrosatellites) are located at the ends of the DNA and are called \emph{telomeres}, wheres others appear in centromeres.

\subsection{Identifying functionally important elements in the human reference genome}

\begin{leftbar}
    ORF -- open reading frame -- a part of the genome that can be transcribed (start codon $ \rightarrow $ functional codons $ \rightarrow $ stop codons).
\end{leftbar}

To annotate genes, we may first mask the repeats (using, for example, the RepeatMasker software).
Then we sequence the \emph{transcriptome} (all RNA molecules) and try to map them back to genome. It becomes increasingly clear, however, that \textbf{not everything that is transcribed is a gene}, for example, promoters and enhancers generate small transcripts when they are active.

Identifying gene switches is much more complicated.
One technique for finding gene switches is // 5:36

\end{document}
