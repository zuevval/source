%XeLaTeX+makeIndex+BibTeX OR LuaLaTeX+...
\documentclass[a4paper,12pt]{article} %14pt - extarticle
\usepackage[utf8]{inputenc} %русский язык, не менять
\usepackage[T2A, T1]{fontenc} %русский язык, не менять
\usepackage[english, russian]{babel} %русский язык, не менять
\usepackage{fontspec} %различные шрифты
\setmainfont{Times New Roman}

%\defaultfontfeatures{Ligatures={TeX},Renderer=Basic}
\usepackage[hyphens]{url} %ссылки \url с переносами
\usepackage{hyperref} %гиперссылки href
\hypersetup{pdfstartview=FitH,  linkcolor=blue, urlcolor=blue, colorlinks=true} %гиперссылки
\usepackage{subfiles}%включение тех-текста
\usepackage{graphicx} %изображения
\usepackage{float}%картинки где угодно
\usepackage{textcomp}

\usepackage{dsfont}%мат. символы
%\usepackage[style=gost-footnote]{biblatex} %ГОСТ https://www.ctan.org/pkg/biblatex-gost С ним только хуже! Не использовать его!

\addto\captionsrussian{\def\refname{Использованные источники}} %не работает с этими пакетами в LuaLaTeX :(

\begin{document}
\title{Устройство для обучения незрячих чтению рельефным шрифтом <<Тренажёр Брайля>>}
\maketitle
\textbf{Читает Валера (слайды 5-12)}\\
Давайте задумаемся о положении слепого в современном обществе. Положение удручающее. За примером далеко ходить не надо. Близко от Политеха был завод <<Контакт>>, где работали слепые в особых условиях. Лет десять назад завод закрыли, затем снесли. На его месте построят жилой массив.\\
Вдобавок ко всем несчастьям огромная проблема - невозможность читать и писать. Впрочем, уже давно француз Луи Брайль изобрёл изобрёл рельефно-точечный шрифт. Можно читать на ощупь выпуклые точки на плотной бумаге, а писать, накалывая с обратной стороны.\\
В мире всё больше незрячих, но всё меньше знают азбуку Брайля. Обучение непростое и небыстрое, не везде есть специальные центры. Мы, студенты, вряд ли прстроим новый завод для незрячих, зато можно обучать слепых азбуке Брайля, а грамотность по Брайлю существенно повышает шансы найти работу.\\
В мастерской ФабЛаб Политех с 2016 года команда, которой руководил сперва Глеб Андреевич Мирошник, затем я, разрабатывает тренажёр Брайля - аппарат с одной ячейкой, имитирующей любой символ шеститочечного алфавита Брайля. Было создано несколько образцов, внизу слайда - созданный зимой. Он соединяется с ПК по USB-кабелю и играет роль приставки, на компьютере - обучающие программы, написанные на языке Python.\\ %фото Глеба Андреевича с сайта и видео работы устройства
В рамках курса мы собрали команду \textit{(Лёша, увы, в Соединённых Штатах)}. Изначально планировали создать два прибора и более развитые программы. Один прибор - улучшенная модель зимнего, другой - полностью автономный, под управлением одноплатного компьютера. Особенность тренажёра - низкая стоимость изготовления, 2-4 тысячи рублей за штуку, поэтому спонсоров не искали.\\ %слайд с командой
В работе пользовались TG, Skype, Discord и виртуальной Kanban-доской на GitHub.\\ %слайд с канбан-доской
Начали с создания ячейки. Шесть севоприводов - шесть точек Брайля. Шестерёнки двигают рейки вверх-вниз. \\ %слайд с ячейкой Брайля - с моделью
\textbf{Читает Саша (слайды 13-15)}\\
Затем подключили джойстик, спаяли шестикнопочную клавиатуру, динамик. На лазерном станке вырезали корпус. Сделали накладки на USB-кабель с надписями по Брайлю - "Верх".\\ %Видео и/или фото в проекциях + кабель
Программисты тем временем сделали приложение - заметки, калькулятор, блиц-опрос. Занялись голосовым распознаванием, которое нужно для ответов в проверочных тестах. \\ %просто код и ветки, м. б. ускоренное видео
К апрелю один аппарат был сделан, и мы поехали с ним в Общество Слепых. \\ %фото устройства
\textbf{Читает Миша (слайды 16-19)}\\
В региональной организации Общества Слепых в целом похвалили работу. Но экспертам не понравилась попытка озвучить уроки живым голосом. Сказали, нужен либо профессиональный диктор, либо синтезатор. (\textit{слайд с фото из ВОС})\\ %видео или фото с Ниной Константиновной (б/звука)
Тогда мы подключили синтезатор RHVoice. Помимо прочего, теперь можно менять скорость воспроизведения.(\textit{видео смены скорости})\\ 
Начали экспериментировать с одноплатным компьютером, но не удалось подключить драйвер сервоприводов. Вот что нам удалось:(\textit{видео с контроллером})\\
Тогда мы сделали устройство снова на Arduino, но добавили клавишу пробела и квадратную кнопку, по нажатию которой можно прослушать помощь - в каком приложении находится пользователь, что можно сделать. Добавили защиту от неправильного подключения питания.\\ %слайд со вторым тренажёром или взрыв-схема
\textbf{Снова Валера}\\
В итоге построены два прототипа, написаны программы - более двух тысяч строк кода на GitHub.\\
Весной проект выдвинули на всероссийский конкурс "Реактор", заняли третье место. Летом планируем создать более совершенную механику, осенью улучшать программы, возможно, напишем приложение на Java для смартфонов.\\
Спасибо за внимание.\\ %слайд с благодарностями и ссылкой на GitHub
%Запасные слайды: Умка-01 и Taptilo; Nano Pi; алафвит Брайля; устройство ячейки, старой ячейки, пьезоэлектрической ячейки. Списки библиотек Python, RHVoice. GitHub - ветки, проект. Евгений и нейросеть
\end{document}