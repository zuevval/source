%XeLaTeX+makeIndex+BibTeX OR LuaLaTeX+...
\documentclass[a4paper,12pt]{article} %14pt - extarticle
\usepackage[utf8]{inputenc} %русский язык, не менять
\usepackage[T2A, T1]{fontenc} %русский язык, не менять
\usepackage[english, russian]{babel} %русский язык, не менять
\usepackage{fontspec} %различные шрифты
\setmainfont{Times New Roman}

%\defaultfontfeatures{Ligatures={TeX},Renderer=Basic}
\usepackage[hyphens]{url} %ссылки \url с переносами
\usepackage{hyperref} %гиперссылки href
\hypersetup{pdfstartview=FitH,  linkcolor=blue, urlcolor=blue, colorlinks=true} %гиперссылки
\usepackage{subfiles}%включение тех-текста
\usepackage{graphicx} %изображения
\usepackage{float}%картинки где угодно
\usepackage{textcomp}


%\usepackage[dvips]{color} %попытка добавить цвета типа OliveGreen. Пропадают все картинки

\usepackage{dsfont}%мат. символы
%\usepackage[style=gost-footnote]{biblatex} %ГОСТ https://www.ctan.org/pkg/biblatex-gost С ним только хуже! Не использовать его!
\newcommand{\ovr}[1]{\overrightarrow{#1}}
\usepackage{listings} %code formatting
\lstset{language=sql,keywordstyle=\color{blue},tabsize=2,breaklines=true,
morekeywords={if, CONCAT, is}}
%хорошая статья об этом пакете: http://mydebianblog.blogspot.com/2012/12/latex.html

\addto\captionsrussian{\def\refname{Использованные источники}}

\begin{document}
\title{Устройство для обучения незрячих чтению рельефным шрифтом <<Тренажёр Брайля>>. Cборник}
\maketitle
\tableofcontents
\section{Текст устного выступления на открытии Школы ФабЛаб 04.02.2019}
Добрый день. Я расскажу вам про тренажёр для обучения чтению по системе Луи Брайля незрячих людей. В качестве вступления послушайте короткую историю о слепых людях.\\
Десять лет назад недалеко от Политеха, стоял небольшой завод под названием Контакт. На этом заводе работали слепые. Они делали игрушки, ещё что-то. Слепых селили в домах неподалёку, и для них была специальная инфраструктура. Но лет десять назад завод был закрыт, а затем и снесён. На его месте собираются возводить жилой дом.\\
Это  я к тому, насколько тяжело инвалидам по зрению жить нормальной жизнью. Но хорошим подспорьем для слепого человека стало бы умение читать и писать специальным рельефным шрифтом Брайля. (слайд). Сейчас появились аудиокниги и прочая электроника, но знание азбуки Брайля до сих пор полезно. Чтение (даже по Брайлю) - это приятно, возможность делать заметки разгружает память, а главное, это очень помогает найти работу, помогает найти хоть какую-то занятость. \\
(слайд) В стенах ФабЛаб'а под руководством Глеба Андреевича Мирошника велись разработки тренажёра Брайля - самоучителя для освоения рельефного шрифта в домашних условиях, без преподавателей и дорогого оборудования. Были построены несколько прототипов; непременный элемент тренажёра - ячейка Брайля (слайд, м. б. с видео), которая выводит любой символ шеститочечной системы. Незрячий ощупывает ячейку и учит либо повторяет символы. \\
Глеб Андреевич в своих прототипах использовал пьезоэлектрические ячейки, которые не встречаются в свободной продаже.  (слайд) Наша команда планирует сделать тренажёр с иной ячейкой, где вместо пьезопластинок точки формируются электромоторами. Получается громоздко, но зато такая ячейка Брайля производится из доступных компонентов. Я думаю, родственники незрячего без труда смогут приобрести подобный тренажёр, а, может, и сделать самостоятельно, если у них есть 3D-принтер. В течение школы мы хотим сделать само устройство и написать несколько программ для него,  главным образом - обучающее приложение с использованием распознавания голоса. Я надеюсь, появление такого несложного тренажёра и программ к нему сделают обучение системе Брайля более лёгким делом.\\
Спасибо ФабЛаб'у за предоставление питательной почвы для проекта. Спасибо моим товарищам, пришедшим работать со мной, особенно Ивану и Евгению из Дальневосточного Федерального университета. Евгений будет не только программистом, но и нашим незрячим экспертом. Команда уже сформирована; если Вам очень хочется присоединиться, лучше поговорить со мной отдельно.\\
Благодарю за внимание.


\section{Описание проекта <<Тренажёр Брайля>>  - заявка на школу ФабЛаб 2019}
\subsection{Краткое описание}
Предлагается собрать и запрограммировать \textbf{тренажёр Брайля} -- прибор для обучения слепых людей \textbf{шрифту Брайля}. Идея шрифта Брайля -- представление буквы в виде группы выпуклых точек, оттиснутых на бумаге или иной поверхности (иногда такие есть на упаковках с лекарствами). Слепые могут читать на ощупь, но должны выучить обозначение каждой буквы.\\
Основной элемент прибора -- электромеханическая \textbf{ячейка Брайля}, которая посредством сервоприводов воспроизводит буквы шеститочечного шрифта Брайля. Планируется, ориентируясь на разработанный заблаговременно прототип ячейки, создать несколько обучающих программ для ПК или одноплатного компьютера Nano Pi. Непосредственно управлять сервоприводами будет контроллер Arduino Uno; планируется написать программу для него. Помимо того, планируется усовершенствовать и доработать ячейку Брайля.

\subsection{Актуальность}
Число инвалидов по зрению в России по состоянию на 2011 год оценивалось в 103 тысячи человек, и предполагается дальнейшее увеличение их количества\cite{stat}. При этом число людей, владеющих шрифтом Брайля, снижается \cite{brailleLower}. Причина как в относительной дороговизне специальных книг, так и в сложности обучения, в неразвитости системы обучения. При этом знание шрифта Брайля позволяет слепому читать, учиться и даже работать.\\
В настоящее время выпускаются дисплеи Брайля -- приставки к компьютеру с компактными пьезоэлектрическими ячейками. Эти устройства позволяют читать до 80 символов Брайля одновременно и вводить текст с помощью особой клавиатуры из восьми клавиш (по одной на каждую точку в букве азбуки Брайля). Основной их недостаток -- цена (в среднем 150 тысяч русских рублей по состоянию на январь 2019 года).

\subsection{Предшествующие работы}
Были предприняты многочисленные попытки создания более доступного дисплея Брайля.  В их числе нитиноловые \cite{sma}, на основе электрической стимуляции кожи \cite{kaji1} \cite{kaji2}, на основе электромоторов \cite{bristol}, круговые \cite{rotDisp}. В 2016 году автор статьи пытался разработать собственную конструкцию кругового дисплея Брайля (рис. \ref{myRoundDisp}), но из-за сложности создания миниатюрной механики проект был свёрнут.\\

В 2016-2017 годах команда под руководством магистра кафедры <<Теоретическая Механика>> Глеба Андреевича Мирошника создала несколько прототипов тренажёра Брайля - дисплея с одной ячейкой, ориентированного на обучение слепых, а не на длительное чтение \cite{gleb}. После ряда попыток создать ячейку Брайля на основе сервоприводов было решено отказаться от них в пользу обычной, более надёжной пьезоэлектрической ячейки. Такой тренажёр компактный, но трудновоспроизводимый: ячейки дисплея Брайля, по имеющимся данным, не продаются отдельно от дисплеев.

\subsection{Цели работы}
Цель - создать надёжный и простой в изготовлении тренажёр Брайля и комплекс программ для него. Предлагается развить идею, реализованной командой Г. А. Мирошника. Ячейка Брайля будет создана с применением только доступных компонентов (сервоприводов, контроллеров Arduino и/или Raspberry Pi, деталей, пригодных к изготовлению на большинстве 3D-принтеров). За счёт открытого распространения материалов проекта предполагается возможным легко воспроизводить дисплей и сделать обучение шрифту Брайля доступным для людей по всему миру.

\subsection{Описание конструкции}
Создан прототип ячейки (рис. \ref{photo1}). Точки символа Брайля - головки подвижных штырьков. Штырьки приклеены к рычагам и двигаются вверх-вниз, отчего точки появляются над поверхностью или убираются внутрь. Движение обеспечивают сервоприводы с насаженными на вал шестернями. Каждый рычаг на противоположном от штырька конце имеет зубчатую рейку, которой передаётся движение шестерни.


\subsection{Текущие задачи и планируемое решение задач}
\subsubsection{Базовые задачи}
\begin{itemize}
\item{}Написать программу для Arduino, которая выводит текст, отправленный через Serial.
\item{}Снабдить дисплей органами управления (кнопки или джойстик). По нажатию кнопки передавать сигнал по Serial.
\item{}Составить два обучающих приложения на Python, общающихся с Arduino через Serial. Одно для запоминания азбуки: воспроизводит на экране буквы и сопровождает звуком; второе - тест на знание азбуки: дисплей выводит букву, пользователь должен произнести её вслух. Компьютер определяет с помощью распознавания голоса, верно ли ответил пользователь.
\item{}Спроектировать и создать корпус устройства
\item{}Разместить материалы проекта в свободном доступе (к примеру, на платформе Github), чтобы предоставить возможность создания подобного прибора всем желающим.
\end{itemize}
\subsubsection{Приблизительный список дополнительных задач}
\begin{itemize}
\item{}Адаптировать программы на Python к контроллеру Raspberry Pi/Nano Pi. Создать отдельный опциональный модуль хост-контроллера с динамиком и микрофоном в собственном корпусе.
\item{}Составить меню тренажёра Брайля (список приложений) с возможностью навигации джойстиком. Названия пунктов меню выводятся с помощью ячейки Брайля или произносятся вслух динамиками компьютера.
\item{}Написать иные программы (обучающие или, возможно, для чтения текста). Пример - часы (показать время с помощью ячейки Брайля).
\item{}Повысить компактность ячейки Брайля: взять подшипники меньшего размера, перекомпоновать сервоприводы (разместить два из них на другой стороне рычагов).
\item{}Попробовать управлять серво с помощью ШИМ контроллера PCA9685
\item{}Создать ячейку с восемью точками (расширенный шрифт Брайля).
\item{}Снабдить тренажёр особой клавиатурой из восьми клавиш. Снабдить встроенным динамиком и, возможно, лампочками для зрячего иструктора (индикатор питания и подключения хост-контроллера).
\item{}Предусмотреть в конструкции механизм для автокалибровки ячейки.
\item{}Уделить внимание не техническим аспектам (презентовать проект, возможно, создать видеоклип/сайт проекта).
\end{itemize}

\section{Глобальные задачи и идеи проекта на GitHub}
\subsection{Миссия проекта}
В России и мире всё больше незрячих, но всё меньше владеют азбукой Брайля (трудно учиться, мало учителей, слепым сложно добираться до специальных центров). Тренажёр Брайля - простой и дешёвый в изготовлении прибор, приставка к компьютеру для обучению шрифту Брайля + комплекс программ для ПК.
Важно, чтобы программы для ПК или одноплатного компьютера были легко адаптируемы к разным типам и количеству ячеек Брайля. Пока что в приборе одна шеститочечная ячейка.
\subsection{глобальные задачи}
\begin{itemize}
\item{} Сделать корпус тренажёра
\item{} Сделать органы управления тренажёром (джойстик)
\item{} Запрограммировать передачу сигналов через USB и приём текста
\item{} Составить обучающее приложение для комптютера, проверку знаний с распознаванием голоса, часы Брайля
\item{} Разместить материалы проекта в свободном доступе для возможного воспроизведения другими людьми
\item{} Сделать несложную программу для самостоятельной работы тренажёра без компьютера
\end{itemize}

\subsection{Идеи по Python и Arduino:}
\begin{itemize}
\item{} Попробовать вместо ПК Nano Pi вместе с PWM Servo Shield (I2C), подключать к нему наушники и микрофон
\item{} приложение -замена игрального кубика (вывод случайного числа от 1 до 6)
\item{} Приложение - калькулятор
\item{} Снабдить устройство собственным динамиком (сигнал при подкл. компьютера, при нажатиях)
\item{} Добавить в программу для Arduino функционал при отсутствии подключения к компьютеру (азбука, если можно, часы)
\end{itemize}

\subsection{Идеи по железу:}
\begin{itemize}
\item{} Печатать колпачки клавиш из гибкого филамента
\item{} Доробатывать текущее поколение ячейки
	\begin{itemize}
	\item{} сделать вторую перемычку по центру (она же для поддержки платы контроллера)
	\item{} сделать крепления серво без винтов
	\item{} сделать ячейку компактнее (перекомпоновать серво)
	\item{} поставить подшипники меньшего размера
	\end{itemize}

\item{} Разрабатывать поколение 2 (с линейным перемещением)
\item{} Разрабатывать поколение 3 (восьмигранный барабан как [здесь](http://www.ecs.umass.edu/ece/sdp/sdp16/team15/sdp16.my-free.website/index.html) (или его модификацию - четверьокружности и серво)). Также можно диск с возможностью набора (как телефонный)
\end{itemize}

\section{на Реактор (заготовка)}
Тренажёр Брайля - устройство для обучения чтению и набору рельефным шрифтом

Предлагается путь решения проблемы чтения и письма среди незрячих людей.
Число инвалидов по зрению в России по состоянию на 2011 год оценивалось в 103 тысячи человек, и предполагается дальнейшее увеличение их количества
Инвалидам по зрению тяжело жить нормальной жизнью. Но хорошим подспорьем для незрячего или слабовидящего человека стало бы умение читать и писать специальным рельефным шрифтом Брайля. Сейчас появились аудиокниги и системы экранного доступа для управления компьютером, но знание азбуки Брайля до сих пор полезно. Чтение (даже по Брайлю) - это приятно, возможность делать заметки разгружает память, а главное, умение читать и писать по Брайлю очень помогает найти работу, хоть какую-то занятость. По статистике, среди взрослых незрячих грамотные находят работу в 90\% случаев, неграмотные - менее чем в 30\%!

Разработан прибор для самостоятельного обучения чтению и набору символов Брайля. Это приставка к компьютеру или отдельное устройство, которое должно заменить инструктора, избавив инвалида по зрению от необходимости добираться на курсы (или заменить курсы, если таких нет в месте, где он живёт).

Обычное обучение проходит так: для запоминания каждую букву предлагают пощупать и вслух произносят, где какая. Тем, кто имеет остаточное зрение, можно для скорейшего заучивания показывать соответствующие обычные, плоскопечатные буквы крупным шрифтом. Помимо того, чтобы «пальцы помнили» символы, ученик должен знать буквы по номерам точек (точки пронумерованы от одного до шести), чтобы набирать на особой шести- или восьмиклавишной клавиатуре.

В стенах мастерской цифрового производства ФабЛаб Политех велись и ведутся разработки тренажёра Брайля - самоучителя для освоения рельефного шрифта в домашних условиях, без преподавателей и дорогого оборудования. Были построены несколько прототипов.

Тренажёр Брайля сконструирован и запрограммирован так, что может воспроизводить вслух тексты уроков и выводить при этом символы Брайля. Символы рельефной азбуки выводит ячейка Брайля - механизм с вертикально расположенными штырьками в гнёздах. Поднимаясь над поверхностью тренажёра, кончики штырьков имитируют выпуклые точки.

Как уже было сказано, прибор может быть самостоятельным или приставкой к компьютеру. Если это приставка, буквы не только выводятся ячейкой Брайля, но и крупным шрифтом показаны на экране компьютера.

Кроме уроков предусмотрены проверочные задания. Нужно потрогать и опознать символ. Используется голосовое распознавание (была написана нейронная сеть на Python с использованием библиотеки Tensorflow).

Для ввода символов сделана шеститочечная клавиатура. Для управления тренажёром предусмотрен четырёхпозиционный джойстик. Управляет всеми узлами контроллер Arduino (в варианте приставки) или Nano Pi (в самостоятельном варианте). На текущий момент сделана только приставка.

Основной элемент тренажёра - ячейка Брайля. Сейчас производятся дисплеи Брайля - устройства для чтения и набора шеститочечных символов с 20-40 ячейками, где штырьки двигаются вверх-вниз пьезопластинами. Такие дисплеи стоят от 100 до 200 тысяч рублей и обычно недоступны незрячим. Тренажёр, в отличие от дисплея, содержит только одну ячейку (этого достаточно для обучения), приводимую в движение сервоприводами. Стоимость тренажёра составит примерно 2000 р (приставка) или 4000 р (отдельный аппарат). Предполагается, что близкие незрячего смогут без труда приобрести аппарат, или, при наличии 3Д-принтера, самостоятельно собрать такой же (в конструкции используются повсеместно распространнённые узлы и материалы).

\subsubsection{Решение (ход работы над проектом)}
Осенью 2016 года, когда задумывался проект, был произведён обзор литературы.
(Здесь я должен поблагодарить Глеба Андреевича Мирошника, магистра кафедры Теоретическая Механика СПбПУ за советы и идеи, а впоследствии за поддержку и развитие проекта). Обнаружены многочисленные попытки создания более простой и доступной (по сравнению с пьезоэлектрическими ячейками) конструкции дисплея Брайля. В их числе нитиноловые (движение обеспечивается сплавами с памятью формы), на основе электрической стимуляции кожи (наиболее в этом направлении продвинулись японцы), на основе электромоторов, круговые (экспериментальный круговой дисплей был создан в Национальном институте стандартов США). 

Зимой 2016 года автор пытался разработать собственную конструкцию кругового дисплея Брайля. В той конструкции 48 штырьков были помещены внутри вращающегося барабана так, что буквы должны были формироваться на боковой поверхности цилиндра. Проворачиваясь вместе с барабаном, словно щётка снегоуборочной машины, штырьки должны были образовывать бегущую строку, а особый механизм на основе сервоприводов - обновлять текст.  Была разработана модель в CAD-программе DS SolidWorks 2014 и написана программа для Arduino, но из-за сложности создания миниатюрной механики проект был свёрнут. Основную деталь конструкции - барабан - не удалось распечатать на FDM 3D-принтере в нужном качестве.

В 2016-2017 годах команда под руководством Г. А. Мирошника создала несколько прототипов тренажёра Брайля  с одной шеститочечной ячейкой, ориентированного на обучение, а не на длительное чтение. После ряда попыток создать ячейку Брайля на основе сервоприводов было решено отказаться от них в пользу обычной, более надёжной пьезоэлектрической ячейки. Такой тренажёр компактный, но трудновоспроизводимый: ячейки дисплея Брайля, по имеющимся данным, не продаются отдельно от дисплеев.

Летом и осенью 2018 года автор разработал и реализовал собственную конструкцию ячейки на основе сервоприводов. С учётом опыта разработки кругового дисплея и прочих попыток была придумана несложная конструкция: шесть сервоприводов, по одному на точку, посредством зубчатой передачи двигают вверх-вниз рычаги-"качели". Противоположное плечо каждого рычага оканчивается штырьком, который ходит вверх-вниз почти вертикально.

Сперва вместо зубчатой передачи была попытка сделать кривошипно-шатунный механизм, но это оказалось не так надёжно. Конструкция с зубчатой передачей оказалась очень надёжной и работает без перебоев.

Детали механизма смоделированы в DS SolidWorks 2014 и распечатаны на 3D-принтере Creality Ender 3 с помощью слайсера Cura 3.6.0. Сервоприводы - Tower Pro SG90. Для крепления рычагов к общей раме используются радиальные подшипники 624ZZ. Профиль шестерёнки, насаженной на вал мотора, и ответный профиль планетарной шестерни были смоделированы в программе Gear Template Generator, после чего в Solidworks эвольвенты восстановлены по точкам отрезками окружностей. Опытным путём установлено, что внутренний диаметр шестерни должен составлять 5.1 мм (для зацепления с валом мотора на внутренней поверхности также сделан 21 зуб высоты 0.5 мм). Шестерёнки были распечатаны с разрешением 0.1 мм, для остальных деталей оказалось достаточно 0.2 мм.

Много времени было потрачено на поиск подходящего материала для изготовления штырьков диаметром 1.5 мм. В 2017 году вышеупомянутый Г. А. Мирошник пробовал использовать тонкий стальной тросик, затем печать на фотополимерном 3D-принтере, но такие штырьки оказались неприятными на ощупь. В нашей конструкции сначала использовались отрезки 1.75мм PLA-прутка для 3D-принтера, отшкуренного до нужного диаметра, но оказались слишком толстыми и ломкими. В декабре 2018 года было найдено решение: штырьки сделали из зубцов массажной расчёски (материал, по предположению, - HDPE, полиэтилен высокой плотности).

В течение Зимней Школы ФабЛаб с 4 по 8 февраля 2019 года на основе ячейки создан полноценный программно-аппаратный комплекс.
Вначале был смоделирован и изготовлен корпус. Панели вырезаны из фанеры на лазерном станке, пластмассовые уголки распечатаны на 3Д-принтере. Кроме ячейки в корпусе поместился контроллер с платой Arduino Nano Sensor Shield, к которому удобно подключены 6 сервоприводов и остальные узлы.
Был подключён четырёхпозиционный джойстик и написаны программы: на C++ (точнее, на его версии для Arduino) - программа, печатающая буквы и цифры, передаваемые через Serial-порт и передающая обратно сигналы от джойстика. На языке Python 3.6 была написана демонстрационная программа для ПК - первые шаги обучения: знакомство с буквами "А" и "Б". По задумке, обучение разбито на уроки, за которыми следуют тесты. Урок разбит на шаги; в каждом шаге надо прослушать сообщение и ощупать текст, выведенный ячейкой. Джойстиком можно переключаться между шагами.
Пятого февраля мы съездили в региональное отделение Всероссийского Общества Слепых в СПб и показали полученное устройство незрячему эксперту Нине Константиновне Балан. Был получен в целом положительный отзыв; как отметила Нина Константиновна, точки, образованные штырьками-зубцами расчёски прятные на ощупь.
Далее был добавлен динамик, озвучивающий нажатия джойстика, и сбоку прибора - кнопка выключения звука нажатий. Было улучшено и расширено обучающее приложение; помимо уроков, добавлены тесты с использованием голосового распознавания. Как и урок, тест разбит на шаги; на каждом шаге надо ощупать текст и, распознав его, произнести вслух после сигнала. Программа на Python при помощи библиотеки speechRecognition обрабатывает ответ и говорит, верно или нет.
С помощью оконной библиотеки PyQT5 был обеспечен параллельный вывод букв крупным шрифтом на экран (как говорится выше, это поможет имеющим остаточное зрение быстрее запомнить буквы).
К концу школы один из нас, Евгений Некрасов из Дальневосточного Федерального Университета, написал нейросеть с использованием библиотек numpy, scipy и tensorflow, которая может распознавать произнесённые вслух буквы. Так в тестах можно обойтись без подключения к интернету, которого требует библиотека speechRecognition.
Помимо обучающего приложения были сделаны ещё два: азбука (воспроизведение алфавита вслух и по Брайлю), часы (ориентировано в перспективе на слепоглухих).
К концу школы ФабЛаб была изготовлена клавиатура, которая обеспечит возможность обучения набору и созданию коротких заметок. Дизайн клавиатуры повторяет конфигурацию клавиш в приборе Pronto - дисплее Брайля, предназначенном для создания заметок. В настоящий момент команда работает над подключением клавиатуры к тренажёру.

\subsubsection{Результаты}
Сконструирован тренажёр Брайля (приставка к компьютеру): ячейка, контроллер Arduino, джойстик, динамик и шесть клавиш в эргономичном корпусе с минимальным количеством резьбовых соединений. Удалось добиться надёжной работы механики.
Написаны три приложения на языке Python 3.6: часы, азбука и, главное, обучающее приложение. В обучающем приложении применяется машинное обучение для распознавания речи.
Получены отзывы от сотрудницы библиотеки для слепых и от незрчего эксперта Всероссийского Общества Слепых. Учтены выданные рекомендации.

Смоделирована и распечатана на 3Д-принтере экспериментальная накладка на USB-кабель с надписью "верх" шрифтом Брайля для удобства подключения к разъёму компьютера.

Удалось создать прибор для обучения азбуке Брайля, затратив на компоненты и материалы всего 1700 рублей, при этом не прибегая только к повсеместно распространённым деталям и технологиям. Поэтому тренажёр может быть легко воспроизведён в любом ФабЛабе Земли.

В продолжение работы будет создана более компактная ячейка Брайля, а в корпусе лицевая панель, скорее всего, будет сделана из оргстекла. Планируется добавить несколько ячеек Брайля сразу (восемь). Также Sensor Shield будет заменён драйвером PWM Servo Shield, более экономно расходующим питание.
Планируется также создать автономную версию на основе одноплатного компьютера Nano Pi.

Все файлы (программы, 3Д-модели) доступны по ссылке: https://github.com/zuevval/braille

%ссылки в закладках браузера
%инструкции в файлах в Braille 2018-19

Я надеюсь, появление такого несложного тренажёра и программ к нему сделают обучение системе Брайля более лёгким делом.


\begin{thebibliography}{99}
\bibitem{gleb}{Тренажёр Брайля. Глеб Андреевич Мирошник. [Электронный ресурс] \url{http://www.brailletrainer.ru}}
\bibitem{sma}{Microtuators of SMA for Braille display system \url{https://www.researchgate.net/publication/232633029_Microtuators_of_SMA_for_Braille_display_system}}
\bibitem{kaji1}{Electro-Tactile Display with Force Feedback. Hiroyuki Kajimoto, Naoki Kawakami, Taro Maeda and Susumu Tachi: Graduate School of Information Science and Technology, The University of Tokyo. [Электронный ресурс]. \url{http://citeseerx.ist.psu.edu/viewdoc/download?doi=10.1.1.493.2786&rep=rep1&type=pdf}}
\bibitem{kaji2}{Electro-Tactile Display with Tactile Primary Color Approach.
Hiroyuki Kajimoto, Naoki Kawakami, Susumu Tachi. Graduate School of
Information Science and Technology, The University of Tokyo.  [Электронный ресурс]. \url{http://files.tachilab.org/publications/intconf2000/kajimoto200410IROS.pdf}}
\bibitem{bristol}{Bristol Braille technology [Электронный ресурс] \url{http://bristolbraille.co.uk}}
\bibitem{rotDisp}{Refreshable braille reader. John W. Roberts, Oliver T. Slattery, David W. Kardos. [Электронный ресурс] \url{https://patents.google.com/patent/US6776619}}
\bibitem{stat}{Оценка динамики и прогноз первичной инвалидности в Амурской области вследствие офтальмопатологии. Выдров А. С. [Электронный ресурс] \url{https://cyberleninka.ru/article/v/otsenka-dinamiki-i-prognoz-pervichnoy-invalidnosti-v-amurskoy-oblasti-vsledstvie-oftalmopatologii}}
\bibitem{brailleLower}{Актуальность рельефно-точечного шрифта по системе брайля в современном мире. [Электронный ресурс] \url{https://beltiz.by/talking/1068-2013-12-19-11-49-03}}
\bibitem{blindLib1}{М. П. Коновалова,  О. Ю. Жарова. Технические средства реабилитации для людей с ограниченными возможностями (на примере опыта работы Калужской областной специальной библиотеки для слепых им. Н. Островского) (окончание). [Электронный ресурс] \url{https://cyberleninka.ru/article/v/tehnicheskie-sredstva-reabilitatsii-dlya-lyudey-s-ogranichennymi-vozmozhnostyami-na-primere-opyta-raboty-kaluzhskoy-oblastnoy-1}}
\bibitem{octagon}{\url{http://faculty.montgomerycollege.edu/auchechu/AbanuloProfile/Senior%20Design/final-presentaton.pdf}}
\bibitem{statUSA}{\url{https://brailleworks.com/braille-literacy-statistics/}}
\bibitem{gloves}{DIY haptic gloves \url{https://www.youtube.com/watch?v=uZKo2RqXKr4}}
\end{thebibliography}
\end{document}