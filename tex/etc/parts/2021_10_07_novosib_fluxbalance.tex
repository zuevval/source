\documentclass[main.tex]{subfiles}
\begin{document}
	
\section{ HiSeq | Анализ потоков }

\subsection{Стехиометрическая матрица как линейное отображение}


\emph{Ядро линейного отображения } -- те векторы, которые отображаются в нулевой вектор.
По теореме о ранге и дефекте можно найти размерность ядра.
Это размерность пространства потоков минус размерность образов.

Известно, что ранг линейного отображения равен рангу матрицы. 

Вводится понятие \emph{множества стационарных потоков} -- все векторы из ядра матрицы.
Т.о. биологический смысл стехиометрической матрицы -- множество всех потоков.
	
\end{document}
