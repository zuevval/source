\documentclass[a4paper,12pt]{article}
\usepackage[subpreambles=true]{standalone}
\usepackage[utf8]{inputenc}
\usepackage[english, russian]{babel}
\usepackage{amssymb,amsmath}
\usepackage{gensymb}
\usepackage{color}
\usepackage{graphicx}
\usepackage[document]{ragged2e}
%\usepackage{longtable}
%\usepackage{rotating}
\usepackage{array}
\usepackage{multirow}
%\usepackage{subfiles}
\usepackage{float}

\renewcommand{\bfdefault}{b}
\newcommand{\ovr}[1]{\overrightarrow{#1}}
\newcommand{\lk}{\guillemotleft}
\newcommand{\rk}{\guillemotright}

%\addto\captionsrussian{\def\refname{}}

\begin{document}
\begin{titlepage}
\begin{centering}

\textbf{Санкт-Петербургский Политехнический университет Петра Великого}
\par
\textbf{Институт прикладной математики и механики}
\par
\textbf{Кафедра \lkТеоретическая механика\rk}
\vspace{10 pt}
\hrule
\vfill

\LARGE{Отчёт по лабораторной работе $4$}

\vfill

\Large \textbf {Нахождение замечательных точек на графике многочлена}
\vfill
\vfill

\normalsize
\begin{flushright}
Выполнил:\hspace{10pt}\par
Зуев Валерий\par
студент 1 курса (гр. 13632/2)\par
Проверил:\hspace{10pt}\par
Лапин Руслан Леонидович\par
Аспирант каферды \lkТеоретическая Механика\rk\par
\end{flushright}

\vfill
\vfill

Санкт-Петербург
\par
2017

\end{centering}
\end{titlepage}

\setcounter{page}{2}
\center{\Large \textbf{Отчёт по работе \lkНахождение замечательных точек на графике многочлена\rk} \par}
\normalfont
\justify
\section{Задача:}
\begin{enumerate}
\item{Создать матрицу случайных чисел от -10 до 10.}
\item{Встроенными методами Matlab найти характеристический полином матрицы, найти его корни.}
\item{Взять производную, найти корни производной (экстремумы функции), построить график производной, отметить корни.}
\item{Взять вторую производную, с ней проделать то же.}
\item{Построить график полинома с отмеченными разным цветом и стилем корнями, экстремумами и точками смены выпуклости, добавить к графику легенду.}
\end{enumerate}
\section{Решение:}
Были созданы три файла: в одном хранится матрица $10\times10$, во втором программа, выполняющая построение графика, поиск корней многочлена, первой и второй производной; из третьего можно изменять содержимое первого файла.\par
\vspace{10pt}
\hrule
\vspace{10pt}
\emph{(считывание таблицы "M" из файла matrix\_for\_polynomial)}\par
load('matrix\_for\_polynomial.m', '-mat', 'M');\par
\emph{(построение характеристического полинома в виде вектора коэффициентов)}\par
p = poly(M);\par
X=-12.5:0.1:10;\par
Y=polyval(p,X);\par
\emph{(Начало построения основного графика)}\par
figure\par
hold on\par
grid minor\par
plot(X,Y)\par
k = 1e10;\par
axis ([-12.5 8 -k +0.5*k])\par
{\color{green}
\%defining arrays of roots to show on the plot}\par
Roots = roots(p);\par
roots\_quantity = size (Roots);\par
YRoots = zeros(roots\_quantity(1,1), 1);\par
{\color{green}
\%trying to get rid of complex roots}\par
for i = 1:roots\_quantity(1,1)\par
\hspace{10pt}if imag(Roots(i,1))==0\par
\hspace{10pt}\hspace{10pt}pl1 = plot(Roots(i,1), YRoots(i,1), '*r');\par
\hspace{10pt}end\par
end\par
\par
{\color{green}
\%same for the derivative of p}\par
p1=polyder(p);\par
Roots1 = roots(p1)\par
roots\_quantity1 = size (Roots1);\par
YRoots1 = polyval(p, Roots1);\par
{\color{green}
\%trying to get rid of complex roots}\par
for i = 1:roots\_quantity1(1,1)\par
\hspace{10pt}if imag(Roots1(i,1))==0\par
\hspace{10pt}\hspace{10pt}pl2 = plot(Roots1(i,1), YRoots1(i,1), 'or');\par
\hspace{10pt}end\par
end\par
\par
{\color{green}
\%once again for the second derivative}\par
p2=polyder(p1);\par
Roots2 = roots(p2)\par
roots\_quantity2 = size (Roots2);\par
YRoots2 = polyval(p, Roots2);\par
{\color{green}
\%trying to get rid of complex roots}\par
for i = 1:roots\_quantity2(1,1)\par
\hspace{10pt}if imag(Roots2(i,1))==0\par
\hspace{10pt}\hspace{10pt}pl3 = plot(Roots2(i,1), YRoots2(i,1), 'sg');\par
\hspace{10pt}end\par
end\par

legend([pl1, pl2, pl3],'roots', 'exteremal points', 'roots of 2nd derivative');\par

\begin{figure}[H]
\center{\includegraphics[width=\textwidth]{graph1.jpg}}
\caption{Вывод программы}
\label{graph}
\end{figure}

\input{add}

\hrule

\newpage
\large \textbf{Программа для изменения коэффициентов матрицы:}\par
\normalsize

{\color{green}
\%M = 10.*rand(10,10)-5;\par
\%save('matrix\_for\_polynomial.m', 'M');\par
\%to change contents of file matrix\_for\_polynomial,\par
\%disable comments for first two lines\par}
load('matrix\_for\_polynomial.m', '-mat', 'M');\par
p = poly(M)\par
X=-12.5:0.1:10;\par
Y=polyval(p,X);\par
figure\par
hold on\par
grid on\par
plot(X,Y)\par
{\color{green}
\%defining arrays of roots to show on the plot}\par
Roots = roots(p);\par
roots\_quantity = size (Roots);\par
YRoots = zeros(roots\_quantity(1,1), 1);\par
{\color{green}
\%trying to get rid of complex roots}\par
for i = 1:roots\_quantity(1,1)\par
\hspace{10pt} if imag(Roots(i,1))==0\par
\hspace{10pt}\hspace{10pt} plot(Roots(i,1), YRoots(i,1), '*')\par
\hspace{10pt} end\par
end\par

\end{document}