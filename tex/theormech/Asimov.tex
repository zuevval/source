\NeedsTeXFormat{LaTeX2e}
\documentclass[a4paper,12pt]{article}
%в две колонки - \documentclass[a4paper, twocolumn,12pt]{article}
\textwidth=16cm
\oddsidemargin=0cm
\topmargin=-1cm
\textheight=25cm
\newcommand{\lk}{\guillemotleft}
\newcommand{\rk}{\guillemotright}

\usepackage[utf8]{inputenc}
\usepackage[english, russian]{babel}
\usepackage{amssymb,amsmath}
\usepackage{gensymb}
\usepackage{array}

\begin{document}
\begin{centering}
GOLD\\
\end{centering}
JONAS WILLARD LOOKED FROM SIDE to side and tapped his baton on the stand before him.
He said, “Understood now? This is just a practice scene, designed to find out if we know what
we’re doing. We’ve gone through this enough times so that I expect a professional performance now. Get
ready. All of you get ready.”\\
He looked again from side to side. There was a person at each of the voice-recorders, and there
were three others working the image projection. A seventh was for the music and an eighth for the allimportant
background. Others waited to one side for their turn.\\
Willard said, “ All right now. Remember this old man has spent his entire adult life as a tyrant. He
is accustomed to having everyone jump at his slightest word, to having everyone tremble at his frown. That
is all gone now but he doesn’t know it. He faces his daughter whom he thinks of only as a bent-headed
obsequious girl who will do anything he says, and he cannot believe that it is an imperious queen that he
now faces. So let’s have the King.”\\
Lear appeared. Tall, white hair and beard, somewhat disheveled, eyes sharp and piercing.
Willard said, “Not bent. Not bent. He's eighty years old but he doesn't think of himself as old. Not
now. Straight. Every inch a king.” The image was adjusted. “That's right. And the voice has to be strong.
No quavering. Not now. Right?”\\
“Right, chief,” said the Lear voice-recorder, nodding\\
“All right. The Queen.”\\
And there she was, almost as tall as Lear, standing straight and rigid as a statue, her draped
clothing in fine array, nothing out of place. Her beauty was as cold and unforgiving as ice.\\
“And the Fool.“\\
A little fellow, thin and fragile, like a frightened teenager but with a face too old for a teenager and
with a sharp look in eyes that seemed so large that they threatened to devour his face.\\
“Good,” said Willard. “Be ready for Albany. He comes in pretty soon. Begin the scene. “ He
tapped the podium again, took a quick glance at the marked-up play before him and said, “Lear! “ and his
baton pointed to the Lear voice-recorder, moving gently to mark the speech cadence that he wanted created.
Lear says, “How now, daughter? What makes that frontlet on? Methinks you are too much o' late i'
th' frown.”\\
The Fool's thin voice, fifelike, piping, interrupts, “Thou wast a pretty fellow when thou hadst no
need to care for her frowning--”\\
Goneril, the Queen, turns slowly to face the Clown as he speaks, her eyes turning momentarily
into balls of lurid light--doing it so momentarily that those watching caught the impression rather than
viewed the fact. The Fool completes his speech in gathering fright and backs his way behind Lear in a blind
search for protection against the searing glance.\\
Goneril proceeds to tell Lear the facts of life and there is the faint crackling of thin ice as she
speaks, while the music plays in soft discords, barely heard.\\
Nor are Goneril's demands so out of line, for she wants an orderly court and there couldn't be one
as long as Lear still thought of himself as tyrant. But Lear is in no mood to recognize reason. He breaks into
a passion and begins railing.\\
Albany enters. He is Goneril's consort--round-faced, innocent, eyes looking about in wonder.
What is happening? He is completely drowned out by his dominating wife and by his raging father-in-law.
It is at this point that Lear breaks into one of the great piercing denunciations in all of literature. He is
overreacting. Goneril has not as yet done anything to deserve this, but Lear knows no restraint. He says:
“Hear, Nature, hear! dear goddess, hear!
Suspend thy purpose, if thou didst intend
To make this creature fruitful.
Into her womb convey sterility;
Dry up in her the organs of increase;
And from her derogate body never spring
A babe to honour her! If she must teem,
Create her child of spleen, that it may live
And be a thwart disnatur’d torment to her.
Let it stamp wrinkles in her brow of youth,
With cadent tears fret channels in her cheeks,
Turn all her mother’s pains and benefits
To laughter and contempt, that she may feel
How sharper than a serpent’s tooth, it is
To have a thankless child! “
The voice-recorder strengthened Lear’s voice for this speech, gave it a distant hiss, his body
became taller and somehow less substantial as though it had been converted into a vengeful Fury.
As for Goneril, she remained untouched throughout, never flinching, never receding, but her
beautiful face, without any change that could be described, seemed to accumulate evil so that by the end of
Lear’s curse, she had the appearance of an archangel still, but an archangel ruined. All possible pity had
been wiped out of the countenance, leaving behind only a devil’s dangerous magnificence.
The Fool remained behind Lear throughout, shuddering. Albany was the very epitome of
confusion, asking useless questions, seeming to want to step between the two antagonists and clearly afraid
to do so.
Willard tapped his baton and said, “ All right. It’s been recorded and I want you all to watch the
scene.” He lifted his baton high and the synthesizer at the rear of the set began what could only be called
the instant replay.
It was watched in silence, and Willard said, “It was good, but I think you’ll grant it was not good
enough. I’m going to ask you all to listen to me, so that I can explain what we’re trying to do.
Computerized theater is not new, as you all know. Voices and images have been built up to beyond what
human beings can do. You don’t have to break your speechifying in order to breathe; the range and quality
of the voices are almost limitless; and the images can change to suit the words and action. Still, the
technique has only been used, so far, for childish purposes. What we intend now is to make the first serious
compu-drama the world has ever seen, and nothing will do--for me, at any rate--but to start at the top. I
want to do the greatest play written by the greatest playwright in history: King Lear by William
Shakespeare.
“I want not a word changed. I want not a word left out. I don’t want to modernize the play. I don’t
want to remove the archaisms, because the play, as written, has its glorious music and any change will
diminish it. But in that case, how do we have it reach the general public? I don’t mean the students, I don’t
mean the intellectuals, I mean everybody. I mean people who’ve never watched Shakespeare before and
whose idea of a good play is a slapstick musical. This play is archaic in spots, and people don’t talk in
iambic pentameter. They are not even accustomed to hearing it on the stage.
“So we’re going to have to translate the archaic and the unusual. The voices, more than human,
will, just by their timbre and changes, interpret the words. The images will shift to reinforce the words.
“Now Goneril’s change in appearance as Lear’s curse proceeded was good. The viewer will gauge
the devastating effect it has on her even though her iron will won’t let it show in words. The viewer will
therefore feel the devastating effect upon himself, too, even if some of the words Lear uses are strange to
him.
“In that connection, we must remember to make the Fool look older with everyone of his
appearances. He’s a weak, sickly fellow to begin with, broken-hearted over the loss of Cordelia, frightened
to death of Goneril and Regan, destroyed by the storm from which Lear, his only protector, can’t protect
him--and I mean by that the storm of Lear’s daughter’s as well as of the raging weather. When he slips out
of the play in Act III. Scene VI, it must be made plain that he is about to die. Shakespeare doesn’t say so, so
the Fool’s face must say so.
“However, we’ve got to do something about Lear. The voicerecorder was on the right track by
having a hissing sound in the voice-track. Lear is spewing venom; he is a man who, having lost power, has
no recourse but vile and extreme words. He is a cobra who cannot strike. But I don’t want the hiss
noticeable until the right time. What I am more interested in is the background.”
The woman in charge of background was Meg Cathcart. She had been creating backgrounds for as
long as the compu-drama technique had existed.
“What do you want in background? “ Cathcart asked, coolly.
“The snake motif,” said Willard. “Give me some of that and there can be less hiss in Lear’s voice.
Of course, I don’t want you to show a snake. The too obvious doesn’t work. I want a snake there that
people can’t see but that they can feel without quite noting why they feel. I want them to know a snake is
there without really knowing it is there, so that it will chill them to the bone, as Lear’s speech should. So
when we do it over, Meg, give us a snake that is not a snake.”
“And how do I do that, Jonas?” said Cathcart, making free with his first name. She knew her
worth and how essential she was.
He said, “I don’t know. If I did I’d be a backgrounder instead of a lousy director. I only know
what I want. You‘ve got to supply it. You’ve got to supply sinuosity, the impression of scales. Until we get
to one point. Notice when Lear says, ‘How sharper than a serpent’s tooth it is to have a thankless child.’
That is power. The whole speech leads up to that and it is one of the most famous quotes in Shakespeare.
And it is sibilant. There is the ‘sh,’ the three s’s in ‘serpent’s’ and in ‘thankless,’ and the two unvoiced ‘th’s
in ‘tooth’ and ‘thankless.’ That can be hissed. If you keep down the hiss as much as possible in the rest of
the speech, you can hiss here, and you should zero in to his face and make it venomous. And for
background, the serpent--which, after all, is now referred to in the words--can make its appearance in
background. A flash of an open mouth and fangs, fangs--We must have the momentary appearance of fangs
as Lear says, ‘a serpent’s tooth.”‘
Willard felt very tired suddenly. “ All right. We’ll try again tomorrow. I want each one of you to
go over the entire scene and try to work out the strategy you intend to use. Only please remember that you
are not the only ones involved. What you do must match the others, so I’ll encourage you to talk to each
other about this--and, most of all, to listen to me because I have no instrument to handle and I alone can see
the playas a whole. And if I seem as tyrannical as Lear at his worst in spots, well, that’s my job.”
Willard was approaching the great storm scene, the most difficult portion of this most difficult
play, and he felt wrung out. Lear has been cast out by his daughters into a raging storm of wind and rain,
with only his Fool for company, and he has gone almost mad at this mistreatment. To him, even the storm
is not as bad as his daughters.
Willard pointed his baton and Lear appeared. A point in another direction and the Fool was there
clinging, disregarded, to Lear’s left leg. Another point and the background, came in, with its impression of
a storm, of a howling wind, of driving rain, of the crackle of thunder and the flash of lightning.
The storm took over, a phenomenon of nature, but even as it did so, the image of Lear extended
and became what seemed mountain-tall. The storm of his emotions matched the storm of the elements, and
his voice gave back to the wind every last howl. His body lost substance and wavered with the wind as
though he himself were a storm cloud, contending on an equal basis with the atmospheric fury. Lear,
having failed with his daughters, defied the storm to do its worst. He called out in a voice that was far more
than human:
Blow, winds, and crack your cheeks! rage! blow!
You cataracts and hurricanoes, spout
Till you have drench’d our steeples, drown’d the cocks!
You sulph’rous and thought-executing fires.
Vaunt-couriers to oak-cleaving thunderbolts,
Singe my white head! And thou, all-shaking thunder,
Strike flat the thick rotundity o’ th’ world.
Crack Nature’s moulds, all germains spill at once.
That make ingrateful man.”
The Fool interrupts, his voice shrilling, and making Lear’s defiance the more heroic by contrast.
He begs Lear to make his way back to the castle and make peace with his daughters, but Lear doesn’t even
hear him. He roars on:
Rumble thy bellyful! Spit, fire! spout, rain!
Nor rain, wind, thunder, fire are my daughters.
I tax not you, you elements, with unkindness.
I never gave you kingdom, call’d you children,
You owe me no subscription. Then let fall
Your horrible pleasure. Here I stand your slave,
A poor, infirm, weak, and despis’d old man....”
The Duke of Kent, Lear’s loyal servant (though the King in a fit of rage has banished him) finds
Lear and tries to lead him to some shelter. After an interlude in the castle of the Duke of Gloucester, the
scene returns to Lear in the storm, and he is brought, or rather dragged, to a hovel.
And then, finally, Lear learns to think of others. He insists that the Fool enter first and then he
lingers outside to think (undoubtedly for the first time in his life) of the plight of those who are not kings
and courtiers.
His image shrank and the wildness of his face smoothed out. His head was lifted to the rain, and
his words seemed detached and to be coming not quite from him, as though he were listening to someone
else read the speech. It was, after all, not the old Lear speaking, but a new and better Lear, refined and
sharpened by suffering. With an anxious Kent watching, and striving to lead him into the hovel, and with
Meg Cathcart managing to work up an impression of beggars merely by producing the fluttering of rags,
Lear says:
“Poor naked wretches, wheresoe’er you are..
That bide the pelting of this pitiless storm.
How shall your houseless heads and unfed sides..
Your loop’d and window’d raggedness, defend you
From seasons such as these? O, I have ta’en
Too little care of this! Take physic, pomp;
Expose thyself to feel what wretches feel,
That thou mayst shake the superflux to them
And show the heavens more just.
“Not bad,” said Wilbur, eventually. “We’re getting the idea.
Only, Meg, rags aren’t enough. Can you manage an impression of hollow eyes? Not blind ones.
The eyes are there, but sunken in.”
“I think I can do that,” said Cathcart.
It was difficult for Willard to believe. The money spent was greater than expected. The time it had
taken was considerably greater than had been expected. And the general weariness was far greater than had
been expected. Still, the project was coming to an end.
He had the reconciliation scene to get through--so simple that it would require the most delicate
touches. There would be no background, no souped-up voices, no images, for at this point Shakespeare
became simple. Nothing beyond simplicity was needed.
Lear was an old man, just an old man. Cordelia, having found him, was a loving daughter, with
none of the majesty of Goneril, none of the cruelty of Regan, just softly endearing.
Lear, his madness burned out of him, is slowly beginning to understand the situation. He scarcely
recognizes Cordelia at first and thinks he is dead and she is a heavenly spirit. Nor does he recognize the
faithful Kent.
When Cordelia tries to bring him back the rest of the way to sanity, he says:
“Pray, do not mock me.
I am a very foolish fond old man.
Fourscore and upward, not an hour more nor less.
And, to deal plainly,
I fear I am not in my perfect mind.
Methinks I should know you, and know this man;
Yet I am doubtful; for I am mainly ignorant
What place this is; and all the skills I have
Remembers not these garments; nor I know not
Where I did lodge last night. Do not laugh at me;
For (as I am a man) I think this lady
To be my child Cordelia.”
Cordelia tells him she is and he says:
“Be your tears wet? Yes, faith. I pray weep not.
If you have poison for me, I will drink it.
I know you do not love me; for your sisters
Have, as I do remember, done me wrong.
You have some cause, they have not.”
All poor Cordelia can say is “No cause, no cause.”
And eventually, Willard was able to draw a deep breath and say, “We’ve done all we can do. The
rest is in the hands of the public.”
It was a year later that Willard, now the most famous man in the entertainment world, met
Gregory Laborian. It had come about almost accidentally and largely because of the activities of a mutual
friend. Willard was not grateful.
He greeted Laborian with what politeness he could manage and cast a cold eye on the time-strip
on the wall.
He said, “I don’t want to seem unpleasant or inhospitable, Mr.--uh--but I’m really a very busy
man, and don’t have much time.”
“I’m sure of it, but that’s why I want to see you. Surely, you want to do another compu-drama.”
“Surely I intend to, but,” and Willard smiled dryly, “ King Lear is a hard act to follow and I don’t
intend to turn out something that will seem like trash in comparison.”
“But what if you never find anything that can match King Lear?”
“I’m sure I never will, but I’ll find something. “
“I have something. “
“Oh?”
“I have a story, a novel, that could be made into a compudrama.”
“Oh, well. I can’t really deal with items that come in over the transom.”
“I’m not offering you something from a slush pile. The novel has been published and it has been
rather highly thought of.”
“I’m sorry. I don’t want to be insulting. But I didn’t recognize your name when you introduced
yourself.”
“Laborian. Gregory Laborian.”
“But I still don’t recognize it. I’ve never read anything by you. I’ve never heard of you.”
Laborian sighed. “I wish you were the only one, but you’re not. Still, I could give you a copy of
my novel to read.”
Willard shook his head. “That’s kind of you, Mr. Laborian, but I don’t want to mislead you. I have
no time to read it. And even if I had the time--I just want you to understand--I don’t have the inclination.”
“I could make it worth your while, Mr. Willard. “
“In what way?”
“I could pay you. I wouldn’t consider it a bribe, merely an offer of money that you would well
deserve if you worked with my novel.”
“I don’t think you understand, Mr. Laborian, how much money it takes to make a first-class
compu-drama. I take it you’re not a multimillionaire.”
“No, I’m not, but I can pay you a hundred thousand globodollars.”
“If that’s a bribe, at least it’s a totally ineffective one. For a hundred thousand globo-dollars, I
couldn’t do a single scene.”
Laborian sighed again. His large brown eyes looked soulful. “I understand, Mr. Willard, but if
you’ll just give me a few more minutes--” (for Willard’s eyes were wandering to the time-strip again.)
“Well, five more minutes. That’s all I can manage really. “
“It’s all I need. I’m not offering the money for making the compu-drama. You know, and I know,
Mr. Willard, that you can go to any of a dozen people in the country and say you are doing a compu-drama
and you’ll get all the money you need. After King Lear, no one will refuse you anything, or even ask you
what you plan to do. I’m offering you one hundred thousand globo-dollars for your own use.”
“Then it is a bribe, and that won’t work with me. Good-bye, Mr. Laborian.”
“Wait. I’m not offering you an electronic switch. I don’t suggest that I place my financial card into
a slot and that you do so, too, and that a hundred thousand globo-dollars be transferred from my account to
yours. I’m talking gold, Mr. Willard.”
Willard had risen from his chair, ready to open the door and usher Laborian out, but now he
hesitated. “What do you mean, gold?”
“I mean that I can lay my hands on a hundred thousand globo-dollars of gold, about fifteen
pounds’ worth, I think. I may not be a multimillionaire, but I’m quite well off and I wouldn’t be stealing it.
It would be my own money and I am entitled to draw it in gold. There is nothing illegal about it. What I am
offering you is a hundred thousand globo-dollars in five-hundred globo-dollar pieces--two hundred of
them. Gold, Mr. Willard.”
Gold! Willard was hesitating. Money, when it was a matter of electronic exchange, meant nothing.
There was no feeling of either wealth or of poverty above a certain level. The world was a matter of plastic
cards (each keyed to a nucleic acid pattern) and of slots, and all the world transferred, transferred,
transferred.
Gold was different. It had a feel. Each piece had a weight. Piled together it had a gleaming beauty.
It was wealth one could appreciate and experience. Willard had never even seen a gold coin, let alone felt
or hefted one. Two hundred of them!
He didn’t need the money. He was not so sure he didn’t need the gold.
He said, with a kind of shamefaced weakness. “What kind of a novel is it that you are talking
about?”
“Science fiction.“
Willard made a face. “I’ve never read science fiction. “
“Then it’s time you expanded your horizons, Mr. Willard. Read mine. If you imagine a gold coin
between every two pages of the book, you will have your two hundred.”
And Willard, rather despising his own weakness, said, “What’s the name of your book?”
“Three in One. “
“And you have a copy?”
“I brought one with me.“
And Willard held out his hand and took it.
That Willard was a busy man was by no means a lie. It took him better than a week to find the
time to read the book, even with two hundred pieces of gold glittering, and luring him on.
Then he sat a while and pondered. Then he phoned Laborian.
The next morning, Laborian was in Williard’s office again.
Willard said, bluntly, “Mr. Laborian, I have read your book.”
Laborian nodded and could not hide the anxiety in his eyes. “I hope you like it, Mr. Willard.”
Willard lifted his hand and rocked it right and left. “So-so. I told you I have not read science
fiction, and I don’t know how good or bad it is of its kind--”
“Does it matter, if you liked it?”
“I’m not sure if I liked it. I’m not used to this sort of thing. We are dealing in this novel with three
sexes.”
“Yes.”
“Which you call a Rational, an Emotional, and a Parental. “
“Yes.”
“But you don’t describe them?”
Laborian looked embarrassed. “I didn’t describe them, Mr. Willard, because I couldn’t. They’re
alien creatures, really alien. I didn’t want to pretend they were alien by simply giving them blue skins or a
pair of antennae or a third eye. I wanted them indescribable, so I didn’t describe them, you see.”
“What you’re saying is that your imagination failed.”
“N--no. I wouldn’t say that. It’s more like not having that kind of imagination. I don’t describe
anyone. If I were to write a story about you and me, I probably wouldn’t bother describing either one of
us.”
Willard stared at Laborian without trying to disguise his contempt. He thought of himself. Middlesized,
soft about the middle, needed to reduce a bit, the beginnings of a double chin, and a mole on his right
wrist. Light brown hair, dark blue eyes, bulbous nose. What was so hard to describe? Anyone could do it. If
you had an imaginary character, think of someone real--and describe.
There was Laborian, dark in complexion, crisp curly black hair, looked as though he needed a
shave, probably looked that way all the time, prominent Adam’s apple, small scar on the right cheek, dark
brown eyes rather large, and his only good feature.
Willard said, “I don’t understand you. What kind of writer are you if you have trouble describing
things? What do you write?”
Laborian said, gently, somewhat as though this was not the first time he had had to defend himself
along those lines, “You’ve read Three in One. I’ve written other novels and they’re all in the same style.
Mostly conversation. I don’t see things when I write; I hear, and for most part, what my characters talk
about are ideas--competing ideas. I’m strong on that and my readers like it.”
“Yes, but where does that leave me? I can’t devise a compudrama based on conversation alone. I
have to create sight and sound and subliminal messages, and you leave me nothing to work on.”
“Are you thinking of doing Three in One, then?”
“Not if you give me nothing to work on. Think, Mr. Laborian, think! This Parental. He’s the dumb
one.”
“Not dumb,” said Laborian, frowning. “Single-minded. He only has room in his mind for children,
real and potential.”
“Blockish! If you didn’t use that actual word for the Parental in the novel, and I don’t remember
offhand whether you did or not, it’s certainly the impression I got. Cubical. Is that what he is?”
“Well, simple. Straight lines. Straight planes. Not cubical. Longer than he is wide.”
“How does he move? Does he have legs?”
“I don’t know. I honestly never gave it any thought.”
“Hmp. And the Rational. He’s the smart one and he’s smooth and quick. What is he? Eggshaped?”
“I’d accept that. I’ve never given that any thought, either, but I’d accept that.”
“And no legs?”
“I haven’t described any.”
“And how about the middle one. Your ‘she’ character--the other two being ‘he’s.”‘
“The Emotional.”
“That’s right. The Emotional. You did better on her.”
“Of course. I did most of my thinking about her. She was trying to save the alien intelligences--us-
-of an alien world, Earth. The reader’s sympathy must be with her, even though she fails.”
“I gather she was more like a cloud, didn’t have any firm shape at all, could attenuate and tighten.”
“Yes, yes. That’s exactly right.”
“Does she flow along the ground or drift through the air?”
Laborian thought, then shook his head. “I don’t know. I would say you would have to suit yourself
when it came to that. “
“I see. And what about the sex?”
Laborian said, with sudden enthusiasm. “That’s a crucial point. I never have any sex in my novels
beyond that which is absolutely necessary and then I manage to refrain from describing it--”
“You don’t like sex?”
“I like sex fine, thank you. I just don’t like it in my novels. Everyone else puts it in and, frankly, I
think that readers find its absence in my novels refreshing; at least, my readers do. And I must explain to
you that my books do very well. I wouldn’t have a hundred thousand dollars to spend if they didn’t.”
“All right. I’m not trying to put you down.”
“However, there are always people who say I don’t include sex because I don’t know how, so--out
of vainglory, I suppose--I wrote this novel just to show that I could do it. The entire novel deals with sex.
Of course, it’s alien sex, not at all like ours.”
“That’s right. That’s why I have to ask you about the mechanics of it. How does it work?”
Laborian looked uncertain for a moment. “They melt.”
“I know that that’s the word you use. Do you mean they come together? Superimpose?”
“I suppose so.”
Willard sighed. “How can you write a book without knowing anything about so fundamental a
part of it?”
“I don’t have to describe it in detail. The reader gets the impression. With subliminal suggestion
so much a part of the compu-drama, how can you ask the question?”
Willard’s lips pressed together. Laborian had him there. “Very well. They superimpose. What do
they look like after they have superimposed? “
Laborian shook his head. “I avoided that. “
“You realize, of course, that I can’t.” Laborian nodded. “Yes.”
Willard heaved another sigh and said, “Look, Mr. Laborian, assuming that I agree to do such a
compu-drama--and I have not yet made up my mind on the matter--I would have to do it entirely my way. I
would tolerate no interference from you. You have ducked so many of your own responsibilities in writing
the book that I can’t allow you to decide suddenly that you want to participate in my creative endeavors.”
“That’s quite understood, Mr. Willard. I only ask that you keep my story and as much of my
dialogue as you can. All of the visual, sonic, and subliminal aspects I am willing to leave entirely in your
hands.”
“You understand that this is not a matter of a verbal agreement which someone in our industry,
about a century and a half ago, described as not worth the paper it was written on. There will have to be a
written contract made firm by my lawyers that will exclude you from participation.”
“My lawyers will be glad to look over it, but I assure you I am not going to quibble.”
“And, “ said Willard severely, “I will want an advance on the money you offered me. I can’t afford
to have you change your mind on me and I am not in the mood for a long lawsuit.”
At this, Laborian frowned. He said, “Mr. Willard, those who know me never question my financial
honesty. You don’t know me so I’ll permit the remark, but please don’t repeat it. How much of an advance
do you wish?”
“Half,” said Willard, briefly.
Laborian said, “I will do better than that. Once you have obtained the necessary commitments
from those who will be willing to put up the money for the compu-drama and once the contract between us
is drawn up, then I will give you every cent of the hundred thousand dollars even before you begin the first
scene of the book.”
Willard’s eyes opened wide and he could not prevent himself from saying, “Why?”
“Because I want to urge you on. What’s more, if the compu-drama turns out to be too hard to do,
if it won’t work, or if you turn out something that will not do--my hard luck--you can keep the hundred
thousand. It’s a risk I’m ready to take.”
“Why? What’s the catch?”
“No catch. I’m gambling on immorality. I’m a popular writer but I have never heard anyone call
me a great one. My books are very likely to die with me. Do Three in One as a compu-drama and do it well
and that at least might live on, and make my name ring down through the ages,” he smiled ruefully, “or at
least some ages. However--”
“Ah,” said Willard. “Now we come to it.”
“Well, yes. I have a dream that I’m willing to risk a great deal for, but I’m not a complete fool. I
will give you the hundred thousand I promised before you start and if the thing doesn’t work out you can
keep it, but the payment will be electronic. It: however, you turn out a product that satisfies me, then you
will return the electronic gift and I will give you the hundred thousand globo-dollars in gold pieces. You
have nothing to lose except that to an artist like yourself, gold must be more dramatic and worthwhile than
blips in a finance-card. “ And Laborian smiled gently.
Willard said, “Understand, Mr. Laborian! I would be taking a risk, too. I risk losing a great deal of
time and effort that I might have devoted to a more likely project. I risk producing a docudrama that will be
a failure and that will tarnish the reputation I have built up with Lear. In my business, you’re only as good
as your most recent product. I will consult various people--”
“On a confidential basis, please.”
“Of course! And I will do a bit of deep consideration. I am willing to go along with your
proposition for now, but you mustn’t think of it as a definite commitment. Not yet. We will talk further.”
Jonas Willard and Meg Cathcart sat together over lunch in Meg’s apartment. They were at their
coffee when Willard said, with apparent reluctance as one who broaches a subject he would rather not,
“Have you read the book? “
“Yes, I have.”
“And what did you think?”
“I don’t know,” said Cathcart peering at him from under the dark, reddish hair she wore clustered
over her forehead. “ At least not enough to judge.”
“You’re not a science fiction buff either, then?”
“Well, I’ve read science fiction, mostly sword and sorcery, but nothing like Three in One. I’ve
heard of Laborian, though. He does what they call ‘hard science fiction.”‘
“It’s hard enough. I don’t see how I can do it. That book, whatever its virtues, just isn’t me.”
Cathcart fixed him with a sharp glance. “How do you know it isn’t you?”
“Listen, it’s important to know what you can’t do.”
“And you were born knowing you can’t do science fiction?”
“I have an instinct in these things.”
“So you say. Why don’t you think what you might do with those three undescribed characters, and
what you would want subliminally, before you let your instinct tell you what you can and can’t do. For
instance, how would you do the Parental, who is referred to constantly as ‘he’ even though it’s the Parental
who bears the children? That struck me as jackassy, if you must know.”
“No, no,” said Willard, at once. “I accept the ‘he.’ Laborian might have invented a third pronoun,
but it would have made no sense and the reader would have gagged on it. Instead, he reserved the pronoun,
‘she,’ for the Emotional. She’s the central character, differing from the other two enormously. The use of
‘she’ for her and only for her focuses the reader’s attention on her, and it’s on her that the reader’s attention
must focus. What’s more, it’s on her that the viewer’s attention must focus in the compu-drama.”
“Then you have been thinking of it. “ She grinned, impishly. “I wouldn’t have known if I hadn’t
needled you.”
Willard stirred uneasily. “ Actually, Laborian said something of the sort, so I can’t lay claim to
complete creativity here. But let’s get back to the Parental. I want to talk about these things to you because
everything is going to depend on subliminal suggestion, if I do try to do this thing. The Parental is a block,
a rectangle.”
“A right parallelepiped, I think they would call it in solid geometry.”
“Come on. I don’t care what they call it in solid geometry. The point is we can’t just have a block.
We have to give it personality. The Parental is a ‘he’ who bears children, so we have to get across an
epicene quality. The voice has to be neither clearly masculine nor feminine. I’m not sure that I have in
mind exactly the timbre and sound I will need, but that will be for the voice-recorder and myself to work
out by trial and error, I think. Of course, the voice isn’t the only thing.”
“What else?”
“The feet. The Parental moves about, but there is no description of any limbs. He has to have the
equivalent of arms; there are things he does. He obtains an energy source that he feeds the Emotional, so
we’ll have to evolve arms that are alien but that are arms. And we need legs. And a number of sturdy,
stumpy legs that move rapidly.”
“Like a caterpillar? Or a centipede?”
Willard winced. “Those aren’t pleasant comparisons, are they?”
“Well, it would be my job to subliminate, if I may use the expression, a centipede, so to speak,
without showing one. Just the notion of a series of legs, a double fading row of parentheses, just on and off
as a kind of visual leitmotiv for the Parental, whenever he appears.”
“I see what you mean. We’ll have to try it out and see what we can get away with. The Rational is
ovoid. Laborian admitted it might be egg-shaped. We can imagine him progressing by rolling but I find that
completely inappropriate. The Rational is mind-proud, dignified. We can’t make him do anything
laughable, and rolling would be laughable.”
“We could have him with a flat bottom slightly curved, and he could slide along it, like a penguin
belly-whopping.”
“Or like a snail on a layer of grease. No. That would be just as bad. I had thought of having three
legs extrude. In other words, when he is at rest, he would be smoothly ovoid and proud of it, but when he is
moving three stubby legs emerge and he can walk on them.”
“Why three?”
“It carries on the three motif; three sexes, you know. It could be a kind of hopping run. The foreleg
digs in and holds firm and the two hind legs come along on each side.”
“Like a three-legged kangaroo?”
“Yes! Can you subliminate a kangaroo?”
“I can try.”
“The Emotional, of course, is the hardest of the three. What can you do with something that may
be nothing but a coherent cloud of gas?”
Cathcart considered. “What about giving the impression of draperies containing nothing. They
would be moving about wraithlike, just as you presented Lear in the storm scene. She would be wind, she
would be air, she would be the filmy, foggy draperies that would represent that.”
Willard felt himself drawn to the suggestion. “Hey, that’s not bad, Meg. For the subliminal effect,
could you do Helen of Troy?”
“Helen of Troy?”
“Yes! To the Rational and Parental, the Emotional is the most beautiful thing ever invented.
They’re crazy about her. There’s this strong, almost unbearable sexual attraction--their kind of sex--and
we’ve got to make the audience aware of it in their terms. If you can somehow get across a statuesque
Greek woman, with bound hair and draperies--the draperies would exactly fit what we’re imagining for the
Emotional--and make it look like the paintings and sculptures everyone is familiar with, that would be the
Emotional’s leitmotiv.”
“You don’t ask simple things. The slightest intrusion of a human figure will destroy the mood.”
“You don’t intrude a human figure. Just the suggestion of one. It’s important. A human figure, in
actual fact, may destroy the mood, but we’ll have to suggest human figures throughout. The audience has to
think of these odd things as human beings. No mistake.”
“I’ll think about it,” said Cathcart, dubiously.
“Which brings us to another thing. The melting. The triple-sex of these things. I gather they
superimpose. I gather from the book that the Emotional is the key to that. The Parental and Rational can’t
melt without her. She’s the essential part of the process. But, of course, that fool, Laborian, doesn’t
describe it in detail. Well, we can’t have the Rational and Parental running toward the Emotional and
jumping on her. That would kill the drama at once no matter what else we might do.”
“I agree.”
“What we must do, then, and this is off the top of my head, is to have the Emotional expand, the
draperies move out and enswathe (if that’s the word) both Parental and Rational. They are obscured by the
draperies and we don’t see exactly how it’s done but they get closer and closer until they superimpose.”
“We’ll have to emphasize the drapery,” said Cathcart. “We’ll have to make it as graceful as
possible in order to get across the beauty of it, and not just the eroticism. We’ll have to have music.”
“Not the Romeo and Juliet overture, please. A slow waltz, perhaps, because the melting takes a
long time. And not a familiar one. I don’t want the audience humming along with it. In fact, it would be
best if it comes in occasional bits so that the audience gets the impression of a waltz, rather than actually
hearing it.”
“We can’t see how to do it, until we try it and see what works.”
“Everything I say now is a first-order suggestion that may have to be yanked about this way and
that under the pressure of actual events. And what about the orgasm? We’ll have to indicate that
somehow.”
“Color.”
“Hmm.”
“Better than sound, Jonas. You can’t have an explosion. I wouldn’t want some kind of eruption,
either. Color. Silent color. That might do it.”
“What color ? I don’t want a blinding flash, either. “
“No. You might try a delicate pink, very slowly darkening, and then toward the end suddenly
becoming a deep, deep red.”
“I’m not sure. We’ll have to try it out. It must be unmistakable and moving and not make the
audience giggle or feel embarrassed. I can see ourselves running through every color change in the
spectrum, and, in the end, finding that it will depend on what you do subliminally. And that brings us to the
triple-beings.”
“The what?”
“You know. After the last melting, the superimposition remains permanent and we have the adult
form that is all three components together. There, I think, we’ll have to make them more human. Not
human, mind you. Just more human. A faint suggestion of human form, not just subliminal, either. We’ll
need a voice that is somehow reminiscent of all three, and I don’t know how the recorder can make that
work. Fortunately, the triple-beings don’t appear much in the story.”
Willard shook his head. “ And that brings us to the rough fact that the compu-drama might not be
a possible project at all.”
“Why not? You seem to have been offering potential solutions of all kinds for the various
problems.”
“Not for the essential part. look! In King Lear, we had human characters, more than human
characters. You had searing emotions. What have we got here? We have funny little cubes and ovals and
drapery. Tell me how my Three in One is going to be different from an animated cartoon?”
“For one thing, an animated cartoon is two-dimensional. Even with elaborate animation it is flat,
and its coloring is without shading. It is invariably satiricial--”
“I know all that. That’s not what I want you to tell me. You’re missing the important point. What a
compu-drama has, that a mere animated cartoon does not, are subliminal suggestions such as can only be
created by a: complex computer in the hands of an imaginative genius. What my compu-drama has that an
animated cartoon doesn’t is you, Meg.”
“Well, I was being modest. “
“Don’t be. I’m trying to tell you that everything--everything--is going to depend on you. We have
a story here that is dead serious. Our Emotional is trying to save Earth out of pure idealism; it’s not her
world. And she doesn’t succeed, and she won’t succeed in my version, either. No cheap, happy ending.”
“Earth isn’t exactly destroyed.”
“No, it isn’t. There’s still time to save it if Laborian ever gets around to doing a sequel, but in this
story the attempt fails. It’s a tragedy and I want it treated as one--as tragic as Lear. No funny voices, no
humorous actions, no satirical touches. Serious. Serious. Serious. And I’m going to depend on you to make
it so. It will be you who makes sure that the audience reacts to the Rational, the Emotional, the Parental, as
though they were human beings. All their peculiarities will have to melt away and they’ll have to be
recognized as intelligent beings on a par with humanity, if not ahead of it. Can you do it?”
Cathcart said dryly, “It looks as though you will insist I can.”
“I do so insist.”
“Then you had better see about getting the ball rolling, and you leave me alone while you’re doing
it. I need time to think. Lots of time.”
The early days of the shooting were an unmitigated disaster. Each member of the crew had his
copy of the book, carefully, almost surgically trimmed, but with no scenes entirely omitted.
“We’re going to stick to the course of the book as closely as we can, and improve it as we go
along just as much as we can,” Willard had announced confidently. “ And the first thing we do is get a hold
on the triple-beings.”
He turned to the head voice-recorder. “How have you been working on that?”
“I’ve tried to fuse the three voices. “
“Let’s hear. All right, everyone quiet.”
“I’ll give you the Parental first,” said the recorder. There came a thin, tenor voice, out of key with
the blockish figure that the Image man had produced. Willard winced slightly at the mismatch, but the
Parental was mismatched--a masculine mother. The Rational, rocking slowly back and forth, had a
somewhat self-important voice; enunciation over-careful, and it was a light baritone.
Willard interrupted. “Less rocking in the Rational. We don’t want the audience to become seasick.
He rocks when he is deep in thought, and not all the time.”
He then nodded his head at Dua’s draperies, which seemed quite successful, as did her clear and
infinitely sweet soprano voice.
“She must never shriek,” said Willard, severely, “not even when she is in a passion.”
“She won’t,” said the recorder. “The trick is, though, to blend the voices in setting up the triplebeing,
in having each one distantly identifiable.”
All three voices sounded softly, the words not clear. They seemed to melt into each other and then
the voice could be heard enunciating.
Willard shook his head in immediate discontent. “No, that won’t do at all. We can’t have three
voices in a kind of intimate patchwork. We’d be making the triple-being a figure of fun. We need one voice
which somehow suggests all three.”
The voice-recorder was clearly offended. “It’s easy to say that. How do you suggest we do it?”
“I do it,” said Willard, brutally, “by ordering you to do it. I’ll tell you when you have it. And
Cathcart--where is Cathcart?”
“Here I am,” she said, emerging from behind her instrumentation. “Where I’m supposed to be.”
“I don’t like the sublimination, Cathcart. I gather you tried to give the impression of cerebral
convolutions.”
“For intelligence. The triple-beings represent the intelligence-peak of these aliens.”
“Yes, I understand, but what you managed to do was to give the impression of worms. You’ll have
to think of something else. And I don’t like the appearance of the triple-being, either. He looks just like a
big Rational.”
“He is like a big Rational,” said one of the imagists.
“Is he described in the book that way?” asked Willard, sharply.
“Not in so many words, but the impression I get--”
“Never mind your impression. I’ll make the decisions.”
Willard grew fouler-tempered as the day wore on. At least twice he had difficulty controlling his
passion, the second time coming when he happened to notice someone watching the proceedings from a
spot at one edge of the lot.
He strode toward him angrily. “What are you doing here?”
It was Laborian, who answered quietly, “Watching. “
“Our contract states--”
“That I am to interfere in the proceedings in no way. It does not say I cannot watch quietly.”
“You’ll get upset if you do. This is the way preparing a compudrama works. There are lots of
glitches to overcome and it would be upsetting to the company to have the author watching and
disapproving.”
“I’m not disapproving. I’m here only to answer questions if you care to ask them.”
“Questions? What kind of questions?”
Laborian shrugged. “I don’t know. Something might puzzle you and you might want a
suggestion.”
“I see,” said Willard, with heavy irony, “so you can teach me my business.”
“No, so I can answer your questions.”
“Well, I have one.”
“Very well,” and Laborian produced a small cassette recorder. “If you’ll just speak into this and
say that you are asking me a question and wish me to answer without prejudicing the contract, we’re in
business.”
Willard paused for a considerable time, staring at Laborian as though he suspected trickery of
some sort, then he spoke into the cassette.
“Very well,” said Laborian. “What’s your question?”
“Did you have anything in mind for the appearance of the triple-being in the book?”
“Not a thing,” said Laborian, cheerfully.
“How could you do that?” Willard’s voice trembled as though he were holding back a final “you
idiot” by main force.
“Easily. What I don’t describe, the reader supplies in his own mind. Each reader does it differently
to suit himself, I presume. That’s the advantage of writing. A compu-drama would have an enormously
larger audience than a book could have, but you must pay for that by having to present an image.”
“I understand that,” said Willard. “So much for the question, then.”
“Not at all. I have a suggestion.”
“Like what?”
“Like a head. Give the triple-being a head. The Parental has no head, nor the Rational, nor the
Emotional, but all three look up to the triple-beings as creatures of intelligence beyond their own. That is
the entire difference between the triple-beings and the three Separates. Intelligence.”
“A head?”
“Yes. We associate intelligence with heads. The head contains the brain, it contains the sense
organs. Omit the head and we cannot believe in intelligence. The headless oysters or clams are mollusks
that seem no more intelligent to us than a spring of grass would be, but the related octopus, also a mollusk,
we accept as possibly intelligent because it has a head--and eyes. Give the triple-being eyes, too.”
Work had, of course, ceased on the set. Everyone had gathered in as closely as they thought
judicious to listen to the conversation between director and author.
Willard said, “What kind of head?”
“Your choice. All you need is a bulge suggesting a head. And eyes. The viewer is sure to get the
idea.”
Willard turned away, shouting, “Well, get back to work. Who called a vacation? Where are the
imagists? Back to the machine and begin trying out heads.”
He turned suddenly and said, in an almost surly fashion, to Laborian. “Thank you! “
“Only if it works,” said Laborian, shrugging.
The rest of the day was spent in testing heads, searching for one that was not a humorous bulge,
and not an unimaginative copy of the human head, and eyes that were not astonished circles or vicious slits.
Then, finally, Willard called a halt and growled, “We’ll try again tomorrow. If anyone gets any brilliant
thoughts overnight, give them to Meg Cathcart. She’ll pass on to me any that are worth it.” And he added,
in an annoyed mutter, “I suppose she’ll have to remain silent.”
Willard was right and wrong. He was right. There were no brilliant ideas handed to him, but he
was wrong for he had one of his own.
He said to Cathcart, “Listen, can you get across a top hat?”
“A what?”
“The sort of thing they wore in Victorian times. Look, when the Parental invades the lair of the
triple-beings to steal an energy source, he’s not an impressive sight in himself, but you told me you could
just get across the idea of a helmet and a long line that will give the notion of a spear. He’ll be on a knightly
quest.”
“Yes, I know,” she said, “but it might not work. We’ll have to try it out.”
“Of course, but that points the direction. If you have just a suggestion of a top hat, it will give the
impression of the triple-being as an aristocrat. The exact shape of the head and eyes becomes less crucial in
that case. Can it be done?”
“Anything can be done. The question is: will it work? “
“We’ll try it.”
And as it happened, one thing led to another. The suggestion of the top hat caused the voicerecorder
to say, “Why not give the triple-being a British accent? “
Willard was caught off-guard. “Why?”
“Well, the British have a language with more tones than we do. At least, the upper classes do. The
American version of English tends to be flat, and that’s true of the Separates, too. If the triple-being spoke
British rather than English, his voice could rise and fall with the words--tenor and baritone and even an
occasional soprano squeak. That’s what we would want to indicate with the three voices out of which his
voice was formed.”
“Can you do that?” said Willard.
“I think so.”
“Then we’ll try. Not bad--if it works.”
It was interesting to see how the entire group found themselves engaged in the Emotional.
The scene in particular where the Emotional was fleeing across the face of the planet, where she
had her brief set-to with the other Emotionals caught at everyone.
Willard said tensely, “This is going to be one of the great dramatic scenes. We’ll put it out as
widely as we can. It’s going to be draperies, draperies, draperies, but they must not be entangled one with
the other. Each one must be distinct. Even when you rush the Emotionals in toward the audience I want
each set of draperies to be a different off-white. And I want Dua’s drapery to be distinct from all of them. I
want her to glitter a little, just to be different, and because she’s our Emotional. Got it?”
“Got it,” said the leading imagist. “We’ll handle it.”
“And another thing. All the other Emotionals twitter. They’re birds. Our Emotional doesn’t
twitter, and she despises the rest because she’s more intelligent than they are and she knows it. And when
she’s fleeing--” he paused, and brooded a bit. “Is there any way we can get away from the ‘Ride of the
Valkyries’?”
“We don’t want to,” said the soundman promptly. “Nothing better for the purpose has ever been
written.”
Cathcart said, “Yes, but we’ll only have snatches of it now and then. Hearing a few bars has the
effect of the whole, and I can insert the hint of tossing manes.”
“Manes?” said Willard, dubiously.“
Absolutely. Three thousand years of experience with horses has pinned us down to the galloping
stallion as the epitome of wild speed. All our mechanical devices are too static, however fast they go. And I
can arrange to have the manes just match, emphasize, and punctuate the flowing of the draperies.”
“That sounds good. We’ll try it.”
Willard knew where the final stumbling block would be found. The last melting. He called the
troupe together to lecture them, partly to make sure they understood what it was they were all doing now,
partly to put off the time of reckoning when they would actually try to put it all into sound, image, and
sublimination.
He said, “ All right, the Emotional’s interest is in saving the other world--Earth--only because she
can’t bear the thought of the meaningless destruction of intelligent beings. She knows the triple-beings are
carrying through a scientific project, necessary for the welfare of her world and caring nothing for the
danger into which it puts the alien world--us.
“She tries to warn the alien world and fails. She knows, at last, that the whole purpose of melting
is to produce a new set of Rational, Emotional and Parental, and then, with that done, there is a final
melting that would turn the original set into a triple-being. Do you have that? It's a sort of larval form of
Separates and an adult form of triples.
“But the Emotional doesn't want to melt. She doesn't want to produce the new generation. Most of
all, she doesn't want to become a triple-being and participate in what she considers their work of
destruction. She is, however, tricked into the final melt and realizes too late that she is not only going to be
a triple-being but a triple-being who will be, more than any other, responsible for the scientific project that
will destroy the other world.
“All this Laborian could describe in words, words, words, in his book, but we've got to do it more
immediately and more forcefully, in images and sublimination as well. That's what we're now going to try
to do.”
They were three days in the trying before Willard was satisfied.
The weary Emotional, uncertain, stretching outward, with Cathcart's sublimination instilling the
feeling of not-sure, not-sure. The Rational and Parental enfolded and coming together, more rapidly than on
previous occasions--hurrying for the superimposition before it might be stopped--and the Emotional
realizing too late the significance of it all and struggling--struggling--
And failure. The drenching feeling of failure as a new triple-being stepped out of the
superimposition, more nearly human than anyone else in the compu-drama--proud, indifferent.
The scientific procedure would go on. Earth would continue the downward slide.
And somehow this was it--this was the nub of everything that Willard was trying to do--that
within the new triple-being the Emotional still existed in part. There was just the wisping of drapery and the
viewer was to know that the defeat was not final after all.
The Emotional would, somehow, still try, lost though it was in a greater being.
They watched the completed compu-drama, all of them, seeing it for the first time as a whole and
not as a collection of parts, wondering if there were places to edit, to reorder. (Not now, thought Willard,
not now. Afterwards, when he had recovered and could look at it more objectively.)
He sat in his chair, slumped. He had put too much of himself into it. It had seemed to him that it
contained everything he wanted it to contain; that it did everything he wanted to have done; but how much
of that was merely wishful thinking?
When it was over and the last tremulous, subliminal cry of the defeated-but-not-yet-defeated
Emotional faded, he said, “Well.”
And Cathcart said, “That’s almost as good as your King Lear was, Jonas.”
There was a general murmur of agreement and Willard cast a cynical eye about him. Wasn’t that
what they would be bound to say, no matter what?
His eye caught that of Gregory Laborian. The writer was expressionless, said nothing.
Willard’s mouth tightened. There at least he could expect an opinion that would be backed, or not
backed, by gold. Willard had his hundred thousand. He would see now whether it would stay electronic.
He said, and his own uncertainty made him sound imperious, “Laborian. I want to see you in my
office.”
They were together alone for the first time since well before the compu-drama had been made.
“Well?” said Willard. “What do you think, Mr. Laborian?”
Laborian smiled. “That woman who runs the subliminal background told you that it was almost as
good as your King Lear was, Mr. Willard.”
“I heard her.”
“She was quite wrong. “
“In your opinion?”
“Yes. My opinion is what counts right now. She was quite wrong. Your Three in One is much
better than your King Lear.”
“Better?” Willard’s weary face broke into a smile.
“Much better. Consider the material you had to work with in doing King Lear. You had William
Shakespeare, producing words that sang, that were music in themselves; William Shakespeare producing
characters who, whether for good or evil, whether strong or weak, whether shrewd or foolish, whether
faithful or traitorous, were all larger than life; William Shakespeare, dealing with two overlapping plots,
reinforcing each other and tearing the viewers to shreds.
“What was your contribution to King Lear? You added dimensions that Shakespeare lacked the
technological knowledge to deal with; that he couldn’t dream of; but the fanciest technologies and all that
your people and your own talents could do could only build somewhat on the greatest literary genius of all
time, working at the peak of his power.
“But in Three in One, Mr. Willard, you were working with my words which didn’t sing; my
characters, which weren’t great; my plot which tore at no one. You dealt with me, a run-of-the-mill writer
and you produced something great, something that will be remembered long after I am dead. One book of
mine, anyway, will live on because of what you have done.
“Give me back my electronic hundred thousand, Mr. Willard, and I will give you this.”
The hundred thousand was shifted back from one financial card to the other and, with an effort,
Laborian then pulled his fat briefcase onto the table and opened it. From it, he drew out a box, fastened
with a small hook. He unfastened it carefully, and lifted the top. Inside it glittered the gold pieces, each one
marked with the planet Earth, the western hemisphere on one side, the eastern on the other. Large gold
pieces, two hundred of them, each worth five hundred globo-dollars.
Willard, awed, plucked out one of the gold pieces. It weighed about one and a quarter ounces. He
threw it up in the air and caught it.
“Beautiful,” he said.
“It’s yours, Mr. Willard,” said Laborian. “Thank you for doing the compu-drama for me. It is
worth every piece of that gold.”
Willard stared at the gold and said, “You made me do the compu-drama of your book with your
offer of this gold. To get this gold, I forced myself beyond my talents. Thank you for that, and you are
right. It was worth every piece of that gold.”
He put the gold piece back in the box and closed it. Then he lifted the box and handed it back to
Laborian.
\end{document}