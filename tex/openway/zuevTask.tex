%XeLaTeX+makeIndex+BibTeX OR LuaLaTeX+...
\documentclass[a4paper,12pt]{article} %14pt - extarticle
\usepackage[utf8]{inputenc} %русский язык, не менять
\usepackage[T2A, T1]{fontenc} %русский язык, не менять
\usepackage[english, russian]{babel} %русский язык, не менять
\usepackage{fontspec} %различные шрифты
\setmainfont{Times New Roman}
\defaultfontfeatures{Ligatures={TeX},Renderer=Basic}
\usepackage{hyperref} %гиперссылки
\hypersetup{pdfstartview=FitH,  linkcolor=blue, urlcolor=blue, colorlinks=true} %гиперссылки
\usepackage{subfiles}%включение тех-текста
\usepackage{graphicx} %изображения
\usepackage{float}%картинки где угодно
\usepackage{textcomp}

\usepackage{tikz} %vector graphics 
\usetikzlibrary{graphs}


\begin{document}
\author{Валерий Зуев \\ \href{mailto:valera.zuev.zva@gmail.com}{valera.zuev.zva@gmail.com}}
\title{Ответ на вступительное задание летней школы OpenWay 2019}
\maketitle

\section{Задание 1}
Ниже перечислены три достижения моей жизни, которым было уделено больше всего сил и принесшие ожидаемый результат.
\begin{enumerate}
	\item \textbf{История семьи}. В течение трёх лет записывал воспоминания своего дедушки-геолога, затем перевёл в электронный формат. Кратко записал воспоминания бабушки о военных годах. Перевёл в электронный формат фронтовые письма прадеда.
	\item \textbf{Водительские права}. Летом 2016 года отец, работающий в США, пригласил маму со мной и сестрой к себе. Он предложил выучить правила дорожного движения, потренироваться на его автомобиле и сдать экзамен на водительские права. В тот год я сдал теоретическую часть, на следующий год снова прилетел в США и досдал практику.
	\item \textbf{победа в Инженерных соревнованиях}. Весной 2018 года с однокурсниками
\end{enumerate}
\section{Задание 2}

\section{Задание 3}

%TODO: make subfile

\begin{tikzpicture}
\graph[nodes={align=center,rectangle,draw=black}, grow down sep, branch right sep] {
	Диакритика? ->
	{
		"Много над E:\\ \`{E}, \'{E}, \^{E}, \"{E}",
		Мало -> "Хоть что-то есть?" -> 
		{
			"\c{C} и \"{E}" -> Точно не французский? -> 
			{ 
				"Ой$\dots$он",
				Вроде нет -> Албанский[not target]
			},
			"Только \"{A} и \"{O} \\ но мноогоо \\ сдвооенных" -> Финский[not target]
		}
	} -> 
	Французский
};
\end{tikzpicture}

\section{Задание 4}

\section{Задание 5}
См. сопровождающие файлы


	
\end{document}