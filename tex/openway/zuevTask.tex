%XeLaTeX+makeIndex+BibTeX OR LuaLaTeX+...
\documentclass[a4paper,12pt]{article} %14pt - extarticle
\usepackage[utf8]{inputenc} %русский язык, не менять
\usepackage[T2A, T1]{fontenc} %русский язык, не менять
\usepackage[english, russian]{babel} %русский язык, не менять
\usepackage{fontspec} %различные шрифты
\setmainfont{Times New Roman}
\defaultfontfeatures{Ligatures={TeX},Renderer=Basic}
\usepackage{hyperref} %гиперссылки
\hypersetup{pdfstartview=FitH,  linkcolor=blue, urlcolor=blue, colorlinks=true} %гиперссылки
\usepackage{subfiles}%включение тех-текста
\usepackage{graphicx} %изображения
\usepackage{float}%картинки где угодно
\usepackage{textcomp}

\usepackage{tikz} %vector graphics 
\usetikzlibrary{graphs}


\begin{document}
\author{Валерий Зуев} %\\ %\href{mailto:valera.zuev.zva@gmail.com}{valera.zuev.zva@gmail.com}}
\title{Ответ на вступительное задание летней школы OpenWay 2019}
\maketitle

\section{Задание 1}
Ниже перечислены три достижения моей жизни, которым было уделено больше всего сил и принесшие ожидаемый результат.
\begin{enumerate}
	\item \textbf{История семьи}. В течение трёх лет записывал воспоминания своего дедушки-геолога, затем перевёл в электронный формат. Кратко записал воспоминания бабушки о военных годах. Перевёл в электронный формат фронтовые письма прадеда.
	\item \textbf{Водительские права}. Летом 2016 года отец, работающий в США, пригласил маму со мной и сестрой к себе. Он предложил выучить правила дорожного движения, потренироваться на его автомобиле и сдать экзамен для получения водительского удостоверения. В тот год я сдал теоретическую часть, на следующий год снова прилетел в США и досдал практику.
	\item \textbf{Победа в Инженерных соревнованиях}. Весной 2018 года с тремя сокурсниками прошли четыре этапа внутриинститутских Инженерных соревнований, где занимались 3Д-моделированием и прототипированием, решили кейс. Обошли более 15 команд с разных курсов и поехали на чемпионат European BEST Engineering Competition в Москву, правда, на этом этапе не вошли в первую тройку.
\end{enumerate}
\newpage
\section{Задание 2}
\textbf{(Вопрос 1). О чём бы Вы хотели написать книгу?} \\
Когда-то мне в голову приходила идея написать цикл научно-популярных энциклопедий о современных и модных технологиях - 3D-печати, квадрокоптерах, биоинформатике. Подобных энциклопедий мне не хватало самому; пока что эти весьма любопытные направления слабо систематизированы, и при изучении хочется иметь какой-то компас, чтобы составить общее представление о предмете. Но, во-первых, для написания такой литературы желательно самому быть экспертом или глубоко вникать в каждую деталь и советоваться с экспертами; во-вторых, из-за бурного прогресса уже лет через пять после написания даже лучшие энциклопедии будут уже неактуальны.\\
О художественной литературе нельзя сказать, что она устаревает. Другое дело, что в мире её весьма много, как стихов, так и прозы. Вряд ли можно внести что-то кардинально новое, оригинальное и качественное, выражаясь словами Дж. Голсуорси, <<в огромный и неустроенный храм литературы>>.\\
Тем не менее, по моему мнению, каждое время должно иметь свою, актуальную литературу, которая отражает мысль времени, его философию и вопросы, историю и дух эпохи.  Мне хотелось бы написать повесть о жизни в современной России, о поколениях, которые уходят и которые живут сейчас, о проблемах людей и трудностях взаимопонимания между теми, кто был воспитан разными режимами. Наверное, это будет отчасти автобиографический труд, в котором я бы хотел изобразить хороших людей, которых я знаю и которых хочу поставить в пример другим. Помимо того, планирую рассказать об экологической ситуации и об отношении людей к защите природы, о социальных вопросах, о людях побогаче и победнее; о том, как люди приходят в неестественное состояние, о сектантах и карьеристах, об эмигрантах и иммигрантах.\\


\section{Задание 3}
\textbf{(Вопрос 1). Опишите архитектуру программного продукта, который Вам приходилось разрабатывать (или с которым приходилось работать).}
Интересный программный продукт, с которым я сталкивался при создании сайта \url{https://unionpaint.ru} - система управления контентом OpenCart. CMS OpenCart - серьёзный и гибкий инструмент с открытым кодом для создания интернет-магазинов. В основном написан на PHP, HTML и CSS; относительно небольшая часть (анимация, обработка нажатий экранных кнопок) написана на Javascript.\\
OpenCart следует парадигме MVCL (model, view, controller, language). Файлы веб-страниц разделены на четыре группы:
\begin{enumerate}
	\item model - отвечающие за обращения к базе данных 
	\item view - графическая часть
	\item controller - логика
	\item language - шаблоны текста, которые подставляются в файлы графического представления
\end{enumerate}
Для файлов графической части введено особое расширение .tpl, но по существу это PHP-код шаблона станицы. Каждому шаблону раздела сайта соответствует свой .php-файл из части controller, который выводит итоговую страницу с учётом параметров сайта; он же подставляет в нужные поля результаты запросов к базе данных и обрабатывает действия пользователя. Иерархию обращений можно изобразить в виде диаграммы:\\
\begin{figure}[H]
	\center{\includegraphics[width=.75\textwidth]{pict/hierarchy.png}}
	\caption{Порядок взаимодействия между файлами сайта}
	\label{hierarchy}
\end{figure}
Отдельная часть <<language>> удобна при переводе разделов сайта на другие языки.\\
Сайт делится на административную часть (\textit{admin}) и открытую часть, <<витрину>> (\textit{catalog}). Соответственно названы корневые директории файловой системы сайта на OpenCart; содержимое обеих размещено в подпапках model, view, control, language. Административная панель отделена от витрины. Любой раздел сайта может иметь административную и пользовательскую часть, административная - для настройки и наполнения пользовательской.\\
Помимо \textit{admin} и \textit{catalog} есть корневая директория \textit{system}, в которой размещены важные конфигурационные файлы, воздействующие на весь сайт, и служебные файлы. Основные конфигурационные параметры (URL-адрес сервера, пути к директориям сайта на диске сервера и параметры базы данных) вынесены в файлы \textit{config.php} и \textit{admin/config.php}.\\
Для персонализации сайта предусмотрены два механизма - темы (\textit{themes)} и дополнения (\textit{extensions}). Тема меняет элементы внешнего облика всего сайта, дополнения - только отдельные разделы (возможно, даже только в административной части - например, дополнение для экспорта каталога в csv-файл). Темы - наборы css-файлов и javascript-сценариев; дополнения - полноценные разделы сайта со своими наборами model, view, control, language - файлов в административной части и витрине. Дополнения устанавливаются с помощью утилиты OCMOD или VQmod: установщику передаётся xml-файл с инструкциями, в каких файлах нужно внести поправки для корректной работы  сайта (например, добавить нужные названия разделов в навигационную панель).

\begin{flushright}
	Использованные материалы: \href{https://docs.ocstore.com/index.php?title=%D0%A0%D1%83%D0%BA%D0%BE%D0%B2%D0%BE%D0%B4%D1%81%D1%82%D0%B2%D0%BE_%D0%A0%D0%B0%D0%B7%D1%80%D0%B0%D0%B1%D0%BE%D1%82%D1%87%D0%B8%D0%BA%D0%B0}{Opencart - руководство разработчика}
\end{flushright}
		

\section{Задание 4}
\textbf{(Вопрос 1). Расскажите о }

\section{Задание 5}
См. сопровождающие файлы


	
\end{document}